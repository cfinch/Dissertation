
\chapter{Fibronectin Adsorption}

Fibronectin (FN) is a physiologically important protein in vertebrates.
It is abundant in plasma and other bodily fluids, and plays an important
role in the extracellular matrix. The structure of fibronectin is
complex. The primary structure of FN is a chain composed of three
types of repeated modules. Although only one gene codes for FN, alternative
splicing of the pre-mRNA results in numerous variants in which modules
are added or deleted \cite{Pankov2002}. X-ray crystallography has
been used to determine the secondary and tertiary structure of the
three types of modules. The complete FN molecule has not been crystallized,
so its secondary and tertiary structure are unknown. It is likely
that the secondary and tertiary structure are highly dependent on
the local environment. In solution, fibronectin exists as a dimer
with two identical subunits linked by disulfide bonds. In the extracellular
matrix, fibronectin is assembled into a fibrillar network\cite{Mao2005}.


\section{Methods and Materials}


\subsection{Experimental Methods}


\subsubsection*{Whispering Gallery Mode Sensor}

A whispering gallery mode sensor system was constructed as described
in {[}{]}. A schematic overview of the system is shown in Figure \ref{fig:WGM System Diagram}.%
\begin{figure}
\includegraphics{Fibronectin/Figures/WGM_System_Overview}

\caption{\label{fig:WGM System Diagram}Schematic diagram of the whispering
gallery mode biosensor}



\end{figure}



\subsubsection{Protein Adsorption Experiments}

Bovine plasma fibronectin in solution was obtained from Sigma-Aldrich
(F1141) and diluted to the desired concentrations with 50mM phosphate
buffered saline (PBS) at pH 7.4.


\subsection{Modeling the Kinetics of Fibronectin Adsorption}

It is well established that single-layer adsorption models can be
stated in the general form\begin{equation}
\frac{d\theta}{dt}=k_{a}c\Phi\left(\theta\right)-k_{d}\theta\label{eq:Single Layer Kinetics}\end{equation}
where $\theta$ is the fraction of the surface covered by adsorbed
particles, $c$ is the concentration of protein in solution near the
surface, $k_{a}$ is the adsorption rate constant, and $k_{d}$ is
the desorption rate constant. The function $\Phi$ represents the
blocking effect of adsorbed particles. For the Langmuir model, the
blocking function is simply $\Phi\left(\theta\right)=1-\theta/\theta_{\infty}$.
The blocking function for the random sequential adsorption (RSA) model
is not available in analytic form. A widely used approximation for
the random sequential adsorption of spherical particles is\begin{equation}
\Phi\left(\theta\right)=\frac{\left(1-\bar{\theta}\right)^{3}}{1.0-0.812\bar{\theta}+0.2335\bar{\theta}^{2}+0.0845\bar{\theta}^{3}}\label{eq:RSA Blocking Function}\end{equation}
where $\bar{\theta}=\theta/\theta_{\infty}$ \cite{Schaaf1989}.


\subsubsection{RSA-type model of adsorption with transition}

Single-layer adsorption models cannot adequately predict the kinetics
of adsorption for many combinations of proteins and surfaces. More
complex models have been developed to account for these experimental
results. Because many surfaces cause proteins to denature upon adsorption,
it is common to model adsorption with a post-adsorption transition.
This process can be described by the kinetic equations\begin{align}
\frac{d\rho_{\alpha}}{dt} & =k_{a}c\Phi_{\alpha}-k_{s}\rho_{\alpha}\Psi_{\alpha\beta}-k_{d}\rho_{\alpha}\label{eq:Adsorption Transition Kinetics}\\
\frac{d\rho_{\beta}}{dt} & =k_{s}\rho_{\alpha}\Psi_{\alpha\beta}\label{eq:Adsorption Transition Kinetics 2}\end{align}
The form of the blocking functions depends upon the assumptions of
the model. When the particles are spherical, scaled particle theory
(SPT) can be used to derive the following blocking functions \cite{Brusatori1999}:
\begin{align}
\Phi_{\alpha} & =\left(1-\theta\right)\exp\left[-\frac{2\left(\overline{\rho_{\alpha}}+\Sigma\overline{\rho_{\beta}}\right)}{1-\theta}-\frac{\overline{\rho_{\alpha}}+\overline{\rho_{\beta}}+\left(\Sigma-1\right)^{2}\overline{\rho_{\alpha}}\overline{\rho_{\beta}}}{\left(1-\theta\right)^{2}}\right]\label{eq:Phi_alpha}\\
\Psi_{\alpha\beta} & =\exp\left[-\frac{2\left(\Sigma-1\right)\left(\overline{\rho_{\alpha}}+\Sigma\overline{\rho_{\beta}}\right)}{1-\theta}-\frac{\left(\Sigma^{2}-1\right)\left[\overline{\rho_{\alpha}}+\overline{\rho_{\beta}}+\left(\Sigma-1\right)^{2}\overline{\rho_{\alpha}}\overline{\rho_{\beta}}\right]}{\left(1-\theta\right)^{2}}\right]\label{eq:Psi_alpha_beta}\end{align}
The following non-dimensional variables are defined: $\overline{\rho_{\alpha}}=\rho_{\alpha}\pi R_{\alpha}^{2}$,
$\overline{\rho_{\beta}}=\rho_{\beta}\pi R_{\alpha}^{2}$, $\theta=\overline{\rho_{\alpha}}+\Sigma^{2}\overline{\rho_{\beta}}$,
and $\Sigma=R_{\beta}/R_{\alpha}$.

Since the blocking functions derived from SPT describe spherical particles,
they can be directly compared to the RSA blocking function. When $k_{s}=0$,
equations \ref{eq:Adsorption Transition Kinetics} and \ref{eq:Adsorption Transition Kinetics 2}
reduce to equation \ref{eq:Single Layer Kinetics}. By setting $\rho_{\beta}=0$
and $\Sigma=1$, equations \ref{eq:RSA Blocking Function} and \ref{eq:Phi_alpha}
can be compared as shown in Figure \ref{fig:RSA and SPT Blocking Functions}.%
\begin{figure}
\includegraphics{Fibronectin/Plots/ASF_comparison}

\caption{\label{fig:RSA and SPT Blocking Functions}Blocking functions from
the RSA model and scaled particle theory}

\end{figure}
 It can be seen that the first-level blocking function derived from
SPT is similar but not identical to the RSA blocking function, especially
at higher values of fractional surface coverage.

To quantitatively compare different multi-layer models, it is necessary
to transform equations from different sources to use a common set
of variables. The surface number densities $\rho_{i}$ used in equations
\ref{eq:Adsorption Transition Kinetics} and \ref{eq:Adsorption Transition Kinetics 2}
can be converted to fractional surface coverage by multiplying by
the area covered by an adsorbed particle in state $i$, $\sigma_{i}$,
to obtain:\begin{align}
\frac{d\theta_{\alpha}}{dt} & =k_{a}\,\sigma_{\alpha}c\Phi_{\alpha}-k_{s}\theta_{\alpha}\Psi_{\alpha\beta}-k_{d}\theta_{\alpha}\label{eq:dtheta_alpha dt}\\
\frac{d\theta_{\beta}}{dt} & =k_{s}\Sigma^{2}\theta_{\alpha}\Psi_{\alpha\beta}\label{eq:dtheta_beta dt}\end{align}



\subsubsection{Langmuir-type model of adsorption with transition}

A different model of adsorption with a post-adsorption transition
has been used to fit the adsorption kinetics of the fibronectin fragment
$\textrm{FNII\ensuremath{I_{7-10}}}$ \cite{Michael2003}. Written
using the same variables as equations \ref{eq:Adsorption Transition Kinetics}
and \ref{eq:Adsorption Transition Kinetics 2}, the equations that
describe this model are\begin{align}
\frac{d\rho_{\alpha}}{dt} & =k_{a}\, c\, A_{av}-k_{s}\,\rho_{\alpha}\, A_{av}-k_{d}\,\rho_{\alpha}\label{eq:Michael kinetics 1}\\
\frac{d\rho_{\beta}}{dt} & =k_{s}\,\rho_{\alpha}\, A{}_{av}\label{eq:Michael kinetics 2}\end{align}
where $Y_{i}$ is the surface density ($ng/cm^{2}$) of adsorbed protein
in each state. $A_{av}$ is the surface area ($cm^{2}$) available
for adsorption, which is given by\[
A_{av}=A_{total}\left(1-f\,\sigma_{1}\,\rho_{1}-f\, b\,\sigma_{1}\,\rho_{2}\right)\]
where $b=\sigma_{\beta}/\sigma_{\alpha}$. Since $\theta_{i}=\sigma_{i}\, f\,\rho_{i}$
and $f=N_{A}/M$ ($molecules/ng$), this expression can be written
$A_{av}=A_{total}\left(1-\theta\right)$, where $\theta=\theta_{\alpha}+\theta_{\beta}$.
Equations \ref{eq:Michael kinetics 1} and \ref{eq:Michael kinetics 2}
can be written in terms of fractional surface coverage by multiplying
both sides by $\sigma_{i}\, f$ to obtain:\begin{align}
\frac{d\theta_{1}}{dt} & =k\,\sigma_{1}\,\sigma A_{total}\, f\, c\left(1-\theta\right)-s\, A_{total}\,\theta{}_{1}\left(1-\theta\right)-r\,\theta_{1}\label{eq:Langmuir two stage 1}\\
\frac{d\theta_{2}}{dt} & =s\, A_{total}\, b\,\theta_{1}\,\left(1-\theta\right)\label{eq:Langmuir two stage 2}\end{align}
It is clear that the blocking function for this model is actually
the Langmuir blocking function, with $\theta_{\infty}=1$.

The kinetics of adsorption based on the Langmuir blocking function
were compared to the kinetics modeled by the SPT blocking functions,
and the results are shown in Figure \ref{fig:SPT vs Langmuir kinetics}.
%
\begin{figure}[h]
\includegraphics{Fibronectin/Plots/SPT_vs_Langmuir_kinetics}

\caption{\label{fig:SPT vs Langmuir kinetics}Comparison of kinetics predicted
by SPT blocking function (a) and Langmuir blocking function (b) for
$k_{a}=1$, $k_{s}=\pi$, $k_{d}=\pi$, $r_{\alpha}=1$, $\Sigma=1.2$,
and $c=1$.}

\end{figure}
 The parameters used for the comparison were taken from Figure 2a
from \cite{Brusatori1999}. Note that Figure 2a from \cite{Brusatori1999}
cannot be directly compared directly with Figure \ref{fig:SPT vs Langmuir kinetics}
in this work, because $\theta_{\beta}\neq\overline{\rho_{\beta}}$.
For the Langmuir model, $\theta_{\infty}=0.547$ was used for consistency
with the RSA and SPT blocking functions.


\subsection{Fitting Kinetic Models to Experimental Data}


\section{Results}


\subsection{Experimental Data}

%
\begin{figure}
\includegraphics[width=1\columnwidth]{Fibronectin/Plots/FN_Experimental_Data}

\caption{\label{fig:FN experiments}Measured adsorption kinetics for fibronectin
on 13F, DETA, and OEG surfaces}



\end{figure}



\subsection{Fitted Models}

%
\begin{table}
\caption{\label{tab:FN on DETA}Parameter values fitted to FN on DETA}


\begin{tabular}{lcccccc}
 & $k_{a}\times10^{6}$ & $k_{s}\times10^{7}$ & $k_{d}\times10^{4}$ & $\sigma_{\alpha}\times10^{12}$ & $\sigma_{\beta}\times10^{12}$ & $SSE$\tabularnewline
\hline
Langmuir & $2.11$ &  & $7.33$ & $2.20$ &  & $127$\tabularnewline
Langmuir two-stage & $ $ & $ $ & $ $ & $ $ & $ $ & $ $\tabularnewline
RSA & $2.05$ &  & $3.16$ & $1.82$ &  & $86.1$\tabularnewline
RSA two-stage & $2.61$ & $5.67$ & $2.53$ & $2.08$ & $110$ & $80.0$\tabularnewline
\end{tabular}
\end{table}


%
\begin{figure}
\includegraphics{Fibronectin/Plots/FN_DETA_RSA_CFD}

\caption{\label{fig:FN DETA RSA}RSA model fitted to experimental data for
FN on DETA}



\end{figure}


%
\begin{table}
\caption{\label{tab:FN on 13F}Parameter values fitted to FN on 13F}


\begin{tabular}{lcccccc}
 & $k_{a}\times10^{6}$ & $k_{s}\times10^{2}$ & $k_{d}\times10^{4}$ & $\sigma_{\alpha}\times10^{12}$ & $\sigma_{\beta}\times10^{12}$ & $SSE$\tabularnewline
\hline
Langmuir & $2.22$ &  & $6.05$ & $2.36$ &  & $212$\tabularnewline
Langmuir two-stage & $ $ & $ $ & $ $ & $ $ & $ $ & $ $\tabularnewline
RSA & $2.24$ &  & $2.49$ & $1.95$ &  & $121$\tabularnewline
RSA two-stage & $1.72$ & $3.88$ & $2.91$ & $1.42$ & $3.38$ & $63.9$\tabularnewline
\end{tabular}
\end{table}


%
\begin{figure}
\includegraphics{Fibronectin/Plots/FN_13F_RSA_CFD}

\caption{\label{fig:FN 13F RSA}RSA model fitted to experimental data for FN
on 13F}

\end{figure}
%
\begin{table}
\caption{\label{tab:FN on OEG}Parameter values fitted to FN on OEG}


\begin{tabular}{lcccc}
 & $k_{a}$ & $k_{d}$ & $\sigma_{\alpha}$ & $SSE$\tabularnewline
\hline
Langmuir & $1.91\times10^{-6}$ & $1.12\times10^{-3}$ & $1.6\times10^{-11}$ & $4.3$\tabularnewline
RSA & $9.22\times10^{-7}$ & $7.20\times10^{-4}$ & $1.25\times10^{-11}$ & $2.8$\tabularnewline
\end{tabular}
\end{table}
%
\begin{figure}
\includegraphics{Fibronectin/Plots/FN_OEG_RSA_CFD}

\caption{\label{fig:FN OEG RSA}RSA model fitted to experimental data for FN
on OEG}

\end{figure}



\section{Discussion}
