
\chapter{A CONTINUUM MODEL OF THE ADSORPTION OF HARD SPHERES}


\section{Introduction}

Computational methods have been applied to model and predict protein
adsorption, but their success has been limited due to the complexity
of the problem. While nanoscale simulation methods like molecular
dynamics (MD) have the potential to predict protein adsorption from
first principles, the small time step required by atomistic techniques
presents a significant obstacle. Protein adsorption occurs on a time
scale of seconds to hours, while it was recently reported that an
advanced MD simulation on a specially designed supercomputer has the
ability to simulate several microseconds of protein behavior \cite{Dror2010}.
As a result, simplified models are widely used. In colloidal models,
the protein is represented by a simplified geometric shape and interactions
are modeled by DLVO forces. Recently, Brownian dynamics simulations
been used to implement colloidal-scale simulations of protein transport
and adsorption \cite{Unni2005,Magan2006,Quinn2008}. 

A fundamental limitation of nanoscale and mesoscale simulation methods
is that they are impractical for modeling transport in practical applications
such as medical implants, engineered tissue constructs, organ models,
and microfluidic devices. CFD simulations are widely used to model
transport in macroscale systems. To predict protein adsorption, simplified
continuum models of adsorption based upon chemical kinetics have been
used as adsorbing boundary conditions in CFD simulations \cite{Glaser1993}.
The Langmuir and RSA adsorption models, as well as numerous others,
have successfully been utilized to model protein adsorption \cite{Rabe2010}. 

Chemical thermodynamics has been used to create a link between continuum
models and discrete particles \cite{Adamczyk1999a}. Adsorbed particles
were assumed to create an energy barrier which becomes higher as the
number of adsorbed particles increases. This barrier incorporates
steric exclusion due to the blocking effect of hard particles and
the longer-ranged repulsive effect of electrostatic interactions.
The flux of particles to the surface is influenced by the size of
the barrier. Previous authors have used the surface boundary layer
approximation (SFBLA) to simplify the model by assuming that the interface
is one-dimensional and the flux through the boundary layer is independent
of the position above the surface. It cannot be assumed that the flux
at the surface is equal to the flux at the interface with the bulk
when modeling the adsorption of hard spheres. A particle may diffuse
into the boundary region, collide with adsorbed particles, and diffuse
back out of the boundary region at a later time. At any instant, the
flux at the surface will be less than or equal to the flux at the
bulk interface.

Brownian dynamics simulations of hard sphere adsorption were utilized
to obtain configurations of adsorbed spheres, which were then analyzed
to obtain the chemical potential near the surface. The principles
of non-equilibrium thermodynamics were used to derive a continuum
model of hard-sphere adsorption in which the flux of particles is
allowed to vary with the distance from the surface. The chemical potential
from the hard-sphere simulations was used to determine the coefficients
of the continuum model, which was solved numerically in the region
near the surface. This model was used as a boundary condition for
a conventional CFD simulation to predict coupled transport and adsorption.
The kinetics of adsorption predicted by the continuum model matched
the kinetics from the Brownian dynamics simulation. This approach
can be used to incorporate the results from a discrete particle simulation
when simulating transport and adsorption in a microscale or macroscale
system.


\section{Methods and Materials}


\subsection{Derivation of the Continuum Model }

The first steps of the derivation of the continuum model follow the
method described in \cite{Adamczyk1999a,Adamczyk2000}, starting with
the continuity equation\begin{equation}
\frac{\partial n}{\partial t}=-\nabla\cdot\mathbf{J}\label{eq:Continuity}\end{equation}
$n$ is the number density of particles in solution and $\mathbf{J}$
is the flux of particles. Adsorption is a process of equilibration
that can be described using non-equilibrium thermodynamics. It has
been postulated that the flux is proportional to the gradient of the
total potential \begin{align}
\mathbf{J} & =-\left(\mathbf{M}\cdot\nabla\mathbf{E}\right)\, n\label{eq:Flux ito Mobility}\end{align}
where$\mathbf{M}$ is the mobility tensor and $\mathbf{E}$ is the
total potential, which can be written as $\mathbf{E}=\mu+\mathbf{\Phi}$.
The chemical potential $\mu$ represents particle-particle interactions,
including interactions between particles in solution and adsorbed
particles. The external potential $\mathbf{\Phi}$ includes effects
such as an electric field due to a charged surface or a gravitational
potential. Using the relation $\mathbf{D}=kT\mathbf{M}$, the flux
can be written in terms of the diffusion tensor\begin{align}
\mathbf{J} & =-\mathbf{D}\cdot\left(\nabla\mu/kT+\nabla\mathbf{\Phi}/kT\right)\, n\label{eq:Flux ito Diffusion}\end{align}


The chemical potential can be written in terms of the activity coefficient
$\gamma$\begin{equation} \label{eq:ChemicalPotential}
\mu = \mu^\standardstate + kT \, \ln \gamma \frac{n}{n^ \standardstate}
\end{equation}\( \mu^\standardstate \) is the potential in the standard state,
which is chosen to be the potential of a particle in solution far
away from other particles. The activity coefficient is a function
of position in space, the number and configuration of particles, and
the particle-particle interaction potential. Expanding the potential
and substituting into the flux equation results in \begin{equation} \label{eq:Flux ito Potential}
\mathbf{J} = -\mathbf{D} \cdot \left[ \frac{\nabla \mu^\standardstate}{kT}
 + \nabla \ln \frac{n}{n^\standardstate} + \nabla \ln \gamma
 + \frac{\nabla \Phi}{kT}
 \right] \, n
\end{equation}Since the potential in the standard state is constant, its gradient
is zero. This expression for the flux was substituted into the continuity
equation to obtain\begin{equation} \label{eq:PDE 1}
\frac{\partial n}{ \partial t} =
 \nabla \cdot \left[ \mathbf{D} \cdot \left(
 \nabla \ln \frac{n}{n^\standardstate}
 + \nabla \ln \gamma + \frac{\nabla \Phi}{kT}
 \right) \, n \right]
\end{equation}For modeling adsorption at an interface, the general equation can
be simplified considerably by making some assumptions about the interface.
It was assumed that the interfacial layer is thin with respect to
the overall geometry. Transport parallel to the interface and convection
were neglected so the equation reduced to a one-dimensional form:\begin{equation} \label{eq:PDE 2}
\frac{\partial n}{ \partial t} =
 \frac{\partial}{ \partial h} \, \left[ D \, 
 \left(
 \frac{\partial}{ \partial h} \ln \frac{n}{n^\standardstate} 
 + \frac{\partial}{ \partial h} \ln \gamma
 + \frac{1}{kT} \frac{\partial}{ \partial h} \Phi
 \right) \, n \right]
\end{equation}$h$ is the distance between the edge of the particle and the surface,
as shown in Figure \ref{fig:Geometry}. %
\begin{figure}
\includegraphics{HardSpheres/Figures/AdsorbingBoundary}

\caption{\label{fig:Geometry}Geometry used to define the hard-sphere adsorbing
boundary condition}


%
\end{figure}


For the case of hard spheres with no surface potential, $\Phi\equiv0$
and the diffusion coefficient near the surface was assumed to be constant.
The notation \( \frac{n}{n^\standardstate} \) will be dropped, and
it will be assumed that $n$ has been normalized. The equation can
be re-arranged to have the form of a generalized diffusion equation\begin{align}
\frac{\partial n}{\partial t} & =D\,\frac{\partial}{\partial h}\left[\left(\frac{1}{n}\frac{\partial n}{\partial h}+\frac{1}{\gamma}\frac{\partial\gamma}{\partial h}\right)\, n\right]\nonumber \\
 & =D\,\left[\frac{\partial^{2}}{\partial h^{2}}n+\frac{1}{\gamma}\frac{\partial\gamma}{\partial h}\frac{\partial n}{\partial h}+n\,\frac{\partial}{\partial h}\left(\frac{1}{\gamma}\frac{\partial\gamma}{\partial h}\right)\right]\label{eq:Generalized Diffusion Eqn}\end{align}
The result is a parabolic partial differential equation with variable
coefficients. Let\begin{align}
k_{1} & =\frac{1}{\gamma}\frac{\partial\gamma}{\partial h}\nonumber \\
k_{2} & =\frac{\partial}{\partial h}\left(\frac{1}{\gamma}\frac{\partial\gamma}{\partial h}\right)=\frac{\partial k_{1}}{\partial h}\label{eq:Variable coefficients}\end{align}
Then\begin{align}
\frac{\partial n}{\partial t} & =D\,\left[\frac{\partial^{2}}{\partial h^{2}}n+k_{1\,}\frac{\partial n}{\partial h}+k_{2}\, n\right]\label{eq:Generalized Diffusion Eqn 1D}\end{align}
 This equation predicts the evolution of number density over time
at every point in a domain for which the activity coefficient is known.
To utilize this equation to predict the surface density of adsorbed
particles over time, the boundary condition at the surface ($h=0$)
can be defined\[
\frac{d\Gamma}{dt}=-J_{s}\left(t\right)\]
$\Gamma$ is the number of adsorbed particles per unit area and $J_{s}$
is the flux at the surface. The total surface density at time $t$
is given by\[
\Gamma\left(t\right)=\int_{0}^{t}J_{s}\left(\tau\right)\, d\tau\]
The choice of boundary condition for the bulk solution depends upon
the nature of the problem to be solved. It is straightforward to couple
the generalized diffusion equation to a conventional CFD simulation
to predict transport-influenced adsorption in arbitrary geometries. 


\subsection{Computation of the Activity Coefficient}

The coefficients of the generalized diffusion equation are functions
of the activity coefficient $\gamma$, which is a function of space
and the number and configuration of adsorbed particles. Computation
of the activity coefficient is critical to obtain a useful model. 


\subsubsection{Brownian Dynamics Simulation}

A Brownian dynamics simulation of irreversible hard-sphere adsorption
was used to obtain configurations of adsorbed particles. The Langevin
position equation \cite{Elimelech1998} was used to update the position
of each particle at each time step: \[
\mathbf{r}_{i}(t+\Delta t)=\mathbf{r}_{i}(t)+\mathbf{g}_{q}\sqrt{2\, D\,\Delta t}\]
$\mathbf{r}_{i}$ is the position of particle $i$, $\Delta t$ is
the simulation time step, $D$ is the diffusion coefficient, and $g_{q}\in\mathbb{R}^{3}$
is a vector of random numbers drawn from a Gaussian distribution with
a mean of zero and a variance of one. At each time step, all particles
in the domain were moved simultaneously, and overlaps were detected.
Any particle which overlapped another was reset to its original position
and moved again using a different random vector, until each particle
found a valid position. The simulation domain was a rectangular box
with height $L$ and width and length $S$. Periodic boundary conditions
were applied on the four sides of the simulation domain so that a
particle that exited one side of the box re-entered on the opposite
side. An adsorbing boundary condition was used for the bottom of the
box. A particle adsorbed when it reached the adsorbing surface without
overlapping any previously adsorbed particles. The configuration of
adsorbed particles was recorded every time a new particle adsorbed.
A perfect adsorbing boundary could also be used to simulate diffusion-limited
adsorption. Adsorbed particles were moved out of the simulation domain
so that blocking did not interfere with the adsorption of additional
particles. 

Two different boundary conditions were used for the top of the box.
To simulate constant near-surface concentration a reflecting boundary
was used for the top of the box. At each time step, the particles
in the box were counted. If there were too few particles in the box,
particles were added at positions drawn from a uniform random distribution,
ensuring that the newly added particles did not overlap with existing
particles. If there were too many particles in the domain, a particle
was chosen at random for deletion. To simulate diffusion on a semi-infinite
domain, an open box top was used to allow the Brownian dynamics simulation
to exchange particles with an infinite bulk solution. This boundary
condition was implemented according to the multi-scale linking algorithm
described in \cite{Magan2004}. 


\subsubsection{Implementation of Brownian Dynamics Simulation}

The simulation was implemented using the Python programming language.
Numerical data, such as the coordinates of the particles, were stored
in NumPy arrays \cite{Oliphant2006}. Routines from the SciPy library
were used for standard operations like interpolation and numerical
integration \cite{Oliphant2007}. Collision detection was implemented
in C for speed, using the weave function from SciPy. Each time a particle
adsorbed on the surface the configuration of adsorbed particles was
recorded, along with the profile of concentration vs. distance from
the adsorbing surface and the fraction of the surface covered by particles
($\theta$). The PyTables package was used to save the simulation
results to binary files in HDF5 format \cite{Alted2002-,HDFGroup2000-}. 


\subsubsection{Controls and Validation for the Brownian dynamics simulation}

The Brownian dynamics simulation was validated by simulating diffusion-limited
adsorption and comparing the results to the analytical solution of
a well-known boundary value problem. A perfect adsorbing boundary
condition (perfect sink) was used so that particles that adsorbed
to the surface did not block the adsorption of additional particles.
The classical diffusion equation in one dimension can be solve analytically
with the boundary conditions $n(h=0,t)=0$ and $n(h\rightarrow\infty,t)=n_{b}$.
Control simulations were performed with three different box widths
(75, 100, 150) to ensure that edge effects were not distorting the
pattern of adsorbed particles. The simulation was also tested with
three time steps ($10^{-5}$, $10^{-6}$ and $10^{-7}sec$) to determine
the largest time step that would produce accurate results.


\subsubsection{Calculation of the Activity Coefficient}

The activity coefficient was determined empirically from the results
of the Brownian dynamics simulations using the Widom particle insertion
method \cite{Widom1963}. In this method, a {}``test'' particle
is introduced into a fixed configuration of particles and the energy
of interaction $\psi$ between the test particle and the surrounding
particles is calculated. The activity $a$ can be computed by taking
the canonical average of many such insertions, using the formula\[
\frac{n}{a}=\left\langle \exp\left(\frac{-\psi}{k_{B}T}\right)\right\rangle \]
For hard spheres, the energy of interaction is either infinite if
the test particle overlaps an existing particle or zero if it does
not. Therefore the activity coefficient $\gamma=a/n$ is also infinite
if the test particle overlaps another, and zero if it does not. To
avoid dealing with infinite quantities, the available volume function
(AVF) was defined as \[
AVF\left(h,\theta\right)=\gamma^{-1}\left(h,\theta\right)\]
The value of the AVF is one if the test particle does not overlap
with a simulation particle and zero if it does overlap. After the
completion of an ensemble of Brownian dynamics simulations, the AVF
was calculated for each run at multiple values of $h$ and $\theta$.
For each $\theta_{i}$ a planar grid of non-overlapping test particles
was constructed in the simulation domain at a given height above the
adsorbing surface. Particles in solution with cannot interact with
adsorbed particles, as shown in Figure \ref{fig:Geometry}. The position
of each test particle was offset by a small random vector in $x-y$
plane and each test particle was checked for overlaps with every simulation
particle. The fraction of test particles without overlaps was recorded
as the value of $AVF\left(h,\theta\right)$. Multiple replicates with
different random displacements from the grid were performed for each
and . The analysis was performed with 50 and 500 replicates, and 50
replicates were found to be sufficient to determine the AVF. The results
from multiple runs of the Brownian dynamics simulation were averaged
to obtain an estimate of the available volume function. The coefficients
of the generalized diffusion equation were computed directly from
the available volume function:\[
k_{1}=\frac{\partial}{\partial h}\log\gamma=\frac{1}{AVF}\,\frac{\partial}{\partial h}AVF\]
\[
k_{2}=\frac{\partial}{\partial h}\left(\frac{1}{\gamma}\frac{\partial\gamma}{\partial h}\right)=\frac{1}{AVF^{2}}\left(\frac{\partial AVF}{\partial h}\right)^{2}-\frac{1}{AVF}\left(\frac{\partial^{2}AVF}{\partial h^{2}}\right)\]



\subsection{Implementation of the Continuum Model}

The control volume formulation \cite{Patankar1980} was used to obtain
a finite difference form of Equation 15 in the region $0\leq h<2$.
Central differencing was used to approximate first derivatives in
space, and a fully implicit scheme was used to approximate time derivatives.
To ensure stability, the source term was linearized so that it was
independent of the value of n. Any particle that touches the surfaces
adsorbs, so the number density of particles at the surface is zero.
The Dirichlet (type 1) boundary condition $n=0$ was used at $h=0$.
For simulations with constant near-surface concentration the type
1 boundary condition $n=n_{b}$ was applied at $h=2$. 


\subsubsection{Coupling the Continuum Model of Hard-Sphere Adsorption to a Conventional
CFD Transport Simulation}

For simulations in which the concentration at $h=2$ was influenced
by diffusion, a second simulation domain was created to model diffusion
in the bulk for $h\geq2$. The classical diffusion equation was discretized
and solved in the bulk domain in the same manner as the generalized
diffusion equation. The generalized diffusion equation was solved
in the interaction region to obtain the net flux, using the value
of number density at $h=2$ from the previous time step. For small
values of $t$ the surface is mostly available for adsorption, so
the net flux is limited by diffusive transport. The net flux is determined
by $J_{c}$ rather than $J_{s}$. Once the surface is mostly blocked,
the net flux is determined by the rate at which particles can find
available space on the surface, so the value of $J_{s}$ should be
used for the net flux. The correct value for the net flux can be computed
by\[
J\left(h=2,t\right)=-min\left(\left|J_{s}\right|,\left|J_{c}\right|\right)\]
 The net flux from the generalized diffusion equation was used as
the left-hand boundary condition to solve the classical diffusion
equation in the bulk, which resulted in a new value for the number
density at the interface. This number density was used to solve the
generalized diffusion equation in the near-surface domain again, and
the iterative process was repeated until the number density at the
interface computed in each domain converged:$\left|n\left(h=2^{-}\right)-n\left(h=2^{+}\right)\right|<\delta$.


\subsubsection{Validation of the Continuum Model}

Equation \ref{eq:Generalized Diffusion Eqn 1D} has the form of a
diffusion equation. In the case that the activity coefficient is constant,
this equation reduces to the classical one-dimensional diffusion equation.
It was verified that the adsorption kinetics predicted by the continuum
model matched the kinetics predicted by the classical diffusion equation
for a perfect adsorbing boundary when the activity coefficient was
held constant ($AVF\left(h,\theta\right)\equiv1$). 


\section*{Results}


\subsection{Brownian Dynamics Simulation Results}

The Brownian dynamics simulation was run with three different number
densities of particles in solution. The number density had a significant
impact on the kinetics of adsorption, as shown in Figure \ref{fig:BrD Adsorption Kinetics}
.%
\begin{figure}
\includegraphics{HardSpheres/Plots/BrD_kinetics}\caption{\label{fig:BrD Adsorption Kinetics}Kinetics of adsorption predicted
by the Brownian dynamics simulation for three different volume fractions.}


%
\end{figure}
 The configurations of adsorbed particles predicted by the Brownian
dynamics simulations were characterized using the pair correlation
function $g\left(r\right)$, which is also known as the radial distribution
function (RDF). The results are shown in Figure \ref{fig:Brownian Dynamics RDF}.
%
\begin{figure}
\includegraphics{HardSpheres/Plots/RDF_BrD_volume_fraction}\caption{\label{fig:Brownian Dynamics RDF}Radial distribution function predicted
by Brownian dynamics simulation for three volume fractions, and the
RDF predicted by the RSA model.}
%
\end{figure}
 Figure \ref{fig:Brownian Dynamics RDF} also shows the pair correlation
function for configurations generated by RSA simulations, which are
essentially identical to those generated by the Brownian dynamics
simulations. A representative plot of the AVF for a volume fraction
of 0.01 is shown in Figure \ref{fig:Available Volume Function}.%
\begin{figure}[H]
\includegraphics{HardSpheres/Plots/AVF_publication_plot}

\caption{\label{fig:Available Volume Function}Plot of the available volume
function. Available surface function for RSA shown as red dashed line.}
%
\end{figure}
 The available surface function (ASF) for the RSA model is also shown
in Figure \ref{fig:Available Volume Function} \cite{Schaaf1989}.
Since Brownian dynamics and RSA simulations produce identical configurations
of adsorbed particles, it is not surprising that the AVF for Brownian
dynamics at $h=0$ is identical to the ASF for RSA. 


\subsection{Continuum Model Results}

The coefficients of the generalized diffusion equation were computed
from the AVF from Brownian dynamics simulations. Representative plots
of the coefficient values are shown in Figure \ref{fig:Coefficient Plots}.%
\begin{figure}[H]
\subfloat[\label{fig:Coefficient a}Coefficient $k_{1}$]{\includegraphics{/home/cfinch/Brownian_Dynamics/Documentation/Hard_Sphere_Paper/Plots/coeff_a_publication_plot}

}\subfloat[\label{fig:Coefficient b}Coefficient $k_{2}$]{\includegraphics{/home/cfinch/Brownian_Dynamics/Documentation/Hard_Sphere_Paper/Plots/coeff_b_publication_plot}

}

\caption{\label{fig:Coefficient Plots}Coefficients of the generalized diffusion
equation, derived from Brownian dynamics results}
%
\end{figure}
 The calculation of coefficient was challenging due to the presence
of $AVF^{-1}$ and $AVF^{-2}$ in Equation \ref{eq:Variable coefficients},
which result in large numbers when the value of the AVF approaches
zero. The continuum model accurately reproduced the kinetics predicted
by the Brownian dynamics simulations, as shown in Figure \ref{fig:Kinetics CFD BrD}.%
\begin{figure}
\includegraphics{HardSpheres/Plots/BrD_Continuum_Kinetics_dt}

\caption{\label{fig:Kinetics CFD BrD}Kinetics of adsorption predicted by Brownian
dynamics simulations and the continuum model for $\phi=0.01$.}
%
\end{figure}



\section{Discussion}

The RDFs for particle configurations generated by Brownian dynamics
simulations are almost identical to the RDF for particles generated
by RSA. This agreement indicates that the adsorption of hard spheres
is a random sequential process that is essentially independent of
transport to the surface. If kinetic predictions are not required,
an RSA simulation can be used to generate surface configurations that
are equivalent to results from hard-sphere Brownian dynamics simulations,
with much less computational effort.

The AVF shown in Figure \ref{fig:Available Volume Function} differs
significantly from the blocking function reported in \cite{Magan2004},
which was estimated by taking the ratio of flux at the surface to
the flux expected for a perfect adsorbing boundary. In this work the
AVF was computed directly by attempting to adsorb an additional particle
onto a surface with adsorbed particles. The method used here is more
likely to obtain an accurate result.
