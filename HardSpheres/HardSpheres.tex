
\chapter{\texorpdfstring{A CONTINUUM MODEL OF THE ADSORPTION OF HARD SPHERES}{CHAPTER \arabic{chapter}. A CONTINUUM MODEL OF THE ADSORPTION OF HARD SPHERES}}


\section{Introduction}

Computational methods have been applied to model and predict protein
adsorption, but their success has been limited due to the complexity
of the problem. While nanoscale simulation methods like molecular
dynamics (MD) have the potential to predict protein adsorption from
first principles, the small (femtosecond) time step required by atomistic
techniques presents a significant obstacle. The adsorption of proteins
and their subsequent rearrangement occurs on a time scale of seconds
to hours~\cite{Latour2005}. MD has been used to simulate the adsorption
of a fragment of fibrinogen with explicit solvent molecules for $5\, ns$
of simulated time~\cite{Agashe2005}. A similar study was carried
out to model the interaction between lysozyme and a graphite surface
for $500\, ps$~\cite{Raffaini2010}. It was recently reported that
an advanced MD simulation on a specially designed supercomputer has
the ability to simulate several microseconds of protein behavior~\cite{Dror2010}.
Because of the limitations of atomistic models, simplified models
are widely used to model protein adsorption. Colloidal models represent
a protein molecule with a simplified geometric shape (typically a
sphere or ellipsoid) and interactions are modeled by DLVO (electrostatic
and van der Waals) forces~\cite{evans1994colloidal,Roth1993,Lenhoff1995}.
Recently, Brownian dynamics simulations have been used to implement
colloidal-scale simulations of protein transport and adsorption~\cite{Unni2005,Magan2006,Quinn2008}. 

A fundamental limitation of nanoscale and mesoscale simulation methods
is that they are impractical for modeling transport in practical applications
such as medical implants, engineered tissue constructs, blood flow
simulations, and microfluidic devices. Computational fluid dynamics
(CFD) simulations are widely used to model transport in macroscale
systems. Simplified continuum models of adsorption based upon chemical
kinetics have been used as adsorbing boundary conditions in CFD simulations~\cite{Jenkins2004,Glaser1993,Edwards1999}.
At first, kinetic models of adsorption were formulated based upon
the assumptions of the Langmuir adsorption model~\cite{Andrade1986}.
More recently random sequential adsorption (RSA) models have been
applied to characterize the blocking of the surface more rigorously
for various particle shapes~\cite{Talbot2000287}. The available
surface for adsorption is usually quantified through the available
surface function $ASF\left(\theta\right)$, where $\theta$ is the
fraction of surface that is covered by adsorbed particles. Many types
of kinetic models are currently used to model protein adsorption~\cite{Rabe2010}. 

Continuum models can be linked to discrete-particle simulations using
the principles of chemical thermodynamics. The available surface function
has been generalized to depend on the height above the adsorbing surface
as well as the fractional surface coverage~\cite{Adamczyk1999}.
Adsorbed particles create an energy barrier which becomes higher as
the number of adsorbed particles increases, reducing the flux of particles
to the surface. This barrier incorporates steric exclusion due to
the blocking effect of hard particles and the longer-ranged repulsive
effect of electrostatic interactions~\cite{Adamczyk2000}. This model
has been implemented by making simplifying assumptions about transport
to the surface. For example, the surface boundary layer approximation
(SFBLA) assumes that the interface is one-dimensional and the flux
through the boundary layer is independent of the position above the
surface~\cite{Adamczyk1999a}. Transport to the interface has been
accounted for by coupling the adsorption model to a diffusive transport
model, but the kinetic coefficients of the adsorption model were approximated
with results from an RSA model~\cite{Adamczyk2000}. 

To fully account for the blocking effect of hard spheres, it cannot
be assumed that the flux at the surface ($J_{s}$) is equal to the
flux at the interface with the continuum ($J_{c}$) . A particle may
diffuse into the boundary region, collide with adsorbed particles,
and diffuse back out of the boundary region at a later time. At any
instant, the flux at the surface will be less than or equal to the
flux at the bulk interface. Brownian dynamics simulations of hard
sphere adsorption were utilized to obtain configurations of adsorbed
spheres, which were then analyzed to obtain the generalized blocking
function $AVF\left(h,\theta\right)$. The principles of non-equilibrium
thermodynamics were used to derive a continuum model of hard-sphere
adsorption in which the flux of particles was allowed to vary with
the distance from the surface. The generalized blocking function from
the Brownian dynamics simulations was used to determine the coefficients
of the continuum model, which was solved numerically in the region
near the surface. This model was used as a boundary condition for
a conventional CFD simulation to predict coupled transport and adsorption.
Good agreement was found between the kinetics obtained from the Brownian
dynamics simulations and the kinetics predicted by the continuum model.


\section{Methods and Materials}


\subsection{Derivation of the Continuum Model }

The first steps of the derivation of the continuum model follow the
method described in \cite{Adamczyk1999,Adamczyk2000,Adamczyk1999a},
starting with the continuity equation\begin{equation}
\frac{\partial n}{\partial t}=-\nabla\cdot\mathbf{J}\label{eq:Continuity}\end{equation}
$n$ is the number density of particles in solution and $\mathbf{J}$
is the flux of particles. Adsorption is a process of equilibration
that can be described using non-equilibrium thermodynamics. In general,
irreversible fluxes tend to be linear functions of thermodynamic gradients
(such as Fick's first law) \cite{degroot1984non}, so it was postulated
that \begin{align}
\mathbf{J} & =-\left(\mathbf{M}\cdot\nabla E\right)\, n\label{eq:Flux ito Mobility}\end{align}
$\mathbf{M}$ is the mobility tensor and $E$ is the total potential,
which can be written as $E=\mu+\Phi$. The chemical potential $\mu$
represents particle-particle interactions, including interactions
between particles in solution and adsorbed particles. The external
potential $\Phi$ includes effects such as an electric field due to
a charged surface or a gravitational potential. Using the relation
$\mathbf{D}=kT\mathbf{M}$, the flux can be written in terms of the
diffusion tensor\begin{align}
\mathbf{J} & =-\mathbf{D}\cdot\left(\nabla\mu/kT+\nabla\Phi/kT\right)\, n\label{eq:Flux ito Diffusion}\end{align}


The chemical potential can be written in terms of the activity coefficient
$\gamma$:\begin{equation} \label{eq:ChemicalPotential}
\mu = \mu^\standardstate + kT \, \ln \gamma \frac{n}{n^ \standardstate}
\end{equation}\( \mu^\standardstate \) is the potential in the standard state,
which is chosen to be the potential of a particle in solution far
away from other particles. The activity coefficient is a function
of position in space, the number and configuration of particles, and
the particle-particle interaction potential. Expanding the potential
and substituting into the flux equation results in \begin{equation} \label{eq:Flux ito Potential}
\mathbf{J} = -\mathbf{D} \cdot \left[ \frac{\nabla \mu^\standardstate}{kT}
 + \nabla \ln \frac{n}{n^\standardstate} + \nabla \ln \gamma
 + \frac{\nabla \Phi}{kT}
 \right] \, n
\end{equation}Since the potential in the standard state is constant, the gradient
of the first term is zero. This expression for the flux was substituted
into the continuity equation to obtain\begin{equation} \label{eq:PDE 1}
\frac{\partial n}{ \partial t} =
 \nabla \cdot \left[ \mathbf{D} \cdot \left(
 \nabla \ln \frac{n}{n^\standardstate}
 + \nabla \ln \gamma + \frac{\nabla \Phi}{kT}
 \right) \, n \right]
\end{equation}For modeling adsorption at an interface, the general equation can
be simplified considerably by making some assumptions about the interface.
It was assumed that the interfacial layer is thin with respect to
the overall geometry. Transport parallel to the interface and convection
were neglected so the equation reduced to a one-dimensional form:\begin{equation} \label{eq:PDE 2}
\frac{\partial n}{ \partial t} =
 \frac{\partial}{ \partial h} \, \left[ D \, 
 \left(
 \frac{\partial}{ \partial h} \ln \frac{n}{n^\standardstate} 
 + \frac{\partial}{ \partial h} \ln \gamma
 + \frac{1}{kT} \frac{\partial}{ \partial h} \Phi
 \right) \, n \right]
\end{equation}$h$ is the distance between the edge of the particle and the surface,
as shown in Figure \ref{fig:Geometry}. %
\begin{figure}
\includegraphics{HardSpheres/Figures/AdsorbingBoundary}

\caption{\label{fig:Geometry}Geometry used to define the hard-sphere adsorbing
boundary condition.}


%
\end{figure}


For the case of hard spheres with no surface potential, $\Phi\equiv0$
and the diffusion coefficient near the surface was assumed to be constant.
The notation \(n  / n^\standardstate\) will be dropped, and it will
be assumed that $n$ has been normalized. The equation can be re-arranged
to have the form of a generalized diffusion equation\begin{align}
\frac{\partial n}{\partial t} & =D\,\frac{\partial}{\partial h}\left[\left(\frac{1}{n}\frac{\partial n}{\partial h}+\frac{1}{\gamma}\frac{\partial\gamma}{\partial h}\right)\, n\right]\nonumber \\
 & =D\,\left[\frac{\partial^{2}}{\partial h^{2}}n+\frac{1}{\gamma}\frac{\partial\gamma}{\partial h}\frac{\partial n}{\partial h}+n\,\frac{\partial}{\partial h}\left(\frac{1}{\gamma}\frac{\partial\gamma}{\partial h}\right)\right]\label{eq:Generalized Diffusion Eqn}\end{align}
Equation \ref{eq:Generalized Diffusion Eqn} is a parabolic partial
differential equation with variable coefficients. Let\begin{align}
k_{1} & =\frac{1}{\gamma}\frac{\partial\gamma}{\partial h}\nonumber \\
k_{2} & =\frac{\partial}{\partial h}\left(\frac{1}{\gamma}\frac{\partial\gamma}{\partial h}\right)=\frac{\partial k_{1}}{\partial h}\label{eq:Variable coefficients}\end{align}
Then\begin{align}
\frac{\partial n}{\partial t} & =D\,\left[\frac{\partial^{2n}}{\partial h^{2}}+k_{1\,}\frac{\partial n}{\partial h}+k_{2}\, n\right]\label{eq:Generalized Diffusion Eqn 1D}\end{align}
 This equation predicts the evolution of number density over time
at every point in a domain for which the activity coefficient is known.
To utilize this equation to predict the surface density of adsorbed
particles over time, the boundary condition at the surface ($h=0$)
can be defined\begin{equation}
\frac{d\Gamma}{dt}=-J_{s}\left(t\right)\label{eq:Flux matching BC}\end{equation}
$\Gamma$ is the number of adsorbed particles per unit area and $J_{s}$
is the flux at the surface. The total surface density of adsorbed
particles at time $t$ is given by\begin{equation}
\Gamma\left(t\right)=\int_{0}^{t}J_{s}\left(\tau\right)\, d\tau\label{eq:Total surface density}\end{equation}
The choice of boundary condition for the bulk solution depends upon
the nature of the problem to be solved. It is straightforward to couple
the generalized diffusion equation to a conventional CFD simulation
to predict transport-influenced adsorption in arbitrary geometries. 


\subsection{Calculation of the Activity Coefficient}

The coefficients of the generalized diffusion equation are functions
of the activity coefficient $\gamma$, which is a function of space
and the number and configuration of adsorbed particles. Computation
of the activity coefficient is critical to obtain a useful model. 


\subsubsection{Brownian Dynamics Simulation}

A Brownian dynamics simulation of irreversible hard-sphere adsorption
was used to obtain configurations of adsorbed particles. The Langevin
position equation \cite{Elimelech1998} was used to update the position
of each particle at each time step: \begin{equation}
\mathbf{r}_{i}(t+\Delta t)=\mathbf{r}_{i}(t)+\mathbf{g}_{q}\sqrt{2\, D\,\Delta t}\label{eq:Langevin equation}\end{equation}
$\mathbf{r}_{i}$ is the position of particle $i$, $\Delta t$ is
the simulation time step, $D$ is the diffusion coefficient, and $g_{q}\in\mathbb{R}^{3}$
is a vector of random numbers drawn from a Gaussian distribution with
a mean of zero and a variance of one. At each time step, all particles
in the domain were moved simultaneously, and overlaps were detected.
Any particle which overlapped another was reset to its original position
and moved again using a different random vector, until each particle
found a valid position. 

The simulation domain was a rectangular box with height $L$ and width
and length $S$. Periodic boundary conditions were applied on the
four sides of the simulation domain so that a particle that exited
one side of the box re-entered on the opposite side. An adsorbing
boundary condition was used for the bottom of the box. A particle
adsorbed when it reached the adsorbing surface without overlapping
any previously adsorbed particles. The configuration of adsorbed particles
was recorded every time a new particle adsorbed. To simulate a perfect
adsorbing boundary for validation purposes, adsorbed particles could
be moved out of the simulation domain so they did not interfere with
the adsorption of additional particles. 

Two different boundary conditions were used for the top of the box.
To simulate constant near-surface concentration a reflecting boundary
was used for the top of the box. At each time step, the particles
in the box were counted. If there were too few particles in the box,
particles were added at positions drawn from a uniform random distribution,
ensuring that the newly added particles did not overlap with existing
particles. If there were too many particles in the domain, a particle
was chosen at random for deletion. To simulate diffusion on a semi-infinite
domain, an open box top was used to allow the Brownian dynamics simulation
to exchange particles with an infinite bulk solution. This boundary
condition was implemented according to the multi-scale linking algorithm
described in \cite{Magan2004}. 


\subsubsection{Implementation of Brownian Dynamics Simulation}

The simulation was implemented using the Python programming language
using the dimensionless variables defined in \cite{Magan2004}. Numerical
data, such as the coordinates of the particles, were stored in NumPy
arrays \cite{Oliphant2006}. Routines from the SciPy library were
used for standard operations like interpolation and numerical integration
\cite{Oliphant2007}. Collision detection was implemented in C for
speed, using the \emph{weave} function from SciPy. The Message Passing
Interface (MPI) was used to run multiple simulations in parallel on
a Linux cluster~\cite{MPI}. Each time a particle adsorbed on the
surface the configuration of adsorbed particles was recorded, along
with the profile of concentration vs. distance from the adsorbing
surface and the fraction of the surface covered by particles ($\theta$).
The PyTables package was used to save the simulation results to binary
files in HDF5 format \cite{Alted2002-,HDFGroup2000-}. 


\subsubsection{Controls and Validation for the Brownian Dynamics Simulation}

The Brownian dynamics simulation was validated by simulating diffusion-limited
adsorption and comparing the results to the analytical solution of
a well-known boundary value problem. A perfect adsorbing boundary
condition (perfect sink) was used so that particles that adsorbed
to the surface did not block the adsorption of additional particles.
The classical diffusion equation in one dimension can be solve analytically
with the boundary conditions $n(h=0,t)=0$ and $n(h\rightarrow\infty,t)=n_{b}$.
Control simulations were performed with three different box widths
(75, 100, 150) to ensure that edge effects were not distorting the
pattern of adsorbed particles. The simulation was also tested with
three time steps ($10^{-5}$, $10^{-6}$ and $10^{-7}sec$) to determine
the largest time step that would produce accurate results.


\subsubsection{Monte Carlo Calculation of the Activity Coefficient}

The activity coefficient was determined empirically from the results
of the Brownian dynamics simulations using the Widom particle insertion
method \cite{Widom1963,Dullens2005}. In this method, a {}``test''
particle is introduced into a fixed configuration of particles and
the energy of interaction $\psi$ between the test particle and the
surrounding particles is calculated. The activity $a$ can be computed
by taking the canonical average of many such insertions, using the
formula\begin{equation}
\frac{n}{a}=\left\langle \exp\left(\frac{-\psi}{k_{B}T}\right)\right\rangle \label{eq:Widom equation}\end{equation}
For hard spheres, the energy of interaction is either infinite if
the test particle overlaps an existing particle or zero if it does
not. Therefore the activity coefficient $\gamma=a/n$ is also infinite
if the test particle overlaps another, and zero if it does not. To
avoid dealing with infinite quantities, the available volume function
(AVF) was defined as \begin{equation}
AVF\left(h,\theta\right)=\gamma^{-1}\left(h,\theta\right)\label{eq:AVF}\end{equation}
The AVF is the three-dimensional equivalent of the available surface
function. The value of the AVF is one if the test particle does not
overlap with a simulation particle and zero if it does overlap. 

After the completion of an ensemble of Brownian dynamics simulations,
the AVF was calculated for each run at multiple values of $h$ and
$\theta$. For each $\theta_{i}$ a planar grid of non-overlapping
test particles was constructed in the simulation domain at a given
height $h_{i}$ above the adsorbing surface. Particles in solution
with $h\geq2$ cannot interact with adsorbed particles, as shown in
Figure \ref{fig:Geometry}. The position of each test particle was
offset by a small random vector in $xy$ plane and each test particle
was checked for overlaps with every simulation particle. The fraction
of test particles without overlaps was recorded as the value of $AVF\left(h_{i},\theta_{i}\right)$.
Multiple replicates with different random displacements from the grid
were performed for each $\theta_{i}$ and $h_{i}$. The analysis was
performed with 50 and 500 replicates, and 50 replicates were found
to be sufficient to determine the AVF. The results from multiple runs
of the Brownian dynamics simulation were averaged to obtain an estimate
of the available volume function. The coefficients of the generalized
diffusion equation were computed directly from the available volume
function:\[
k_{1}=\frac{\partial}{\partial h}\log\gamma=\frac{-1}{AVF}\,\frac{\partial AVF}{\partial h}\]
\[
k_{2}=\frac{\partial}{\partial h}\left(\frac{1}{\gamma}\frac{\partial\gamma}{\partial h}\right)=\frac{1}{AVF^{2}}\left(\frac{\partial AVF}{\partial h}\right)^{2}-\frac{1}{AVF}\left(\frac{\partial^{2}AVF}{\partial h^{2}}\right)\]



\subsubsection{Exact Calculation of the Available Volume Function}

%
\begin{figure}
\includegraphics{HardSpheres/Figures/Joined_Particles_and_Images_9May2010}

\caption{\label{fig:Computational geometry}Example showing how computational
geometry can be used to obtain the available volume function. Unavailable
space due to image particles is shown in red, and overlaps between
particles and images are shown in green.}


%
\end{figure}
An alternative method was developed to compute the available volume
function and confirm the results of the implementation of Widom's
method. The Computational Geometry Algorithm Library (CGAL) is collection
of open-source tools for computational geometry~\cite{CGAL}. At
a given height above the surface $h_{i},$ the generalized polygon
class from CGAL was used to compute the union of all the space which\emph{
}could not be occupied by the center of a particle at that particular
height. {}``Image'' particles were created to accurately represent
space occupied by particles at the edge of the domain that {}``wrapped''
to the other side of the domain due to the periodic boundary conditions
in the Brownian dynamics simulation. An example of the union of unavailable
space, with image particles, is shown in Figure~\ref{fig:Computational geometry}.
The fraction of space available for adsorption at a given $h_{i}$
and $\theta_{i}$ could be directly calculated by subtracting the
unavailable area from the total area and dividing the result by the
total area. For computational efficiency, the algorithm started with
the first surface arrangement $\theta_{0}$, calculated the available
area, added the particles that adsorbed at $\theta_{1}$ and calculated
the available area, and so on. This was repeated for each $h_{i}$,
and parallel computing was used to run multiple values of $h_{i}$
in parallel.


\subsection{Implementation of the Continuum Model}

The control volume formulation \cite{Patankar1980} was used to obtain
a finite difference form of Equation \ref{eq:Generalized Diffusion Eqn}
in the region $0\leq h<2$. Central differencing was used to approximate
first derivatives in space, and a fully implicit scheme was used to
approximate time derivatives. To ensure stability, the source term
was linearized so that it was independent of the value of $n$. Any
particle that touched the surfaces adsorbed, so the number density
of particles at the surface was always zero. The Dirichlet (type 1)
boundary condition $n=0$ was used at $h=0$. For simulations with
constant near-surface concentration the type 1 boundary condition
$n=n_{b}$ was applied at $h=2$. 


\subsubsection{Coupling the Continuum Model of Hard-Sphere Adsorption to a Conventional
CFD Transport Simulation}

For simulations in which the concentration at $h=2$ was influenced
by diffusion, a second simulation domain was created to model diffusion
in the bulk for $h\geq2$. The classical diffusion equation was discretized
and solved in the bulk domain in the same manner as the generalized
diffusion equation. The generalized diffusion equation was solved
in the interaction region to obtain the net flux, using the value
of number density at $h=2$ from the previous time step. For small
values of $t$ the surface is mostly available for adsorption, so
the net flux is limited by diffusive transport. The net flux is determined
by $J_{c}$ rather than $J_{s}$. Once the surface is mostly blocked,
the net flux is determined by the rate at which particles can find
available space on the surface, so the value of $J_{s}$ should be
used for the net flux. The correct value for the net flux can be computed
by\begin{equation}
J\left(h=2,t\right)=-min\left(\left|J_{s}\right|,\left|J_{c}\right|\right)\label{eq:Net flux}\end{equation}
 The net flux from the generalized diffusion equation was used as
the left-hand boundary condition to solve the classical diffusion
equation in the bulk, which resulted in a new value for the number
density at the interface. This number density was used to solve the
generalized diffusion equation in the near-surface domain again, and
the iterative process was repeated until the number density at the
interface computed in each domain converged:$\left|n\left(h=2^{-}\right)-n\left(h=2^{+}\right)\right|<\delta$.


\subsubsection{Validation of the Continuum Model}

Equation \ref{eq:Generalized Diffusion Eqn 1D} has the form of a
diffusion equation. In the case that the activity coefficient is constant,
this equation reduces to the classical one-dimensional diffusion equation.
It was verified that the adsorption kinetics predicted by the continuum
model matched the kinetics predicted by the classical diffusion equation
for a perfect adsorbing boundary when the activity coefficient was
held constant ($AVF\left(h,\theta\right)\equiv1$). 


\section*{Results}


\subsection{Brownian Dynamics Simulation Results}

The Brownian dynamics simulation was run with three different number
densities of particles in solution. The number density had a significant
impact on the kinetics of adsorption, as shown in Figure \ref{fig:BrD Adsorption Kinetics}
.%
\begin{figure}
\includegraphics{HardSpheres/Plots/BrD_kinetics}\caption{\label{fig:BrD Adsorption Kinetics}Kinetics of adsorption predicted
by the Brownian dynamics simulation for three different volume fractions.}


%
\end{figure}
 The configurations of adsorbed particles predicted by the Brownian
dynamics simulations were characterized using the pair correlation
function $g\left(r\right)$, which is also known as the radial distribution
function (RDF). The results are shown in Figure \ref{fig:Brownian Dynamics RDF}.
%
\begin{figure}
\includegraphics{HardSpheres/Plots/RDF_BrD_volume_fraction}\caption{\label{fig:Brownian Dynamics RDF}Radial distribution function predicted
by Brownian dynamics simulation for three volume fractions, and the
RDF predicted by the RSA model.}
%
\end{figure}
 Figure \ref{fig:Brownian Dynamics RDF} also shows the pair correlation
function for configurations generated by RSA simulations. A representative
plot of the AVF for a volume fraction of 0.01 is shown in Figure \ref{fig:Available Volume Function}.%
\begin{figure}[H]
\includegraphics{HardSpheres/Plots/AVF_publication_plot}

\caption{\label{fig:Available Volume Function}Plot of the available volume
function. Available surface function for RSA shown as red dashed line.}
%
\end{figure}
 The available surface function (ASF) for the RSA model is also shown
in Figure \ref{fig:Available Volume Function} \cite{Schaaf1989}.
The AVF calculated by the Monte Carlo implementation of Widom's method
matched the results from the computational geometry calculation. The
computational geometry method require more computational resources.


\subsection{Continuum Model Results}

The coefficients of the generalized diffusion equation were computed
from the AVF from Brownian dynamics simulations. Representative plots
of the coefficient values are shown in Figure \ref{fig:Coefficient Plots}.%
\begin{figure}[H]
\subfloat[\label{fig:Coefficient a}Coefficient $k_{1}$]{\includegraphics{/home/cfinch/Brownian_Dynamics/Documentation/Hard_Sphere_Paper/Plots/coeff_a_publication_plot}

}\subfloat[\label{fig:Coefficient b}Coefficient $k_{2}$]{\includegraphics{/home/cfinch/Brownian_Dynamics/Documentation/Hard_Sphere_Paper/Plots/coeff_b_publication_plot}

}

\caption{\label{fig:Coefficient Plots}Coefficients of the generalized diffusion
equation, derived from Brownian dynamics results.}
%
\end{figure}
 The calculation of coefficient was challenging due to the presence
of $AVF^{-1}$ and $AVF^{-2}$ in Equation \ref{eq:Variable coefficients},
which result in large numbers when the value of the AVF approaches
zero. The continuum model accurately reproduced the kinetics predicted
by the Brownian dynamics simulations, as shown in Figure \ref{fig:Kinetics CFD BrD}.%
\begin{figure}
\includegraphics{HardSpheres/Plots/BrD_Continuum_Kinetics_dt}

\caption{\label{fig:Kinetics CFD BrD}Kinetics of adsorption predicted by Brownian
dynamics simulations and the continuum model for $\phi=0.01$.}
%
\end{figure}



\section{Discussion}

The RDFs for particle configurations generated by Brownian dynamics
simulations are almost identical to the RDF for particles generated
by RSA. This agreement indicates that the adsorption of hard spheres
is a random sequential process that is essentially independent of
transport to the surface. If kinetic predictions are not required,
an RSA simulation can be used to generate surface configurations that
are equivalent to results from hard-sphere Brownian dynamics simulations,
with much less computational effort. Since the pair correlation function
shows that Brownian dynamics and RSA simulations produce identical
configurations of adsorbed particles, it is not surprising that the
AVF for Brownian dynamics at $h=0$ is identical to the ASF for random
sequential adsorption of circular disks.

The AVF shown in Figure \ref{fig:Available Volume Function} differs
significantly from the blocking function reported in~\cite{Magan2004},
which was estimated by taking the ratio of flux at the surface to
the flux expected for a perfect adsorbing boundary. In this work the
AVF was computed directly by attempting to adsorb an additional particle
onto a surface with adsorbed particles and verified with a direct
calculation using computational geometry. The method used here is
more likely to obtain an accurate result.

The kinetics of adsorption predicted by the continuum model matched
the kinetics predicted by Brownian dynamics simulations when the Brownian
time step was sufficiently small. The continuum model required about
one minute to run on a desktop PC, while the Brownian dynamics simulation
ran for up to several days on a high-performance cluster to obtain
equivalent results. The continuum model can be scaled to larger domain
sizes much more effectively than Brownian dynamic simulations, which
require detecting collisions between particles. Collision detection
scales as $N^{2}$ using a brute-force approach, although algorithms
have been developed that produce linear scaling with $N$~\cite{Taylor2010}.


\subsection{Future Applications}

This work focused on the adsorption of hard spheres so that the results
of the continuum adsorption model could be rigorously validated by
comparison to results from the RSA model and the classical diffusion
equation. However, the theoretical foundation of the model is directly
applicable to particles with arbitrary force fields. The Widom method
of obtaining the generalized blocking function can also be directly
applied to configurations of particles with long-range {}``soft''
interaction potentials. Brownian dynamics simulations of particles
that interact by means of DLVO forces have already been demonstrated
\cite{Magan2006,Quinn2008,Unni2005}. An advantage of Brownian dynamics
is that it is straightforward to add additional forces and second-order
effects, such as hydrodynamic interactions, without changing the basic
algorithm \cite{Unni2005}. A practical challenge to using this method
for particles with long-range interactions is that more computational
resources will be required for running the Brownian dynamics simulations
and characterizing the chemical potential near the surface. Continuing
advances in simulation algorithms, parallelization techniques, and
microprocessors will make Brownian dynamics simulations even more
useful in the near future. 


\subsection{Conclusions}

The continuum model presented here reproduces the diffusion and adsorption
of hard spheres predicted by Brownian dynamics simulations, while
requiring significantly reduced computational resources. Configurations
of adsorbed particles from Brownian dynamics were analyzed to obtain
an available volume function, which extends the available surface
function into three dimensions. The available volume function was
used to calculate values for the coefficients of the generalized diffusion
equation, and to validate the kinetics predicted by the continuum
model. The continuum model can be coupled to a general-purpose CFD
solver to predict transport and adsorption in practical applications
such as biosensors and models of organs. The method presented here
can be extended to make a priori predictions of protein adsorption
if the particle-particle and particle-surface interaction potentials
can be characterized. 
