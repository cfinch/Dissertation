
\chapter{\texorpdfstring{GENERAL DISCUSSION}{CHAPTER \arabic{chapter}. GENERAL DISCUSSION}}

Modeling and simulation were used to enable advances in nano-scale
surface science and bioengineering that would have been difficult,
time-consuming or impossible with experimental methods alone. Similarly,
modeling and simulation is of limited utility unless quantitative
experiments can be performed to inspire new hypotheses and validate
the results of simulations. Chapter~2 described the derivation of
a new continuum model of hard-sphere adsorption that allows the results
of mesoscale Brownian dynamics simulations to be incorporated into
continuum CFD simulations of practical devices. Although the work
in this chapter was purely theoretical, careful steps were taken to
validate the Brownian dynamics simulation and the continuum CFD simulation.
The results of each simulation were compared against known analytical
results for well-defined test cases, and the results of the two simulations
were compared against one another. This chapter also illustrated the
use of experimental controls in simulation work, by checking the effect
of simulation parameters such as the size of the time step and simulation
domain to ensure that artifacts are not present in the simulation
results. This multi-scale adsorption model is a step towards developing
simulation tools that can make accurate \emph{a priori }predictions
of protein adsorption in microfluidic lab-on-a-chip devices.

The close coupling of experiments and simulations was continued in
Chapter~3, which began with a description of a WGM biosensor to study
the kinetics of protein adsorption on functionalized glass surfaces
relevant for tissue culture. The biosensor does not require any specialized
equipment, and the device components (DFB laser, photodetector, optical
fiber, and simple microfluidics) are inexpensive compared to commercial
systems and can be readily assembled in any laboratory environment.
Despite its simplicity, the sensitivity of the setup rivals that of
a state-of-the art surface plasmon resonance sensor ($\sim1\, pg/mm^{2}$),
with the additional advantage that silica microspheres can be easily
fabricated and conveniently modified with various silane surface coatings
by exploiting established silanol surface chemistries. This is an
alternative to other methods, such as SPR, that require coating the
sensor with gold and are mostly limited to surface modification with
thiols. 

The WGM sensor allowed for the first time the detailed kinetic analysis
of protein adsorption on silane monolayers and enabled some explanation
of the differences between silane and thiol surface modifications.
The WGM sensor was used to obtain the most comprehensive set of kinetic
data that has been reported for the adsorption of fibronectin at low
solution concentrations, and the first quantitative measurement of
the kinetics of glucose oxidase adsorption on silane surfaces. CFD
was used in the design of the flow cell, resulting in a sensor system
which predominantly measures the kinetics of adsorption rather than
the kinetics of transport the sensor. In the case of fibronectin adsorption,
CFD analysis of transport limitations in the flow cell allowed accurate
kinetic parameters to be extracted from the data set in the presence
of transport limitations. Measuring adsorption kinetics at multiple
concentrations allowed a single set of kinetic constants to be fitted
to multiple curves, which increased the likelihood of obtaining a
unique set of fitted parameters. Fitting multiple models to each data
set and evaluating the quality of the fit allowed various hypotheses
about the mechanisms of adsorption to be tested. The use of modeling
and simulation greatly enhanced the conclusions that could be drawn
from a set of experiments.

The results of the WGM biosensor measurements, model fitting and the
cell culture experiments indicate that similar amounts of FN adsorb
on hydrophobic surfaces and charged hydrophilic surfaces. When combined
with antibody data from other studies, our data supports the conclusion
that FN denatures after adsorption on hydrophobic surfaces, leading
to a loss of biological activity. Modeling the adsorption of GO indicated
that the mechanism of adsorption strongly depends on the surface chemistry
of the adsorbing surface. One of the most important goals in biomaterials
research is to discover surfaces that resist protein adsorption. Surfaces
similar to SiPEG (OEG-thiols) have been shown to be the most protein-resistant
class of surfaces~\cite{Ostuni2001}, yet GO and FN both adsorb on
SiPEG in small quantities and the mechanism of protein resistance
remains unknown. Models fitted to protein adsorption on SiPEG seem
to indicate that this surface may actually induce denaturation in
adsorbed proteins. The evidence presented here is tantalizing but
insufficient to draw definitive conclusions. Further improvements
to the sensitivity of WGM biosensors will help answer this and other
difficult questions in biomaterials research.

The relevance of transport modeling in the design of biomicrofluidic
devices was demonstrated in Chapter~4, which presented the design
optimization process for an \emph{in vitro} alveolus. Because of laminar
flow and the no-slip boundary condition, transport in the original
design was not at all {}``plug-like,'' with substantial stagnation
zones in each chamber. Microfluidic structures were added to each
chamber to minimize the stagnation zones and enable a plug of fluid
to move from one chamber to the next with minimal dispersion. The
results from the full device demonstrate that the two types of chamber
can be used as building blocks for devices with multiple chambers.
While each chamber contributes to the widening of the residence time
distribution, the optimized post and baffle designs can effectively
be used to improve overall flow patterns in a multitude of devices
of similar geometry. The use of CFD simulations enabled the optimization
of the device by allowing many design iterations to be performed without
the expense and delay of fabricating a prototype at each iteration.
The CFD simulations were validated qualitatively and quantitatively
with dye visualization experiments in prototype devices. The validation
protocol required only food dye and an optical microscope with a camera\nobreakdash-tools
which are available in almost every laboratory. This project demonstrates
that simulation tools that can quantitatively predict transport and
protein adsorption will enable the rational design of microfluidic
devices for biomedical applications.

The \emph{in vitro} alveolus also demonstrates the importance of quantitative
protein adsorption data. Although it is well known that PDMS must
be modified before use as a cell culture substrate, the modification
protocols currently in use are empirically derived. This project would
benefit greatly from a deeper understanding of how proteins and cells
interact with PDMS. Silicone is typically treated with oxygen plasma
to make it hydrophilic, and sometimes protein is adsorbed on it, before
plating cells. Based on the fibronectin experiments reported here,
it can be hypothesized that FN adsorbed on hydrophilic PDMS retains
a biologically active conformation, while FN adsorbed on untreated
hydrophobic PDMS may not be biologically active. Whispering gallery
mode technology can easily be applied to study the adsorption of extracellular
matrix proteins on PDMS. High-Q WGM resonators made out of PDMS have
already been demonstrated~\cite{Martin2004}. Because PDMS is easy
to mold, it may also be possible to integrate the leaky waveguide
and resonator into one monolithic device that can be mass-produced,
eliminating the time-consuming and tedious task of flame-drawing a
waveguide and securing it into the flow cell with silicone adhesive.
The combination of quantitative measurements and modeling and simulation
will enable new advances in biomedical science.
