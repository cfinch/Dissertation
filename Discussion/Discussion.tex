%% LyX 1.6.10 created this file.  For more info, see http://www.lyx.org/.
%% Do not edit unless you really know what you are doing.
\documentclass[letterpaper,english,PhD]{UCFthesis}
\usepackage[T1]{fontenc}
\usepackage[latin9]{inputenc}
\setcounter{secnumdepth}{3}
\setcounter{tocdepth}{3}
\usepackage{amsmath}
\usepackage{amssymb}

\makeatletter

%%%%%%%%%%%%%%%%%%%%%%%%%%%%%% LyX specific LaTeX commands.
\pdfpageheight\paperheight
\pdfpagewidth\paperwidth


\makeatother

\usepackage{babel}

\begin{document}

\chapter{DISCUSSION}

We have constructed a bench-top WGM biosensor system integrated with
a flow cell to study the kinetics of protein adsorption on functionalized
glass surfaces relevant for tissue culture. Assembly of the table-top
biosensing setup does not require any specialized equipment, and the
device components (DFB laser, photodetector, optical fiber, simple
microfluidics) are inexpensive as compared to commercial systems and
can be readily assembled in any laboratory environment. Despite its
simplicity, the sensitivity of the setup rivals that of a state-of-the
art surface plasmon resonance sensor (\textasciitilde{} 1 pg/mm2 mass
loading), with the additional advantage that silica microspheres can
be easily fabricated and conveniently modified with various silane
surface coatings by exploiting established silanol surface chemistries.
This is an alternative to other methods, such as SPR, that require
coating the sensor with gold and are mostly limited to surface modification
with thiols. It also allowed for the first time the detailed kinetic
analysis of protein adsorption on silane monolayers and enabled some
explanation of the differences between silane and thiol surface modifications.
The WGM sensor was used to obtain the most comprehensive set of kinetic
data that has been reported for the adsorption of fibronectin at low
solution concentrations. Measuring adsorption kinetics at multiple
concentrations allowed a single set of kinetic constants to be fitted
for multiple concentrations, which increased the likelihood of obtaining
a unique set of fitted parameters. The results of the model fitting
and the cell culture experiments indicate that similar amounts of
FN adsorb on hydrophobic surfaces and charged hydrophilic surfaces.
When combined with antibody data from other studies, our data supports
the conclusion that FN denatures after adsorption on hydrophobic surfaces,
leading to a loss of biological activity. Thus, it demonstrates that
a much more sophisticated analysis of how protein adsorption can affect
cellular response to surfaces can be undertaken with this system.
Further improvements to the sensitivity will help answer difficult
questions in biomaterials research, such as improving our understanding
of cell-surface interactions and surfaces that resist protein adsorption.

A design optimization process has been presented that can be used
to optimize the design of a continuous-flow or stop-flow microreactor
so that it operates more like an ideal plug-flow reactor. Plug-like
flow patterns can minimize undesired mixing between compartments and
non-homogeneous concentration profiles across a channel. The results
from the three chamber device demonstrate that the two types of chambers
can be used as building blocks for devices with multiple chambers.
While each chamber contributes to the widening of the residence time
distribution, the optimized post and baffle designs can effectively
be used to improve overall flow patterns in a multitude of devices
of similar geometry. The use of computational fluid dynamics simulations
enabled the optimization of the device by allowing many design iterations
to be performed without the expense and delay of fabricating a prototype
at each iteration. The CFD simulations were validated qualitatively
and quantitatively with dye visualization experiments in prototype
devices. This combination of simulation and experiments was applied
to the design of microfluidic chambers for a bioreactor that models
the alveolus of the lung, demonstrating how the optimization process
can enhance the performance of body-on-a-chip systems for drug and
toxicity studies. 
\end{document}
