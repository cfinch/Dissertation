
\chapter{INTRODUCTION}

Microfluidics is the science and technology of manipulating small
quantities of fluid. One prominent author considers devices with fluid
volumes less than $10^{-9}\, L$ to be microfluidic \cite{Whitesides2006},
while the MEMS Handbook defines a microfluidic channel as having characeristic
dimensions between $1\,\mu m$ and $1\, mm$~\cite{Sharp2002}. Rather
than attempting to agree upon an arbitrary upper limit for the volume
or characteristic dimension of the system, a more practical approach
is to classify a system based upon its characteristics. The defining
characteristics of microfluidic systems are the inability to induce
turbulent flow, a high ratio of surface area to volume, and high shear
rates. At sufficiently small length scales, the flow of a fluid is
completely laminar and it is virtually impossible to induce turbulence.
This means that parallel streams of fluids in a microfluidic channel
will mix only by diffusion, which is a considerably slower process
than the turbulent mixing that can be found in macroscale systems.
Therefore, the creation of mixers is one of the defining challenges
of microfluidics. Because a microfluidic device has so much surface
area relative to its volume, a large fraction of the fluid has the
potential to interact with the surface. Interfacial phenomena such
as protein adsorption are much more significant in a microfluidic
device compared to a conventional one. Since fluid that is in contact
with a surface is generally stationary and turbulence does not occur,
diffusion is the only way that solute can be transported near the
surface. This phenomenon may impose a significant challenge if the
purpose of the device is to deliver nutrients to a layer of cells
or analyte to the surface of sensor. The no-slip boundary condition
on the walls, combined with small channel dimensions, can lead to
high shear rates near the walls. The unique properties of microfluidic
devices lead to both opportunities and challenges when applying the
technology to practical problems.

Microfluidic technology is having a significant impact on the field
of biology, enabling the creation of {}``lab on a chip'' systems
(also know as micro total analysis systems, or \textgreek{m}TAS.)
One important application of microfluidic systems is to culture, manipulate
and analyze single cells or very small populations of cells~\cite{Yi2006}.
The ability to isolate and study a single cell enables systems biologists
to study cell signaling without the noise generated by a heterogeneous
population of cells~\cite{Breslauer2006}. Microfluidics also offers
the promise of developing high-throughput cell-based assays~\cite{Borland2008}.
Microfluidic chambers can also be used to create defined chemical
gradients to study chemotaxis, the movement of cells in response to
chemical cues. Flow cytometry has become a practical tool for clinical
diagnosis, and microfluidic flow cytometers enable measurements to
be performed with fewer cells~\cite{Chung2007}. This is especially
important when taking tissue samples from a fetus or infant. A key
advantage of microfluidic systems is that they can be mass-produced
using conventional microfabrication techniques, which could lead to
inexpensive, disposable analysis chips that take the place of conventional
assays. A recent special issue of \emph{Lab on a Chip} was devoted
to the application of microfluidics to point-of-care (POC) diagnostics~\cite{Sia2008}.


\section{Transport in Microfluidic Systems}

Various dimensionless numbers have been derived to help characterize
fluid systems through dimensional analysis. One of the best-known
is the Reynolds number, which describes the ratio of inertial forces
to viscous (damping) forces~\cite{fox2006}. A large Reynolds number
indicates that inertial forces dominate and the flow may be turbulent.
A small Reynolds number indicates that viscous damping is strong enough
to prevent turbulence from occuring, resulting in laminar flow. The
Reynolds number is defined as\[
Re=\frac{\bar{v}L}{\nu}\]
$\bar{v}$ is the average velocity, $L$ is the characteristic dimension
and $\nu$ is the kinematic viscosity of the system. Because both
the average velocity and dimensions of microfluidic systems tend to
be small, the Reynolds number is also small, corresponding to the
observation that flow in microfluidic systems is almost always laminar.

If a volume of fluid is sufficiently small, there may not be enough
molecules in the system to satisfy the assumptions of continuum models.
The Knudsen number is a dimensionless group that can be used to assess
the validity of the the continuum approximation. For a gas, the Knudsen
number is defined as\[
Kn=\frac{\lambda}{L}\]
$\lambda$ is the mean free path of a molecule in the gas. Molecules
in a liquid do not have a mean free path becaue their motion is highly
constrained by their neighbors, so the lattice spacing $\delta$ can
be used instead of $\lambda$ to compute the Knudsen number~\cite{Sharp2002}.
The lattice spacing for liquid water is about $0.3\, nm$. Knudsen
numbers below $10^{-3}$ indicate that the continuum approximation
is valid, while in the range $10^{-3}<Kn<10^{-1}$ continuum models
can be used with slip boundary conditions. For $Kn>10^{-1}$ entirely
different modeling procedures must be used to obtain accurate results.


\section{Protein Adsorption}

Non-specific binding (adsorption) of biomolecules, such as proteins,
at solid-liquid interfaces affects the function of materials and devices
intended for use with physiological fluids and tissues. Microfluidic
devices are especially prone to protein adsorption because of the
large amount of surface area relative to volume of the device~\cite{Walker2004}.
When designing a microfluidic system to deliver analyte to a cell
or sensor, the adsorption of analyte to the walls of tubing or channels
must be taken into account to ensure that the desired amount of analyte
actually reaches its destination. Protein adsorption is the first
step in many important biological processes, including the attachment
of cells to bioengineered surfaces, the coagulation of blood, and
the response of the immune system to an implanted device~\cite{Rabe2010}.
Adsorption of biomolecules can lead to the formation of biofilms of
bacteria~\cite{Cheng2007} or blood components~\cite{Sun2003} that
can lead to infection or clogging. Therefore, biomolecule adsorption
is a crucial factor in determining the long-term efficacy of lab-on-chip
systems, implants, and medical devices that contact blood or other
biological fluids~\cite{Roach2007,Latour2005,Ramsden2007}. For example,
advances in neurally-controlled prosthetics have been limited by the
body's inflammatory response to implanted sensors~\cite{Lai2007}. 

The inherent variability of protein sequence and structure makes the
prediction of protein adsorption from first principles an intractable
problem \cite{Wilson2004}. Thus, it has been necessary to devise
experimental solutions for making quantitative observations that can
be used to assess the biocompatibility of materials. Early research
on protein adsorption focused on measuring the surface concentration
of adsorbed protein at equilibrium. A material with high surface area
was allowed to soak in a protein solution, and the amount of adsorbed
protein was inferred from the loss of protein in solution. By measuring
the surface concentration for various solution concentrations, it
is possible to plot an {}``isotherm'' that provides some information
about the thermodynamics of adsorption. By the mid-1980s new optical
methods such as total internal reflection fluorescence (TIRF) spectroscopy~\cite{Axelrod1983}
and ellipsometry enabled researchers to measure the kinetics of adsorption~\cite{Andrade1986}.
In the 1990s optical waveguide light spectrocopy (OWLS)~\cite{Kurrat1997}
and surface plasmon resonance (SPR)~\cite{Mrksich1995} instruments
became available. Despite significant progress in our understanding
of protein adsorption, only a handful of protein/surface combinations
have been thoroughly studied and the general problem of predicting
and controlling protein adsorption remains unsolved~\cite{Rabe2010,Gray2004}.


\section{Overview}

This work describes theoretical advances in the modeling and simulation
of microfluidic systems and demonstrates the practical application
of those techniques. A new multi-scale model of the adsorption of
hard spheres was formulated to bridge the gap between simulations
of discrete particles and continuum fluid dynamics. A whispering gallery
mode (WGM) biosensor was constructed and used to measure the kinetics
of adsorption for two types of proteins on four different surfaces.
Computational fluid dynamics was used to analyze the transport of
proteins in the flow cell of the biosensor. Kinetic models of protein
adsorption that take transport limitations into account were fitted
to the experimental data and used to draw conclusions about the mechanisms
of adsorption. Transport simulations were then applied to the practical
problem of optimizing the design of a microfluidic bioreactor to enable
{}``plugs'' of fluid to flow from one chamber to the next with minimal
dispersion. Experiments were used to validate the transport simulations.
The combination of quantitative modeling and simulation with experimental
work leads to results that could not be achieved using either method
by itself.
