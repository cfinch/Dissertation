
\chapter{INTRODUCTION}

Microfluidic technology has had a major impact on 


\section{Protein Adsorption}

Non-specific binding (adsorption) of biomolecules at solid-liquid
interfaces is a phenomenon that affects the function of materials
and devices intended for use with physiological fluids and tissues
\cite{Andrade1986}. Therefore, understanding biomolecule adsorption
is a crucial factor in determining detection limits, biocompatibility,
and long-term efficacy of lab-on-chip, microfluidic, and Bio-MEMS
devices due to loss of analyte by adsorption or fouling of microchannels
and active sensors. Proteins are particularly notorious for their
ability to non-specifically stick to materials. Adsorption induced
changes in the structure of a protein molecule is a critical factor
in determining its subsequent function. However, the inherent variability
of protein sequence and structure makes the prediction of protein
adsorption from first principles an intractable problem \cite{Wilson2004}.
Thus, it is necessary to devise experimental solutions for making
quantitative observations that can be used to assess the biocompatibility
of materials and create reduced-order models for predicting protein
adsorption.


\subsection{Experimental Measurements of the Kinetics of Protein Adsorption}

Literature survey


\section{Modeling Protein Adsorption}

Literature survey
