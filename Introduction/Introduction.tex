
\chapter{INTRODUCTION}

Microfluidics is the science and technology of manipulating small
quantities of fluid. One prominent author considers fluid volumes
less than $10^{-9}L$ to be microfluidic \cite{Whitesides2006}, while
many published microfluidic systems contain up to several mililiters
of fluid. Rather than attempting to agree upon an arbitrary upper
limit for the volume of the system, a more practical approach is to
classify a system based upon its characteristics. The two defining
characteristics of microfluidic systems are the inability to induce
turbulent flow and the high ratio of surface area to volume. At sufficiently
small length scales, the flow of a fluid is completely laminar and
it is virtually impossible to induce turbulence. This means that parallel
streams of fluids in a microfluidic channel will mix only by diffusion,
which is a considerably slower process than it would be in a macroscale
system. Therefore, the creation o mixers is one of the defining challenges
of microfluidics. Because a microfluidic device has so much surface
area relative to its volume, a large fraction of the fluid has the
potential to interact with the surface. Interfacial phenomena such
as protein adsorption are much more significant in a microfluidic
device compared to a conventional one. Since fluid that is in contact
with a surface is generally stationary and turbulence does not occur,
diffusion is the only way that solute can be transported near the
surface. This phenomenon may impose a significant challenge if the
purpose of the device is to deliver nutrients to a layer of cells
or analyte to the surface of sensor.

Microfluidic technology has been applied to many fields.

Microfluidic technology has had a significant impact on the field
of biology.

A general overview of the field of microfluidics can be found in a
recent special issue of \emph{Lab on a Chip} \cite{Santiago2009}.


\section{Transport in Microfluidic Systems}


\section{Protein Adsorption}

Non-specific binding (adsorption) of biomolecules at solid-liquid
interfaces affects the function of materials and devices intended
for use with physiological fluids and tissues \cite{Andrade1986}.
Microfluidic devices are especially prone to protein adsorption because
of the large amount of surface area relative to volume of the device.
Protein adsorption is the first step in many important biological
processes, including the attachment of cells to bioengineered surfaces,
the coagulation of blood, and the response of the immune system to
an implanted device. Therefore, biomolecule adsorption is a crucial
factor in determining the long-term efficacy of lab-on-chip systems,
implants, and medical devices that contact blood or other biological
fluids. Non-specific binding also affects the detection limits of
biosensors when analyte is lost by adsorption to the walls of the
channel. Adsorption of biomolecules is also the first step in the
formation of unwanted biofilms that can lead to clogging. Proteins
are particularly notorious for their ability to non-specifically stick
to materials. Adsorption induced changes in the structure of a protein
molecule are critical to determine its subsequent function. However,
the inherent variability of protein sequence and structure makes the
prediction of protein adsorption from first principles an intractable
problem \cite{Wilson2004}. Thus, it is necessary to devise experimental
solutions for making quantitative observations that can be used to
assess the biocompatibility of materials.


\subsection{Experimental Measurements of Protein Adsorption}

There is a vast body of experimental research on protein adsorption. 


\subsection{Models of Protein Adsorption}

Although experimental measurements are indispensible for studying
protein adsorption, the development of models is critical to understand
and interpret experimental results.




\section{Summary}

This work describes theoretical advances in the modeling of microfluidic
systems and demonstrates the practical application of those techniques.
A new continuum model of the adsorption of hard spheres was formulated
to bridge the gap between simulations of discrete particles and continuum
fluid dynamics. A whispering gallery mode (WGM) biosensor was constructed
and used to measure the kinetics of protein adsorption on four types
of surfaces. Computational fluid dynamics was used to analyze the
transport of proteins in the flow cell of the biosensor. Kinetic models
of protein adsorption that take transport limitations into account
were fitted to the experimental data and used to draw conclusions
about the mechanisms of adsorption. Transport simulations were then
applied to the practical design problem of optimizing flow in a microfluidic
bioreactor. Experiments were used to validate the transport simulations.
The combination of quantitative modeling and simulation with experimental
work leads to results that could not be achieved by either method
by itself.
