
\chapter{CFD SIMULATION OF TRANSPORT IN A MICROFLUIDIC BIOREACTOR}


\section{Introduction}

The principles of transport and adsorption described in the previous
chapters were applied to the design of a microreactor that models
the function of an alveolus. Alveoli are a micro-scale structures
in the lung that facilitate the exchange of oxygen and carbon dioxide
between the blood and inhaled air. Each alveolus is typically $200-250\,\mu m$
in diameter, and the 300-480 million alveoli in the lungs of an adult
human provide a total surface area of $50-100\, m^{2}$ for gas exchange~\cite{Levitsky2007}.
The alveolar membrane consists of a single layer of epithelial cells
in contact with the air and a single layer of endothelial cells that
forms a capillary for blood flow. Development of an \emph{in vitro}
model of the alveolus-capillary interface would be important to biomedical
science, as this barrier tissue is an point of entry for toxins and
plays a role in compromised gas exchange in disease states. A recent
study found that only 11\% of the drugs that passed preclinical trials
eventually became approved for use by either U.S. or European regulatory
agencies, and that the major causes of drugs failing clinical trials
were lack of efficacy and toxicity \cite{Kola2004}. The majority
of these failures occurred in the expensive late phases of clinical
trials. With the potential to overcome the failures of traditional
cell-based drug assays in predicting physiological behavior, body-on-a-chip
devices have become an increasing area of interest in pharmaceutical
testing \cite{Sung2010}. Body-on-a-chip devices have been created
based on pharmacokinetic models in attempts to supplement traditional
cell culture methods that are not always capable of modeling tissue-tissue
interactions that can occur when the body is exposed to pharmaceuticals
\cite{Esch2011}. 

Creating an \emph{in vitro} model of the alveolus poses unique challenges
due to the gas-liquid environment, layers of disparate functional
cell types, and the transport of nutrients, cells, and dissolved gas
species through the tissue. One lung-tissue based device was produced
out of poly(dimethyl siloxane) (PDMS) to recreate the structure, mechanics,
and certain aspects of transport that occur in lung tissue \cite{Huh2010}.
The elastomeric PDMS material used in the device was highly permeable
to both oxygen and carbon dioxide, which precluded the ability to
create physiologically relevant gas concentrations in the liquid phase
of the environment. Additionally, the high gas diffusion through the
PDMS prevented measurement of transport of these gas species through
the tissue, a fundamental function of the alveolus-capillary interface
in the lung. To overcome these limitations silicon based microfluidic
technology was utilized to create an \emph{in vitro} body-on-a-chip
model of the lung with a long-term goal of quantifying the rate of
gas transport across the model alveolar membrane. To replicate the
physiological function of the alveolus, liquid that enters the model
alveolus with a gas composition similar to that of venous blood should
exit the chamber with a gas composition similar to that of arterial
blood. To be able to detect small changes in the rate of gas exchange
across the model alveolar membrane, the alveolar chamber should be
designed such that the gas composition of the liquid reaches arterial
values just before the liquid exits the chamber. 

After leaving the alveolar chamber, the liquid will pass through another
chamber containing dissolved-gas sensors. These sensors require several
minutes to register small changes in gas levels, so the system was
designed to be operated in either continuous or stop-flow mode. The
microfluidics must be designed so that a {}``plug'' of fluid moves
from the alveolar chamber into the sensor chamber with relatively
little mixing in the the longitudinal direction. The residence time
distribution of an ideal plug-flow system is a Dirac delta function,
so the width of the residence time distribution can be used to characterize
how similar a system is to plug flow. Each chamber must be designed
so that liquid flowing through different parts of the chamber experiences
a similar residence time within the chamber. This objective is equivalent
to minimizing the width of the residence time distribution, or approximating
plug flow.

Microreactors have recently been employed for synthetic organic chemistry,
nanoparticle synthesis, and ultra-fast DNA amplification \cite{DeMello2006}.
However, the combination of the no-slip boundary condition and laminar
flow in microfluidic devices produces flow patterns that deviate substantially
from the desired plug flow, presenting a challenge to the design of
continuous-flow microreactors \cite{Hartman2011}. Several innovative
methods have been developed to create plug flow in a microreactor,
such as creating a multiphase flow with dispersed droplets or {}``slugs''
of water containing reagents in a stream of oil \cite{Song2003} or
gas \cite{Guenther2005}. For single-phase flows, a passive microfluidic
structure known as a herringbone mixer can be added to a microfluidic
channel to reduce the width of the residence time distribution to
approximate plug flow \cite{Cantu-Perez2010}. Ideally, combinations
of simulation based design and experimental validation of the results
would be the best approach for successful system development. However,
general methods for minimizing the residence time distribution in
a continuous-flow microreactor by a combination of simulation and
experiment has not been reported. 

In experimental attempts to optimize microfluidic systems, the primary
method employed for visualizing flow patterns in the devices has involved
fluorescent or colored dyes \cite{Carlotto2011,Ryu2011,Campbell2004,Hessel2005}.
Visualization experiments have focused mainly on the mixing of two
components, though this method has been used to a limited extent to
measure residence times in certain devices \cite{Hornung2009}. Both
qualitative visual representations of flow and quantitative measurement
of concentrations can be obtained using dyes, though quantitative
concentration measurements are less common than qualitative representations.
Theoretical studies have been performed to predict the residence time
distributions of various types of microfluidic structures. Analytical
models have been used to characterize the hydrodynamic dispersion
in microchannels \cite{Dutta2006}. However, these methods are limited
to devices with simple, well-defined geometries. Computational fluid
dynamics (CFD) simulations must be used to simulate practical microfluidic
devices with more complex geometries. Studies have shown that CFD
simulations can predict convective and diffusive transport in microfluidic
systems with a high level of accuracy \cite{Znidarsic-Plazl2007}.
Recently, CFD simulations have been used to predict the residence
time distribution of a microfluidic mixing device by applying a pulse
of a {}``tracer'' at the inlet of the device and monitoring the
average concentration at the outlet \cite{Adeosun2009}. CFD has also
been used in conjunction with a particle tracking simulation to model
the residence time distribution of a microfluidic channel containing
a mixing structure \cite{Cantu-Perez2010}. Although CFD simulations
have been used to analyze the residence time distribution of microfluidic
devices, the use of CFD to optimize these devices has not been widely
reported.

Protein adsorption also plays an important role in the microfluidic
model of the alveolus. A PDMS membrane was chosen as the substrate
for culturing monolayers of epithelial and endothelial cells because
of its high permeability to oxygen and carbon dioxide. PDMS is regarded
as a poor substrate for cell culture, so it is standard practice to
treat the polymer with an oxygen plama and adsorb fibronectin onto
the membrane prior to tissue culture~\cite{Powers2010}. PDMS can
be easily molded to create intricate shapes. This capability will
eventually be utililized to create porous membranes that will allow
nutrients, chemical signals and even macrophages to pass between the
epithelial and endothelial cell layers. 

The general field of microreactor design would benefit from a new
methodology combining experimental and simulation-based development.
This capability would be particularly germane to biomedical engineering
of new systems for drug discovery and toxicology. CFD simulations
were used as part of an iterative design process to quantify the convective
and diffusive transport of dissolved gas in the liquid side of the
alveolus and achieve the best possible approximation to plug flow.
The flexibility of the design process was demonstrated by the fabrication
of an alveolar chamber that could accommodate a standard tissue culture
support and a polygonal chamber for more general applications. The
CFD predictions for different design iterations were validated with
quantitative and qualitative experimental measurements using dye visualization.
The design techniques presented here can be utilized to optimize the
performance of microreactors for any application.


\section{Materials and Methods}


\subsection{Device Fabrication }

Silicon was chosen as the substrate for the microfluidic device due
to its impermeability to gas and the ability to incorporate microstructures
using standard microfabrication techniques, and to allow future incorporation
of other tissue compartments. A planar design was chosen to avoid
drilling or etching through the silicon substrate. Devices were fabricated
using silicon wafers with a thickness of $500\mu m$. Wafers were
liquid primed with P20 and spun with Shipley Microposit S1818 photoresist
at 3000 rpm for 60 seconds, followed by a softbake at $115^{\circ}C$
for 60 seconds on a hotplate. Exposure was performed on an ABM contact
aligner and the photoresist was developed with 726MIF developer. Silicon
wafers were etched with a Unaxis 770 deep reactive ion etcher (DRIE)
to a depth of $125\mu m$. The remaining photoresist was stripped,
and the devices were separated with a dicing saw. Devices that incorporated
all three chambers on the same chip as well as isolated chambers for
each component were fabricated and tested. 

Housings were machined from acrylic plates to produce the outside
portions of the microfluidic device. In the bottom section, a rectangular
area was milled to approximately 0.8 mm deep into which a sheet of
PDMS elastomer with nominal thickness of 0.6 mm was placed. The silicon
device was placed on top of the elastomer sheet so that the sheet
pushed the silicon device into the flat surface of the housing top
to produce a tight seal. Inlet and outlet holes were drilled into
the housing top to direct fluid into the inlet of the silicon device
and allow the liquid to escape from the outlet. The two housing pieces
each had a series of threaded screw holes and were held together with
twelve screws. The combined gasket and silicon device thickness was
larger than the depth of the rectangular recess on the bottom housing
piece so that the screws could be tightened to produce a seal between
the top housing piece and the silicon device. 


\subsection{CFD Simulation Methodology}

The CFD-ACE+ suite of simulation tools was used to predict the performance
of the device. The Reynolds and Knudsen numbers were computed to ensure
that flow would be entirely laminar and that the continuum approximations
inherent in CFD would be valid for the system. CFD-GEOM was used to
define the geometry and create a regular mesh throughout the liquid
volume. The velocity derivatives were approximated with upwind differencing
and a 2nd order limiter was used to approximate the derivatives for
computing concentration. The system was simulated in steady-state
with pure water to determine the flow field (it was assumed that dilute
solutions of small molecules would not have a significant impact on
the viscosity of the water.) A transient simulation of gas transport
was then performed, using the flow field from the steady-state simulation
to avoid re-calculating the flow field at each time step. Initially,
no dissolved gas was present in the water. To characterize the residence
time distribution of each design, the inlet boundary condition introduced
water saturated with carbon dioxide (mass fraction of 0.03366) at
a volumetric flow rate of $35\,\mu l\, min^{-1}$, approximating a
step function at the inlet of the device at t=0. The results of this
simulation protocol should also be valid for dissolved oxygen, which
has a diffusion coefficient similar to carbon dioxide ($2\times10^{-9}m^{2}s^{-1}$
for oxygen \cite{Jamnongwong2010} and $1.7\times10^{-9}m^{2}s^{-1}$
for carbon dioxide \cite{Tamimi1994}). The Euler method was used
for time stepping, and the time step was calculated automatically
using the Auto Time Step feature in CFD-ACE-GUI. The minimum time
step was set to $100\,\mu s$. The time step was automatically adjusted
based upon the rate of change of solution variables. The typical maximum
time step was 1 ms, and the simulation results were saved at regular
time intervals. The results were visualized using CFD-VIEW. To quantify
the average concentration across the inlet or outlet channel at a
specific point in time a cut plane was created across the channel
and the CFD-VIEW calculator was used to average the concentration
across the plane. The residence time distribution for the device was
calculated by taking the derivative with respect to time of the normalized
concentration at the outlet. 

Once the device was optimized to minimize the width of the residence
time distribution for dissolved gases in water, another set of simulations
was performed to predict the transport of dye for comparison with
visualization experiments. The diffusion coefficient of Brilliant
Blue FCF dye has not been reported, so it was estimated based on known
diffusion coefficients of similar dye molecules. The molar masses
and diffusion coefficients of Fluorescein ($332g\, mol^{-1}$), Rhodamine
6G ($471g\, mol^{-1}$), Congo Red ($697g\, mol^{-1}$), and Trypan
Blue ($961g\, mol^{-1}$) were plotted on a scatter plot \cite{Nakagaki1950,Inglesby2001,Culbertson2002}.
A roughly linear relationship was observed between molar mass and
diffusion coefficient. Based on this relationship, the diffusion coefficient
of Brilliant Blue FCF was estimated to be approximately $2\times10^{-10}m^{2}s^{-1}$.


\subsection{Design Methodology}

%
\begin{figure}
\begin{tabular}{ccc}
\includegraphics{Bioreactor/Figures/alv_v0} & \includegraphics{Bioreactor/Figures/alv_v1} & \includegraphics{Bioreactor/Figures/alv_v2}\tabularnewline
\includegraphics{Bioreactor/Figures/alv_v3} & \includegraphics{Bioreactor/Figures/alv_v4} & \includegraphics{Bioreactor/Figures/alv_v5}\tabularnewline
\end{tabular}

\caption{\label{fig:Alveolar chamber designs}Progressive refinement of the
design of the alveolar chamber. The earliest design is at the upper
left, and the final design is at the lower right.}
%
\end{figure}
The chamber dimensions and internal structures were optimized by successive
computational fluid dynamics modeling of the system. The iterative
design process revolved around minimizing the width of the residence
time distribution. A CFD simulation was performed for each new design
iteration and the residence time distribution was calculated to evaluate
that iteration. The alveolar chamber was designed to accommodate a
circular cell culture membrane, such as a Transwell insert (Corning
Life Sciences), with a diameter of $6.5\, mm$. The progression of
designs is shown in Figure~\ref{fig:Alveolar chamber designs}.Baffles
in the shape of circular arcs were added to the chamber to support
the membrane and reduce the width of the residence time distribution.
The baffles divide the incoming liquid into several streams such that
the velocity of the liquid in each stream is roughly proportional
to the length of the path it takes through the chamber. Specifically,
the baffles were designed so that liquid following the longest path
around the perimeter of the chamber would have the same residence
time as liquid that traversed the shortest path through the center
of the chamber. The arc length of each baffle was kept as short as
possible, since the no-slip boundary condition on the baffles contributed
to axial dispersion within each channel. The outer baffle was designed
to be long enough to prevent a stagnation zone from forming near the
outlet of the chamber. The inner baffles were made progressively shorter.
A small gap was added near the end of each baffle to prevent stagnation
on the channel inside of the baffle. 

%
\begin{figure}
\begin{tabular}{ccc}
\includegraphics{Bioreactor/Figures/cond_v1a} & \includegraphics{Bioreactor/Figures/cond_v1b} & \includegraphics{Bioreactor/Figures/cond_v1c}\tabularnewline
\includegraphics{Bioreactor/Figures/cond_v1d} & \includegraphics{Bioreactor/Figures/cond_v1e} & \includegraphics{Bioreactor/Figures/cond_v1f}\tabularnewline
\end{tabular}

\caption{\label{fig:Conditioning chamber designs}Progressive refinement of
the design of the conditioning chamber. The earliest design is at
the upper left, and the final design is at the lower right.}
%
\end{figure}
The design used for the bubble trap and conditioning chamber was chosen
to be a rectangle with tapering inlet and outlet sections. To allow
the device to operate in a stop-flow mode, these chambers were designed
to have the same volume as the alveolar chamber. The angle of the
taper and the chamber width were adjusted to minimize the width of
the residence time distribution while maintaining a volume equal to
that of the circular chamber. A staggered array of posts was then
created in each chamber to modify the flow characteristics. The design
iterations of the conditioning chamber are shown in Figure~\ref{fig:Conditioning chamber designs}.
The vertical spacing between posts and the number of columns of posts
were adjusted in the CFD model until the flow through the chamber
was as close as possible to plug flow. Because the chamber wall reduced
the velocity of fluid flowing between the top row of posts and the
chamber wall, the spacing between the top row of posts and the chamber
wall had to be larger than the spacing between posts to avoid introducing
a stagnation zone. 


\subsection{Dye Visualization Experiments }

Blue food dye, containing Brilliant Blue FCF (FD\&C Blue 1) dye, was
used for visualizing the flow of liquid through the microfluidic device.
The concentration of dye in a phosphate buffered saline (PBS) solution
used in the experiments was selected based on preliminary data of
light intensity captured by the camera versus dilution of the dye
solution. The solution concentration was chosen to maximize sensitivity
to changes in concentration throughout the range of fully dilute to
non-diluted from the initial concentration. For the initial solution
concentration chosen, the intensity versus concentration profile was
approximately linear, though this relationship would not hold if a
higher concentration of dye was chosen. Visualization experiments
were performed by pumping the dye solution into the devices initially
containing 1X PBS with no dye and visualizing the movement of the
blue dye solution in the chambers. The dye was pumped at a rate of
$35\,\mu l\, min^{-1}$ using a syringe pump. The individual chamber
designs (polygonal chamber and round alveolar chamber) were tested
for dye visualization both with and without posts or baffles to evaluate
the effect of the simulation-driven baffle or post structures on the
flow pattern. Additionally, the alveolar chamber in the final combined
device\nobreakdash-incorporating two polygonal chambers and one alveolar
chamber\nobreakdash-was tested to assess the flow pattern in the
circular chamber in the practical device. The flow patterns in the
various devices were also used for comparison to the CFD predictions. 

Dimensional analysis was utilized to understand how the device would
perform with dissolved gas, which has a diffusion coefficient that
is roughly an order of magnitude higher than that of the dye. The
P�clet number is the dimensionless number that represents the ratio
of convective transport to diffusive transport in a fluidic system.
The P�clet number in a channel is given by $Pe=LU/D$, where $L$
is the characteristic dimension of the channel, $U$ is the average
velocity in the channel, and $D$ is the diffusion coefficient of
the solute. For a rectangular channel, the hydraulic diameter is used
as the characteristic dimension. Using the average velocity in the
channels between the chambers calculated by the CFD simulation, the
P�clet number for dissolved gas with a diffusion coefficient of was
computed to be 770 for a volumetric flow rate of $35\,\mu l\, min^{-1}$.
Because the diffusion coefficient of the dye is one order of magnitude
lower than that of dissolved gas, experiments with dye were performed
with a volumetric flow rate of $3.5\,\mu l\, min^{-1}$ to obtain
a P�clet number comparable to that of dissolved gas. Experiments were
also performed with a flow rate of $350\,\mu l\, min^{-1}$ to characterize
the performance of the system over a range of two orders of magnitude. 


\subsection{Image Analysis }

Images and video of the flowing liquid in the round alveolar chamber
were captured using a Hitachi KP-20a CCD microscope video camera attached
to a Zeiss Stemi DV4 stereo microscope. Because the blue dye absorbs
light primarily in the red region of the visual spectrum, the red
channel of the captured images was isolated and used for image analysis
using ImageJ \cite{Rasband1997-2011}. Video analysis was performed
for flow experiments in which a solution of dye was pumped into the
device. Video was obtained at 5 frames per second from the time dilute
dye could be observed near the inlet of the chamber of interest until
the entire chamber had been filled with dye for several seconds. The
average red channel intensity was measured across a line of pixels
at the inlet of the chamber and at the outlet of the chamber for each
video frame. For each frame, these average intensities for the inlet
and outlet were normalized to four points outside of the chamber unexposed
to dye to account for any variations in overall light intensity. For
both the inlet and outlet, the average red channel intensity of the
first ten frames (corresponding to 0\% saturation) and the last ten
frames (corresponding to 100\% saturation) were computed. The average
concentration across the inlet and outlet for each frame was calculated
by linearly interpolating the average intensity between the 0\% and
100\% saturation intensities found from the beginning and end of the
experiment. The reported concentrations represent the average of three
experiments under the same conditions, with the curves aligned in
time to the points in which the respective inlet concentration reached
15\% saturation. 


\section{Results}

%
\begin{figure}
\subfloat[\label{fig:RT alveolus}Alveolar chamber]{\includegraphics[width=3in]{Bioreactor/Figures/RT_alveolus}}\subfloat[\label{fig:RT Conditioner}Conditioner]{\includegraphics[width=3in]{Bioreactor/Figures/RT_conditioner}

}

\subfloat[\label{fig:RT 3 chambers}Alveolus on full chip]{\includegraphics[width=3in]{Bioreactor/Figures/RT_3chambers}

}

\caption{\label{fig:Residence time dist CFD}Residence time distributions predicted
by CFD for carbon dioxide-saturated water.}
%
\end{figure}
The design goal was to obtain the best possible approximation to plug
flow so that a bolus of liquid would flow from one chamber into the
next with minimal mixing between the chambers. The residence time
distributions predicted by the CFD simulation for the initial and
final designs of the round alveolar chamber and the polygonal chamber
are shown in Figure~\ref{fig:RT alveolus} and Figure~\ref{fig:RT Conditioner},
respectively. Figure~\ref{fig:RT 3 chambers} shows the predicted
combined residence time distribution for first two chambers of the
device (bubble trap and conditioning chamber) and for the first three
chambers of the device (bubble trap, conditioning chamber, and alveolus).


\subsection{Experimental Validation of CFD Simulations}

%
\begin{figure}
\includegraphics[width=4in]{Bioreactor/Figures/Microreactor}

\caption{\label{fig:Microreactor}Microreactor design~(a), fabricated silicon
chip~(b), and chip in acrylic housing~(c)}
%
\end{figure}
The overall layout of the device is shown in Figure~\ref{fig:Microreactor}a.
The microfluidic device fabricated in silicon is shown in Figure~\ref{fig:Microreactor}b.
The silicon device in the acrylic housing is shown in Figure~\ref{fig:Microreactor}c.
The first chamber after the inlet is a bubble trap \cite{Sung2009},
followed by a chamber that will eventually be used to condition the
liquid to contain physiologically relevant concentrations of dissolved
gas. Another bubble trap is located downstream (to the right) of the
alveolar chamber to prevent bubbles from accumulating in the sensor
chambers, followed by two chambers that will eventually house dissolved-gas
sensors. For additional chip designs were fabricated (not shown) to
test each chamber design in isolation: the alveolar chamber with and
without baffles and the conditioning chamber with and without posts.

%
\begin{figure}
\subfloat[\label{fig:Alveolus no baffles}Without baffles]{\includegraphics[width=3in]{Bioreactor/Figures/Alveolus_without_baffles}

}\subfloat[\label{fig:Alveolus with baffles}With baffles]{\includegraphics[width=3in]{Bioreactor/Figures/Alveolus_with_baffles}

}

\caption{\label{fig:Alveolus CFD vs Dye}Visualization of the flow in the alveolus
predicted by CFD (left columns) and imaged with dye (right columns)}
%
\end{figure}
The flow pattern inside an isolated alveolar chamber with and without
baffles is shown in Figure~\ref{fig:Alveolus CFD vs Dye}, including
both CFD results and experimental visualization. %
\begin{figure}
\subfloat[\label{fig:Conditioner without posts}Without posts]{\includegraphics[width=3in]{Bioreactor/Figures/Conditioner_without_baffles}

}\subfloat[\label{fig:Conditioner with posts}With posts]{\includegraphics[width=3in]{Bioreactor/Figures/Conditioner_with_baffles}

}

\caption{\label{fig:Conditioner CFD Dye}Visualization of the flow in the conditioning
chamber predicted by CFD (left columns) and imaged with dye (right
columns) }
%
\end{figure}
 Figure~\ref{fig:Conditioner CFD Dye} indicates the amount of dye
present in different locations within the isolated conditioning chamber/bubble
trap at various times. The flow patterns for the conditioning chambers
with and without the post arrangement are shown along with the flow
patterns expected from the CFD results. The visual scale of the CFD
results was normalized so that pure water was represented as white
and the maximum dye concentration was black. Images from the visualization
experiments of the alveolar chamber in the combined three chamber
device are compared to the images from the isolated alveolar chamber
and CFD results in Figure~\ref{fig:3 Chambers CFD Dye}. %
\begin{figure}
\includegraphics[width=3in]{Bioreactor/Figures/3_Chambers_baffles}

\caption{\label{fig:3 Chambers CFD Dye}Visualization of the flow in the alveolar
chamber in the full chip predicted by CFD (left column) and imaged
with dye (right column)}


%
\end{figure}


Figure~\ref{fig:Alveolus outlet} indicates the dye concentration
at the outlet of the alveolar chamber as a function of time when a
concentration step function was applied at the chamber inlet. The
outlet concentrations for the conditioning chamber designs, as analyzed
from the video captures, are shown in Figure~\ref{fig:Conditioner outlet},
along with the concentrations expected from the CFD results for these
chambers. Figure~\ref{fig:3 chambers outlet} includes the outlet
concentration of the alveolar chamber in the three chamber design
and the CFD modeling of this device. Dotted lines indicate plus or
minus one standard deviation from the experimental average, while
a dashed line indicates the concentration predicted by CFD. Figure~\ref{fig:Outlet 3 flow rates}
indicates the concentration at the outlet of the first three chambers
for flow rates equal to one tenth and ten times the nominal flow rate.
%
\begin{figure}
\includegraphics{Bioreactor/Figures/Outlet_alveolus}

\caption{\label{fig:Alveolus outlet}Saturation at the outlet of the alveolus}


%
\end{figure}
%
\begin{figure}
\includegraphics{Bioreactor/Figures/Outlet_conditioner}

\caption{\label{fig:Conditioner outlet}Saturation at the outlet of the conditioning
chamber}
%
\end{figure}
%
\begin{figure}
\includegraphics{Bioreactor/Figures/Outlet_3chambers}

\caption{\label{fig:3 chambers outlet}Saturation at the outlet of the alveolus
on the full chip}
%
\end{figure}
%
\begin{figure}
\includegraphics{Bioreactor/Figures/Outlet_three_flow_rates}

\caption{\label{fig:Outlet 3 flow rates}Relative concentration at the outlet
of the alveolus in the full chip for three different flow rates}


%
\end{figure}



\section{Discussion}

The residence time distributions in Figures \ref{fig:RT alveolus}
and \ref{fig:RT Conditioner} indicate that the design optimization
process clearly reduced the width of the residence time distribution
for each chamber, improving the approximation to plug flow. The residence
time distribution of the round chamber without baffles exhibited a
long {}``tail'' which was caused by the stagnation zones that formed
near the outlet of the chamber (Figure \ref{fig:Alveolus no baffles}).
The addition of baffles forced liquid from the inlet to flow around
the outer perimeter of the chamber, greatly reducing these stagnation
zones and suppressing the {}``tail'' of the residence time distribution.
The residence time distribution of the polygonal chamber without posts
was narrower than the residence time distribution of the round chamber
without baffles. Without the constraint imposed by the circular tissue
culture membrane, the shape of the polygonal chamber could be optimized
for plug-like flow without additional structures. However, the addition
of posts further narrowed the residence time distribution. Figure
\ref{fig:RT 3 chambers} demonstrates that each successive chamber
further widens the residence time distribution so that a plug of fluid
becomes more disperse as it progresses through the device. 


\subsection{Dye Visualization Experiments}

The experimental flow patterns confirmed the CFD results that introducing
baffles into the alveolar chamber reduced stagnation zones to produce
a more plug-flow like flow pattern. At early time points, the baffles
prevented a stream of dyed liquid in the center from speeding toward
the outlet, as shown in time points $3.5\, s$ and $6.5\, s$. Especially
at the later time points, the baffles reduced liquid stagnation near
the outer walls of the chamber. Stagnation zones in the alveolar chamber
without baffles were clearly visible at $11.5\, s$, and stagnant
liquid remained at $16.5\, s$ immediately adjacent to the walls.
In contrast, the liquid near the walls of the alveolar chamber with
baffles was much more uniform in concentration at $11.5\, s$, and
the entire chamber had a nearly uniform concentration at $16.5\, s$.
The baffle design clearly improved the flow patterns in the round
alveolar chamber, though further improvements may be possible. The
streaks of higher and lower concentration liquid, which are prominent
at $6.5\, s$, indicated that the flow pattern on the exit side of
the baffles was not completely uniform. It should be noted that the
chambers were optimized for the transport of gases that have diffusion
coefficients an order of magnitude higher than the diffusion coefficient
of the dye used to visualize the flow patterns. Therefore, the stagnation
zones observed in the dye injection experiments are likely to be much
smaller when dissolved gas is used in place of dye.

The dye visualization of the flow patterns in the isolated conditioning
chamber (Figure \ref{fig:Conditioner CFD Dye}) confirmed that the
post layout improves the uniformity of the flow pattern in the conditioner,
like the baffles improved the flow pattern in the alveolar chamber.
Specifically, at $6.5\, s$ and $9.5\, s$ after the introduction
of dye into the inlet the condition chamber without posts exhibited
large regions of stagnant liquid near the chamber walls, represented
by lighter color. In contrast, the chamber with posts did not experience
these stagnant zones. The CFD simulation predicted that the chamber
without posts should be nearly completely filled with dyed liquid
by $6.5\, s$, while experiments showed that filling the chamber required
somewhat more time. 

In Figure \ref{fig:3 Chambers CFD Dye}, the flow pattern for the
liquid in the alveolus of the full device followed the CFD results
as well as matched the trends of the isolated alveolar chamber with
baffles. The dispersion of dye in the inlet of the alveolus due to
the previous two chambers can be observed in Figure \ref{fig:3 Chambers CFD Dye}.
Because of the non-uniform concentration of dye across the inlet,
the alveolar chamber in the full device requires more time to reach
a uniform concentration compared to the single alveolar chamber. This
trend can also be seen in the images from the CFD simulation. 

The images of the flow patterns (Figures \ref{fig:Alveolus CFD vs Dye}-\ref{fig:3 Chambers CFD Dye})
include information about the distribution of the outlet concentrations
over the width of the chambers. In the chambers without posts or baffles,
there was variation in concentration across the width of the outlet
before full saturation was reached. The liquid near the walls had
a lower concentration of dye than the liquid in the center of the
channel, due to the laminar flow within the chamber and the channels.
In both Figures \ref{fig:Alveolus CFD vs Dye} and \ref{fig:Conditioner CFD Dye},
the posts or baffles greatly reduced this variation across the channel
width, indicating more plug-like flow.

The results from the quantitative evaluation of the outlet profiles
in the isolated alveolar chamber (Figure \ref{fig:Alveolus outlet})
further confirmed the results from the CFD simulation. In both the
CFD and in the experimental visualization, the start of the rise in
concentration of the liquid at the outlet was delayed by the presence
of the baffles. Experimentally, the outlet concentrations reached
saturation at approximately the same time with and without baffles.
The delay in the onset of the increase in outlet concentration coupled
with the fact that the two outlets reach saturation at nearly the
same points indicated that the liquid in the alveolar chamber with
baffles exhibited a narrower distribution of residence times.

The concentration profiles for the conditioning chambers with and
without posts (Figure \ref{fig:Conditioner outlet}) reinforced the
conclusion that the post layout improved the flow patterns to produce
more plug-like flow. The outlet concentrations in both chambers began
to rise at approximately the same time, but the two profiles began
to deviate at approximately the half-saturation point. In this chamber,
the post layout improved the flow pattern by reducing the stagnation
zones, creating a steeper concentration profile at the outlet. Toward
the later times, the outlet concentration profile for the chamber
without posts was delayed by approximately 3 seconds relative to the
profile of the chamber with posts.

The concentration profile at the outlet of the alveolar chamber in
the full device (Figure \ref{fig:3 chambers outlet}) was actually
steeper than the CFD results, indicating a narrower residence time
distribution within the entire chip than was predicted. The inlet
concentration for the alveolus was influenced by the previous two
chambers, so the inlet concentration for the alveolar chamber was
not a step function for either the CFD or the experimental results.
The influence of all three chambers on the flow pattern was incorporated
into both the CFD and experimental results for the three-chamber design.
The fact that the experimental concentration profile at the outlet
of the three-chamber design was only slightly shifted from the isolated
alveolar chamber indicates the entire design was quite effective at
approximating plug flow. 

The full device was imaged and the concentrations at the outlet of
the alveolar chamber were analyzed using flow rates covering three
orders of magnitude to vary the ratio of convective to diffusive mass
transfer (i.e. different P�clet numbers) (Figure \ref{fig:Outlet 3 flow rates}).
As expected, faster flow rates produced less steep concentration profiles
in the normalized time scale, since the higher convective transport
rates limited the amount of diffusion that occurred during the normalized
time period. The slowest flow rate ($3.5\,\mu l\, min^{-1}$) simulates
the outlet concentration profile that would be produced using dissolved
\ce{CO2} (with approximately one order of magnitude higher diffusion
coefficient than dye) at a flow rate of $35\,\mu l\, min^{-1}$, since
the ratio of convective transfer to diffusive transfer is about the
same for both conditions. This condition (\ce{CO2} dissolved in a
liquid flowing at $35\,\mu l\, min^{-1}$) is appropriate for a lung-on-a-chip
device of similar geometry. Thus, the different flow rate outlet profiles
can be used to estimate the profiles that would be produced for other
diffusion coefficients.


\subsection{Conclusions}

When comparing the original designs to the optimized designs, the
improvements shown by the experimental measurements closely matched
the improvements predicted by the simulations. These results demonstrated
that the CFD-based design process was effective at reducing the number
of experimental iterations required. The slight deviation of the simulation
results from the experimental results was likely due to differences
in inlet concentration profiles. The inlet concentration was modeled
as a step function in the CFD simulations. In the experiments, the
concentration at the inlet of the chamber was not a perfect step function.
The dye was injected through a channel in the device housing which
was not included in the CFD model. In the functional lung device the
conditioning chamber will fix the concentration of dissolved gas in
the liquid before it enters the alveolus, so any dispersion introduced
by the channel before the conditioning chamber will not be significant. 
