%% LyX 1.6.10 created this file.  For more info, see http://www.lyx.org/.
%% Do not edit unless you really know what you are doing.
\documentclass[letterpaper,english,letterpaper, english]{UCFthesis}
\usepackage[T1]{fontenc}
\usepackage[latin9]{inputenc}
\setcounter{secnumdepth}{0}
\usepackage{array}
\usepackage{float}
\usepackage{amsmath}
\usepackage{graphicx}
\usepackage{amssymb}

\makeatletter

%%%%%%%%%%%%%%%%%%%%%%%%%%%%%% LyX specific LaTeX commands.
\pdfpageheight\paperheight
\pdfpagewidth\paperwidth

\DeclareRobustCommand{\greektext}{%
  \fontencoding{LGR}\selectfont\def\encodingdefault{LGR}}
\DeclareRobustCommand{\textgreek}[1]{\leavevmode{\greektext #1}}
\DeclareFontEncoding{LGR}{}{}
\DeclareTextSymbol{\~}{LGR}{126}
%% Because html converters don't know tabularnewline
\providecommand{\tabularnewline}{\\}
%% A simple dot to overcome graphicx limitations
\newcommand{\lyxdot}{.}


%%%%%%%%%%%%%%%%%%%%%%%%%%%%%% User specified LaTeX commands.
\usepackage{amsmath}
\usepackage{graphicx}
\usepackage{amssymb}
\usepackage{babel}
\usepackage{bpchem}
\usepackage[version=3]{mhchem}

\hypersetup{
    bookmarks=true,         % show bookmarks bar?
    unicode=false,          % non-Latin characters in Acrobat�s bookmarks
    pdftoolbar=true,        % show Acrobat's toolbar?
    pdfmenubar=true,        % show Acrobat's menu?
    pdffitwindow=true,     % window fit to page when opened
    pdfstartview={FitV},    % fits the page to the window
    pdfpagelayout={SinglePage}, % Displays a single page; advancing flips the page
    pdftitle={MODELING TRANSPORT AND ADSORPTION IN MICROFLUIDIC SYSTEMS},    % title
    pdfauthor={Craig Finch},     % author
    pdfcreator={LaTeX},   % creator of the document
    pdfkeywords={keyword1} {key2} {key3}, % list of keywords
    pdfnewwindow=true,      % links in new window
    colorlinks=false,       % false: boxed links; true: colored links
    linkcolor=red,          % color of internal links
    citecolor=black,        % color of links to bibliography
    filecolor=magenta,      % color of file links
    urlcolor=blue           % color of external links
}

\title{Modeling Transport and Protein Adsorption in Microfluidic Systems}
\author{Craig Finch}
\priordegrees{B. S. University of Illinois, 1997 \\ M. S. University of Central Florida, 2001}
\degree{Doctor of Philosophy}
\department{Modeling and Simulation}
\college{College of Sciences}
\term{Fall Term\\2011}
\advisor{James J. Hickman}

\providecommand*\standardstate{%
  {%
    \ensuremath{\protect\cst@standardstate}%
  }%
}
\newcommand*\cst@standardstate{%
  \mathpalette\cst@standardstate@aux\circ
}
\newcommand*\cst@standardstate@aux[2]{%
  \ooalign{%
    \hfil
    $#1-$%
    \hfil
    \cr
    \hfil
    $#1#2$%
    \hfil
    \cr
  }%
}

\let\oldthebibliography=\thebibliography
\let\endoldthebibliography=\endthebibliography
\renewenvironment{thebibliography}[1]{%
\begin{oldthebibliography}{#1}%
\setlength{\itemsep}{-2mm}%
}%
{%
\end{oldthebibliography}%
}

\@ifundefined{showcaptionsetup}{}{%
 \PassOptionsToPackage{caption=false}{subfig}}
\usepackage{subfig}
\makeatother

\usepackage{babel}

\begin{document}
\pagenumbering{roman}
\maketitle
\begin{UCFcopyright} 2011 Craig Finch \end{UCFcopyright} 

\begin{abstract}
Mass transport limitations and surface interactions are important phenomena in microfluidic devices. The flow of water is laminar at small scales, and the absence of turbulent mixing can lead to transport limitations. Microscale devices have a high ratio of surface area to volume, and proteins are known to adsorb preferentially at interfaces. Protein adsorption plays a significant role in biology by mediating critical processes such as the attachment of cells to surfaces, the immune response and the coagulation of blood. Simulation tools that can quantitatively predict transport and protein adsorption will enable the rational design of microfluidic devices for biomedical applications.

Two-dimensional random sequential adsorption (RSA) models are widely used to model the adsorption of proteins on surfaces. As Brownian dynamics simulations have become popular for modeling protein adsorption, the interface model has changed from two-dimensional to three-dimensional. Brownian dynamics simulations were used to model the diffusive transport of hard-sphere particles in a liquid and the adsorption of the particles onto a uniform surface. The configuration of the adsorbed particles was analyzed to quantify the chemical potential near the surface, which was used to derive a continuum model of adsorption that incorporates the results from the Brownian dynamics simulations. The equations of the continuum model were discretized and coupled to a conventional computational fluid dynamics (CFD) simulation of diffusive transport to the surface.  The kinetics of adsorption predicted by the continuum model closely matched the results from the Brownian dynamics simulation.  This new model allows the results from mesoscale simulations to be used as a boundary condition for micro- or macro-scale CFD simulations of transport and protein adsorption in practical devices.

Kinetic models were used to interpret experimental measurements of the kinetics of protein adsorption. A Whispering Gallery Mode (WGM) biosensor was constructed and used to measure the adsorption of fibronectin (FN) and glucose oxidase (GO) onto several types alkysilane self-assembled monolayers (SAMs). Computational fluid dynamics was used to model the transport of protein in the flow cell of the biosensor and the results were used to improve the accuracy of the kinetic models. Multiple kinetic models were fitted to the experimental data.  The fitted parameter values and the quality of fit of the various models were analyzed to test hypotheses about the mechanisms of adsorption. Cells were cultured on silane surfaces coated with FN to assess its biological activity, and a colorimetric assay was used to determine the enzymatic activity of the adsorbed glucose oxidase. The results of the GO activity assay were compared to the activity predicted by the models. The WGM biosensor, transport simulation and kinetic model fitting enabled new insights into the adsorption of proteins on functionalized surfaces at solution concentrations that were previously unattainable.

The process of CFD simulation and experimental validation was used to design a microfluidic bioreactor that mimics the function of an alveolus. The design goal was to move a ``plug'' of liquid from the alveolus to a sensor chamber. Microreactors experience significant deviations from plug flow due to the high ratio of surface area to volume and the no-slip boundary condition at the walls of the chamber.  Iterative CFD simulations were performed to optimize microfluidic structures to minimize the width of the residence time distributions of two types of chambers. Qualitative and quantitative visualization experiments with a dye indicator demonstrated that the CFD simulations accurately predicted the residence time distributions of the chambers.  The results show that the combination of simulations and experiments can be used to optimize the performance of microreactors for \textit{in vitro} tissue engineered systems for drug and toxin studies.
\end{abstract}

\begin{acknowledgments}
This would not have been possible without support from my advisers, Dr. James J. Hickman and Dr. Thomas Clarke, and the head of the Modeling and Simulation graduate program, Dr. Peter Kincaid. I thank Kerry Wilson, Philip Anderson, Christopher Long and Wesley Anderson for providing experimental data and co-authoring several papers. Frank Vollmer introduced our group to whispering gallery mode biosensing. Vaibhav Thakore offered many helpful suggestions during many technical discussions.  

I thank the Institute for Simulation and Training (IST), the I\textsuperscript{2} Lab, and the NanoScience Technology Center at the University of Central Florida, as well the National Institutes for Health (grant \#RO1EB005459) and the United States Army (grant \#W81XWH-10-1-0542) for financial support.  IST also donated computing resources on the Stokes cluster.
\end{acknowledgments}

\clearpage \label{toc_label} \pdfbookmark[0]{TABLE OF CONTENTS}{toc_label} \tableofcontents
\newpage \phantomsection \label{listoffig} \addcontentsline{toc}{chapter}{LIST OF FIGURES} \listoffigures
\newpage \phantomsection \label{listoftab} \addcontentsline{toc}{chapter}{LIST OF TABLES} \listoftables
\newpage \pagenumbering{arabic}
\chapter{INTRODUCTION}

Microfluidics is the science and technology of manipulating small
quantities of fluid. One prominent author considers devices with fluid
volumes less than $10^{-9}\, L$ to be microfluidic \cite{Whitesides2006},
while the MEMS Handbook defines a microfluidic channel as having characeristic
dimensions between $1\,\mu m$ and $1\, mm$~\cite{Sharp2002}. Rather
than attempting to agree upon an arbitrary upper limit for the volume
or characteristic dimension of the system, a more practical approach
is to classify a system based upon its characteristics. The defining
characteristics of microfluidic systems are the inability to induce
turbulent flow, a high ratio of surface area to volume, and high shear
rates. At sufficiently small length scales, the flow of a fluid is
completely laminar and it is virtually impossible to induce turbulence.
This means that parallel streams of fluids in a microfluidic channel
will mix only by diffusion, which is a considerably slower process
than the turbulent mixing that can be found in macroscale systems.
Therefore, the creation of mixers is one of the defining challenges
of microfluidics. Because a microfluidic device has so much surface
area relative to its volume, a large fraction of the fluid has the
potential to interact with the surface. Interfacial phenomena such
as protein adsorption are much more significant in a microfluidic
device compared to a conventional one. Since fluid that is in contact
with a surface is generally stationary and turbulence does not occur,
diffusion is the only way that solute can be transported near the
surface. This phenomenon may impose a significant challenge if the
purpose of the device is to deliver nutrients to a layer of cells
or analyte to the surface of sensor. The no-slip boundary condition
on the walls, combined with small channel dimensions, can lead to
high shear rates near the walls. The unique properties of microfluidic
devices lead to both opportunities and challenges when applying the
technology to practical problems.

Microfluidic technology is having a significant impact on the field
of biology, enabling the creation of {}``lab on a chip'' systems
(also know as micro total analysis systems, or \textgreek{m}TAS.)
One important application of microfluidic systems is to culture, manipulate
and analyze single cells or very small populations of cells~\cite{Yi2006}.
The ability to isolate and study a single cell enables systems biologists
to study cell signaling without the noise generated by a heterogeneous
population of cells~\cite{Breslauer2006}. Microfluidics also offers
the promise of developing high-throughput cell-based assays~\cite{Borland2008}.
Microfluidic chambers can also be used to create defined chemical
gradients to study chemotaxis, the movement of cells in response to
chemical cues. Flow cytometry has become a practical tool for clinical
diagnosis, and microfluidic flow cytometers enable measurements to
be performed with fewer cells~\cite{Chung2007}. This is especially
important when taking tissue samples from a fetus or infant. A key
advantage of microfluidic systems is that they can be mass-produced
using conventional microfabrication techniques, which could lead to
inexpensive, disposable analysis chips that take the place of conventional
assays. A recent special issue of \emph{Lab on a Chip} was devoted
to the application of microfluidics to point-of-care (POC) diagnostics~\cite{Sia2008}.


\section{Transport in Microfluidic Systems}

Various dimensionless numbers have been derived to help characterize
fluid systems through dimensional analysis. One of the best-known
is the Reynolds number, which describes the ratio of inertial forces
to viscous (damping) forces~\cite{fox2006}. A large Reynolds number
indicates that inertial forces dominate and the flow may be turbulent.
A small Reynolds number indicates that viscous damping is strong enough
to prevent turbulence from occuring, resulting in laminar flow. The
Reynolds number is defined as\[
Re=\frac{\bar{v}L}{\nu}\]
$\bar{v}$ is the average velocity, $L$ is the characteristic dimension
and $\nu$ is the kinematic viscosity of the system. Because both
the average velocity and dimensions of microfluidic systems tend to
be small, the Reynolds number is also small, corresponding to the
observation that flow in microfluidic systems is almost always laminar.

If a volume of fluid is sufficiently small, there may not be enough
molecules in the system to satisfy the assumptions of continuum models.
The Knudsen number is a dimensionless group that can be used to assess
the validity of the the continuum approximation. For a gas, the Knudsen
number is defined as\[
Kn=\frac{\lambda}{L}\]
$\lambda$ is the mean free path of a molecule in the gas. Molecules
in a liquid do not have a mean free path becaue their motion is highly
constrained by their neighbors, so the lattice spacing $\delta$ can
be used instead of $\lambda$ to compute the Knudsen number~\cite{Sharp2002}.
The lattice spacing for liquid water is about $0.3\, nm$. Knudsen
numbers below $10^{-3}$ indicate that the continuum approximation
is valid, while in the range $10^{-3}<Kn<10^{-1}$ continuum models
can be used with slip boundary conditions. For $Kn>10^{-1}$ entirely
different modeling procedures must be used to obtain accurate results.


\section{Protein Adsorption}

Non-specific binding (adsorption) of biomolecules, such as proteins,
at solid-liquid interfaces affects the function of materials and devices
intended for use with physiological fluids and tissues. Microfluidic
devices are especially prone to protein adsorption because of the
large amount of surface area relative to volume of the device~\cite{Walker2004}.
When designing a microfluidic system to deliver analyte to a cell
or sensor, the adsorption of analyte to the walls of tubing or channels
must be taken into account to ensure that the desired amount of analyte
actually reaches its destination. Protein adsorption is the first
step in many important biological processes, including the attachment
of cells to bioengineered surfaces, the coagulation of blood, and
the response of the immune system to an implanted device~\cite{Rabe2010}.
Adsorption of biomolecules can lead to the formation of biofilms of
bacteria~\cite{Cheng2007} or blood components~\cite{Sun2003} that
can lead to infection or clogging. Therefore, biomolecule adsorption
is a crucial factor in determining the long-term efficacy of lab-on-chip
systems, implants, and medical devices that contact blood or other
biological fluids~\cite{Roach2007,Latour2005,Ramsden2007}. For example,
advances in neurally-controlled prosthetics have been limited by the
body's inflammatory response to implanted sensors~\cite{Lai2007}. 

The inherent variability of protein sequence and structure makes the
prediction of protein adsorption from first principles an intractable
problem \cite{Wilson2004}. Thus, it has been necessary to devise
experimental solutions for making quantitative observations that can
be used to assess the biocompatibility of materials. Early research
on protein adsorption focused on measuring the surface concentration
of adsorbed protein at equilibrium. A material with high surface area
was allowed to soak in a protein solution, and the amount of adsorbed
protein was inferred from the loss of protein in solution. By measuring
the surface concentration for various solution concentrations, it
is possible to plot an {}``isotherm'' that provides some information
about the thermodynamics of adsorption. By the mid-1980s new optical
methods such as total internal reflection fluorescence (TIRF) spectroscopy~\cite{Axelrod1983}
and ellipsometry enabled researchers to measure the kinetics of adsorption~\cite{Andrade1986}.
In the 1990s optical waveguide light spectrocopy (OWLS)~\cite{Kurrat1997}
and surface plasmon resonance (SPR)~\cite{Mrksich1995} instruments
became available. Despite significant progress in our understanding
of protein adsorption, only a handful of protein/surface combinations
have been thoroughly studied and the general problem of predicting
and controlling protein adsorption remains unsolved~\cite{Rabe2010,Gray2004}.


\section{Overview}

This work describes theoretical advances in the modeling and simulation
of microfluidic systems and demonstrates the practical application
of those techniques. A new multi-scale model of the adsorption of
hard spheres was formulated to bridge the gap between simulations
of discrete particles and continuum fluid dynamics. A whispering gallery
mode (WGM) biosensor was constructed and used to measure the kinetics
of adsorption for two types of proteins on four different surfaces.
Computational fluid dynamics was used to analyze the transport of
proteins in the flow cell of the biosensor. Kinetic models of protein
adsorption that take transport limitations into account were fitted
to the experimental data and used to draw conclusions about the mechanisms
of adsorption. Transport simulations were then applied to the practical
problem of optimizing the design of a microfluidic bioreactor to enable
{}``plugs'' of fluid to flow from one chamber to the next with minimal
dispersion. Experiments were used to validate the transport simulations.
The combination of quantitative modeling and simulation with experimental
work leads to results that could not be achieved using either method
by itself.


%% LyX 1.6.10 created this file.  For more info, see http://www.lyx.org/.
%% Do not edit unless you really know what you are doing.
\documentclass[letterpaper,english,PhD]{UCFthesis}
\usepackage[T1]{fontenc}
\usepackage[latin9]{inputenc}
\setcounter{secnumdepth}{3}
\setcounter{tocdepth}{3}
\usepackage{float}
\usepackage{amsmath}
\usepackage{graphicx}
\usepackage{amssymb}

\makeatletter

%%%%%%%%%%%%%%%%%%%%%%%%%%%%%% LyX specific LaTeX commands.
\pdfpageheight\paperheight
\pdfpagewidth\paperwidth

%% A simple dot to overcome graphicx limitations
\newcommand{\lyxdot}{.}


\@ifundefined{showcaptionsetup}{}{%
 \PassOptionsToPackage{caption=false}{subfig}}
\usepackage{subfig}
\makeatother

\usepackage{babel}

\begin{document}

\chapter{A CONTINUUM MODEL OF THE ADSORPTION OF HARD SPHERES}


\section{Introduction}

Computational methods have been applied to model and predict protein
adsorption, but their success has been limited due to the complexity
of the problem. While nanoscale simulation methods like molecular
dynamics (MD) have the potential to predict protein adsorption from
first principles, the small time step required by atomistic techniques
presents a significant obstacle. Protein adsorption occurs on a time
scale of seconds to hours, while it was recently reported that an
advanced MD simulation on a specially designed supercomputer has the
ability to simulate several microseconds of protein behavior \cite{Dror2010}.
As a result, simplified models are widely used. In colloidal models,
the protein is represented by a simplified geometric shape and interactions
are modeled by DLVO forces. Recently, Brownian dynamics simulations
been used to implement colloidal-scale simulations of protein transport
and adsorption \cite{Unni2005,Magan2006,Quinn2008}. 

A fundamental limitation of nanoscale and mesoscale simulation methods
is that they are impractical for modeling transport in practical applications
such as medical implants, engineered tissue constructs, organ models,
and microfluidic devices. CFD simulations are widely used to model
transport in macroscale systems. To predict protein adsorption, simplified
continuum models of adsorption based upon chemical kinetics have been
used as adsorbing boundary conditions in CFD simulations \cite{Glaser1993}.
The Langmuir and RSA adsorption models, as well as numerous others,
have successfully been utilized to model protein adsorption \cite{Rabe2010}. 

Chemical thermodynamics has been used to create a link between continuum
models and discrete particles \cite{Adamczyk1999a}. Adsorbed particles
were assumed to create an energy barrier which becomes higher as the
number of adsorbed particles increases. This barrier incorporates
steric exclusion due to the blocking effect of hard particles and
the longer-ranged repulsive effect of electrostatic interactions.
The flux of particles to the surface is influenced by the size of
the barrier. Previous authors have used the surface boundary layer
approximation (SFBLA) to simplify the model by assuming that the interface
is one-dimensional and the flux through the boundary layer is independent
of the position above the surface. It cannot be assumed that the flux
at the surface is equal to the flux at the interface with the bulk
when modeling the adsorption of hard spheres. A particle may diffuse
into the boundary region, collide with adsorbed particles, and diffuse
back out of the boundary region at a later time. At any instant, the
flux at the surface will be less than or equal to the flux at the
bulk interface.

Brownian dynamics simulations of hard sphere adsorption were utilized
to obtain configurations of adsorbed spheres, which were then analyzed
to obtain the chemical potential near the surface. The principles
of non-equilibrium thermodynamics were used to derive a continuum
model of hard-sphere adsorption in which the flux of particles is
allowed to vary with the distance from the surface. The chemical potential
from the hard-sphere simulations was used to determine the coefficients
of the continuum model, which was solved numerically in the region
near the surface. This model was used as a boundary condition for
a conventional CFD simulation to predict coupled transport and adsorption.
The kinetics of adsorption predicted by the continuum model matched
the kinetics from the Brownian dynamics simulation. This approach
can be used to incorporate the results from a discrete particle simulation
when simulating transport and adsorption in a microscale or macroscale
system.


\section{Methods and Materials}


\subsection{Derivation of the Continuum Model }

The first steps of the derivation of the continuum model follow the
method described in \cite{Adamczyk1999a,Adamczyk2000}, starting with
the continuity equation\begin{equation}
\frac{\partial n}{\partial t}=-\nabla\cdot\mathbf{J}\label{eq:Continuity}\end{equation}
$n$ is the number density of particles in solution and $\mathbf{J}$
is the flux of particles. Adsorption is a process of equilibration
that can be described using non-equilibrium thermodynamics. It has
been postulated that the flux is proportional to the gradient of the
total potential \begin{align}
\mathbf{J} & =-\left(\mathbf{M}\cdot\nabla\mathbf{E}\right)\, n\label{eq:Flux ito Mobility}\end{align}
where$\mathbf{M}$ is the mobility tensor and $\mathbf{E}$ is the
total potential, which can be written as $\mathbf{E}=\mu+\mathbf{\Phi}$.
The chemical potential $\mu$ represents particle-particle interactions,
including interactions between particles in solution and adsorbed
particles. The external potential $\mathbf{\Phi}$ includes effects
such as an electric field due to a charged surface or a gravitational
potential. Using the relation $\mathbf{D}=kT\mathbf{M}$, the flux
can be written in terms of the diffusion tensor\begin{align}
\mathbf{J} & =-\mathbf{D}\cdot\left(\nabla\mu/kT+\nabla\mathbf{\Phi}/kT\right)\, n\label{eq:Flux ito Diffusion}\end{align}


The chemical potential can be written in terms of the activity coefficient
$\gamma$\begin{equation} \label{eq:ChemicalPotential}
\mu = \mu^\standardstate + kT \, \ln \gamma \frac{n}{n^ \standardstate}
\end{equation}\( \mu^\standardstate \) is the potential in the standard state,
which is chosen to be the potential of a particle in solution far
away from other particles. The activity coefficient is a function
of position in space, the number and configuration of particles, and
the particle-particle interaction potential. Expanding the potential
and substituting into the flux equation results in \begin{equation} \label{eq:Flux ito Potential}
\mathbf{J} = -\mathbf{D} \cdot \left[ \frac{\nabla \mu^\standardstate}{kT}
 + \nabla \ln \frac{n}{n^\standardstate} + \nabla \ln \gamma
 + \frac{\nabla \Phi}{kT}
 \right] \, n
\end{equation}Since the potential in the standard state is constant, its gradient
is zero. This expression for the flux was substituted into the continuity
equation to obtain\begin{equation} \label{eq:PDE 1}
\frac{\partial n}{ \partial t} =
 \nabla \cdot \left[ \mathbf{D} \cdot \left(
 \nabla \ln \frac{n}{n^\standardstate}
 + \nabla \ln \gamma + \frac{\nabla \Phi}{kT}
 \right) \, n \right]
\end{equation}For modeling adsorption at an interface, the general equation can
be simplified considerably by making some assumptions about the interface.
It was assumed that the interfacial layer is thin with respect to
the overall geometry. Transport parallel to the interface and convection
were neglected so the equation reduced to a one-dimensional form:\begin{equation} \label{eq:PDE 2}
\frac{\partial n}{ \partial t} =
 \frac{\partial}{ \partial h} \, \left[ D \, 
 \left(
 \frac{\partial}{ \partial h} \ln \frac{n}{n^\standardstate} 
 + \frac{\partial}{ \partial h} \ln \gamma
 + \frac{1}{kT} \frac{\partial}{ \partial h} \Phi
 \right) \, n \right]
\end{equation}$h$ is the distance between the edge of the particle and the surface,
as shown in Figure \ref{fig:Geometry}. %
\begin{figure}
\includegraphics{Figures/AdsorbingBoundary}

\caption{\label{fig:Geometry}Geometry used to define the hard-sphere adsorbing
boundary condition}


%
\end{figure}


For the case of hard spheres with no surface potential, $\Phi\equiv0$
and the diffusion coefficient near the surface was assumed to be constant.
The notation \( \frac{n}{n^\standardstate} \) will be dropped, and
it will be assumed that $n$ has been normalized. The equation can
be re-arranged to have the form of a generalized diffusion equation\begin{align}
\frac{\partial n}{\partial t} & =D\,\frac{\partial}{\partial h}\left[\left(\frac{1}{n}\frac{\partial n}{\partial h}+\frac{1}{\gamma}\frac{\partial\gamma}{\partial h}\right)\, n\right]\nonumber \\
 & =D\,\left[\frac{\partial^{2}}{\partial h^{2}}n+\frac{1}{\gamma}\frac{\partial\gamma}{\partial h}\frac{\partial n}{\partial h}+n\,\frac{\partial}{\partial h}\left(\frac{1}{\gamma}\frac{\partial\gamma}{\partial h}\right)\right]\label{eq:Generalized Diffusion Eqn}\end{align}
The result is a parabolic partial differential equation with variable
coefficients. Let\begin{align}
k_{1} & =\frac{1}{\gamma}\frac{\partial\gamma}{\partial h}\nonumber \\
k_{2} & =\frac{\partial}{\partial h}\left(\frac{1}{\gamma}\frac{\partial\gamma}{\partial h}\right)=\frac{\partial k_{1}}{\partial h}\label{eq:Variable coefficients}\end{align}
Then\begin{align}
\frac{\partial n}{\partial t} & =D\,\left[\frac{\partial^{2}}{\partial h^{2}}n+k_{1\,}\frac{\partial n}{\partial h}+k_{2}\, n\right]\label{eq:Generalized Diffusion Eqn 1D}\end{align}
 This equation predicts the evolution of number density over time
at every point in a domain for which the activity coefficient is known.
To utilize this equation to predict the surface density of adsorbed
particles over time, the boundary condition at the surface ($h=0$)
can be defined\[
\frac{d\Gamma}{dt}=-J_{s}\left(t\right)\]
$\Gamma$ is the number of adsorbed particles per unit area and $J_{s}$
is the flux at the surface. The total surface density at time $t$
is given by\[
\Gamma\left(t\right)=\int_{0}^{t}J_{s}\left(\tau\right)\, d\tau\]
The choice of boundary condition for the bulk solution depends upon
the nature of the problem to be solved. It is straightforward to couple
the generalized diffusion equation to a conventional CFD simulation
to predict transport-influenced adsorption in arbitrary geometries. 


\subsection{Computation of the Activity Coefficient}

The coefficients of the generalized diffusion equation are functions
of the activity coefficient $\gamma$, which is a function of space
and the number and configuration of adsorbed particles. Computation
of the activity coefficient is critical to obtain a useful model. 


\subsubsection{Brownian Dynamics Simulation}

A Brownian dynamics simulation of irreversible hard-sphere adsorption
was used to obtain configurations of adsorbed particles. The Langevin
position equation \cite{Elimelech1998} was used to update the position
of each particle at each time step: \[
\mathbf{r}_{i}(t+\Delta t)=\mathbf{r}_{i}(t)+\mathbf{g}_{q}\sqrt{2\, D\,\Delta t}\]
$\mathbf{r}_{i}$ is the position of particle $i$, $\Delta t$ is
the simulation time step, $D$ is the diffusion coefficient, and $g_{q}\in\mathbb{R}^{3}$
is a vector of random numbers drawn from a Gaussian distribution with
a mean of zero and a variance of one. At each time step, all particles
in the domain were moved simultaneously, and overlaps were detected.
Any particle which overlapped another was reset to its original position
and moved again using a different random vector, until each particle
found a valid position. The simulation domain was a rectangular box
with height $L$ and width and length $S$. Periodic boundary conditions
were applied on the four sides of the simulation domain so that a
particle that exited one side of the box re-entered on the opposite
side. An adsorbing boundary condition was used for the bottom of the
box. A particle adsorbed when it reached the adsorbing surface without
overlapping any previously adsorbed particles. The configuration of
adsorbed particles was recorded every time a new particle adsorbed.
A perfect adsorbing boundary could also be used to simulate diffusion-limited
adsorption. Adsorbed particles were moved out of the simulation domain
so that blocking did not interfere with the adsorption of additional
particles. 

Two different boundary conditions were used for the top of the box.
To simulate constant near-surface concentration a reflecting boundary
was used for the top of the box. At each time step, the particles
in the box were counted. If there were too few particles in the box,
particles were added at positions drawn from a uniform random distribution,
ensuring that the newly added particles did not overlap with existing
particles. If there were too many particles in the domain, a particle
was chosen at random for deletion. To simulate diffusion on a semi-infinite
domain, an open box top was used to allow the Brownian dynamics simulation
to exchange particles with an infinite bulk solution. This boundary
condition was implemented according to the multi-scale linking algorithm
described in \cite{Magan2004}. 


\subsubsection{Implementation of Brownian Dynamics Simulation}

The simulation was implemented using the Python programming language.
Numerical data, such as the coordinates of the particles, were stored
in NumPy arrays \cite{Oliphant2006}. Routines from the SciPy library
were used for standard operations like interpolation and numerical
integration \cite{Oliphant2007}. Collision detection was implemented
in C for speed, using the weave function from SciPy. Each time a particle
adsorbed on the surface the configuration of adsorbed particles was
recorded, along with the profile of concentration vs. distance from
the adsorbing surface and the fraction of the surface covered by particles
($\theta$). The PyTables package was used to save the simulation
results to binary files in HDF5 format \cite{Alted2002-,HDFGroup2000-}. 


\subsubsection{Controls and Validation for the Brownian dynamics simulation}

The Brownian dynamics simulation was validated by simulating diffusion-limited
adsorption and comparing the results to the analytical solution of
a well-known boundary value problem. A perfect adsorbing boundary
condition (perfect sink) was used so that particles that adsorbed
to the surface did not block the adsorption of additional particles.
The classical diffusion equation in one dimension can be solve analytically
with the boundary conditions $n(h=0,t)=0$ and $n(h\rightarrow\infty,t)=n_{b}$.
Control simulations were performed with three different box widths
(75, 100, 150) to ensure that edge effects were not distorting the
pattern of adsorbed particles. The simulation was also tested with
three time steps ($10^{-5}$, $10^{-6}$ and $10^{-7}sec$) to determine
the largest time step that would produce accurate results.


\subsubsection{Calculation of the Activity Coefficient}

The activity coefficient was determined empirically from the results
of the Brownian dynamics simulations using the Widom particle insertion
method \cite{Widom1963}. In this method, a {}``test'' particle
is introduced into a fixed configuration of particles and the energy
of interaction $\psi$ between the test particle and the surrounding
particles is calculated. The activity $a$ can be computed by taking
the canonical average of many such insertions, using the formula\[
\frac{n}{a}=\left\langle \exp\left(\frac{-\psi}{k_{B}T}\right)\right\rangle \]
For hard spheres, the energy of interaction is either infinite if
the test particle overlaps an existing particle or zero if it does
not. Therefore the activity coefficient $\gamma=a/n$ is also infinite
if the test particle overlaps another, and zero if it does not. To
avoid dealing with infinite quantities, the available volume function
(AVF) was defined as \[
AVF\left(h,\theta\right)=\gamma^{-1}\left(h,\theta\right)\]
The value of the AVF is one if the test particle does not overlap
with a simulation particle and zero if it does overlap. After the
completion of an ensemble of Brownian dynamics simulations, the AVF
was calculated for each run at multiple values of $h$ and $\theta$.
For each $\theta_{i}$ a planar grid of non-overlapping test particles
was constructed in the simulation domain at a given height above the
adsorbing surface. Particles in solution with cannot interact with
adsorbed particles, as shown in Figure \ref{fig:Geometry}. The position
of each test particle was offset by a small random vector in $x-y$
plane and each test particle was checked for overlaps with every simulation
particle. The fraction of test particles without overlaps was recorded
as the value of $AVF\left(h,\theta\right)$. Multiple replicates with
different random displacements from the grid were performed for each
and . The analysis was performed with 50 and 500 replicates, and 50
replicates were found to be sufficient to determine the AVF. The results
from multiple runs of the Brownian dynamics simulation were averaged
to obtain an estimate of the available volume function. The coefficients
of the generalized diffusion equation were computed directly from
the available volume function:\[
k_{1}=\frac{\partial}{\partial h}\log\gamma=\frac{1}{AVF}\,\frac{\partial}{\partial h}AVF\]
\[
k_{2}=\frac{\partial}{\partial h}\left(\frac{1}{\gamma}\frac{\partial\gamma}{\partial h}\right)=\frac{1}{AVF^{2}}\left(\frac{\partial AVF}{\partial h}\right)^{2}-\frac{1}{AVF}\left(\frac{\partial^{2}AVF}{\partial h^{2}}\right)\]



\subsection{Implementation of the Continuum Model}

The control volume formulation \cite{Patankar1980} was used to obtain
a finite difference form of Equation 15 in the region $0\leq h<2$.
Central differencing was used to approximate first derivatives in
space, and a fully implicit scheme was used to approximate time derivatives.
To ensure stability, the source term was linearized so that it was
independent of the value of n. Any particle that touches the surfaces
adsorbs, so the number density of particles at the surface is zero.
The Dirichlet (type 1) boundary condition $n=0$ was used at $h=0$.
For simulations with constant near-surface concentration the type
1 boundary condition $n=n_{b}$ was applied at $h=2$. 


\subsubsection{Coupling the Continuum Model of Hard-Sphere Adsorption to a Conventional
CFD Transport Simulation}

For simulations in which the concentration at $h=2$ was influenced
by diffusion, a second simulation domain was created to model diffusion
in the bulk for $h\geq2$. The classical diffusion equation was discretized
and solved in the bulk domain in the same manner as the generalized
diffusion equation. The generalized diffusion equation was solved
in the interaction region to obtain the net flux, using the value
of number density at $h=2$ from the previous time step. For small
values of $t$ the surface is mostly available for adsorption, so
the net flux is limited by diffusive transport. The net flux is determined
by $J_{c}$ rather than $J_{s}$. Once the surface is mostly blocked,
the net flux is determined by the rate at which particles can find
available space on the surface, so the value of $J_{s}$ should be
used for the net flux. The correct value for the net flux can be computed
by\[
J\left(h=2,t\right)=-min\left(\left|J_{s}\right|,\left|J_{c}\right|\right)\]
 The net flux from the generalized diffusion equation was used as
the left-hand boundary condition to solve the classical diffusion
equation in the bulk, which resulted in a new value for the number
density at the interface. This number density was used to solve the
generalized diffusion equation in the near-surface domain again, and
the iterative process was repeated until the number density at the
interface computed in each domain converged:$\left|n\left(h=2^{-}\right)-n\left(h=2^{+}\right)\right|<\delta$.


\subsubsection{Validation of the continuum model}

Equation \ref{eq:Generalized Diffusion Eqn 1D} has the form of a
diffusion equation. In the case that the activity coefficient is constant,
this equation reduces to the classical one-dimensional diffusion equation.
It was verified that the adsorption kinetics predicted by the continuum
model matched the kinetics predicted by the classical diffusion equation
for a perfect adsorbing boundary when the activity coefficient was
held constant ($AVF\left(h,\theta\right)\equiv1$). 


\section*{Results}


\subsection{Brownian dynamics simulation results}

The Brownian dynamics simulation was run with three different number
densities of particles in solution. The number density had a significant
impact on the kinetics of adsorption, as shown in Figure \ref{fig:BrD Adsorption Kinetics}
.%
\begin{figure}
\includegraphics{Plots/BrD_kinetics}\caption{\label{fig:BrD Adsorption Kinetics}Kinetics of adsorption predicted
by the Brownian dynamics simulation for three different volume fractions.}


%
\end{figure}
 The configurations of adsorbed particles predicted by the Brownian
dynamics simulations were characterized using the pair correlation
function $g\left(r\right)$, which is also known as the radial distribution
function (RDF). The results are shown in Figure \ref{fig:Brownian Dynamics RDF}.
%
\begin{figure}
\includegraphics{Plots/RDF_BrD_volume_fraction}\caption{\label{fig:Brownian Dynamics RDF}Radial distribution function predicted
by Brownian dynamics simulation for three volume fractions, and the
RDF predicted by the RSA model.}
%
\end{figure}
 Figure \ref{fig:Brownian Dynamics RDF} also shows the pair correlation
function for configurations generated by RSA simulations, which are
essentially identical to those generated by the Brownian dynamics
simulations. A representative plot of the AVF for a volume fraction
of 0.01 is shown in Figure \ref{fig:Available Volume Function}.%
\begin{figure}[H]
\includegraphics{Plots/AVF_publication_plot}

\caption{\label{fig:Available Volume Function}Plot of the available volume
function. Available surface function for RSA shown as red dashed line.}
%
\end{figure}
 The available surface function (ASF) for the RSA model is also shown
in Figure \ref{fig:Available Volume Function} \cite{Schaaf1989}.
Since Brownian dynamics and RSA simulations produce identical configurations
of adsorbed particles, it is not surprising that the AVF for Brownian
dynamics at $h=0$ is identical to the ASF for RSA. 


\subsection{Continuum model results}

The coefficients of the generalized diffusion equation were computed
from the AVF from Brownian dynamics simulations. Representative plots
of the coefficient values are shown in Figure \ref{fig:Coefficient Plots}.%
\begin{figure}[H]
\subfloat[\label{fig:Coefficient a}Coefficient $k_{1}$]{\includegraphics{/home/cfinch/Brownian_Dynamics/Documentation/Hard_Sphere_Paper/Plots/coeff_a_publication_plot}

}\subfloat[\label{fig:Coefficient b}Coefficient $k_{2}$]{\includegraphics{/home/cfinch/Brownian_Dynamics/Documentation/Hard_Sphere_Paper/Plots/coeff_b_publication_plot}

}

\caption{\label{fig:Coefficient Plots}Coefficients of the generalized diffusion
equation, derived from Brownian dynamics results}
%
\end{figure}
 The calculation of coefficient was challenging due to the presence
of $AVF^{-1}$ and $AVF^{-2}$ in Equation \ref{eq:Variable coefficients},
which result in large numbers when the value of the AVF approaches
zero. The continuum model accurately reproduced the kinetics predicted
by the Brownian dynamics simulations, as shown in Figure \ref{fig:Kinetics CFD BrD}.%
\begin{figure}
\includegraphics{Plots/BrD_Continuum_Kinetics_dt}

\caption{\label{fig:Kinetics CFD BrD}Kinetics of adsorption predicted by Brownian
dynamics simulations and the continuum model for $\phi=0.01$.}
%
\end{figure}



\section{Discussion}

The RDFs for particle configurations generated by Brownian dynamics
simulations are almost identical to the RDF for particles generated
by RSA. This agreement indicates that the adsorption of hard spheres
is a random sequential process that is essentially independent of
transport to the surface. If kinetic predictions are not required,
an RSA simulation can be used to generate surface configurations that
are equivalent to results from hard-sphere Brownian dynamics simulations,
with much less computational effort.

The AVF shown in Figure \ref{fig:Available Volume Function} differs
significantly from the blocking function reported in \cite{Magan2004},
which was estimated by taking the ratio of flux at the surface to
the flux expected for a perfect adsorbing boundary. In this work the
AVF was computed directly by attempting to adsorb an additional particle
onto a surface with adsorbed particles. The method used here is more
likely to obtain an accurate result.
\end{document}



\chapter{\texorpdfstring{MODELING THE KINETICS OF PROTEIN ADSORPTION}{CHAPTER \arabic{chapter}. MODELING THE KINETICS OF PROTEIN ADSORPTION}}


\section{Introduction}

The previous chapter advanced the theory of protein adsorption by
developing a continuum model that describes the adsorption of three-dimensional
particles at an interface. In this chapter, existing continuum models
of protein adsorption will be used to interpret experimental measurements
of the kinetics of protein adsorption on surfaces with varying properties.
Kinetic adsorption models from multiple sources will be formulated
in a common mathematical framework and quantitatively compared. Computational
fluid dynamics simulations will be used to characterize transport
limitations in the flow cell of a whispering gallery mode (WGM) biosensor.
The kinetic models will be fitted to experimental data from the WGM
biosensor, taking transport limitations into account, and used to
test hypotheses about the experimental results.


\subsection{Alkylsilane Surface Modification}

Many studies to date have focused on the adsorption of proteins to
alkanethiol self-assembled monolayers (SAMs), which are used to functionalize
noble metal surfaces (typically gold). These SAMs are convenient in
that they are relatively easy to prepare, present highly ordered monolayers
with well-defined composition, and are compatible with integrated
electrodes and other sensor systems utilizing metal-coated surfaces,
such as surface plasmon resonance (SPR) sensors. In-depth discussions
of alkanethiol SAMS are easily found in the literature \cite{Love2005}.
Less attention, however, has been given to alkylsilane monolayers,
which are used to functionalize glass or silica surfaces. This may
be because they lack the highly ordered packing formed by alkanethiol
SAMs (resulting in less well defined surfaces), or a perceived difficulty
in the preparation of well-characterized alkylsilane surfaces. This
is somewhat unfortunate, as alkylsilanes represent a broad and useful
class of compounds that are used in an increasing variety of biomedical
and biotechnological applications. 

Alkylsilane monolayers are used to modify the surface chemistry of
glass and silica surfaces to control the adhesion of proteins and
cells. Laser ablation has been used to pattern alkylsilane surfaces
to create cytophobic and cytophilic regions that direct the attachment
and growth of cells~\cite{Stenger1992}. \IUPAC{(3-trimethoxy\|silyl\|propyl)diethyl\|triamine}
(DETA) is used as a cytophilic cell culture substrate. \IUPAC{1,1,2,2-perfluoro\|octyl\|trichloro\|silane}
(13F) is a hydrophobic perfluorinated SAM that has been used to define
cytophobic regions. More recently, tethered chains of polyethylene
glycol (SiPEG) have been used as cytophobic SAMs in place of 13F~\cite{Wilson2011a}.
Surfaces modified with PEG, which is also known as \IUPAC{oligo(ethylene glycol)}
(OEG) or \IUPAC{polyethylene oxide} (PEO), have been extensively
studied because of their resistance to protein adsorption \cite{Gombotz1991}.

A biosensor system utilizing whispering gallery mode technology, where
the active sensor is typically a silica disk, ring, toroid, or sphere/spheroid,
can provide new insights into the adsorption behavior of biomolecules
onto alkylsilane-modified surfaces. Glass and silicon oxide surfaces
are much more common than metal surfaces in cell culture applications,
so silane surface modification is of greater practical importance
than thiol surface modification for tissue engineering. To fully understand
the behaviour of cells and tissue constructs, it is critical to understand
the underlying processes that govern the adsorption of biomolecules
to silica substrates and the alkylsilane coatings used to functionalize
them.


\subsection{Whispering Gallery Mode Biosensing}

WGM biosensing is based on monitoring the frequency shift of an optical
resonance excited inside a dielectric resonator \cite{Vollmer2005,Vollmer2002}.
In our implementation, near-infrared light is evanescently coupled
to a glass microsphere with a radius of 50-200 \textgreek{m}m from
a tapered optical fiber, which is connected to a tunable distributed
feedback (DFB) laser at one end and a photodetector at the other.
The laser and detector are used to precisely monitor changes in the
resonant wavelength of the microsphere. As proteins or other material
accrete at the surface of the microsphere, the effective radius of
the sphere increases, resulting in a red shift of the resonant wavelength
that can be quantified and used to calculate the average surface density
of adsorbed material. Even with a simple experimental configuration
\cite{Vollmer2002} it has been shown that a detection limit of $\sim1\, pg/mm^{2}$
can be readily achieved. This is ten times more sensitive than an
SPR biosensor and theoretical calculations predict the ultimate detection
limit of the method to be close to the single molecule level \cite{Armani2003,Vollmer2008,Vollmer2008a,Arnold2003}.
These qualities make WGM biosensing an ideal method for studying protein
adsorption, as the dynamic range of the method allows measurements
to be performed in concentration regimes that previously were unattainable.
To date, optical resonators of this kind have been applied to a variety
of biosensing applications with great effect. In addition to the inherent
sensitivity of the method, standard CMOS technology can be applied
to fabricate arrays of resonators on silicon wafers that provide a
scalable multiplex sensing capability for detecting multiple biological
or chemical markers from a single sample in parallel \cite{Luchansky2010,Washburn2009}.
Furthermore it has been shown that these measurements can be performed
in complex samples such a blood plasma and serum \cite{Luchansky2011}.
This provides an added level of complexity as {}``real-world'' samples
such a plasma and serum contain a mixture of hundreds of proteins,
which may non-specifically bind to a sensor giving inaccurate readings
or false positive measurements. 


\subsection{Fibronectin}

The unique capabilities of the WGM sensor were utilized to quantify
the kinetics of the adsorption of fibronectin (FN) onto alkylsilane
surfaces. Cell studies were then done to evaluate the biological activity
of the adsorbed FN \cite{Keselowsky2004}. Fibronectin is an important
protein in the extracellular matrix (ECM) that mediates the interaction
of cells with surfaces, but its activity has been shown to be influenced
by its surface structure \cite{Lan2005,Michael2003}. Since the adsorption
of FN has been widely studied, the results from the WGM instrument
could be compared to an extensive amount of data. Fibronectin (FN)
is a physiologically important protein in vertebrates. It is abundant
in plasma and other bodily fluids, and plays an important role in
the extracellular matrix. The structure of fibronectin is complex.
The primary structure of FN is a chain composed of three types of
repeated modules. Although only one gene codes for FN, alternative
splicing of the pre-mRNA results in numerous variants in which modules
are added or deleted \cite{Pankov2002}. X-ray crystallography has
been used to determine the secondary and tertiary structure of the
three types of modules. The complete FN molecule has not been crystallized,
so its secondary and tertiary structure are unknown. It is likely
that the secondary and tertiary structure are highly dependent on
the local environment. In solution, fibronectin exists as a dimer
with two identical subunits linked by disulphide bonds. In the extracellular
matrix, fibronectin is assembled into a fibrillar network \cite{Mao2005}.


\subsection{Glucose Oxidase}

The WGM biosensor was also used to quantify the kinetics of adsorption
of glucose oxidase (GO) onto alkylsilane surfaces. The activity of
the adsorbed enzyme was measured to provide more information about
its conformation on the surface. Glucose oxidase is an important and
useful enzyme from a technological standpoint. GO which has been adsorbed
or covalently attached to electrodes forms the basis for amperometric
glucose sensors \cite{Wang2007}. Advances in portable blood glucose
sensors have enabled diabetics to monitor and control their blood
glucose levels, minimizing the risk of complications from the disease
\cite{Oliver2009}. Perhaps because of its important role in biosensors,
glucose oxidase has been thoroughly characterized. This makes it an
ideal candidate for testing and validating the accuracy of the WGM
biosensor system. 

GO is a dimeric glycoprotein that is composed of two identical subunits~\cite{Wohlfahrt1999}.
The crystal structure of GO from \emph{Aspergillus niger} is available
at the Protein Data Bank (1CF3), and its bounding box is $6.0\, nm$
x $6.2\, nm$ x $7.7\, nm$. Reported values for the molecular mass
of GO range from $152\, kDa$~\cite{Keilin1948} to $186\, kDa$~\cite{Swoboda1965},
depending upon the method of purification that was used. The isoelectric
point of GO is 4.2~\cite{Pazur1964}. Many previous studies have
focused on the effect of different surface chemistries on the structure
and activity of adsorbed glucose oxidase~\cite{Fears2009,Dong1997}.
Atomic force microscopy (AFM) studies have been performed to determine
the size and shape of GO adsorbed on various surfaces~\cite{Muguruma2006,Otsuka2004}.
Although the immobilization of glucose oxidase has been widely studied,
only qualitative results have recently been reported for the kinetics
of adsorption~\cite{Muguruma2006}.


\subsection{Fitting Kinetic Models to Protein Adsorption Data}

As mentioned in the previous chapter, kinetic models have been formulated
as boundary conditions for use in computational fluid dynamics (CFD)
simulations of the transport and adsorption of proteins~\cite{Glaser1993,Edwards1999}.
Kinetic models have also been fitted to experimental measurements
of adsorption kinetics. An RSA-type model with a simple approximation
of transport limitations was fitted to kinetic data from an OWLS adsorption
sensor~\cite{Kurrat1994}. More recently, a model of adsorption with
a post-adsorption transition \cite{Lundstroem1984} was fitted to
a comprehensive set of kinetic data from a surface plasmon resonance
sensor~\cite{Michael2003}. The concentration near the surface was
assumed to be constant. These two approaches were combined in a study
in which a Langmuir-type kinetic model coupled to a CFD simulation
was fitted to experimental measurements of adsorption kinetics in
microcapillaries~\cite{Jenkins2004}. This study was unique in that
the entire CFD model, including transport and adsorption, was included
in the fitting procedure.


\subsection{Overview}

A novel WGM biosensor was constructed and used to quantitatively study
the kinetics of adsorption of GO and FN at varying concentrations
onto alkylsilane monolayers presenting well-defined surface chemistries:
DETA, 13F, and SiPEG. To determine the biological activity of the
adsorbed FN, neuronal and skeletal muscle cells were cultured on the
modified surfaces in a serum-free culture system \cite{Brewer1995,Das2007}.
The kinetics of adsorption of glucose oxidase were also measured on
the silane monolayers and bare glass with the WGM biosensor, and its
enzymatic activity on each surface was determined with a standard
assay kit. The WGM biosensor incorporated a flow cell which minimized
the effect of transport limitations on protein adsorption. This, along
with the inherent sensitivity of the method, allowed the kinetics
of adsorption of FN to be measured at concentrations lower than those
that have previously been reported \cite{Michael2003}. Multiple kinetic
models of protein adsorption were fitted to the measured kinetic curves,
and the resulting parameters were used to draw conclusions about the
mechanisms of adsorption. To maximize the accuracy of the fitted kinetic
parameters, computational fluid dynamics simulations were used to
quantify the limitations in the transport of protein to the sensor
surface. The results demonstrate that the combination of WGM biosensing,
CFD, and kinetic models of adsorption provides a unique capability
to quantify protein adsorption on silane-modified surfaces for the
purpose of understanding the interactions between tissues and tailored
interfaces.


\section{Methods and Materials}


\subsection{Experimental Methods}

A whispering gallery mode sensor system was constructed as described
in \cite{Wilson2011,Wilson2009}. A schematic overview of the system
is shown in Figure \ref{fig:WGM System Diagram}.%
\begin{figure}
\includegraphics{Protein_Adsorption/Figures/WGM_System_Overview}

\caption{\label{fig:WGM System Diagram}Schematic diagram of the whispering
gallery mode biosensor}


%
\end{figure}
 After the assembly of the flow cell, PBS solution was flowed through
the tubing and flow cell until the system reached thermal equilibrium.
Protein solution at the appropriate concentration was flowed through
the system with a peristaltic pump at a volumetric flow rate of $150\,\mu l/hr$.
LabVIEW (National Instruments, Austin, TX) was used to control the
sweep of the laser wavelength and acquire data. The data acquisition
software tracked the location of each resonant valley in the acquired
spectrum using a peak fitting algorithm. All valleys with a FWHM (full
width at half maximum) value below a certain threshold were tracked,
and the position of each valley minimum was determined by fitting
a Bessel function. The position of each resonance over time was saved
to a binary file to be analyzed later.

Data analysis software was written using the Python programming language.
The binary file created by LabVIEW for each experiment was loaded
into the software and the spectral location ($nm$) of each resonance
was reconstructed over time from the raw data. One resonance, with
a continuous trace and the lowest FWHM value, was chosen for further
analysis. A linear baseline subtraction was applied to correct for
baseline drift. The refractive index of adsorbed protein (about 1.45)
is similar to that of glass~\cite{Akimoto1999}, so adsorbing proteins
effectively increase the diameter of the spheroidal resonator. A method
based on first order perturbation theory \cite{Vollmer2002,Arnold2003}
was used to calculate the surface concentration of the adsorbed species
$\Gamma\,\left(molecules\, cm^{-2}\right)$ based on the measured
change in resonant wavelength $\Delta\lambda$:\begin{equation}
\frac{\Delta\lambda}{\lambda}=\frac{\alpha_{ex}\Gamma}{\epsilon_{0}\left(n_{s}^{2}-n_{m}^{2}\right)R}\label{eq:WGM data analysis}\end{equation}
$\lambda$ is the nominal wavelength of the resonance ($1310\, nm$),
$\Delta\lambda$ is the wavelength shift of the resonance, $n_{s}$
is the refractive index of the spheroid (1.46)~\cite{SMF28e}, $n_{m}$
is the refractive index of the medium surrounding the sphere (1.3357)~\cite{Akimoto1999},
$\alpha_{ex}$ is the excess polarizability of the protein molecule,
$\epsilon_{0}$ is the permittivity of free space and $R$ is the
radius of the spheroid. The excess polarizability of the protein can
be calculated from the refractive index increment $dn/dc$:\begin{equation}
\alpha_{ex}=4\pi\epsilon_{0}\frac{n_{s}}{2\pi}\frac{dn}{dc}m\label{eq:Excess polarizability}\end{equation}
where $m=N_{A}/M$ is the mass (in grams) of a single protein molecule
and $M$ is the molar mass. Equation \ref{eq:Excess polarizability}
can be substituted into equation \ref{eq:WGM data analysis} and solved
for surface concentration to obtain\begin{equation}
\Gamma=\frac{\Delta\lambda}{\lambda}\frac{\left(n_{s}^{2}-n_{b}^{2}\right)R}{2\, n_{b}\, dn/dc}\frac{N_{A}}{M}\label{eq:WGM data analysis 2}\end{equation}
The molar mass was not needed to calculate surface density $\rho\,\left(g\, cm^{-2}\right)$:
\begin{equation}
\rho=\Gamma\frac{N_{A}}{M}=\frac{\Delta\lambda}{\lambda}\frac{\left(n_{s}^{2}-n_{b}^{2}\right)R}{2\, n_{b}\, dn/dc}\label{eq:WGM data analysis 3}\end{equation}
The generally accepted value for the refractive index increment of
a dilute protein solution ($0.184\, cm^{3}g^{-1}$) was used to compute
the excess polarizability~\cite{Vollmer2002,Voros2004}. Spheroid
radii were measured from images taken using brightfield microscopy.
Two runs for each concentration were averaged for the DETA and 13F
surfaces, and a single run was used for each concentration on the
SiPEG surface.


\subsubsection{Surface Preparation}

A single mode optical fiber with an acrylate polymer coating, $9\,\mu m$
core and $125\,\mu m$ cladding (SMF-28e+, Corning Inc., Corning,
NY) was used to fabricate the resonators \cite{Vollmer2002}. The
acrylate coating was first removed using a fiber optic stripper and
the stripped region was cleaned with isopropyl alcohol (iPA) to remove
any residual acrylate. The end of the stripped fiber was then placed
in the flame of a nitrous-butane Microflame torch (Azuremoon trading
company, Cordova, TN). A nitrous-butane flame was used due to the
very high temperatures needed to melt the glass and form the resonator
($>700^{\circ}C$). The tip of the fiber was placed in the flame until
the glass glowed bright white and began to melt. The surface tension
of the molten glass caused it to form into a spheroidal droplet. As
the tip melted, the fiber was rotated to ensure that the resonator
remained centered on the stalk of the fiber. Resonator radii used
for these studies ranged from $125\,\mu m$ to $175\,\mu m$.

Glass resonators and glass cover slips (for control measurements)
were modified with silane surface chemistry to achieve the desired
surface properties. The glass was first treated with an oxygen plasma
for 20 minutes. All silane solutions were prepared at a concentration
of 0.1\% (vol:vol) in distilled toluene in an MBraun glove box (Stratham,
NH) under anhydrous, low oxygen conditions. Storing and preparing
solutions in this way prevented solution phase polymerization of the
silane as nascent water vapor and atmospheric oxygen can react with
the monomer. Preparation of 13F surfaces was performed in the glove
box due to the extreme reactivity of the monomer. Microspheres were
immersed in the 0.1\% \IUPAC{1,1,2,2-perfluoro\|octyl\|trichloro\|silane}
solution for 30 minutes. 5 minutes prior to completion of the reaction,
the beaker was removed from the glove box and placed in a chemical
fume hood. Both the SiPEG and DETA surface modifications were performed
in a chemical fume hood. The SiPEG coating was achieved by starting
with a 0.1\% solution (vol:vol) of \IUPAC{2-[Methoxy\|poly(ethyleneoxy)propyl]trimethoxy\|silane}
(Gelest, Tullytown, PA) and adding concentrated HCl to create a 0.08\%
HCl solution (vol:vol). The resonators and cover slips were immersed
in the resultant solution for 1 hour. DETA coated substrates were
immersed in a 0.1\% solution (vol:vol) of \IUPAC{(3-trimethoxy\|silyl\|propyl)diethyl\|triamine}
(Gelest) in toluene, gently heated to $65^{\circ}C$ over 30 minutes
and then allowed to cool for 15-20 minutes. Substrates were rinsed
three times with fresh toluene and heated again to $65^{\circ}C$
in fresh toluene for 30 minutes. Upon completion of the reactions,
the microspheres and cover slips were washed three times in dry toluene
and stored in a desiccator until needed. Control cover slips were
analyzed by XPS and contact angle goniometry, and the results were
consistent with previously published results \cite{Wilson2011a}.


\subsubsection{Fibronectin}

The kinetics of FN adsorption were measured with the WGM biosensor
as described in \cite{Wilson2011}. The experiments will be briefly
summarized here. Bovine plasma fibronectin in solution (F1141, Sigma-Aldrich,
St. Louis, MO) was diluted with 50mM phosphate buffered saline (PBS)
at pH 7.4. Concentrations of $10\,\mu g/ml$, $5\,\mu g/ml$, $1\,\mu g/ml$,
$0.5\,\mu g/ml$ and $0.25\,\mu g/ml$ were used for the experiments.
To determine the biological activity of the protein adsorbed on the
various silanes, cell culture experiments were performed on silane-coated
cover slips that had been treated with $1\,\mu g/ml$ of FN in PBS
(pH 7.4). Embryonic hippocampal neurons and skeletal myoblasts were
cultured on the SiPEG, DETA, and 13F surfaces in a serum-free culture
system \cite{Brewer1995,Das2007}. After plating, cultures were maintained
in a water-jacketed incubator at $37^{\circ}C$ (85\% relative humidity)
and 5\% \BPChem{CO\_2} for seven days. Phase-contrast microscopy
images were taken during the course of the culture to document the
morphology of the cells and, in the case of skeletal myoblasts, the
differentiation of the cells into functional myotubes. A live-dead
assay (Invitrogen, Carlsbad, CA) was performed at day 7 to determine
the amount of living versus dead cells on the cover slips.

Skeletal muscle was dissected from the hind limb thighs of a rat fetus
at embryonic day 18 (Charles River Laboratories, Wilmington, MA) as
previously described \cite{Wilson2011}. Purified myocytes were plated
at a density of 500-800 cells per square millimeter onto FN-coated
cover slips. Myocytes were allowed to attach for 1 hour after which
time 3 ml of culture medium (neurobasal media containing B-27 (Invitrogen,
Carlsbad, CA{]}), Glutamax (Invitrogen, Carlsbad, CA), and Penicillin/Streptavidin)
was added. Culture medium was exchanged every 4 days. 

Embryonic hippocampal neurons were prepared according to a previously
published protocol \cite{Wilson2011}. Rat pups at embryonic day 18
were dissected from timed pregnant rats and the hippocampi were isolated
from the embryonic brains. Isolated hippocampal neurons were resuspended
in culture medium (Neurobasal / B27 / Glutamax\textbackslash{}u2122
/ Antibiotic-antimycotic) and plated at a density of 75 cells per
square millimeter. After plating, 3 ml of culture media was added.
Half of the culture medium was changed every 3-4 days. 


\subsubsection{Glucose Oxidase}

Glucose oxidase from \emph{aspergillis niger} was obtained from Sigma-Aldrich
(G7141, St. Louis, MO) and diluted to $100\,\mu g/ml$ and $10\,\mu g/ml$
with PBS at pH 7.4. The kinetics of GO adsorption were measured with
the WGM biosensor using the same protocol as the FN experiments. For
the activity assay, glass resonators were placed in a 96-well plate
and incubated for two hours in the glucose oxidase solution. The resonators
were then rinsed three times with PBS, placed in a new 96-well plate
and allowed to soak for two hours in $100\,\mu l$ of PBS to remove
any reversibly-bound enzyme. The Amplex Red glucose oxidase activity
assay (Invitrogen, Carlsbad, CA) was used to test the activity of
the enzyme adsorbed to the resonator. $50\,\mu l$ of the PBS was
removed from each well and $50\,\mu l$ of Amplex Red solution was
added. A Synergy HT plate reader (Bio-Tek, Winooski, VT) was used
read the absorbance of each well at a wavelength of $530\, nm$. A
standard curve, created using known concentrations of GO, was used
to translate the absorbance measurements into activity units.


\subsection{Analysis of Transport in the WGM Biosensor}

The rate of adsorption on a surface may be limited by the rate at
which the chemistry of adsorption takes place, the rate of transport
to the surface, or both. Computational fluid dynamics was used to
model the transport of protein in the flow cell of the WGM biosensor.
Before beginning, the Reynolds and Knudsen numbers were calculated
confirm that flow in the device would be entirely laminar and that
conventional CFD simulation methods were appropriate. CFD simulations
were run with a CFD-ACE+ multiphysics solver (ESI Software, Huntsville,
AL) to determine the influence of transport on the adsorption of protein
on the resonator surface. For all CFD simulations, the size of the
mesh and the time step were refined until the simulation results did
not change significantly. Upwind differencing was used to approximate
velocity derivatives, a 2nd order limiter was used to approximate
concentration derivatives, and the fully implicit Euler method was
used for time stepping in transient simulations. A constant-velocity
boundary condition was used at the inlet, a fixed-pressure boundary
was used at the outlet, and no-slip boundary conditions were applied
on all the surfaces.

A multi-scale approach was used to obtain high-resolution results
while keeping the simulation run time reasonable. A model of the entire
flow cell, including the $40\, cm$ of tubing between the three-way
stopcock and the flow cell, was created using CFD-GEOM and discretized
using a structured mesh. The model is shown in Figure~\ref{fig:CFD flow cell}.%
\begin{figure}
\includegraphics{Protein_Adsorption/Figures/CFD_geometry_flow_cell}\includegraphics{Protein_Adsorption/Figures/CFD_geometry_flow_cell_mesh}

\caption{\label{fig:CFD flow cell}Model of the flow cell with tubing used
for the first stage of CFD simulations. A close-up of the mesh is
shown at right. The inlet tube has been truncated in these images.}


%
\end{figure}
 This model did not include the resonator and waveguide, as they have
a minimal effect on the overall flow in the channel. A steady-state
simulation was performed first to calculate the flow field. Transient
simulations were then performed in which the concentration at the
inlet of the tubing was suddenly increased from zero to the target
concentration. The concentration of protein was monitored at a point
in the flow cell at the location of the resonator, and the simulation
was run until the concentration at this point reached the target value.
It was assumed that the low concentrations of protein used in these
experiments did not affect the flow field significantly, so the transient
simulations utilized the flow field from the steady-state simulation
to reduce computation time. 

Two additional CFD models were created to simulate the transport of
protein near the resonator. A detailed three-dimensional model was
used to model the flow field around the resonator and the waveguide
as shown in Figure~\ref{fig:CFD 3D resonator}.%
\begin{figure}
\includegraphics{Protein_Adsorption/Figures/CFD_3D_mesh}

\caption{\label{fig:CFD 3D resonator}3D model of the resonator and waveguide
in the flow cell of the WGM biosensor.}


%
\end{figure}
 This model ran too slowly to be used for transient simulations of
adsorption, so a simplified model was created with a two-dimensional
axisymmetric geometry as shown in Figure~\ref{fig:CFD Mesh 2D}.%
\begin{figure}
\includegraphics{Protein_Adsorption/Figures/CFD_geometry_2D_mesh}

\includegraphics{Protein_Adsorption/Figures/CFD_geometry_2D_mesh_closeup}

\caption{\label{fig:CFD Mesh 2D}Overall mesh for the two-dimensional CFD model
and a close-up of the mesh on the resonator.}


%
\end{figure}
 Because the axisymmetric geometry models a channel with a circular
cross section while the actual channel has a square cross section,
the average flow velocity of the inlet boundary condition was adjusted
so the velocity near the resonator matched the results from the three-dimensional
simulation. Data from the simulation of the whole flow cell was used
to set the concentration over time at the inlet of the axisymmetric
model. A surface reaction was defined on the surface of the resonator
using the Langmuir model built in to the Biochemistry module of ACE+.
The purpose of this reaction was to deplete the protein near the resonator
at an appropriate rate, rather than to model the actual chemistry
of adsorption. An association rate constant of $1.44\times10^{6}\, L\, mol{}^{-1}s^{-1}$,
a dissociation rate constant of $8.9\times10^{-4}\, s^{-1}$, and
a maximum density of adsorption sites of $4.4\times10^{-9}\, mol\, m^{-2}$
were used to match the maximum adsorption rate that was observed in
the experimental data for FN on DETA and 13F. A steady-state simulation
was used to establish the flow field and a transient simulation was
used to determine the concentration of protein near the surface of
the resonator. The adsorption simulation mandated a fairly small time
step (on the order of $10^{-4}\, s$), so it was impractical to run
the simulation to equilibrium due to the large number of time steps
required. The system was simulated for 150-300 seconds (of simulated
time). It was assumed that the near-surface concentration increased
linearly from the end of the CFD simulation to the target concentration.


\subsection{Modeling the Kinetics of Protein Adsorption}

It is well established that single-layer adsorption models can be
stated in the general form\begin{equation}
\frac{d\theta}{dt}=k_{a}c\,\Phi\left(\theta\right)-k_{d}\theta\label{eq:Single Layer Kinetics}\end{equation}
where $\theta$ is the fraction of the surface covered by adsorbed
particles, $c$ is the concentration of protein in solution near the
surface, $k_{a}$ is the adsorption rate constant, and $k_{d}$ is
the desorption rate constant \cite{Andrade1986}. The function $\Phi\left(\theta\right)$
represents the blocking effect of adsorbed particles. For the Langmuir
model, the blocking function is simply $\Phi\left(\theta\right)=1-\theta/\theta_{\infty}$.
The blocking function for the random sequential adsorption (RSA) model
is not available in analytic form. A widely used approximation for
the random sequential adsorption of spherical particles is\begin{equation}
\Phi\left(\theta\right)=\frac{\left(1-\bar{\theta}\right)^{3}}{1.0-0.812\bar{\,\theta}+0.2335\bar{\,\theta}^{2}+0.0845\bar{\,\theta}^{3}}\label{eq:RSA Blocking Function}\end{equation}
where $\bar{\theta}=\theta/\theta_{\infty}$ \cite{Schaaf1989}. Adsorption
models of this form assume that proteins behave like hard particles,
with the RSA model making a further assumption that the particles
pack like spheres.


\subsubsection{RSA-Type Model of Adsorption with Transition}

%
\begin{figure}
\subfloat[\label{fig:Two stage adsorption}Two-stage adsorption]{\includegraphics{Protein_Adsorption/Figures/two_stage_figure}

}\subfloat[\label{fig:Two layer adsorption}Two-layer adsorption]{\includegraphics{Protein_Adsorption/Figures/two_layer_figure}



}

\caption{\label{fig:Adsorption Schematics}Schematic representations of adsorption
with post-adsorption transition (a) and two-layer adsorption (b).}


%
\end{figure}
Single-step adsorption models cannot adequately predict the kinetics
of adsorption for many combinations of proteins and surfaces. More
complex models have been developed to account for these experimental
results. Because many surfaces cause proteins to denature upon adsorption,
it is common to model adsorption with a post-adsorption transition.
A schematic representation of this process is shown in Figure \ref{fig:Two stage adsorption}.
An adsorbing protein is treated as a hard sphere (state $\alpha$).
After adsorption, the sphere may transition to a second state $\beta$
with a different radius if there is available space on the surface.
This process can be described by the kinetic equations\begin{align}
\frac{d\rho_{\alpha}}{dt} & =k_{a}c\,\Phi_{\alpha}-k_{s}\rho_{\alpha}\Psi_{\alpha\beta}-k_{d}\rho_{\alpha}\label{eq:Adsorption Transition Kinetics}\\
\frac{d\rho_{\beta}}{dt} & =k_{s}\rho_{\alpha}\Psi_{\alpha\beta}\label{eq:Adsorption Transition Kinetics 2}\end{align}
The blocking function $\Phi_{\alpha}$ represents the probability
that a protein adsorbing from solution is able to find space to adsorb,
while $\Psi_{\alpha\beta}$ represents the probability that an adsorbed
protein in state $\alpha$ has space to transition to state $\beta$.
The form of the blocking functions depends upon the assumptions of
the model. When the particles are spherical with initial radius $R_{\alpha}$
and final radius $R_{\beta}$, scaled particle theory (SPT) can be
used to derive the following blocking functions \cite{Brusatori1999}:
\begin{align}
\Phi_{\alpha} & =\left(1-\theta\right)\exp\left[-\frac{2\left(\overline{\rho_{\alpha}}+\Sigma\overline{\rho_{\beta}}\right)}{1-\theta}-\frac{\overline{\rho_{\alpha}}+\overline{\rho_{\beta}}+\left(\Sigma-1\right)^{2}\overline{\rho_{\alpha}}\overline{\rho_{\beta}}}{\left(1-\theta\right)^{2}}\right]\label{eq:Phi_alpha}\\
\Psi_{\alpha\beta} & =\exp\left[-\frac{2\left(\Sigma-1\right)\left(\overline{\rho_{\alpha}}+\Sigma\overline{\rho_{\beta}}\right)}{1-\theta}-\frac{\left(\Sigma^{2}-1\right)\left[\overline{\rho_{\alpha}}+\overline{\rho_{\beta}}+\left(\Sigma-1\right)^{2}\overline{\rho_{\alpha}}\overline{\rho_{\beta}}\right]}{\left(1-\theta\right)^{2}}\right]\label{eq:Psi_alpha_beta}\end{align}
The following non-dimensional variables were defined: $\overline{\rho_{\alpha}}=\rho_{\alpha}\pi R_{\alpha}^{2}$,
$\overline{\rho_{\beta}}=\rho_{\beta}\pi R_{\alpha}^{2}$, $\theta=\overline{\rho_{\alpha}}+\Sigma^{2}\overline{\rho_{\beta}}$,
and $\Sigma=R_{\beta}/R_{\alpha}$. Since the blocking functions derived
from SPT describe spherical particles, they can be directly compared
to the RSA blocking function. When $k_{s}=0$, equations \ref{eq:Adsorption Transition Kinetics}
and \ref{eq:Adsorption Transition Kinetics 2} reduce to equation
\ref{eq:Single Layer Kinetics}. By setting $\rho_{\beta}=0$ and
$\Sigma=1$, equations \ref{eq:RSA Blocking Function} and \ref{eq:Phi_alpha}
can be compared.

To quantitatively compare different multi-layer models, it is necessary
to transform equations from different sources to use a common set
of variables. The surface number densities $\rho_{i}$ used in equations
\ref{eq:Adsorption Transition Kinetics} and \ref{eq:Adsorption Transition Kinetics 2}
can be converted to fractional surface coverage by multiplying by
the area covered by an adsorbed particle in state $i$, $\sigma_{i}$,
to obtain:\begin{align}
\frac{d\theta_{\alpha}}{dt} & =k_{a}\,\sigma_{\alpha}c\,\Phi_{\alpha}-k_{s}\theta_{\alpha}\Psi_{\alpha\beta}-k_{d}\theta_{\alpha}\label{eq:dtheta_alpha dt}\\
\frac{d\theta_{\beta}}{dt} & =k_{s}\Sigma^{2}\theta_{\alpha}\Psi_{\alpha\beta}\label{eq:dtheta_beta dt}\end{align}



\subsubsection{Langmuir-Type Model of Adsorption with Transition}

A different model of adsorption with a post-adsorption transition
has been used to fit the adsorption kinetics of the fibronectin fragment
\BPChem{FNIII\_{7-10}} \cite{Lundstroem1984,Michael2003}. Written
using the same variables as equations \ref{eq:Adsorption Transition Kinetics}
and \ref{eq:Adsorption Transition Kinetics 2}, the equations that
describe this model are\begin{align}
\frac{d\rho_{\alpha}}{dt} & =k_{a}\, c\, A_{av}-k_{s}\,\rho_{\alpha}\, A_{av}-k_{d}\,\rho_{\alpha}\label{eq:Michael kinetics 1}\\
\frac{d\rho_{\beta}}{dt} & =k_{s}\,\rho_{\alpha}\, A{}_{av}\label{eq:Michael kinetics 2}\end{align}
where $Y_{i}$ is the surface density ($ng/cm^{2}$) of adsorbed protein
in each state. $A_{av}$ is the surface area ($cm^{2}$) available
for adsorption, which is given by\[
A_{av}=A_{total}\left(1-f\,\sigma_{1}\,\rho_{1}-f\, b\,\sigma_{1}\,\rho_{2}\right)\]
where $b=\sigma_{\beta}/\sigma_{\alpha}$. Since $\theta_{i}=\sigma_{i}\, f\,\rho_{i}$
and $f=N_{A}/M$ ($molecules/ng$), this expression can be written
$A_{av}=A_{total}\left(1-\theta\right)$, where $\theta=\theta_{\alpha}+\theta_{\beta}$.
Equations \ref{eq:Michael kinetics 1} and \ref{eq:Michael kinetics 2}
can be written in terms of fractional surface coverage by multiplying
both sides by $\sigma_{i}\, f$ to obtain:\begin{align}
\frac{d\theta_{1}}{dt} & =k\,\sigma_{1}\,\sigma A_{total}\, f\, c\left(1-\theta\right)-s\, A_{total}\,\theta{}_{1}\left(1-\theta\right)-r\,\theta_{1}\label{eq:Langmuir two stage 1}\\
\frac{d\theta_{2}}{dt} & =s\, A_{total}\, b\,\theta_{1}\,\left(1-\theta\right)\label{eq:Langmuir two stage 2}\end{align}
It is clear that the blocking function for this model is actually
the Langmuir blocking function, with $\theta_{\infty}=1$.


\subsubsection{Langmuir-Type Model of Two-Layer Adsorption}

A two-layer adsorption model was formulated, based on the assumptions
of the Langmuir adsorption model. A schematic representation of this
model is shown in Figure \ref{fig:Two layer adsorption}.~The two-layer
adsorption model can be represented by the chemical equations \[ \cee{A + B <->[k_{a1},k_{d1}] AB} \]
and \[ \cee{A + AB <->[k_{a2},k_{d2}] AAB} \] A represents a molecule
in solution, B represents an available adsorption site, AB represents
a single molecule adsorbed on the surface, and AAB represents a {}``stack''
of two adsorbed molecules. The kinetics of adsorption can be modeled
by a set of coupled ordinary differential equations:\begin{align}
\frac{d\theta_{AB}}{dt} & =k_{a1}c\,\left(\theta_{\infty}-\theta_{AB}-\theta_{AAB}\right)-k_{d1}\theta_{AB}-k_{a2}c\,\theta_{AB}\label{eq:Two Layer AB}\\
\frac{d\theta_{AAB}}{dt} & =k_{a2}c\,\theta_{AB}-k_{d2}-\theta_{AAB}\label{eq:Two Layer AAB}\end{align}
$\theta_{AB}$ is the fraction of the surface covered by a single
particle and $\theta_{AAB}$ is the fraction of the surface covered
by two layers of particles. $\theta=\theta_{AB}+\theta_{AAB}$ is
the total fractional surface coverage. Although it is straightforward
to solve this set of equations analytically, the resulting formulae
are complicated, and the analysis of the solutions is beyond the scope
of this work. 


\subsubsection{Calculating the Surface Density of Active GO }

Since the Langmuir and RSA models only allow protein to exist in one
state, all of the adsorbed protein must be treated as active or inactive.
For the two adsorption models that incorporate a post-adsorption transition,
enzyme in the initial state was assumed to be active, and enzyme in
the denatured state was assumed to be inactive. For the two-layer
model, it was assumed that enzyme molecules in the upper layer prevented
the substrate solution from interacting with molecules adsorbed in
the lower layer. Enzyme in the lower layer was therefore treated as
inactive. The activity of adsorbed enzyme in contact with the surface
which is not screened by an upper layer depends upon the nature of
the surface. It was assumed that hydrophilic surfaces (glass and DETA)
did not induce denaturation upon adsorption, so single-layer protein
was assumed to be active. The surface density of active enzyme was
calculated~$\rho_{active}=\rho_{AB}+\rho_{AAB}$. For 13F (a hydrophobic
surface), it was assumed that adsorption induces denaturation and
destroys the activity of the enzyme. Only enzyme in the upper layer
was considered to be active, so $\rho_{active}=\rho_{AAB}$ The surface
densities of active protein and total protein predicted by the models
were plotted and compared to experimental results. 


\subsection{Implementation of Models and Fitting to Experimental Data}

A general single-layer adsorption simulation based on Equation \ref{eq:Single Layer Kinetics}
was implemented using the Python programming language. Various blocking
functions, such as the Langmuir and RSA blocking functions, could
be plugged into the simulation. Another simulation was created based
on equations \ref{eq:dtheta_alpha dt} and \ref{eq:dtheta_beta dt}
that could utilize either the Langmuir or SPT-derived blocking function.
Equations \ref{eq:Langmuir two stage 1} and \ref{eq:Two Layer AAB}
were used to create a two-layer adsorption simulation. The differential
equations were solved numerically using the \emph{odeint} routine
from SciPy~\cite{Jones2001-}. All equations were solved in terms
of fractional surface coverage to avoid numerical difficulties that
may occur when working with small floating-point numbers. For comparison
with experimental data, the fractional surface coverage predicted
by each model was converted to surface density ($ng\, cm^{-1}$) using
$\rho_{i}=\theta_{i}\,\sigma_{i}^{-1}\, f^{-1}$. 

Each model was fitted to the experimental data from the WGM biosensor
for glass, DETA, 13F, and SiPEG by adjusting the parameters until
the best possible fit was achieved, according to the least-squares
criterion. The concentration of protein in solution near the surface
was assumed to be constant. For each surface, kinetic curves for all
solution concentrations were fitted simultaneously with a single set
of parameters using the \emph{leastsq} fitting routine from SciPy,
which uses a modified version of the Levenberg-Marquardt algorithm.
Third-order splines were used to interpolate the averaged experimental
data to the same time points used in the model. The quality of fit
was quantified by computing the sum of squared errors (SSE) for the
total surface concentration of adsorbed protein measured by the WGM
sensor and the surface concentration predicted by the model. The SSE
for each experiment was divided by the total number of data points
in the data set so that the quality of fit could be compared between
data sets with different numbers of time points. 


\section{Results}

A comparison of the blocking functions of the Langmuir and RSA models
is shown in Figure~\ref{fig:Blocking function comparison}.%
\begin{figure}
\includegraphics{Protein_Adsorption/Plots/ASF_comparison}

\caption{\label{fig:Blocking function comparison}Blocking functions from the
RSA model ($\phi_{FIT,3}$) and scaled particle theory ($\Phi_{\alpha}$)}
%
\end{figure}
 The blocking function for protein in the initial state for the model
derived from scaled particle theory is also shown. It can be seen
that the first-level blocking function derived from SPT is similar
but not identical to the RSA blocking function, especially at higher
values of fractional surface coverage.

The kinetics of adsorption based on the Langmuir blocking function
were compared to the kinetics modeled by the SPT blocking functions,
and the results are shown in Figure~\,\ref{fig:SPT vs Langmuir kinetics}.
%
\begin{figure}[h]
\subfloat[\label{fig:SPT kinetics}Blocking function from scaled particle theory]{\includegraphics[scale=0.75]{Protein_Adsorption/Plots/SPT_kinetics}}\subfloat[\label{fig:Langmuir two-stage kinetics}Langmuir two-stage model]{\includegraphics[scale=0.75]{Protein_Adsorption/Plots/Langmuir_twostage_kinetics}}

\caption{\label{fig:SPT vs Langmuir kinetics}Comparison of kinetics predicted
by SPT blocking function (a) and Langmuir blocking function (b) for
$k_{a}=1$, $k_{s}=\pi$, $k_{d}=\pi$, $r_{\alpha}=1$, $\Sigma=1.2$,
and $c=1$.}
%
\end{figure}
 The parameters used for the comparison were taken from Figure 2a
from \cite{Brusatori1999}. Note that Figure 2a from \cite{Brusatori1999}
cannot be directly compared directly with Figure \ref{fig:SPT vs Langmuir kinetics}
in this work, because $\theta_{\beta}\neq\overline{\rho_{\beta}}$.


\subsection{Transport Analysis of the WGM Biosensor}

The steady-state velocity magnitude predicted by the model of the
whole flow cell is shown in Figure~\ref{fig:CFD 3D velocity}.%
\begin{figure}
\includegraphics{Protein_Adsorption/Figures/CFD_3D_velocity}\includegraphics{Protein_Adsorption/Figures/CFD_3D_velocity_xcut}

\caption{\label{fig:CFD 3D velocity}The magnitude of velocity predicted by
CFD simulations in the vicinity of the WGM resonator.}
%
\end{figure}
It can be seen that the flow field around the resonator was symmetric
about the long axis of the resonator, with only a minor perturbation
caused by the presence of the waveguide. This configuration ensured
that the shear rate was constant in the region where the evanescent
wave was excited, minimizing any shear rate effects on the adsorption
of protein from solution. These results also confirmed that the axisymmetric
model was a reasonable choice for simulating the depletion region
near the surface of the resonator.


\subsubsection{Fibronectin}

The evolution of the concentration of FN in solution near the surface
of the WGM resonator over time is shown in Figure \ref{fig:CFD Bulk Conc}.%
\begin{figure}
\subfloat[\label{fig:CFD FN near surf conc}Near-surface concentration]{\includegraphics{Protein_Adsorption/Plots/CFD_FN_DETA_near_surf_conc}



}\subfloat[\label{fig:CFD FN near surf conc rescaled}Same data, rescaled to
show lower concentrations]{\includegraphics{Protein_Adsorption/Plots/CFD_FN_DETA_near_surf_conc_rescaled}



}

\includegraphics{Protein_Adsorption/Plots/CFD_FN_legend}\caption{\label{fig:CFD Bulk Conc}CFD prediction of the concentration of FN
very close to the surface of the resonator.}


%
\end{figure}
 At $10\,\mu g/ml$ the concentration near the surface of the resonator
required about 150 seconds to reach its final value. Figure \ref{fig:CFD FN near surf conc rescaled}
indicates that the near-surface concentration takes longer to reach
its final value as the bulk concentration is decreased. Transport
was not analyzed for the SiPEG surface because its low affinity for
protein was not expected to deplete protein in solution near the surface
significantly. The surface density of adsorbed FN predicted by the
CFD model is shown in Figure \ref{fig:CFD near surface conc}, along
with the average experimental data.%
\begin{figure}
\subfloat[\label{fig:CFD FN DETA surf conc}DETA]{\includegraphics{Protein_Adsorption/Plots/CFD_FN_DETA_surf_conc}}\subfloat[\label{fig:CFD FN 13F surf conc}13F]{\includegraphics{Protein_Adsorption/Plots/CFD_FN_13F_surf_conc}}

\includegraphics{Protein_Adsorption/Plots/CFD_FN_legend}

\caption{\label{fig:CFD near surface conc}CFD predictions and experimental
measurements of the surface density of adsorbed FN. Thick lines indicate
CFD predictions, while thin lines indicate average experimental data.}


%
\end{figure}



\subsubsection{Glucose Oxidase}

A similar analysis was performed for the transport of glucose oxidase
in the flow cell. The predicted concentration near the surface of
the resonator with adsorption parameters for GO on glass is shown
in Figure \ref{fig:CFD GO near surf conc}.%
\begin{figure}
\includegraphics{Protein_Adsorption/Plots/CFD_GO_glass_near_surf_conc}

\caption{\label{fig:CFD GO near surf conc}CFD prediction of the concentration
of GO near the surface of the resonator.}
%
\end{figure}
 The results for DETA and 13F are virtually identical. To verify that
the correct kinetic constants were used in the Langmuir adsorption
model, the surface density of adsorbed protein over time predicted
by the CFD simulations was plotted along with the measured surface
density. These results are shown in Figure \ref{fig:CFD GO near surf conc}.%
\begin{figure}
\subfloat[\label{fig:CFD GO DETA surf conc}DETA]{\includegraphics{Protein_Adsorption/Plots/CFD_GO_DETA_surf_conc}}\subfloat[\label{fig:CFD GO 13F surf conc}13F]{\includegraphics{Protein_Adsorption/Plots/CFD_GO_13F_surf_conc}

}

\subfloat[\label{fig:CFD GO glass surf conc}Glass]{\includegraphics{Protein_Adsorption/Plots/CFD_GO_glass_surf_conc}

}\includegraphics{Protein_Adsorption/Plots/CFD_surf_conc_legend}\caption{\label{fig:CFD GO surface density}Surface density of adsorbed GO
predicted by CFD simulation and measured by WGM biosensor.}
%
\end{figure}
 Transport was not analyzed for the SiPEG surface because its low
affinity for protein was not expected to deplete protein in solution
near the surface.


\subsection{Modeling the Adsorption of Fibronectin on Silane Surfaces}

The kinetics of adsorption of fibronectin on 13F, DETA, and SiPEG
are shown in Figure \ref{fig:FN experiments}.%
\begin{figure}
\includegraphics[width=1\columnwidth]{Protein_Adsorption/Plots/FN_Experimental_Data}

\caption{\label{fig:FN experiments}Measured adsorption kinetics for fibronectin
on 13F, DETA, and SiPEG surfaces.}


%
\end{figure}
 The kinetics predicted by the models fitted to adsorption on DETA
are shown in Figure \ref{fig:FN DETA fitted}, and the fitted parameter
values are shown in Table \ref{tab:FN on DETA params}.%
\begin{table}
\caption{\label{tab:FN on DETA params}Fitted parameter values for FN on DETA.}
\begin{tabular}{>{\raggedright}p{0.75in}cccccc}
 & $k_{a}\left(cm^{3}\, ng^{-1}\, s^{-1}\right)$ & $k_{s}\left(s^{-1}\right)$ & $k_{d}\left(s^{-1}\right)$ & $\sigma_{\alpha}\left(nm^{2}\right)$ & $\sigma_{\beta}\left(nm^{2}\right)$ & $SSE$\tabularnewline[\doublerulesep]
\cline{2-7} 
\noalign{\vskip\doublerulesep}
RSA & $2.05\times10^{-6}$ &  & $3.16\times10^{-4}$ & $182$ &  & $86.1$\tabularnewline
\noalign{\vskip\doublerulesep}
RSA with transition & $2.37\times10^{-6}$ & $1.14\times10^{-4}$ & $2.12\times10^{-4}$ & $203$ & $380$ & $86.0$\tabularnewline
\noalign{\vskip\doublerulesep}
\end{tabular}%
\end{table}
%
\begin{figure}
\includegraphics{Protein_Adsorption/Plots/FN_DETA_RSA_CFD}

\caption{\label{fig:FN DETA fitted}RSA model fitted to experimental data for
FN on DETA.}


%
\end{figure}
 The kinetics predicted by models fitted to FN adsorption on 13F are
shown in Figure \ref{fig:FN 13F fitted}, and the parameter values
for the fitted models are shown in Table \ref{tab:FN OEG fitted params}.%
\begin{table}
\caption{\label{tab:FN on 13F}Fitted parameter values for FN on 13F.}
\begin{tabular}{>{\raggedright}p{0.75in}cccccc}
 & $k_{a}\left(cm^{3}\, ng^{-1}\, s^{-1}\right)$ & $k_{s}\left(s^{-1}\right)$ & $k_{d}\left(s^{-1}\right)$ & $\sigma_{\alpha}\left(nm^{2}\right)$ & $\sigma_{\beta}\left(nm^{2}\right)$ & $SSE$\tabularnewline[\doublerulesep]
\cline{2-7} 
\noalign{\vskip\doublerulesep}
RSA & $2.24\times10^{-6}$ &  & $2.49\times10^{-4}$ & $196$ &  & $121$\tabularnewline
\noalign{\vskip\doublerulesep}
RSA with transition & $1.99\times10^{-6}$ & $8.00\times10^{-3}$ & $9.95\times10^{-5}$ & $176$ & $303$ & $85.5$\tabularnewline
\noalign{\vskip\doublerulesep}
\end{tabular}%
\end{table}
%
\begin{figure}
\includegraphics{Protein_Adsorption/Plots/FN_13F_VanTassel_CFD}

\caption{\label{fig:FN 13F fitted}Adsorption model with post-adsorption transition
fitted to experimental data for FN on 13F.}
%
\end{figure}
 The kinetics predicted by models fitted to FN adsorption on SiPEG
are shown in Figure \ref{fig:FN OEG fitted}, and the parameter values
for the fitted models are shown in Table \ref{tab:FN OEG fitted params}.%
\begin{table}
\caption{\label{tab:FN OEG fitted params}Parameter values fitted to FN on
SiPEG.}
\begin{tabular}{>{\raggedright}p{0.75in}cccccc}
 & $k_{a}\left(cm^{3}\, ng^{-1}\, s^{-1}\right)$ & $k_{s}\left(s^{-1}\right)$ & $k_{d}\left(s^{-1}\right)$ & $\sigma_{\alpha}\left(nm^{2}\right)$ & $\sigma_{\beta}\left(nm^{2}\right)$ & $SSE$\tabularnewline[\doublerulesep]
\cline{2-7} 
\noalign{\vskip\doublerulesep}
RSA & $9.25\times10^{-7}$ &  & $7.14\times10^{-4}$ & $1240$ &  & $2.7$\tabularnewline
\noalign{\vskip\doublerulesep}
RSA with transition & $1.12\times10^{-6}$ & $1.16\times10^{-2}$ & $1.04\times10^{-4}$ & $1240$ & $3420$ & $1.86$\tabularnewline
\noalign{\vskip\doublerulesep}
\end{tabular}%
\end{table}
%
\begin{figure}
\includegraphics{Protein_Adsorption/Plots/FN_OEG_RSA_CFD}

\caption{\label{fig:FN OEG fitted}RSA model fitted to experimental data for
FN on SiPEG.}
%
\end{figure}
 Results of the cell culture experiments are shown in Table~\ref{tab:Cell counts on FN}.%
\begin{table}
\caption{\label{tab:Cell counts on FN}Cell counts ($mm^{-2}$) for embryonic
hippocampal neurons and embryonic skeletal muscle cultured on silane
surfaces.}


\begin{tabular}{cccccc}
Cell Type & Parameter & N & DETA & 13F & SiPEG\tabularnewline[\doublerulesep]
\hline
\noalign{\vskip\doublerulesep}
Hippocampal & Live & $9$ & $212\pm102$ & $1\pm3$ & $4\pm5$\tabularnewline
\noalign{\vskip\doublerulesep}
Hippocampal & Dead & $9$ & $340\pm75$ & $218\pm81$ & $265\pm86$\tabularnewline
\noalign{\vskip\doublerulesep}
Muscle & Live & $4$ & $178\pm43$ & $50\pm32$ & $0\pm0$\tabularnewline
\noalign{\vskip\doublerulesep}
Muscle & Dead & $4$ & $63\pm66$ & $18\pm14$ & $111\pm59$\tabularnewline
\noalign{\vskip\doublerulesep}
Muscle & Myotubes & $4$ & $35\pm13$ & $0\pm0$ & $0\pm0$\tabularnewline
\noalign{\vskip\doublerulesep}
\end{tabular}%
\end{table}



\subsection{Modeling the Adsorption of Glucose Oxidase on Silane Surfaces}

The experimentally measured kinetic curves for glucose oxidase on
glass, DETA, 13F, and SiPEG, along with the model that achieved the
best fit to each data set, are shown in Figures \ref{fig:GO glass plot},
\ref{fig:GO on DETA}, \ref{fig:GO on 13F} and \ref{fig:GO on PEG},
respectively. The initial rate of adsorption was highest on the DETA
and 13F surfaces and lowest on the SiPEG surface. The surface density
of adsorbed protein reached the highest saturation value on the DETA
surface at $100\,\mu g/ml$, while the highest saturation at $10\,\mu g/ml$
occurred on 13F. On the 13F surface the saturation surface density
at $10\,\mu g/ml$ was nearly as high as $100\,\mu g/ml$. Adsorption
on the glass surface showed an {}``overshoot'' profile in which
the surface density reached a maximum and then decreased gradually
with time. The parameter values for the five models fitted to the
experimental measurements are shown in Tables \ref{tab:GO params glass},
\ref{tab:GO params DETA}, and \ref{tab:GO params 13F}, respectively.
Only the RSA and Langmuir models were fitted to GO on SiPEG, and the
fitted parameter values are shown in Table \ref{tab:GO params PEG}.

%
\begin{figure}
\includegraphics{Protein_Adsorption/Plots/GO_glass}\caption{\label{fig:GO glass plot}Langmuir model with post-adsorption transition
fitted to data for GO on glass from the WGM biosensor.}
%
\end{figure}
%
\begin{figure}
\includegraphics{Protein_Adsorption/Plots/GO_DETA}

\caption{\label{fig:GO on DETA}Langmuir two-layer adsorption model fitted
to data for GO on DETA from the WGM biosensor.}
%
\end{figure}
%
\begin{figure}
\includegraphics{Protein_Adsorption/Plots/GO_13F}

\caption{\label{fig:GO on 13F}Langmuir model with post-adsorption transition
fitted to data for GO on 13F from the WGM biosensor.}
%
\end{figure}
%
\begin{figure}
\includegraphics{Protein_Adsorption/Plots/GO_PEG}

\caption{\label{fig:GO on PEG}RSA model fitted to data for GO on SiPEG from
the WGM biosensor.}
%
\end{figure}
%
\begin{figure}
\includegraphics{Protein_Adsorption/Plots/GO_glass_activity}

\caption{\label{fig:GO glass activity}Experimental measurements and model
predictions of GO activity on a glass surface. The {*} denotes the
model with the best fit to the kinetic data.}


%
\end{figure}
%
\begin{figure}
\includegraphics{Protein_Adsorption/Plots/GO_DETA_activity}

\caption{\label{fig:GO DETA activity}Experimental measurements and model predictions
of GO activity on a DETA surface. The {*} denotes the model with the
best fit to the kinetic data.}
%
\end{figure}
%
\begin{figure}
\includegraphics{Protein_Adsorption/Plots/GO_13F_activity}

\caption{\label{fig:GO 13F activity}Experimental measurements and model predictions
of GO activity on a 13F surface. The {*} denotes the model with the
best fit to the kinetic data.}
%
\end{figure}
%
\begin{table}
\caption{\label{tab:GO params glass}Parameters fitted to data for GO on glass.}
\begin{tabular}{>{\raggedright}p{0.75in}cccccc}
 & $k_{a}\left(cm^{3}\, ng^{-1}\, s^{-1}\right)$ & $k_{s}\left(s^{-1}\right)$ & $k_{d}\left(s^{-1}\right)$ & $\sigma_{\alpha}\left(nm^{2}\right)$ & $\sigma_{\beta}\left(nm^{2}\right)$ & $SSE$\tabularnewline[\doublerulesep]
\cline{2-7} 
\noalign{\vskip\doublerulesep}
Langmuir & $3.33\times10^{-8}$ &  & $1.15\times10^{-3}$ & $257$ &  & $6.15$\tabularnewline
Langmuir with transition & $1.35\times10^{-8}$ & $1.48\times10^{-3}$ & $1.75\times10^{-3}$ & $122$ & $414$ & $0.73$\tabularnewline
\noalign{\vskip\doublerulesep}
RSA & $1.16\times10^{-8}$ &  & $1.34\times10^{-3}$ & $80.8$ &  & $7.81$\tabularnewline
\noalign{\vskip\doublerulesep}
RSA with transition & $3.80\times10^{-9}$ & $2.20\times10^{-3}$ & $1.99\times10^{-3}$ & $31.5$ & $349$ & $1.19$\tabularnewline
\noalign{\vskip\doublerulesep}
\noalign{\vskip\doublerulesep}
 & $k_{a1}$ & $k_{a2}$ & $k_{d1}$ & $k_{d2}$ & $\sigma$ & $SSE$\tabularnewline
Two layer model & $1.85\times10^{-8}$ & $5.46\times10^{-8}$ & $2.18\times10^{-4}$ & $4.60\times10^{-3}$ & $224$ & $3.15$\tabularnewline
\end{tabular}%
\end{table}
%
\begin{table}
\caption{\label{tab:GO params DETA}Parameters fitted to data for GO on DETA.}
\begin{tabular}{>{\raggedright}p{0.75in}cccccc}
 & $k_{a}\left(cm^{3}\, ng^{-1}\, s^{-1}\right)$ & $k_{s}\left(s^{-1}\right)$ & $k_{d}\left(s^{-1}\right)$ & $\sigma_{\alpha}\left(nm^{2}\right)$ & $\sigma_{\beta}\left(nm^{2}\right)$ & $SSE$\tabularnewline[\doublerulesep]
\cline{2-7} 
\noalign{\vskip\doublerulesep}
Langmuir & $3.10\times10^{-8}$ &  & $1.83\times10^{-4}$ & $190$ &  & $66.1$\tabularnewline
\noalign{\vskip\doublerulesep}
Langmuir with transition & $1.15\times10^{-7}$ & $9.41\times10^{-3}$ & $3.22\times10^{-2}$ & $161$ & $161$ & $38.7$\tabularnewline
\noalign{\vskip\doublerulesep}
RSA & $1.47\times10^{-8}$ &  & $2.15\times10^{-4}$ & $72.6$ &  & $39.3$\tabularnewline
\noalign{\vskip\doublerulesep}
RSA with transition & $1.19\times10^{-8}$ & $0.412$ & $2.33\times10^{-4}$ & $28.0$ & $457$ & $6.47$\tabularnewline
\noalign{\vskip\doublerulesep}
\noalign{\vskip\doublerulesep}
 & $k_{a1}$ & $k_{a2}$ & $k_{d1}$ & $k_{d2}$ & $\sigma$ & $SSE$\tabularnewline
\noalign{\vskip\doublerulesep}
Two layer model & $1.44\times10^{-7}$ & $1.26\times10^{-8}$ & $1.19\times10^{-3}$ & $3.42\times10^{-9}$ & $363$ & $12.7$\tabularnewline
\end{tabular}%
\end{table}
%
\begin{table}
\caption{\label{tab:GO params 13F}Parameters fitted to data for GO on 13F.}
\begin{tabular}{>{\raggedright}p{0.75in}cccccc}
 & $k_{a}\left(cm^{3}\, ng^{-1}\, s^{-1}\right)$ & $k_{s}\left(s^{-1}\right)$ & $k_{d}\left(s^{-1}\right)$ & $\sigma_{\alpha}\left(nm^{2}\right)$ & $\sigma_{\beta}\left(nm^{2}\right)$ & $SSE$\tabularnewline[\doublerulesep]
\cline{2-7} 
\noalign{\vskip\doublerulesep}
Langmuir & $7.20\times10^{-8}$ &  & $1.17\times10^{-4}$ & $247$ &  & $2.45$\tabularnewline
Langmuir with transition & $5.96\times10^{-8}$ & $1.14\times10^{-2}$ & $6.74\times10^{-5}$ & $209$ & $291$ & $2.23$\tabularnewline
\noalign{\vskip\doublerulesep}
RSA & $3.79\times10^{-8}$ &  & $3.81\times10^{-5}$ & $103$ &  & $12.8$\tabularnewline
\noalign{\vskip\doublerulesep}
RSA with transition & $1.27\times10^{-9}$ & $0.377$ & $9.73\times10^{-3}$ & $4.6$ & $175$ & $2.3$\tabularnewline
\noalign{\vskip\doublerulesep}
\noalign{\vskip\doublerulesep}
 & $k_{a1}$ & $k_{a2}$ & $k_{d1}$ & $k_{d2}$ & $\sigma$ & $SSE$\tabularnewline
Two layer model & $7.92\times10^{-8}$ & $7.18\times10^{-7}$ & $1.51\times10^{-4}$ & $1.28\times10^{-4}$ & $496$ & $1.82$\tabularnewline
\end{tabular}%
\end{table}
%
\begin{table}
\caption{\label{tab:GO params PEG}Fitted parameters for GO on SiPEG.}
\begin{tabular}{>{\raggedright}p{0.75in}cccc}
 & $k_{a}\left(cm^{3}\, ng^{-1}\, s^{-1}\right)$ & $k_{d}\left(s^{-1}\right)$ & $\sigma\,\left(nm^{2}\right)$ & $SSE$\tabularnewline[\doublerulesep]
\cline{2-5} 
\noalign{\vskip\doublerulesep}
Langmuir & $4.06\times10^{-8}$ & $3.67\times10^{-4}$ & $684$ & $0.81$\tabularnewline
\noalign{\vskip\doublerulesep}
RSA & $2.18\times10^{-8}$ & $2.55\times10^{-4}$ & $482$ & $0.73$\tabularnewline
\noalign{\vskip\doublerulesep}
\end{tabular}

%
\end{table}


For GO on glass, the Langmuir-derived models fitted the data slightly
better than the RSA-derived models. The Langmuir-type model with post-adsorption
transition provided the best fit to the kinetic data. Both models
with post-adsorption transition predicted the amount of active protein
at $10\,\mu g/ml$, but not at $100\,\mu g/ml$, as shown in Figure
\ref{fig:GO glass activity}. The activity predicted by both models
at $100\,\mu g/ml$ matched the experimental value much more closely
if it was assumed that the protein was active in both the initial
and final states. For GO on DETA, the RSA-derived models fitted the
data better than the Langmuir-type models. The best fit to the kinetic
curves was achieved by the RSA-type model with post-adsorption transition.
However, this model did not predict the amount of active enzyme at
$10\,\mu g/ml$ or $100\,\mu g/ml$, as shown in Figure \ref{fig:GO DETA activity}
The two-layer model does not fit the kinetic data quite as well, but
it does predict the amount of active enzyme very well. For GO on 13F,
the two-layer model and the Langmuir-type model with post-adsorption
transition fitted the kinetic data well. Although the two-layer model
fitted the kinetic data slightly better, Figure \ref{fig:GO 13F activity}
indicates that it does not accurately predict the amount of active
adsorbed protein. The Langmuir-type model with a post-adsorption transition
predicted the amount of active protein very well.


\section{Discussion}

Although the whispering gallery mode biosensor we constructed is more
sophisticated than any WGM biosensor that has been previously reported,
it is still a prototype system with significant room for improvement.
The sensitivity of the instrument could be improved by reducing the
effect of temperature on the resonant wavelengths of the spheroid.
This could be accomplished by actively stabilizing the temperature
of the instrument or measuring the temperature constantly and compensating
for the temperature drift in software. Temperature stabilization should
eliminate the need to perform a linear baseline subtraction on the
data. The instrument could be made significantly easier to use if
the resonator and waveguide did not have to be fabricated separately
and assembled by hand.

The methods used to convert the raw data (resonance wavelength over
time) to final data (adsorbed surface density over time) could also
be improved. Equation \ref{eq:WGM data analysis 3} depends upon several
constants taken from literature, such as the refractive index of adsorbed
protein and the refractive index increment of protein in solution.
The refractive index of adsorbed protein is often taken to be constant~\cite{Akimoto1999}.
More recent measurements have shown that the refractive index actually
depends upon the density of the adsorbed layer, which is influenced
by the size of the protein, the solvent conditions and the relaxation
of the protein on the surface~\cite{Voros2004}. Refractive index
increments for many proteins have been tabulated, and generally fall
into the range of $0.18-0.20\, cm^{3}g^{-1}$~\cite{Fasman1989}.
However, the refractive index increment depends on the wavelength
and solvent conditions, and the infrared laser used in the WGM biosensor
has a significantly longer wavelength ($1310\, nm$) than the wavelengths
at which the refractive index increment is usually measured ($\sim600nm$)~\cite{Rocco1983}.
The absolute surface densities reported by the instrument could be
made more accurate by measuring these values for the particular proteins
and solution conditions used in the protein adsorption experiments.


\subsection{Comparison of Adsorption Models}

Langmuir-type and RSA-type models are based upon different assumptions
about adsorption behavior. RSA models assume that proteins act like
hard spheres. Because the spheres adsorb randomly on the surface,
the surface packing is inefficient and the fractional surface coverage
reaches its maximum value around 0.547. Although it is well known
that proteins such as fibronectin are nothing like hard spheres, the
RSA model is surprising useful for fitting adsorption data for a wide
variety of proteins. The Langmuir model makes no assumptions about
the shape of the adsorbing particles and assumes that the surface
available for adsorption is a linear function of the surface coverage.
Adsorption stops when a limiting surface coverage has been reached,
which may be less than 100\% surface coverage if there is a limited
surface concentration of adsorption sites. Reality is probably somewhere
in between the two extremes, with proteins adsorbing in a random sequential
fashion but packing somewhat more effectively than spheres due their
flexibility. Fitting the RSA and Langmuir models to the same data
set and seeing which model fits better can indicate whether a particular
protein is hard or flexible (on a particular surface under specific
solvent conditions).

Two different kinetic models have been proposed in the literature
to model the adsorption of protein with a post-adsorption transition.
The model by Brusatori and Van Tassel \cite{Brusatori1999} is analogous
to the RSA model for adsorbing disks because it was assumed that the
particles were circular before and after the transition. Ideally,
the blocking function used in this model (equations~\ref{eq:Phi_alpha}
and~\ref{eq:Psi_alpha_beta}) should produce the same results as
the RSA blocking function (equation\,\ref{eq:RSA Blocking Function})
when the radius of particles in state~$\beta$ is identical to the
radius of particles in state~$\alpha$. Figure~\ref{fig:Blocking function comparison}
demonstrates that there is a small difference between the blocking
functions. This difference is probably because neither blocking function
is exact; the equation for the RSA blocking function is fitted to
simulation data and the two-stage blocking function is also an approximation.
If the RSA model and the transition model are fitted to the same data
set, the fitted parameters will not be identical, even if the transition
rate constant is very small.

When the model used by Michael et al. \cite{Michael2003} was written
in terms of a different set of variables, it was revealed that the
model uses a Langmuir-like blocking function. This result shows that
the relationship between the kinetic models stated and Michael et
al. and Brusatori and Van Tassel is analogous to the relationship
between the RSA model and the Langmuir model. This is illustrated
in Figure~\ref{fig:Adsorption Model Matrix}.%
\begin{figure}
\includegraphics{Protein_Adsorption/Figures/Adsorption_Model_Matrix}

\caption{\label{fig:Adsorption Model Matrix}Relationship between adsorption
models.}


%
\end{figure}
 Although the kinetics of the total fractional surface coverage $\theta$
are similar for both models, Figure~\ref{fig:SPT vs Langmuir kinetics}
shows that the kinetics of $\theta_{\alpha}$ and $\theta_{\beta}$
are very different for the two models. This difference has significant
implications for modeling experimental data if a protein in state~$\alpha$
functions differently than a protein in state~$\beta$.


\subsection{Transport Analysis}

Although the flow cell was designed to minimize transport limitations,
CFD simulations showed that the near-surface concentration of fibronectin
was influenced by transport limitations early in the adsorption process
on DETA and 13F surfaces. Initially, transport was limited by the
large amount of buffer that had to be displaced from the flow system
before the protein solution could reach the resonator at full concentration.
This limitation was due to the prototype nature of the system, and
can be easily eliminated in future systems. After the protein solution
in the bulk of the flow cell reached its target concentration, the
rate of adsorption onto the resonator was high enough to deplete the
fibronectin in a boundary layer near the resonator. Because of the
no-slip boundary condition at the surface and the absence of turbulent
mixing, diffusion was the only way that fibronectin could be transported
to the surface to adsorb. Fibronectin is a relatively large protein
with a correspondingly low diffusion coefficient, so its rate of diffusive
transport was rather low. This type of limitation is virtually unavoidable
in microfluidic systems when the rate of adsorption is high and the
solution concentration is low. The near-surface concentration predicted
by CFD was used when fitting the adsorption models to the fibronectin
data, enabling accurate kinetic parameters to be extracted even in
the presence of transport limitations.

The shape of the kinetic curves predicted by CFD simulations for the
adsorption of FN did not match the shape of the experimentally measured
curves during the first minute or so of adsorption. The experimental
curves have a {}``sigmoidal'' shape in which the rate of adsorption
increases during the first 30 seconds, while the maximum rate of adsorption
occurs at $t=0$ for the predicted curves. Since the displacement
of buffer from the system and the development of the near-surface
depletion layer have been modeled, another mechanism must be hypothesized
to account for this discrepancy. It is likely that a significant amount
of protein adsorbed to the silicone tubing between the reservoir and
the flow cell and to the channel walls in the flow cell. The rate
of adsorption to these surfaces would be highest when protein solution
was first introduced into the system, leading to a depletion of protein
in the solution that reached the resonator. These surfaces would quickly
saturate, reducing the rate of adsorption to the tubing and allowing
the protein solution to reach the resonator at the desired concentration.

In the experiments reported here, transport limitation had much less
influence on the adsorption of glucose oxidase. GO is a smaller protein
than FN, its initial rate of adsorption is lower and the molar concentrations
of GO used in these experiments were higher than the concentrations
of FN. GO reached 99\% of full concentration at the resonator surface
within 20 seconds, so the near-surface depletion had minimal impact
on the kinetics. Constant concentration was assumed when fitting the
kinetic adsorption models to the GO data. The association rate constants
from fitted models (Tables \ref{fig:FN DETA fitted}-\ref{fig:FN OEG fitted}
and \ref{tab:GO params glass}-\ref{tab:GO params PEG}) demonstrate
that GO has a much lower affinity for the surfaces than FN, which
explains why its maximum rate of adsorption is lower and transport
limitation is much less significant.

Our modeling method was approximate in that the evolution of the near-surface
concentration over time was computed once, based on the estimated
adsorption rate, and was not modified during the curve-fitting procedure.
A more accurate method would be to incorporate the CFD model into
the fitting routine so that the near-surface concentration would be
updated as the adsorption rate changed \cite{Jenkins2004}. However,
this method is only practical when the CFD simulation is simple enough
to run very quickly. It was also assumed that the near-surface concentration
increased linearly from sixty seconds until the target concentration
was reached. Although this probably had a slight impact on the modeled
kinetics, it was better than assuming that the concentration near
the resonator remained constant.


\subsection{Fibronectin Adsorption}

%
\begin{table}
\begin{tabular}{>{\centering}p{1in}>{\centering}p{1in}>{\centering}p{0.8in}>{\centering}p{0.8in}>{\centering}p{0.65in}>{\centering}p{1.25in}}
Solution Concentration ($\mu g/ml$) & Hydrophobic Surface & Hydrophilic Charged Surface & Hydrophilic Neutral Surface & Ref & Notes\tabularnewline[\doublerulesep]
\hline
\noalign{\vskip\doublerulesep}
$10$ & $200$ (13F) & $190$ & $25$ & This work & \tabularnewline
\noalign{\vskip\doublerulesep}
$10$ & $170$ (\ce{CH3}) & $170$ & $30$ & \cite{Keselowsky2003} & \tabularnewline
\noalign{\vskip\doublerulesep}
$10$ & $160$ (\ce{CH3}) &  & $30$ & \cite{Capadona2003} & \tabularnewline
\noalign{\vskip\doublerulesep}
$10$ & $140$ & $140$ & $50$ & \cite{Michael2003} & \ce{FN III_{7-10}} fragment\tabularnewline
\noalign{\vskip\doublerulesep}
$10$ & $110$ & $70$ & $25$ & \cite{Lee2006} & \tabularnewline
\noalign{\vskip\doublerulesep}
$10$ & $\sim175$ & $\sim175$ &  & \cite{Baujard-Lamotte2008} & Saturation value inferred\tabularnewline
\noalign{\vskip\doublerulesep}
$1$ & $135$ (13F) & $137$ & $14$ & This work & \tabularnewline
\noalign{\vskip\doublerulesep}
$1$ & $20$(\ce{CH3}) & $20$ & $10$ & \cite{Keselowsky2003} & \tabularnewline
\noalign{\vskip\doublerulesep}
$1$ & $20$(\ce{CH3}) &  & $5$ & \cite{Capadona2003} & \tabularnewline
\noalign{\vskip\doublerulesep}
$1$ & $30$ & $25$ & $10$ & \cite{Michael2003} & \tabularnewline
\noalign{\vskip\doublerulesep}
$1$ & $10$ & $10$ & $10$ & \cite{Lee2006} & \tabularnewline
\noalign{\vskip\doublerulesep}
\end{tabular}

\caption{\label{tab:FN saturation values}Saturation surface density of adsorbed
fibronectin ($ng/cm^{2})$ from this work and previous studies reported
in the literature.}
%
\end{table}
The saturation surface concentration of FN, as measured by the WGM
system, was compared to previously published results for solution
concentrations of $1\,\mu g/ml$ and $10\,\mu g/ml$ as shown in Table
\ref{tab:FN saturation values}. The WGM measurements compare favorably
with previously published results for a solution concentration of
$10\,\mu g/ml$. Although different measurement techniques and surface
preparations were used, the surface concentrations on hydrophobic
and hydrophilic surfaces were fairly similar among the various references
(reference \cite{Lee2006} being the only exception). The saturation
surface concentration of FN on hydrophilic neutral surfaces measured
by the WGM sensor also indicated excellent agreement with previously
published results. However, the amount of adsorbed protein measured
by the WGM system at lower concentrations was significantly greater
than the amount reported in previously published results for hydrophobic
and hydrophilic charged surfaces. In contrast, the surface concentration
value for a neutral hydrophobic surface was in good agreement with
previous results. The discrepancy between the WGM sensor results and
the other methods at low concentrations may be explained by differences
in the measurement system, the surface chemistry, or the adsorption
process. Since the limiting surface coverages measured by the WGM
sensor agree well with other techniques for $10\,\mu g/ml$ and for
neutral hydrophilic surfaces at $1\,\mu g/ml$, the results from the
WGM sensor can be considered reliable. It is likely that if systematic
errors were inherent to the WGM method, those errors would be reflected
throughout all solution concentrations measured. 

One possible interpretation of the higher saturation values measured
at $1\,\mu g/ml$ and lower on 13F and DETA could be the relative
packing order of silane monolayers compared to alkanethiol monolayers.
Alkanethiol SAMs are known to create highly ordered monolayers due
to their tight packing on highly ordered gold films \cite{Prime1991}.
Because of this tight packing only the terminal functional groups
of the alkanethiol are presented at the surface, resulting in highly
defined surface chemistries. Alkylsilane monolayers, which are formed
on silica surfaces, are less tightly packed and therefore form less
ordered monolayers, which can potentially present more than just the
terminal functional group. It has been hypothesized that this may
result from interaction of electron donating groups of the silane
side chain with silanol groups at the surface resulting in reaction
site-limited substrates \cite{Stenger1992}. This can lead to incomplete
monolayers that allow interaction of protein with the unreacted substrate
or allow sufficient degrees of freedom for the silane side-chains
to adopt multiple conformations, creating less ordered monolayers
that can rearrange to accommodate the native protein structure. At
the high concentrations the amount of protein available to bind would
swamp out these effects but they would be present at the lower concentrations.
This could explain why the differences between the silane chemistry
used in this work and the alkanethiol chemistry used in previous work
\cite{Keselowsky2003} did not show up at $10\,\mu g/ml$. Thus, this
additional degree of freedom would allow the long side chains to rearrange
to accommodate greater protein interaction for structural stabilization
and higher coverage or to expose new surface sites for increased protein
adsorption. This would not be possible with the tightly packed alkane
thiol monolayers. This effect could also have major consequences for
protein function, as described later. 

Another significant difference between the WGM measurement system
and previous work was that the protein was deposited on the WGM resonator
under flow conditions and the measurement was continued until saturation
was reached, while previous measurements were made after exposure
to a static FN solution for a fixed amount of time (30-60 minutes.)
The combination of high-affinity surfaces and low solution concentration
is conducive to transport-limited adsorption, which could explain
the discrepancy between WGM and static experiments for hydrophobic
and charged hydrophilic surfaces. In contrast, neutral hydrophilic
surfaces have a much lower affinity for protein, so depletion of protein
near the surface would be much less of a factor, resulting in good
agreement between the WGM and static measurements. 


\subsubsection{Fibronectin on DETA and 13F}

The fitted parameters of the RSA model were very similar for FN adsorption
on DETA and 13F surfaces, reflecting the similar shapes of the experimental
curves. The RSA model fitted the DETA data well, indicating that the
assumptions of the RSA model were valid for the process of adsorption
on DETA. This result was confirmed by the fitting results for the
two-stage adsorption model. The mean squared error for the model with
transition was only slightly lower than the error for the fitted RSA
model. We conclude that fibronectin adsorbs on DETA with a well-defined
footprint, which does not change significantly after adsorption. This
result is consistent with the well-established theory that proteins
generally experience less denaturation on a hydrophilic surface than
on a hydrophobic surface \cite{Latour2005}. The experimental results
for FN on 13F showed significant deviations from the RSA model in
the saturation region, especially at higher solution concentrations.
The two-stage model allows particles to change size after adsorption,
which significantly improved the fit of the model to the data for
FN on 13F. The fitted values for the association constant were quite
similar for DETA and 13F, but the transition rate constant $k_{s}$
for the 13F surface was an order of magnitude larger than $k_{s}$
for the DETA surface. The results from the fitting process indicated
that FN denatured after adsorption on 13F, which had been previously
postulated for certain hydrophobic surfaces \cite{Lan2005,Sivaraman2009}. 


\subsubsection{Fibronectin on SiPEG}

The RSA model fitted the SiPEG data well. The association rate constant
for FN adsorption on SiPEG was lower than the values for DETA and
13F while the dissociation rate constant was higher, which is expected
for a protein-resistant surface. This result is consistent with findings
that SiPEG is an electrostatically neutral surface that does not exhibit
coulombic attraction for proteins in solution. Surprisingly, the fitted
radius of FN adsorbed on SiPEG was more than twice the fitted radius
of FN adsorbed on DETA or 13F. For the two-stage model, the transition
rate constant for adsorption on SiPEG was significantly higher than
for the other surfaces. The fitted pre-transition radius and post-transition
radius of adsorbed FN were also larger for SiPEG than DETA or 13F.
The large radius predicted by the RSA model and the significant transition
predicted by the two-stage model seemed to indicate that FN denatures
after it adsorbs to PEG. This prediction was not consistent with the
well-known observation that proteins in contact with hydrophobic surfaces
tend to denature, while proteins in contact with hydrophilic, charged
surfaces tend to retain their native conformations. However, it also
may indicate that the SiPEG surface could be promoting the denaturation
of adsorbed proteins, which could explain why it is a cell-resistant
surface despite being hydrophilic. 

Although the SSE of the fitted two-stage model was about 30\% lower
than the SSE for the RSA model, the absolute change in SSE was relatively
small, and may not be significant. It is possible that the two extra
variable parameters (transition rate constant and post-transition
radius) are redundant for the SiPEG surface, in which case their fitted
values should not be considered significant. It is also possible that
the radius predicted by the fitting process for SiPEG is an artifact
caused by fitting the data with a model that is not well suited to
the surface chemistry. Given the assumptions of the RSA model, surface
coverage can reach saturation in only two ways: either the rate of
desorption equals the rate of adsorption, or there is no space left
on the surface for another protein to adsorb. The second case may
not apply to an adsorption-resistant surface like SiPEG. However,
combinations of parameters and that fitted the initial adsorption
kinetics did not predict the low saturation level of protein observed
in our experiments. One possible explanation is that FN adsorbed to
a small number of defects in the SiPEG monolayer, which could explain
both the rapid initial adsorption and the small amount of adsorbed
protein when the surface is saturated. If this were the case, a site-limited
adsorption model like the Langmuir model may be better for modeling
adsorption on SiPEG. Our prototype instrument did not have the sensitivity
to perform a more thorough study of adsorption on SiPEG at low solution
concentrations. Future systems based on whispering gallery mode technology
have the potential to study the adsorption of proteins on SiPEG surfaces
in greater detail, which could lead to greater understanding as to
why SiPEG resists protein adsorption. 


\subsubsection{Cell Growth and Survival on FN-Coated Alkylsilane Monolayers}

Both the embryonic hippocampal neurons and myocytes showed significantly
better survival on DETA surfaces than 13F surfaces. However, measurements
with the WGM biosensor indicated that the amount of adsorbed protein
measured on the 13F surfaces was comparable to that of DETA, indicating
that the conformation of adsorbed FN, and its function, was just as
important as the quantity of FN for cell survival. This is consistent
with the postulate made above that the silane monolayers are able
to rearrange to accommodate more protein and that the DETA surface,
being charged and hydrophilic, could accommodate the functional conformation
of the FN so little or no denaturation would occur. Conversely, the
hydrophobic side chains of the 13F could rearrange to allow for the
adsorption of more protein but would also promote the exposure of
the protein's hydrophobic core, thus denaturing the protein and deactivating
its biological activity as postulated previously for hydrophobic surfaces
\cite{Keselowsky2003}. 

Results from the skeletal myocyte culture (Table~\ref{tab:Cell counts on FN})
provided further information about the bioactivity of absorbed FN.
Skeletal myocytes are precursor cells that fuse and differentiate
into contractile myotubes. This differentiation is mediated by, among
other factors, the interaction of the $\alpha5\beta1$ integrin receptors
on the surface of the myocytes with the cell binding domain of the
FN molecule \cite{Michael2003}. Without this interaction, myotubes
do not form. The muscle cell culture on 13F indicated that while a
significant number of cells survived, no myotubes formed. The number
of dead cells was actually less than that of SiPEG or DETA, and the
fact that so many cells survived on the 13F substrate indicates that
there was enough protein adsorbed to the surface to promote adhesion.
However, the lack of myotube formation indicates that FN adsorbed
on 13F had reduced biological activity due to its denaturation and
did not activate the $\alpha5\beta1$ integrin signaling pathways
necessary for myotube differentiation. These proliferation and differentiation
results are consistent with previously reported results \cite{Michael2003}.
The lack of survival of cells on SiPEG surfaces can be attributed
to the small amount of adsorbed FN and the possibility that the protein
was also denatured. 


\subsection{Glucose Oxidase Adsorption}

The highest saturation surface density of adsorbed protein was found
on the hydrophobic 13F surface, and the lowest was found on the hydrophilic
SiPEG surface. Intermediate amounts of protein adsorbed on the hydrophilic
charged surfaces, glass and DETA. These results are consistent with
the general consensus of previous protein adsorption studies~\cite{Rabe2010}.
The initial kinetics of adsorption were slowest on the SiPEG surface,
followed by glass. The 13F surface had the highest rate of adsorption
at $100\,\mu g/ml$, but the initial rate of adsorption on the DETA
surface was slightly higher at $10\,\mu g/ml$.

In a previous study, glucose oxidase was adsorbed onto plasma-polymerized
thin films of hexamethyldisiloxane (HDMS) on silica substrates~\cite{Muguruma2006}.
HDMS is a hydrophobic polymer, and its surface properties were modified
by exposure to nitrogen and oxygen plasma, as shown in Table \ref{tab:Surface Properties}.
Adsorption was measured with a quartz crystal microbalance (QCM),
although only the saturation values were quantified. The size and
shape of adsorbed GO was also measured by AFM. It was assumed that
GO in solution could be approximated by an ellipsoid with a major
axis of $10-14\, nm$ and a minor axis of $6-8\, nm$. It was found
that GO adsorbed to native HDMS surfaces with the major axis parallel
to the surface, while GO adsorbed to HDMS-O and HDMS-N with the major
axis normal the surface. The maximum area occupied by an adsorbed
molecule (its ''footprint'') would be $88\, nm^{2}$ when adsorbed
with major axis parallel to surface, while the minimum area would
be $28\, nm^{2}$ when adsorbed with the major axis normal to surface.
Previous AFM studies of GO adsorbed on gold also found that it could
be represented by an ellipse with a major axis of $14-18\, nm$ and
a minor axis of $5-8\, nm$~\cite{Quinto1998}. Estimates of protein
size from crystal structure data or AFM measurements on gold surfaces
should be considered {}``order of magnitude'' estimates, since the
size of proteins is heavily dependent upon their environment. 

%
\begin{table}
\caption{\label{tab:Surface Properties}Properties of surfaces used in these
experiments and similar surfaces used in previous experiments.}
\begin{tabular}{lccc}
 & Water contact angle & Zeta potential ($mV$) & Reference\tabularnewline[\doublerulesep]
\cline{2-4} 
\noalign{\vskip\doublerulesep}
Glass & $<5^{\circ}$ & $-25$ & \cite{Kirby2004,Wilson2011a}\tabularnewline
\noalign{\vskip\doublerulesep}
DETA & $49\pm2^{\circ}$ & $10$ & \cite{Wilson2011a,Metwalli2006}\tabularnewline
\noalign{\vskip\doublerulesep}
13F & $94\pm2^{\circ}$ & $-15$ & \cite{Stenger1992,Tandon2008}\tabularnewline
\noalign{\vskip\doublerulesep}
SiPEG & $38\pm2^{\circ}$ &  & \cite{Wilson2011a}\tabularnewline
\noalign{\vskip\doublerulesep}
HDMS & $>90^{\circ}$ &  & \cite{Muguruma2006}\tabularnewline
\noalign{\vskip\doublerulesep}
HDMS-N & $<50^{\circ}$ & $20$ & \cite{Muguruma2006}\tabularnewline
\noalign{\vskip\doublerulesep}
HDMS-O & Hydrophilic & $-40$ & \cite{Muguruma2006}\tabularnewline
\noalign{\vskip\doublerulesep}
\end{tabular}%
\end{table}



\subsubsection{Glucose Oxidase on Glass}

The Langmuir-type models fitted the kinetics of GO on glass slightly
better than the RSA-type models, and the two-stage models fitted the
data better than the single-stage models. The enzyme activity data
indicated that GO adsorbed on bare glass retained about 25\% of its
activity at a solution concentration of $10\,\mu g/ml$. When the
solution concentration was increased to $100\,\mu g/ml$, almost all
of the adsorbed enzyme remained active. Both of the two-stage models
predicted the amount of active enzyme correctly at $10\,\mu g/ml$,
but under-predicted the amount of active enzyme at $100\,\mu g/ml$
with the assumption that enzyme in state $\alpha$ was active and
enzyme in state $\beta$ was inactive. If it was assumed that enzyme
in both states remains active, both of the two-stage models would
accurately predict the amount of active enzyme at $100\,\mu g/ml$.
It has been shown that some {}``soft'' enzymes lose most of their
structure and activity when adsorbed to glass, while {}``hard''
enzymes retain their structure and function after adsorption~\cite{Zoungrana1997,Welzel2002}.
An alternative hypothesis is that an initial layer of enzyme adsorbed
to the surface in a highly spread-out state, resulting in a very low
surface density and low activity, followed by a second layer of adsorbed
enzyme that retained its activity. This could explain the low fraction
of active enzyme at $10\,\mu g/ml$, and the high fraction of active
enzyme at $100\,\mu g/ml$. Modeling this process would require a
model that incorporated both a post-adsorption transition and multi-layer
adsorption.


\subsubsection{Glucose Oxidase on DETA}

The model fits to the kinetic data for GO on DETA were not as good
as the fits to the kinetic data for GO on glass. The RSA-type model
with a post-adsorption transition provided the best fit to the kinetic
data, as determined by the sum of squared error. It was assumed that
the DETA surface did not induce proteins to denature upon adsorption,
and this assumption was consistent with the data from the enzyme activity
assay at $10\,\mu g/ml$. When the solution concentration was increased
to $100\,\mu g/ml$, approximately half of the adsorbed GO apparently
lost its activity. It is unlikely that this reduction was caused by
denaturation, since adsorbed proteins are more likely to retain their
native conformation when the rate of adsorption is higher~\cite{Latour2005}.
Further, the fitted parameter values for the two-stage RSA-type model
were fairly implausible. The fitted transition rate constant was two
orders of magnitude higher than the transition rate constant for the
glass surface, and the surface area occupied by a molecule in state
$\beta$ was 16 times larger than a molecule in state $\alpha$. A
more likely hypothesis for the 50\% reduction in activity is that
a second layer of adsorbed enzyme formed, preventing the substrate
solution from reaching the lower layer of enzyme. The two-layer model
provided the best prediction of the amount of active adsorbed enzyme
at both concentrations, although it did not fit the kinetic data quite
as well as the RSA-type two-stage model. 


\subsubsection{Glucose Oxidase on 13F}

The kinetic data for GO on 13F was fitted well by both the Langmuir-type
model with a post-adsorption transition and the two-layer model. Although
the SSE was slightly lower for the two-layer model, the Langmuir-type
model provided a much better prediction of the amount of active adsorbed
enzyme. The experimental data showed that almost none of the adsorbed
enzyme was active at a solution concentration of $10\,\mu g/ml$.
When the solution concentration was increased to $100\,\mu g/ml$,
about 75\% of the adsorbed enzyme retained its activity. GO that adsorbs
from a higher solution concentration may experience less denaturation
because it is sterically hindered from spreading out on the surface
by the high density of adsorbed enzyme molecules. Surprisingly, the
fitted parameters for the Langmuir-type model with a post-adsorption
transition indicated the area occupied by a molecule in the final
state was not much larger than that of a molecule in the initial state.


\subsubsection{Glucose Oxidase on SiPEG}

Despite its sensitivity, the WGM sensor was unable to detect the adsorption
of GO on an SiPEG surface reliably at a solution concentration of
$10\,\mu g/ml$. At $100\,\mu g/ml$, GO adsorbed on the SiPEG surface,
but at a relatively low surface density. The enzyme activity assay
did not detect any significant amount of active enzyme for either
solution concentration. This may be because the amount of activity
was below the detection limit of the assay, or because the enzyme
did not retain its activity on SiPEG. It is well known that SiPEG
surfaces resist protein adsorption, but denaturation on SiPEG has
not been previously reported. Further investigation is necessary to
answer this question. Both the RSA and Langmuir single-layer models
fitted this data set well.


\subsection{Fitting Kinetic Models to Experimental Data}

Fitting different models to the experimental data enabled us to test
hypotheses about the mechanisms of adsorption on various surfaces.
Each model was based upon different assumptions about the actual behavior
of the protein at the surface. Fitting each model to the experimental
data and evaluating the quality of fit provided a quantitative methodology
to compare hypotheses and determine which assumptions were most realistic
for a particular combination of protein and surface. The fitted parameter
values also provided critical information about the validity of each
model. It is possible that a good fit can be achieved with an unrealistic
parameter value or combination of parameter values (such as a negative
value for a kinetic constant or radius). Therefore, it is important
to understand the physical process that is being modeled and review
the fitted parameter values to ensure their validity. The least-squares
fitting algorithm used in this work did not incorporate constraints
on parameter values. Many algorithms are available for constrained
optimization~\cite{Gill1981,Jones2001-}. When several models fitted
the data equally well, the quantitative enzyme activity data for adsorbed
glucose oxidase provided a valuable test to support or reject a particular
model. 

It was assumed that the kinetic constants did not vary across the
limited concentration range in this study. Therefore, a single set
of parameters was fitted to multiple concentrations for a single surface.
Fitting more concentrations, while holding the number of fitted parameters
constant, increased the possibility of finding a unique combination
of parameters that minimized the mean-squared error. A model with
too many parameters can have multiple parameter sets with equivalent
optimal fits, much like an under-determined system of linear equations
can have multiple solutions. Although a lower SSE could have been
achieved by fitting data at each concentration individually, the likelihood
of finding non-unique parameters would have increased.

Finding a unique combination of parameters that provides a {}``best
fit'' to the data is equivalent to finding a global minimum in an
optimization problem. While this is not too difficult for the basic
Langmuir and RSA models (which have only three parameters each), it
becomes more challenging for the multi-stage and multi-layer models
(with five parameters each). It is easy to formulate even more complex
adsorption models, but the addition of each parameter increases the
difficulty of finding a global minimum in the error function. The
danger of a very complex model with many variable parameters is that
it can be fitted to almost any data set but few useful conclusions
can be drawn from the results! If models with more parameters must
be used, it will be necessary to find additional experimental methods
that can provide additional data or decrease the size of the search
space. It may also be necessary to use novel nonlinear optimization
methods, such as genetic algorithms, to optimize the fit of the model
to the data.


\subsection{Conclusions}

The combination of simulations and experiments described in this chapter
allowed protein adsorption to be studied in a way that would not be
possible with either technique by itself. Utilization of the WGM sensor
enabled, for the first time, quantitative analysis of protein adsorption
on silane monolayers, which are commonly used as substrates for cell
culture. The sensitivity of this technology can be readily enhanced
by a number of methods, such as fabricating smaller microspheres and
using a laser with a shorter wavelength \cite{Vollmer2008a} or coating
the glass microsphere with a high-index wave-guiding layer \cite{Teraoka2006}.
WGM biosensors have the potential to help answer difficult questions
in biomaterials research. For example, PEG/OEG monolayers are resistant
to protein adsorption but have not been shown to be useful as long-term
biocompatible coatings for implants and medical devices. The field
of bioengineering would benefit from the development of a surface
coating that reduces or eliminates the immune response to a foreign
body, such as an implanted medical device.

The work described here combined well-known modeling techniques in
a novel way to achieve a detailed understanding of protein adsorption
on silane surfaces. Computational fluid dynamics simulations were
used to design and characterize the flow cell of the WGM biosensor,
while most experimental studies have used simple approximate models
of transport limitations or neglected them altogether. The results
of the CFD simulations were incorporated into the process of fitting
kinetic models to fibronectin adsorption data (this step was unnecessary
for glucose oxidase.) Four existing adsorption models and one new
model were formulated in terms of the same variables and fitted to
the experimental data, and the quality of fit and fitted parameter
values were compared. It is common to fit a single model to experimental
data. Fitting five models and comparing the results resulted in a
thorough and comprehensive study of protein adsorption on surfaces
used for cell culture.



\chapter{CFD SIMULATION OF TRANSPORT IN A MICROFLUIDIC BIOREACTOR}


\section{Introduction}

The principles of transport and adsorption described in the previous
chapters were applied to the design of a microreactor that models
the function of an alveolus. Alveoli are a micro-scale structures
in the lung that facilitate the exchange of oxygen and carbon dioxide
between the blood and inhaled air. Each alveolus is typically $200-250\,\mu m$
in diameter, and the 300-480 million alveoli in the lungs of an adult
human provide a total surface area of $50-100\, m^{2}$ for gas exchange~\cite{Levitsky2007}.
The alveolar membrane consists of a single layer of epithelial cells
in contact with the air and a single layer of endothelial cells that
forms a capillary for blood flow. Development of an \emph{in vitro}
model of the alveolus-capillary interface would be important to biomedical
science, as this barrier tissue is an point of entry for toxins and
plays a role in compromised gas exchange in disease states. A recent
study found that only 11\% of the drugs that passed preclinical trials
eventually became approved for use by either U.S. or European regulatory
agencies, and that the major causes of drugs failing clinical trials
were lack of efficacy and toxicity \cite{Kola2004}. The majority
of these failures occurred in the expensive late phases of clinical
trials. With the potential to overcome the failures of traditional
cell-based drug assays in predicting physiological behavior, body-on-a-chip
devices have become an increasing area of interest in pharmaceutical
testing \cite{Sung2010}. Body-on-a-chip devices have been created
based on pharmacokinetic models in attempts to supplement traditional
cell culture methods that are not always capable of modeling tissue-tissue
interactions that can occur when the body is exposed to pharmaceuticals
\cite{Esch2011}. 

Creating an \emph{in vitro} model of the alveolus poses unique challenges
due to the gas-liquid environment, layers of disparate functional
cell types, and the transport of nutrients, cells, and dissolved gas
species through the tissue. One lung-tissue based device was produced
out of poly(dimethyl siloxane) (PDMS) to recreate the structure, mechanics,
and certain aspects of transport that occur in lung tissue \cite{Huh2010}.
The elastomeric PDMS material used in the device was highly permeable
to both oxygen and carbon dioxide, which precluded the ability to
create physiologically relevant gas concentrations in the liquid phase
of the environment. Additionally, the high gas diffusion through the
PDMS prevented measurement of transport of these gas species through
the tissue, a fundamental function of the alveolus-capillary interface
in the lung. To overcome these limitations silicon based microfluidic
technology was utilized to create an \emph{in vitro} body-on-a-chip
model of the lung with a long-term goal of quantifying the rate of
gas transport across the model alveolar membrane. To replicate the
physiological function of the alveolus, liquid that enters the model
alveolus with a gas composition similar to that of venous blood should
exit the chamber with a gas composition similar to that of arterial
blood. To be able to detect small changes in the rate of gas exchange
across the model alveolar membrane, the alveolar chamber should be
designed such that the gas composition of the liquid reaches arterial
values just before the liquid exits the chamber. 

After leaving the alveolar chamber, the liquid will pass through another
chamber containing dissolved-gas sensors. These sensors require several
minutes to register small changes in gas levels, so the system was
designed to be operated in either continuous or stop-flow mode. The
microfluidics must be designed so that a {}``plug'' of fluid moves
from the alveolar chamber into the sensor chamber with relatively
little mixing in the the longitudinal direction. The residence time
distribution of an ideal plug-flow system is a Dirac delta function,
so the width of the residence time distribution can be used to characterize
how similar a system is to plug flow. Each chamber must be designed
so that liquid flowing through different parts of the chamber experiences
a similar residence time within the chamber. This objective is equivalent
to minimizing the width of the residence time distribution, or approximating
plug flow.

Microreactors have recently been employed for synthetic organic chemistry,
nanoparticle synthesis, and ultra-fast DNA amplification \cite{DeMello2006}.
However, the combination of the no-slip boundary condition and laminar
flow in microfluidic devices produces flow patterns that deviate substantially
from the desired plug flow, presenting a challenge to the design of
continuous-flow microreactors \cite{Hartman2011}. Several innovative
methods have been developed to create plug flow in a microreactor,
such as creating a multiphase flow with dispersed droplets or {}``slugs''
of water containing reagents in a stream of oil \cite{Song2003} or
gas \cite{Guenther2005}. For single-phase flows, a passive microfluidic
structure known as a herringbone mixer can be added to a microfluidic
channel to reduce the width of the residence time distribution to
approximate plug flow \cite{Cantu-Perez2010}. Ideally, combinations
of simulation based design and experimental validation of the results
would be the best approach for successful system development. However,
general methods for minimizing the residence time distribution in
a continuous-flow microreactor by a combination of simulation and
experiment has not been reported. 

In experimental attempts to optimize microfluidic systems, the primary
method employed for visualizing flow patterns in the devices has involved
fluorescent or colored dyes \cite{Carlotto2011,Ryu2011,Campbell2004,Hessel2005}.
Visualization experiments have focused mainly on the mixing of two
components, though this method has been used to a limited extent to
measure residence times in certain devices \cite{Hornung2009}. Both
qualitative visual representations of flow and quantitative measurement
of concentrations can be obtained using dyes, though quantitative
concentration measurements are less common than qualitative representations.
Theoretical studies have been performed to predict the residence time
distributions of various types of microfluidic structures. Analytical
models have been used to characterize the hydrodynamic dispersion
in microchannels \cite{Dutta2006}. However, these methods are limited
to devices with simple, well-defined geometries. Computational fluid
dynamics (CFD) simulations must be used to simulate practical microfluidic
devices with more complex geometries. Studies have shown that CFD
simulations can predict convective and diffusive transport in microfluidic
systems with a high level of accuracy \cite{Znidarsic-Plazl2007}.
Recently, CFD simulations have been used to predict the residence
time distribution of a microfluidic mixing device by applying a pulse
of a {}``tracer'' at the inlet of the device and monitoring the
average concentration at the outlet \cite{Adeosun2009}. CFD has also
been used in conjunction with a particle tracking simulation to model
the residence time distribution of a microfluidic channel containing
a mixing structure \cite{Cantu-Perez2010}. Although CFD simulations
have been used to analyze the residence time distribution of microfluidic
devices, the use of CFD to optimize these devices has not been widely
reported.

Protein adsorption also plays an important role in the microfluidic
model of the alveolus. A PDMS membrane was chosen as the substrate
for culturing monolayers of epithelial and endothelial cells because
of its high permeability to oxygen and carbon dioxide. PDMS is regarded
as a poor substrate for cell culture, so it is standard practice to
treat the polymer with an oxygen plama and adsorb fibronectin onto
the membrane prior to tissue culture~\cite{Powers2010}. PDMS can
be easily molded to create intricate shapes. This capability will
eventually be utililized to create porous membranes that will allow
nutrients, chemical signals and even macrophages to pass between the
epithelial and endothelial cell layers. 

The general field of microreactor design would benefit from a new
methodology combining experimental and simulation-based development.
This capability would be particularly germane to biomedical engineering
of new systems for drug discovery and toxicology. CFD simulations
were used as part of an iterative design process to quantify the convective
and diffusive transport of dissolved gas in the liquid side of the
alveolus and achieve the best possible approximation to plug flow.
The flexibility of the design process was demonstrated by the fabrication
of an alveolar chamber that could accommodate a standard tissue culture
support and a polygonal chamber for more general applications. The
CFD predictions for different design iterations were validated with
quantitative and qualitative experimental measurements using dye visualization.
The design techniques presented here can be utilized to optimize the
performance of microreactors for any application.


\section{Materials and Methods}


\subsection{Device Fabrication }

Silicon was chosen as the substrate for the microfluidic device due
to its impermeability to gas and the ability to incorporate microstructures
using standard microfabrication techniques, and to allow future incorporation
of other tissue compartments. A planar design was chosen to avoid
drilling or etching through the silicon substrate. Devices were fabricated
using silicon wafers with a thickness of $500\mu m$. Wafers were
liquid primed with P20 and spun with Shipley Microposit S1818 photoresist
at 3000 rpm for 60 seconds, followed by a softbake at $115^{\circ}C$
for 60 seconds on a hotplate. Exposure was performed on an ABM contact
aligner and the photoresist was developed with 726MIF developer. Silicon
wafers were etched with a Unaxis 770 deep reactive ion etcher (DRIE)
to a depth of $125\mu m$. The remaining photoresist was stripped,
and the devices were separated with a dicing saw. Devices that incorporated
all three chambers on the same chip as well as isolated chambers for
each component were fabricated and tested. 

Housings were machined from acrylic plates to produce the outside
portions of the microfluidic device. In the bottom section, a rectangular
area was milled to approximately 0.8 mm deep into which a sheet of
PDMS elastomer with nominal thickness of 0.6 mm was placed. The silicon
device was placed on top of the elastomer sheet so that the sheet
pushed the silicon device into the flat surface of the housing top
to produce a tight seal. Inlet and outlet holes were drilled into
the housing top to direct fluid into the inlet of the silicon device
and allow the liquid to escape from the outlet. The two housing pieces
each had a series of threaded screw holes and were held together with
twelve screws. The combined gasket and silicon device thickness was
larger than the depth of the rectangular recess on the bottom housing
piece so that the screws could be tightened to produce a seal between
the top housing piece and the silicon device. 


\subsection{CFD Simulation Methodology}

The CFD-ACE+ suite of simulation tools was used to predict the performance
of the device. CFD-GEOM was used to define the geometry and create
a regular mesh throughout the liquid volume. The velocity derivatives
were approximated with upwind differencing and a 2nd order limiter
was used to approximate the derivatives for computing concentration.
The system was simulated in steady-state with pure water to determine
the flow field (it was assumed that dilute solutions of small molecules
would not have a significant impact on the viscosity of the water.)
A transient simulation of gas transport was then performed, using
the flow field from the steady-state simulation to avoid re-calculating
the flow field at each time step. Initially, no dissolved gas was
present in the water. To characterize the residence time distribution
of each design, the inlet boundary condition introduced water saturated
with carbon dioxide (mass fraction of 0.03366) at a volumetric flow
rate of $35\,\mu l\, min^{-1}$, approximating a step function at
the inlet of the device at t=0. The results of this simulation protocol
should also be valid for dissolved oxygen, which has a diffusion coefficient
similar to carbon dioxide ($2\times10^{-9}m^{2}s^{-1}$ for oxygen
\cite{Jamnongwong2010} and $1.7\times10^{-9}m^{2}s^{-1}$ for carbon
dioxide \cite{Tamimi1994}). The Euler method was used for time stepping,
and the time step was calculated automatically using the Auto Time
Step feature in CFD-ACE-GUI. The minimum time step was set to $100\,\mu s$.
The time step was automatically adjusted based upon the rate of change
of solution variables. The typical maximum time step was 1 ms, and
the simulation results were saved at regular time intervals. The results
were visualized using CFD-VIEW. To quantify the average concentration
across the inlet or outlet channel at a specific point in time a cut
plane was created across the channel and the CFD-VIEW calculator was
used to average the concentration across the plane. The residence
time distribution for the device was calculated by taking the derivative
with respect to time of the normalized concentration at the outlet. 

Once the device was optimized to minimize the width of the residence
time distribution for dissolved gases in water, another set of simulations
was performed to predict the transport of dye for comparison with
visualization experiments. The diffusion coefficient of Brilliant
Blue FCF dye has not been reported, so it was estimated based on known
diffusion coefficients of similar dye molecules. The molar masses
and diffusion coefficients of Fluorescein ($332g\, mol^{-1}$), Rhodamine
6G ($471g\, mol^{-1}$), Congo Red ($697g\, mol^{-1}$), and Trypan
Blue ($961g\, mol^{-1}$) were plotted on a scatter plot \cite{Nakagaki1950,Inglesby2001,Culbertson2002}.
A roughly linear relationship was observed between molar mass and
diffusion coefficient. Based on this relationship, the diffusion coefficient
of Brilliant Blue FCF was estimated to be approximately $2\times10^{-10}m^{2}s^{-1}$.


\subsection{Design Methodology}

The chamber dimensions and internal structures were optimized by successive
computational fluid dynamics modeling of the system. The iterative
design process revolved around minimizing the width of the residence
time distribution. A CFD simulation was performed for each new design
iteration and the residence time distribution was calculated to evaluate
that iteration. The alveolar chamber was designed to accommodate a
circular cell culture membrane, such as a Transwell insert (Corning
Life Sciences), with a diameter of $6.5\, mm$. Baffles in the shape
of circular arcs were added to the chamber to support the membrane
and reduce the width of the residence time distribution. The baffles
divide the incoming liquid into several streams such that the velocity
of the liquid in each stream is roughly proportional to the length
of the path it takes through the chamber. Specifically, the baffles
were designed so that liquid following the longest path around the
perimeter of the chamber would have the same residence time as liquid
that traversed the shortest path through the center of the chamber.
The arc length of each baffle was kept as short as possible, since
the no-slip boundary condition on the baffles contributed to axial
dispersion within each channel. The outer baffle was designed to be
long enough to prevent a stagnation zone from forming near the outlet
of the chamber. The inner baffles were made progressively shorter.
A small gap was added near the end of each baffle to prevent stagnation
on the channel inside of the baffle. 

The design used for the bubble trap and conditioning chamber was chosen
to be a rectangle with tapering inlet and outlet sections. To allow
the device to operate in a stop-flow mode, these chambers were designed
to have the same volume as the alveolar chamber. The angle of the
taper and the chamber width were adjusted to minimize the width of
the residence time distribution while maintaining a volume equal to
that of the circular chamber. A staggered array of posts was then
created in each chamber to modify the flow characteristics. The vertical
spacing between posts and the number of columns of posts were adjusted
in the CFD model until the flow through the chamber was as close as
possible to plug flow. Because the chamber wall reduced the velocity
of fluid flowing between the top row of posts and the chamber wall,
the spacing between the top row of posts and the chamber wall had
to be larger than the spacing between posts to avoid introducing a
stagnation zone. 


\subsection{Dye Visualization Experiments }

Blue food dye, containing Brilliant Blue FCF (FD\&C Blue 1) dye, was
used for visualizing the flow of liquid through the microfluidic device.
The concentration of dye in a phosphate buffered saline (PBS) solution
used in the experiments was selected based on preliminary data of
light intensity captured by the camera versus dilution of the dye
solution. The solution concentration was chosen to maximize sensitivity
to changes in concentration throughout the range of fully dilute to
non-diluted from the initial concentration. For the initial solution
concentration chosen, the intensity versus concentration profile was
approximately linear, though this relationship would not hold if a
higher concentration of dye was chosen. Visualization experiments
were performed by pumping the dye solution into the devices initially
containing 1X PBS with no dye and visualizing the movement of the
blue dye solution in the chambers. The dye was pumped at a rate of
$35\,\mu l\, min^{-1}$ using a syringe pump. The individual chamber
designs (polygonal chamber and round alveolar chamber) were tested
for dye visualization both with and without posts or baffles to evaluate
the effect of the simulation-driven baffle or post structures on the
flow pattern. Additionally, the alveolar chamber in the final combined
device\nobreakdash-incorporating two polygonal chambers and one alveolar
chamber\nobreakdash-was tested to assess the flow pattern in the
circular chamber in the practical device. The flow patterns in the
various devices were also used for comparison to the CFD predictions. 

Dimensional analysis was utilized to understand how the device would
perform with dissolved gas, which has a diffusion coefficient that
is roughly an order of magnitude higher than that of the dye. The
P�clet number is the dimensionless number that represents the ratio
of convective transport to diffusive transport in a fluidic system.
The P�clet number in a channel is given by $Pe=LU/D$, where $L$
is the characteristic dimension of the channel, $U$ is the average
velocity in the channel, and $D$ is the diffusion coefficient of
the solute. For a rectangular channel, the hydraulic diameter is used
as the characteristic dimension. Using the average velocity in the
channels between the chambers calculated by the CFD simulation, the
P�clet number for dissolved gas with a diffusion coefficient of was
computed to be 770 for a volumetric flow rate of $35\,\mu l\, min^{-1}$.
Because the diffusion coefficient of the dye is one order of magnitude
lower than that of dissolved gas, experiments with dye were performed
with a volumetric flow rate of $3.5\,\mu l\, min^{-1}$ to obtain
a P�clet number comparable to that of dissolved gas. Experiments were
also performed with a flow rate of $350\,\mu l\, min^{-1}$ to characterize
the performance of the system over a range of two orders of magnitude. 


\subsection{Image Analysis }

Images and video of the flowing liquid in the round alveolar chamber
were captured using a Hitachi KP-20a CCD microscope video camera attached
to a Zeiss Stemi DV4 stereo microscope. Because the blue dye absorbs
light primarily in the red region of the visual spectrum, the red
channel of the captured images was isolated and used for image analysis
using ImageJ \cite{Rasband1997-2011}. Video analysis was performed
for flow experiments in which a solution of dye was pumped into the
device. Video was obtained at 5 frames per second from the time dilute
dye could be observed near the inlet of the chamber of interest until
the entire chamber had been filled with dye for several seconds. The
average red channel intensity was measured across a line of pixels
at the inlet of the chamber and at the outlet of the chamber for each
video frame. For each frame, these average intensities for the inlet
and outlet were normalized to four points outside of the chamber unexposed
to dye to account for any variations in overall light intensity. For
both the inlet and outlet, the average red channel intensity of the
first ten frames (corresponding to 0\% saturation) and the last ten
frames (corresponding to 100\% saturation) were computed. The average
concentration across the inlet and outlet for each frame was calculated
by linearly interpolating the average intensity between the 0\% and
100\% saturation intensities found from the beginning and end of the
experiment. The reported concentrations represent the average of three
experiments under the same conditions, with the curves aligned in
time to the points in which the respective inlet concentration reached
15\% saturation. 


\section{Results}

%
\begin{figure}
\subfloat[\label{fig:RT alveolus}Alveolar chamber]{\includegraphics[width=3in]{Bioreactor/Figures/RT_alveolus}}\subfloat[\label{fig:RT Conditioner}Conditioner]{\includegraphics[width=3in]{Bioreactor/Figures/RT_conditioner}

}

\subfloat[\label{fig:RT 3 chambers}Alveolus on full chip]{\includegraphics[width=3in]{Bioreactor/Figures/RT_3chambers}

}

\caption{\label{fig:Residence time dist CFD}Residence time distributions predicted
by CFD for carbon dioxide-saturated water.}
%
\end{figure}
The design goal was to obtain the best possible approximation to plug
flow so that a bolus of liquid would flow from one chamber into the
next with minimal mixing between the chambers. The residence time
distributions predicted by the CFD simulation for the initial and
final designs of the round alveolar chamber and the polygonal chamber
are shown in Figure~\ref{fig:RT alveolus} and Figure~\ref{fig:RT Conditioner},
respectively. Figure~\ref{fig:RT 3 chambers} shows the predicted
combined residence time distribution for first two chambers of the
device (bubble trap and conditioning chamber) and for the first three
chambers of the device (bubble trap, conditioning chamber, and alveolus).


\subsection{Experimental Validation of CFD Simulations}

%
\begin{figure}
\includegraphics[width=4in]{Bioreactor/Figures/Microreactor}

\caption{\label{fig:Microreactor}Microreactor design~(a), fabricated silicon
chip~(b), and chip in acrylic housing~(c)}
%
\end{figure}
The overall layout of the device is shown in Figure~\ref{fig:Microreactor}a.
The microfluidic device fabricated in silicon is shown in Figure~\ref{fig:Microreactor}b.
The silicon device in the acrylic housing is shown in Figure~\ref{fig:Microreactor}c.
The first chamber after the inlet is a bubble trap \cite{Sung2009},
followed by a chamber that will eventually be used to condition the
liquid to contain physiologically relevant concentrations of dissolved
gas. Another bubble trap is located downstream (to the right) of the
alveolar chamber to prevent bubbles from accumulating in the sensor
chambers, followed by two chambers that will eventually house dissolved-gas
sensors. For additional chip designs were fabricated (not shown) to
test each chamber design in isolation: the alveolar chamber with and
without baffles and the conditioning chamber with and without posts.

%
\begin{figure}
\subfloat[\label{fig:Alveolus no baffles}Without baffles]{\includegraphics[width=3in]{Bioreactor/Figures/Alveolus_without_baffles}

}\subfloat[\label{fig:Alveolus with baffles}With baffles]{\includegraphics[width=3in]{Bioreactor/Figures/Alveolus_with_baffles}

}

\caption{\label{fig:Alveolus CFD vs Dye}Visualization of the flow in the alveolus
predicted by CFD (left columns) and imaged with dye (right columns)}
%
\end{figure}
The flow pattern inside an isolated alveolar chamber with and without
baffles is shown in Figure~\ref{fig:Alveolus CFD vs Dye}, including
both CFD results and experimental visualization. %
\begin{figure}
\subfloat[\label{fig:Conditioner without posts}Without posts]{\includegraphics[width=3in]{Bioreactor/Figures/Conditioner_without_baffles}

}\subfloat[\label{fig:Conditioner with posts}With posts]{\includegraphics[width=3in]{Bioreactor/Figures/Conditioner_with_baffles}

}

\caption{\label{fig:Conditioner CFD Dye}Visualization of the flow in the conditioning
chamber predicted by CFD (left columns) and imaged with dye (right
columns) }
%
\end{figure}
 Figure~\ref{fig:Conditioner CFD Dye} indicates the amount of dye
present in different locations within the isolated conditioning chamber/bubble
trap at various times. The flow patterns for the conditioning chambers
with and without the post arrangement are shown along with the flow
patterns expected from the CFD results. The visual scale of the CFD
results was normalized so that pure water was represented as white
and the maximum dye concentration was black. Images from the visualization
experiments of the alveolar chamber in the combined three chamber
device are compared to the images from the isolated alveolar chamber
and CFD results in Figure~\ref{fig:3 Chambers CFD Dye}. %
\begin{figure}
\includegraphics[width=3in]{Bioreactor/Figures/3_Chambers_baffles}

\caption{\label{fig:3 Chambers CFD Dye}Visualization of the flow in the alveolar
chamber in the full chip predicted by CFD (left column) and imaged
with dye (right column)}


%
\end{figure}


Figure~\ref{fig:Alveolus outlet} indicates the dye concentration
at the outlet of the alveolar chamber as a function of time when a
concentration step function was applied at the chamber inlet. The
outlet concentrations for the conditioning chamber designs, as analyzed
from the video captures, are shown in Figure~\ref{fig:Conditioner outlet},
along with the concentrations expected from the CFD results for these
chambers. Figure~\ref{fig:3 chambers outlet} includes the outlet
concentration of the alveolar chamber in the three chamber design
and the CFD modeling of this device. Dotted lines indicate plus or
minus one standard deviation from the experimental average, while
a dashed line indicates the concentration predicted by CFD. Figure~\ref{fig:Outlet 3 flow rates}
indicates the concentration at the outlet of the first three chambers
for flow rates equal to one tenth and ten times the nominal flow rate.
%
\begin{figure}
\includegraphics{Bioreactor/Figures/Outlet_alveolus}

\caption{\label{fig:Alveolus outlet}Saturation at the outlet of the alveolus}


%
\end{figure}
%
\begin{figure}
\includegraphics{Bioreactor/Figures/Outlet_conditioner}

\caption{\label{fig:Conditioner outlet}Saturation at the outlet of the conditioning
chamber}
%
\end{figure}
%
\begin{figure}
\includegraphics{Bioreactor/Figures/Outlet_3chambers}

\caption{\label{fig:3 chambers outlet}Saturation at the outlet of the alveolus
on the full chip}
%
\end{figure}
%
\begin{figure}
\includegraphics{Bioreactor/Figures/Outlet_three_flow_rates}

\caption{\label{fig:Outlet 3 flow rates}Relative concentration at the outlet
of the alveolus in the full chip for three different flow rates}


%
\end{figure}



\section{Discussion}

The residence time distributions in Figures \ref{fig:RT alveolus}
and \ref{fig:RT Conditioner} indicate that the design optimization
process clearly reduced the width of the residence time distribution
for each chamber, improving the approximation to plug flow. The residence
time distribution of the round chamber without baffles exhibited a
long {}``tail'' which was caused by the stagnation zones that formed
near the outlet of the chamber (Figure \ref{fig:Alveolus no baffles}).
The addition of baffles forced liquid from the inlet to flow around
the outer perimeter of the chamber, greatly reducing these stagnation
zones and suppressing the {}``tail'' of the residence time distribution.
The residence time distribution of the polygonal chamber without posts
was narrower than the residence time distribution of the round chamber
without baffles. Without the constraint imposed by the circular tissue
culture membrane, the shape of the polygonal chamber could be optimized
for plug-like flow without additional structures. However, the addition
of posts further narrowed the residence time distribution. Figure
\ref{fig:RT 3 chambers} demonstrates that each successive chamber
further widens the residence time distribution so that a plug of fluid
becomes more disperse as it progresses through the device. 


\subsection{Dye Visualization Experiments}

The experimental flow patterns confirmed the CFD results that introducing
baffles into the alveolar chamber reduced stagnation zones to produce
a more plug-flow like flow pattern. At early time points, the baffles
prevented a stream of dyed liquid in the center from speeding toward
the outlet, as shown in time points $3.5\, s$ and $6.5\, s$. Especially
at the later time points, the baffles reduced liquid stagnation near
the outer walls of the chamber. Stagnation zones in the alveolar chamber
without baffles were clearly visible at $11.5\, s$, and stagnant
liquid remained at $16.5\, s$ immediately adjacent to the walls.
In contrast, the liquid near the walls of the alveolar chamber with
baffles was much more uniform in concentration at $11.5\, s$, and
the entire chamber had a nearly uniform concentration at $16.5\, s$.
The baffle design clearly improved the flow patterns in the round
alveolar chamber, though further improvements may be possible. The
streaks of higher and lower concentration liquid, which are prominent
at $6.5\, s$, indicated that the flow pattern on the exit side of
the baffles was not completely uniform. It should be noted that the
chambers were optimized for the transport of gases that have diffusion
coefficients an order of magnitude higher than the diffusion coefficient
of the dye used to visualize the flow patterns. Therefore, the stagnation
zones observed in the dye injection experiments are likely to be much
smaller when dissolved gas is used in place of dye.

The dye visualization of the flow patterns in the isolated conditioning
chamber (Figure \ref{fig:Conditioner CFD Dye}) confirmed that the
post layout improves the uniformity of the flow pattern in the conditioner,
like the baffles improved the flow pattern in the alveolar chamber.
Specifically, at $6.5\, s$ and $9.5\, s$ after the introduction
of dye into the inlet the condition chamber without posts exhibited
large regions of stagnant liquid near the chamber walls, represented
by lighter color. In contrast, the chamber with posts did not experience
these stagnant zones. The CFD simulation predicted that the chamber
without posts should be nearly completely filled with dyed liquid
by $6.5\, s$, while experiments showed that filling the chamber required
somewhat more time. 

In Figure \ref{fig:3 Chambers CFD Dye}, the flow pattern for the
liquid in the alveolus of the full device followed the CFD results
as well as matched the trends of the isolated alveolar chamber with
baffles. The dispersion of dye in the inlet of the alveolus due to
the previous two chambers can be observed in Figure \ref{fig:3 Chambers CFD Dye}.
Because of the non-uniform concentration of dye across the inlet,
the alveolar chamber in the full device requires more time to reach
a uniform concentration compared to the single alveolar chamber. This
trend can also be seen in the images from the CFD simulation. 

The images of the flow patterns (Figures \ref{fig:Alveolus CFD vs Dye}-\ref{fig:3 Chambers CFD Dye})
include information about the distribution of the outlet concentrations
over the width of the chambers. In the chambers without posts or baffles,
there was variation in concentration across the width of the outlet
before full saturation was reached. The liquid near the walls had
a lower concentration of dye than the liquid in the center of the
channel, due to the laminar flow within the chamber and the channels.
In both Figures \ref{fig:Alveolus CFD vs Dye} and \ref{fig:Conditioner CFD Dye},
the posts or baffles greatly reduced this variation across the channel
width, indicating more plug-like flow.

The results from the quantitative evaluation of the outlet profiles
in the isolated alveolar chamber (Figure \ref{fig:Alveolus outlet})
further confirmed the results from the CFD simulation. In both the
CFD and in the experimental visualization, the start of the rise in
concentration of the liquid at the outlet was delayed by the presence
of the baffles. Experimentally, the outlet concentrations reached
saturation at approximately the same time with and without baffles.
The delay in the onset of the increase in outlet concentration coupled
with the fact that the two outlets reach saturation at nearly the
same points indicated that the liquid in the alveolar chamber with
baffles exhibited a narrower distribution of residence times.

The concentration profiles for the conditioning chambers with and
without posts (Figure \ref{fig:Conditioner outlet}) reinforced the
conclusion that the post layout improved the flow patterns to produce
more plug-like flow. The outlet concentrations in both chambers began
to rise at approximately the same time, but the two profiles began
to deviate at approximately the half-saturation point. In this chamber,
the post layout improved the flow pattern by reducing the stagnation
zones, creating a steeper concentration profile at the outlet. Toward
the later times, the outlet concentration profile for the chamber
without posts was delayed by approximately 3 seconds relative to the
profile of the chamber with posts.

The concentration profile at the outlet of the alveolar chamber in
the full device (Figure \ref{fig:3 chambers outlet}) was actually
steeper than the CFD results, indicating a narrower residence time
distribution within the entire chip than was predicted. The inlet
concentration for the alveolus was influenced by the previous two
chambers, so the inlet concentration for the alveolar chamber was
not a step function for either the CFD or the experimental results.
The influence of all three chambers on the flow pattern was incorporated
into both the CFD and experimental results for the three-chamber design.
The fact that the experimental concentration profile at the outlet
of the three-chamber design was only slightly shifted from the isolated
alveolar chamber indicates the entire design was quite effective at
approximating plug flow. 

The full device was imaged and the concentrations at the outlet of
the alveolar chamber were analyzed using flow rates covering three
orders of magnitude to vary the ratio of convective to diffusive mass
transfer (i.e. different P�clet numbers) (Figure \ref{fig:Outlet 3 flow rates}).
As expected, faster flow rates produced less steep concentration profiles
in the normalized time scale, since the higher convective transport
rates limited the amount of diffusion that occurred during the normalized
time period. The slowest flow rate ($3.5\,\mu l\, min^{-1}$) simulates
the outlet concentration profile that would be produced using dissolved
\ce{CO2} (with approximately one order of magnitude higher diffusion
coefficient than dye) at a flow rate of $35\,\mu l\, min^{-1}$, since
the ratio of convective transfer to diffusive transfer is about the
same for both conditions. This condition (\ce{CO2} dissolved in a
liquid flowing at $35\,\mu l\, min^{-1}$) is appropriate for a lung-on-a-chip
device of similar geometry. Thus, the different flow rate outlet profiles
can be used to estimate the profiles that would be produced for other
diffusion coefficients. 

When comparing the original designs to the optimized designs, the
improvements shown by the experimental measurements closely matched
the improvements predicted by the simulations. These results demonstrated
that the CFD-based design process was effective at reducing the number
of experimental iterations required. The slight deviation of the simulation
results from the experimental results was likely due to differences
in inlet concentration profiles. The inlet concentration was modeled
as a step function in the CFD simulations. In the experiments, the
concentration at the inlet of the chamber was not a perfect step function.
The dye was injected through a channel in the device housing which
was not included in the CFD model. In the functional lung device the
conditioning chamber will fix the concentration of dissolved gas in
the liquid before it enters the alveolus, so any dispersion introduced
by the channel before the conditioning chamber will not be significant. 


%% LyX 1.6.10 created this file.  For more info, see http://www.lyx.org/.
%% Do not edit unless you really know what you are doing.
\documentclass[letterpaper,english,PhD]{UCFthesis}
\usepackage[T1]{fontenc}
\usepackage[latin9]{inputenc}
\setcounter{secnumdepth}{3}
\setcounter{tocdepth}{3}
\usepackage{amsmath}
\usepackage{amssymb}

\makeatletter

%%%%%%%%%%%%%%%%%%%%%%%%%%%%%% LyX specific LaTeX commands.
\pdfpageheight\paperheight
\pdfpagewidth\paperwidth


\makeatother

\usepackage{babel}

\begin{document}

\chapter{DISCUSSION}

We have constructed a bench-top WGM biosensor system integrated with
a flow cell to study the kinetics of protein adsorption on functionalized
glass surfaces relevant for tissue culture. Assembly of the table-top
biosensing setup does not require any specialized equipment, and the
device components (DFB laser, photodetector, optical fiber, simple
microfluidics) are inexpensive as compared to commercial systems and
can be readily assembled in any laboratory environment. Despite its
simplicity, the sensitivity of the setup rivals that of a state-of-the
art surface plasmon resonance sensor (\textasciitilde{} 1 pg/mm2 mass
loading), with the additional advantage that silica microspheres can
be easily fabricated and conveniently modified with various silane
surface coatings by exploiting established silanol surface chemistries.
This is an alternative to other methods, such as SPR, that require
coating the sensor with gold and are mostly limited to surface modification
with thiols. It also allowed for the first time the detailed kinetic
analysis of protein adsorption on silane monolayers and enabled some
explanation of the differences between silane and thiol surface modifications.
The WGM sensor was used to obtain the most comprehensive set of kinetic
data that has been reported for the adsorption of fibronectin at low
solution concentrations. Measuring adsorption kinetics at multiple
concentrations allowed a single set of kinetic constants to be fitted
for multiple concentrations, which increased the likelihood of obtaining
a unique set of fitted parameters. The results of the model fitting
and the cell culture experiments indicate that similar amounts of
FN adsorb on hydrophobic surfaces and charged hydrophilic surfaces.
When combined with antibody data from other studies, our data supports
the conclusion that FN denatures after adsorption on hydrophobic surfaces,
leading to a loss of biological activity. Thus, it demonstrates that
a much more sophisticated analysis of how protein adsorption can affect
cellular response to surfaces can be undertaken with this system.
Further improvements to the sensitivity will help answer difficult
questions in biomaterials research, such as improving our understanding
of cell-surface interactions and surfaces that resist protein adsorption.

A design optimization process has been presented that can be used
to optimize the design of a continuous-flow or stop-flow microreactor
so that it operates more like an ideal plug-flow reactor. Plug-like
flow patterns can minimize undesired mixing between compartments and
non-homogeneous concentration profiles across a channel. The results
from the three chamber device demonstrate that the two types of chambers
can be used as building blocks for devices with multiple chambers.
While each chamber contributes to the widening of the residence time
distribution, the optimized post and baffle designs can effectively
be used to improve overall flow patterns in a multitude of devices
of similar geometry. The use of computational fluid dynamics simulations
enabled the optimization of the device by allowing many design iterations
to be performed without the expense and delay of fabricating a prototype
at each iteration. The CFD simulations were validated qualitatively
and quantitatively with dye visualization experiments in prototype
devices. This combination of simulation and experiments was applied
to the design of microfluidic chambers for a bioreactor that models
the alveolus of the lung, demonstrating how the optimization process
can enhance the performance of body-on-a-chip systems for drug and
toxicity studies. 
\end{document}


\newpage \phantomsection \label{listofref} \addcontentsline{toc}{chapter}{LIST OF REFERENCES}

\bibliographystyle{ieeetr}
\bibliography{Dissertation_References}

\end{document}
