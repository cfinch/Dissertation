
\chapter{MODELING THE KINETICS OF PROTEIN ADSORPTION}


\section{Introduction}

Many studies to date have focused on the adsorption of proteins to
alkanethiol self-assembled monolayers (SAMs). Alkanethiol SAMs are
used to functionalize noble metal surfaces (typically gold). These
SAMs are convenient in that they are relatively easy to prepare, present
highly ordered monolayers with well-defined composition, and are compatible
with integrated electrodes and other sensor systems utilizing metal-coated
surfaces, such as surface plasmon resonance (SPR) sensors. In-depth
discussions of alkanethiol SAMS are easily found in the literature
\cite{Love2005}. Less attention, however, has been given to alkylsilane
monolayers, which are used to functionalize glass or silica surfaces.
This may be because they lack the highly ordered packing formed by
alkanethiol SAMs (resulting in less well defined surfaces), or a perceived
difficulty in the preparation of well-characterized alkylsilane surfaces.
This is somewhat unfortunate, as alkylsilanes represent a broad and
useful class of compounds that are used in an increasing variety of
biomedical and biotechnological applications. 

Alkylsilane monolayers are used to modify the surface chemistry of
glass and silica surfaces to control the adhesion of proteins and
cells. Laser ablation has been used to pattern alkylsilane surfaces
to create cytophobic and cytophilic regions that direct the attachment
and growth of cells~\cite{Stenger1992}. \IUPAC{(3-trimethoxy\|silyl\|propyl)diethyl\|triamine}
(DETA) is used as a cytophilic cell culture substrate. \IUPAC{1,1,2,2-perfluoro\|octyl\|trichloro\|silane}
(13F) is a hydrophobic perfluorinated SAM that has been used to define
cytophobic regions. More recently, tethered chains of polyethylene
glycol (SiPEG) have been used as cytophobic SAMs in place of 13F~\cite{Wilson2011a}.
Surfaces modified with PEG, which is also known as \IUPAC{oligo(ethylene glycol)}
(OEG) or \IUPAC{polyethylene oxide} (PEO), have been extensively
studied because of their resistance to protein adsorption \cite{Gombotz1991}.

A biosensor system utilizing whispering gallery mode (WGM) technology,
where the active sensor is typically a silica disk, ring, toroid,
or sphere/spheroid, can provide new insights into the adsorption behavior
of biomolecules onto alkylsilane-modified surfaces. Glass and silicon
oxide surfaces are much more common than metal surfaces in cell culture
applications, so silane surface modification is of greater practical
importance than thiol surface modification for tissue engineering.
To fully understand the behaviour of cells and tissue constructs,
it is critical to understand the underlying processes that govern
the adsorption of biomolecules to silica substrates and the alkylsilane
coatings used to functionalize them.


\subsection{Whispering Gallery Mode Biosensing}

WGM biosensing is based on monitoring the frequency shift of an optical
resonance excited inside a dielectric resonator \cite{Vollmer2005,Vollmer2002}.
In our implementation, near-infrared light is evanescently coupled
to a glass microsphere with a radius of 50-200 \textgreek{m}m from
a tapered optical fiber, which is connected to a tunable distributed
feedback (DFB) laser at one end and a photodetector at the other.
The laser and detector are used to precisely monitor changes in the
resonant wavelength of the microsphere. As proteins or other material
accrete at the surface of the microsphere, the effective radius of
the sphere increases, resulting in a red shift of the resonant wavelength
that can be quantified and used to calculate the average surface density
of adsorbed material. Even with a simple experimental configuration
\cite{Vollmer2002} it has been shown that a detection limit of $\sim1\, pg/mm^{2}$
can be readily achieved. This is ten times more sensitive than an
SPR biosensor and theoretical calculations predict the ultimate detection
limit of the method to be close to the single molecule level \cite{Armani2003,Vollmer2008,Vollmer2008a,Arnold2003}.
These qualities make WGM biosensing an ideal method for studying protein
adsorption, as the dynamic range of the method allows measurements
to be performed in concentration regimes that previously were unattainable.
To date, optical resonators of this kind have been applied to a variety
of biosensing applications with great effect. In addition to the inherent
sensitivity of the method, standard CMOS technology can be applied
to fabricate arrays of resonators on silicon wafers that provide a
scalable multiplex sensing capability for detecting multiple biological
or chemical markers from a single sample in parallel \cite{Luchansky2010,Washburn2009}.
Furthermore it has been shown that these measurements can be performed
in complex samples such a blood plasma and serum \cite{Luchansky2011}.
This provides an added level of complexity as {}``real-world'' samples
such a plasma and serum contain a mixture of hundreds of proteins,
which may non-specifically bind to a sensor giving inaccurate readings
or false positive measurements. 


\subsection{Fibronectin}

The unique capabilities of the WGM sensor were utilized to quantify
the kinetics of the adsorption of fibronectin (FN) onto alkylsilane
surfaces. Cell studies were then done to evaluate the biological activity
of the adsorbed FN \cite{Keselowsky2004}. Fibronectin is an important
protein in the extracellular matrix (ECM) that mediates the interaction
of cells with surfaces, but its activity has been shown to be influenced
by its surface structure \cite{Lan2005,Michael2003}. Since the adsorption
of FN has been widely studied, the results from the WGM instrument
could be compared to an extensive amount of data. Fibronectin (FN)
is a physiologically important protein in vertebrates. It is abundant
in plasma and other bodily fluids, and plays an important role in
the extracellular matrix. The structure of fibronectin is complex.
The primary structure of FN is a chain composed of three types of
repeated modules. Although only one gene codes for FN, alternative
splicing of the pre-mRNA results in numerous variants in which modules
are added or deleted \cite{Pankov2002}. X-ray crystallography has
been used to determine the secondary and tertiary structure of the
three types of modules. The complete FN molecule has not been crystallized,
so its secondary and tertiary structure are unknown. It is likely
that the secondary and tertiary structure are highly dependent on
the local environment. In solution, fibronectin exists as a dimer
with two identical subunits linked by disulphide bonds. In the extracellular
matrix, fibronectin is assembled into a fibrillar network \cite{Mao2005}.


\subsection{Glucose Oxidase}

The WGM biosensor was also used to quantify the kinetics of adsorption
of glucose oxidase (GO) onto alkylsilane surfaces. The activity of
the adsorbed enzyme was measured to provide more information about
its conformation on the surface. Glucose oxidase is an important and
useful enzyme from a technological standpoint. GO which has been adsorbed
or covalently attached to electrodes forms the basis for amperometric
glucose sensors \cite{Wang2007}. Advances in portable blood glucose
sensors have enabled diabetics to monitor and control their blood
glucose levels, minimizing the risk of complications from the disease
\cite{Oliver2009}. Perhaps because of its important role in biosensors,
glucose oxidase has been thoroughly characterized. This makes it an
ideal candidate for testing and validating the accuracy of the WGM
biosensor system. 

GO is a dimeric glycoprotein that is composed of two identical subunits~\cite{Wohlfahrt1999}.
The crystal structure of GO from \emph{Aspergillus niger} is available
at the Protein Data Bank (1CF3), and its bounding box is $6.0\, nm$
x $6.2\, nm$ x $7.7\, nm$. Reported values for the molecular mass
of GO range from $152\, kDa$~\cite{Keilin1948} to $186\, kDa$~\cite{Swoboda1965},
depending upon the method of purification that was used. The isoelectric
point of GO is 4.2~\cite{Pazur1964}. Many previous studies have
focused on the effect of different surface chemistries on the structure
and activity of adsorbed glucose oxidase~\cite{Fears2009,Dong1997}.
Atomic force microscopy (AFM) studies have been performed to determine
the size and shape of GO adsorbed on various surfaces~\cite{Muguruma2006,Otsuka2004}.
Although the immobilization of glucose oxidase has been widely studied,
only qualitative results have recently been reported for the kinetics
of adsorption~\cite{Muguruma2006}.


\subsection{Fitting Kinetic Models to Protein Adsorption Data}

As mentioned in the previous chapter, kinetic models have been formulated
as boundary conditions for use in computational fluid dynamics (CFD)
simulations of the transport and adsorption of proteins~\cite{Glaser1993,Edwards1999}.
Kinetic models have also been fitted to experimental measurements
of adsorption kinetics. An RSA-type model with a simple approximation
of transport limitations was fitted to kinetic data from an OWLS adsorption
sensor~\cite{Kurrat1994}. More recently, a model of adsorption with
a post-adsorption transition \cite{Lundstroem1984} was fitted to
a comprehensive set of kinetic data from a surface plasmon resonance
sensor~\cite{Michael2003}. The concentration near the surface was
assumed to be constant. These two approaches were combined in a study
in which a Langmuir-type kinetic model coupled to a CFD simulation
was fitted to experimental measurements of adsorption kinetics in
microcapillaries~\cite{Jenkins2004}. This study was unique in that
the entire CFD model, including transport and adsorption, was included
in the fitting procedure.


\subsection{Overview}

A novel WGM biosensor was constructed and used to quantitatively study
the kinetics of adsorption of GO and FN at varying concentrations
onto alkylsilane monolayers presenting well-defined surface chemistries:
DETA, 13F, and SiPEG. To determine the biological activity of the
adsorbed FN, neuronal and skeletal muscle cells were cultured on the
modified surfaces in a serum-free culture system \cite{Brewer1995,Das2007}.
The kinetics of adsorption of glucose oxidase were also measured on
the silane monolayers and bare glass with the WGM biosensor, and its
enzymatic activity on each surface was determined with a standard
assay kit. The WGM biosensor incorporated a flow cell which minimized
the effect of transport limitations on protein adsorption. This, along
with the inherent sensitivity of the method, allowed the kinetics
of adsorption of FN to be measured at concentrations lower than those
that have previously been reported \cite{Michael2003}. Multiple kinetic
models of protein adsorption were fitted to the measured kinetic curves,
and the resulting parameters were used to draw conclusions about the
mechanisms of adsorption. To maximize the accuracy of the fitted kinetic
parameters, computational fluid dynamics simulations were used to
quantify the limitations in the transport of protein to the sensor
surface. The results demonstrate that the combination of WGM biosensing,
CFD, and kinetic models of adsorption provides a unique capability
to quantify protein adsorption on silane-modified surfaces for the
purpose of understanding the interactions between tissues and tailored
interfaces.


\section{Methods and Materials}


\subsection{Experimental Methods}

A whispering gallery mode sensor system was constructed as described
in \cite{Wilson2011,Wilson2009}. A schematic overview of the system
is shown in Figure \ref{fig:WGM System Diagram}.%
\begin{figure}
\includegraphics{Protein_Adsorption/Figures/WGM_System_Overview}

\caption{\label{fig:WGM System Diagram}Schematic diagram of the whispering
gallery mode biosensor}


%
\end{figure}
 After the assembly of the flow cell, PBS solution was flowed through
the tubing and flow cell until the system reached thermal equilibrium.
Protein solution at the appropriate concentration was flowed through
the system with a peristaltic pump at a volumetric flow rate of $150\,\mu l/hr$.
LabVIEW (National Instruments, Austin, TX) was used to control the
sweep of the laser wavelength and acquire data. The data acquisition
software tracked the location of each resonant valley in the acquired
spectrum using a peak fitting algorithm. All valleys with a FWHM (full
width at half maximum) value below a certain threshold were tracked,
and the position of each valley minimum was determined by fitting
a Bessel function. The position of each resonance over time was saved
to a binary file to be analyzed later.

Data analysis software was written using the Python programming language.
The binary file created by LabVIEW for each experiment was loaded
into the software and the spectral location ($nm$) of each resonance
was reconstructed over time from the raw data. One resonance, with
a continuous trace and the lowest FWHM value, was chosen for further
analysis. A linear baseline subtraction was applied to correct for
baseline drift. A method based on first order perturbation theory
\cite{Vollmer2002,Arnold2003} was used to calculate the surface concentration
of the adsorbed species $\sigma_{s}$ based on the measured change
in resonant wavelength $\Delta\lambda$:\begin{equation}
\frac{\Delta\lambda}{\lambda}=\frac{\alpha_{ex}\sigma_{s}}{\epsilon_{0}\left(n_{s}^{2}-n_{m}^{2}\right)R}\label{eq:WGM data analysis}\end{equation}
$\lambda$ is the nominal wavelength of the resonance, $\Delta\lambda$
is the wavelength shift of the resonance, $n_{s}$ is the refractive
index of the spheroid (1.46), $n_{m}$ is the refractive index of
the medium surrounding the sphere (1.3357), $\alpha_{ex}$ is the
excess polarizability of the protein molecule ($0.184\, cm^{3}g^{-1}$),
$\epsilon_{0}$ is the permittivity of free space and $R$ is the
radius of the spheroid. Spheroid radii were measured from images taken
using brightfield microscopy. Two runs for each concentration were
averaged for the DETA and 13F surfaces, and a single run was used
for each concentration on the SiPEG surface.


\subsubsection{Surface Preparation}

A single mode optical fiber with an acrylate polymer coating, $9\,\mu m$
core and $125\,\mu m$ cladding (SMF-28e+, Corning Inc., Corning,
NY) was used to fabricate the resonators \cite{Vollmer2002}. The
acrylate coating was first removed using a fiber optic stripper and
the stripped region was cleaned with isopropyl alcohol (iPA) to remove
any residual acrylate. The end of the stripped fiber was then placed
in the flame of a nitrous-butane Microflame torch (Azuremoon trading
company, Cordova, TN). A nitrous-butane flame was used due to the
very high temperatures needed to melt the glass and form the resonator
($>700^{\circ}C$). The tip of the fiber was placed in the flame until
the glass glowed bright white and began to melt. The surface tension
of the molten glass caused it to form into a spheroidal droplet. As
the tip melted, the fiber was rotated to ensure that the resonator
remained centered on the stalk of the fiber. Resonator radii used
for these studies ranged from $125\,\mu m$ to $175\,\mu m$.

Glass resonators and glass cover slips (for control measurements)
were modified with silane surface chemistry to achieve the desired
surface properties. The glass was first treated with an oxygen plasma
for 20 minutes. All silane solutions were prepared at a concentration
of 0.1\% (vol:vol) in distilled toluene in an MBraun glove box (Stratham,
NH) under anhydrous, low oxygen conditions. Storing and preparing
solutions in this way prevented solution phase polymerization of the
silane as nascent water vapor and atmospheric oxygen can react with
the monomer. Preparation of 13F surfaces was performed in the glove
box due to the extreme reactivity of the monomer. Microspheres were
immersed in the 0.1\% \IUPAC{1,1,2,2-perfluoro\|octyl\|trichloro\|silane}
solution for 30 minutes. 5 minutes prior to completion of the reaction,
the beaker was removed from the glove box and placed in a chemical
fume hood. Both the SiPEG and DETA surface modifications were performed
in a chemical fume hood. The SiPEG coating was achieved by starting
with a 0.1\% solution (vol:vol) of \IUPAC{2-[Methoxy\|poly(ethyleneoxy)propyl]trimethoxy\|silane}
(Gelest, Tullytown, PA) and adding concentrated HCl to create a 0.08\%
HCl solution (vol:vol). The resonators and cover slips were immersed
in the resultant solution for 1 hour. DETA coated substrates were
immersed in a 0.1\% solution (vol:vol) of \IUPAC{(3-trimethoxy\|silyl\|propyl)diethyl\|triamine}
(Gelest) in toluene, gently heated to $65^{\circ}C$ over 30 minutes
and then allowed to cool for 15-20 minutes. Substrates were rinsed
three times with fresh toluene and heated again to $65^{\circ}C$
in fresh toluene for 30 minutes. Upon completion of the reactions,
the microspheres and cover slips were washed three times in dry toluene
and stored in a desiccator until needed. Control cover slips were
analyzed by XPS and contact angle goniometry, and the results were
consistent with previously published results \cite{Wilson2011a}.


\subsubsection{Fibronectin}

The kinetics of FN adsorption were measured with the WGM biosensor
as described in \cite{Wilson2011}. The experiments will be briefly
summarized here. Bovine plasma fibronectin in solution (F1141, Sigma-Aldrich,
St. Louis, MO) was diluted with 50mM phosphate buffered saline (PBS)
at pH 7.4. Concentrations of $10\,\mu g/ml$, $5\,\mu g/ml$, $1\,\mu g/ml$,
$0.5\,\mu g/ml$ and $0.25\,\mu g/ml$ were used for the experiments.
To determine the biological activity of the protein adsorbed on the
various silanes, cell culture experiments were performed on silane-coated
cover slips that had been treated with $1\,\mu g/ml$ of FN in PBS
(pH 7.4). Embryonic hippocampal neurons and skeletal myoblasts were
cultured on the SiPEG, DETA, and 13F surfaces in a serum-free culture
system \cite{Brewer1995,Das2007}. After plating, cultures were maintained
in a water-jacketed incubator at $37^{\circ}C$ (85\% relative humidity)
and 5\% \BPChem{CO\_2} for seven days. Phase-contrast microscopy
images were taken during the course of the culture to document the
morphology of the cells and, in the case of skeletal myoblasts, the
differentiation of the cells into functional myotubes. A live-dead
assay (Invitrogen, Carlsbad, CA) was performed at day 7 to determine
the amount of living versus dead cells on the cover slips.

Skeletal muscle was dissected from the hind limb thighs of a rat fetus
at embryonic day 18 (Charles River Laboratories, Wilmington, MA) as
previously described \cite{Wilson2011}. Purified myocytes were plated
at a density of 500-800 cells per square millimeter onto FN-coated
cover slips. Myocytes were allowed to attach for 1 hour after which
time 3 ml of culture medium (neurobasal media containing B-27 (Invitrogen,
Carlsbad, CA{]}), Glutamax (Invitrogen, Carlsbad, CA), and Penicillin/Streptavidin)
was added. Culture medium was exchanged every 4 days. 

Embryonic hippocampal neurons were prepared according to a previously
published protocol \cite{Wilson2011}. Rat pups at embryonic day 18
were dissected from timed pregnant rats and the hippocampi were isolated
from the embryonic brains. Isolated hippocampal neurons were resuspended
in culture medium (Neurobasal / B27 / Glutamax\textbackslash{}u2122
/ Antibiotic-antimycotic) and plated at a density of 75 cells per
square millimeter. After plating, 3 ml of culture media was added.
Half of the culture medium was changed every 3-4 days. 


\subsubsection{Glucose Oxidase}

Glucose oxidase from \emph{aspergillis niger} was obtained from Sigma-Aldrich
(G7141, St. Louis, MO) and diluted to $100\,\mu g/ml$ and $10\,\mu g/ml$
with PBS at pH 7.4. The kinetics of GO adsorption were measured with
the WGM biosensor using the same protocol as the FN experiments. For
the activity assay, glass resonators were placed in a 96-well plate
and incubated for two hours in the glucose oxidase solution. The resonators
were then rinsed three times with PBS, placed in a new 96-well plate
and allowed to soak for two hours in $100\,\mu l$ of PBS to remove
any reversibly-bound enzyme. The Amplex Red glucose oxidase activity
assay (Invitrogen, Carlsbad, CA) was used to test the activity of
the enzyme adsorbed to the resonator. $50\,\mu l$ of the PBS was
removed from each well and $50\,\mu l$ of Amplex Red solution was
added. A Synergy HT plate reader (Bio-Tek, Winooski, VT) was used
read the absorbance of each well at a wavelength of $530\, nm$. A
standard curve, created using known concentrations of GO, was used
to translate the absorbance measurements into activity units.


\subsection{Analysis of Transport in the WGM Biosensor}

The rate of adsorption on a surface may be limited by the rate at
which the chemistry of adsorption takes place, the rate of transport
to the surface, or both. Computational fluid dynamics was used to
model the transport of protein in the flow cell of the WGM biosensor.
Before beginning, the Reynolds and Knudsen numbers were calculated
confirm that flow in the device would be entirely laminar and that
conventional CFD simulation methods were appropriate. CFD simulations
were run with a CFD-ACE+ multiphysics solver (ESI Software, Huntsville,
AL) to determine the influence of transport on the adsorption of protein
on the resonator surface. For all CFD simulations, the size of the
mesh and the time step were refined until the simulation results did
not change significantly. Upwind differencing was used to approximate
velocity derivatives, a 2nd order limiter was used to approximate
concentration derivatives, and the Euler method was used for time
stepping in transient simulations. 

A multi-scale approach was used to obtain high-resolution results
while keeping the simulation run time reasonable. A model of the entire
flow cell, including the $40\, cm$ of tubing between the three-way
stopcock and the flow cell, was created using CFD-GEOM and discretized
using a structured mesh. The model is shown in Figure~\ref{fig:CFD flow cell}.%
\begin{figure}
\includegraphics{Protein_Adsorption/Figures/CFD_geometry_flow_cell}\includegraphics{Protein_Adsorption/Figures/CFD_geometry_flow_cell_mesh}

\caption{\label{fig:CFD flow cell}Model of the flow cell with tubing used
for the first stage of CFD simulations. A close-up of the mesh is
shown at right. The inlet tube has been truncated in these images.}


%
\end{figure}
 This model did not include the resonator and waveguide, as they have
a minimal effect on the overall flow in the channel. A steady-state
simulation was performed first to calculate the flow field. Transient
simulations were then performed in which the concentration at the
inlet of the tubing was suddenly increased from zero to the target
concentration. The concentration of protein was monitored at a point
in the flow cell at the location of the resonator, and the simulation
was run until the concentration at this point reached the target value.
It was assumed that the low concentrations of protein used in these
experiments did not affect the flow field significantly, so the transient
simulations utilized the flow field from the steady-state simulation
to reduce computation time. 

Two additional CFD models were created to simulate the transport of
protein near the resonator. A detailed three-dimensional model was
used to model the flow field around the resonator and the waveguide
as shown in Figure~\ref{fig:CFD 3D resonator}.%
\begin{figure}
\includegraphics{Protein_Adsorption/Figures/CFD_3D_mesh}

\caption{\label{fig:CFD 3D resonator}3D model of the resonator and waveguide
in the flow cell of the WGM biosensor.}


%
\end{figure}
 This model ran too slowly to be used for transient simulations of
adsorption, so a simplified model was created with a two-dimensional
axisymmetric geometry as shown in Figure~\ref{fig:CFD Mesh 2D}.%
\begin{figure}
\includegraphics{Protein_Adsorption/Figures/CFD_geometry_2D_mesh}

\includegraphics{Protein_Adsorption/Figures/CFD_geometry_2D_mesh_closeup}

\caption{\label{fig:CFD Mesh 2D}Overall mesh for the two-dimensional CFD model
and a close-up of the mesh on the resonator.}


%
\end{figure}
 Because the axisymmetric geometry models a channel with a circular
cross section while the actual channel has a square cross section,
the flow velocity at the inlet was adjusted so the velocity near the
resonator matched the results from the three-dimensional simulation.
Data from the simulation of the whole flow cell was used to set the
concentration over time at the inlet of the axisymmetric model. A
surface reaction was defined on the surface of the resonator using
the Langmuir model built in to the Biochemistry module of ACE+. The
purpose of this reaction was to deplete the protein near the resonator
at an appropriate rate, rather than to model the actual chemistry
of adsorption. An association rate constant of $1.44\times10^{6}\, L\, mol{}^{-1}s^{-1}$,
a dissociation rate constant of $8.9\times10^{-4}\, s^{-1}$, and
a maximum density of adsorption sites of $4.4\times10^{-9}\, mol\, m^{-2}$
were used to match the maximum adsorption rate that was observed in
the experimental data for FN on DETA and 13F. A steady-state simulation
was used to establish the flow field and a transient simulation was
used to determine the concentration of protein near the surface of
the resonator. The adsorption simulation mandated a fairly small time
step (on the order of $10^{-4}\, s$), so it was impractical to run
the simulation to equilibrium due to the large number of time steps
required. The system was simulated for 150-300 seconds (of simulated
time). It was assumed that the near-surface concentration increased
linearly from the end of the CFD simulation to the target concentration.


\subsection{Modeling the Kinetics of Protein Adsorption}

It is well established that single-layer adsorption models can be
stated in the general form\begin{equation}
\frac{d\theta}{dt}=k_{a}c\,\Phi\left(\theta\right)-k_{d}\theta\label{eq:Single Layer Kinetics}\end{equation}
where $\theta$ is the fraction of the surface covered by adsorbed
particles, $c$ is the concentration of protein in solution near the
surface, $k_{a}$ is the adsorption rate constant, and $k_{d}$ is
the desorption rate constant \cite{Andrade1986}. The function $\Phi$
represents the blocking effect of adsorbed particles. For the Langmuir
model, the blocking function is simply $\Phi\left(\theta\right)=1-\theta/\theta_{\infty}$.
The blocking function for the random sequential adsorption (RSA) model
is not available in analytic form. A widely used approximation for
the random sequential adsorption of spherical particles is\begin{equation}
\Phi\left(\theta\right)=\frac{\left(1-\bar{\theta}\right)^{3}}{1.0-0.812\bar{\,\theta}+0.2335\bar{\,\theta}^{2}+0.0845\bar{\,\theta}^{3}}\label{eq:RSA Blocking Function}\end{equation}
where $\bar{\theta}=\theta/\theta_{\infty}$ \cite{Schaaf1989}.


\subsubsection{RSA-Type Model of Adsorption with Transition}

Single-layer adsorption models cannot adequately predict the kinetics
of adsorption for many combinations of proteins and surfaces. More
complex models have been developed to account for these experimental
results. Because many surfaces cause proteins to denature upon adsorption,
it is common to model adsorption with a post-adsorption transition.
This process can be described by the kinetic equations\begin{align}
\frac{d\rho_{\alpha}}{dt} & =k_{a}c\,\Phi_{\alpha}-k_{s}\rho_{\alpha}\Psi_{\alpha\beta}-k_{d}\rho_{\alpha}\label{eq:Adsorption Transition Kinetics}\\
\frac{d\rho_{\beta}}{dt} & =k_{s}\rho_{\alpha}\Psi_{\alpha\beta}\label{eq:Adsorption Transition Kinetics 2}\end{align}
The form of the blocking functions depends upon the assumptions of
the model. When the particles are spherical with initial radius $R_{\alpha}$
and final radius $R_{\beta}$, scaled particle theory (SPT) can be
used to derive the following blocking functions \cite{Brusatori1999}:
\begin{align}
\Phi_{\alpha} & =\left(1-\theta\right)\exp\left[-\frac{2\left(\overline{\rho_{\alpha}}+\Sigma\overline{\rho_{\beta}}\right)}{1-\theta}-\frac{\overline{\rho_{\alpha}}+\overline{\rho_{\beta}}+\left(\Sigma-1\right)^{2}\overline{\rho_{\alpha}}\overline{\rho_{\beta}}}{\left(1-\theta\right)^{2}}\right]\label{eq:Phi_alpha}\\
\Psi_{\alpha\beta} & =\exp\left[-\frac{2\left(\Sigma-1\right)\left(\overline{\rho_{\alpha}}+\Sigma\overline{\rho_{\beta}}\right)}{1-\theta}-\frac{\left(\Sigma^{2}-1\right)\left[\overline{\rho_{\alpha}}+\overline{\rho_{\beta}}+\left(\Sigma-1\right)^{2}\overline{\rho_{\alpha}}\overline{\rho_{\beta}}\right]}{\left(1-\theta\right)^{2}}\right]\label{eq:Psi_alpha_beta}\end{align}
The following non-dimensional variables are defined: $\overline{\rho_{\alpha}}=\rho_{\alpha}\pi R_{\alpha}^{2}$,
$\overline{\rho_{\beta}}=\rho_{\beta}\pi R_{\alpha}^{2}$, $\theta=\overline{\rho_{\alpha}}+\Sigma^{2}\overline{\rho_{\beta}}$,
and $\Sigma=R_{\beta}/R_{\alpha}$. Since the blocking functions derived
from SPT describe spherical particles, they can be directly compared
to the RSA blocking function. When $k_{s}=0$, equations \ref{eq:Adsorption Transition Kinetics}
and \ref{eq:Adsorption Transition Kinetics 2} reduce to equation
\ref{eq:Single Layer Kinetics}. By setting $\rho_{\beta}=0$ and
$\Sigma=1$, equations \ref{eq:RSA Blocking Function} and \ref{eq:Phi_alpha}
can be compared.

To quantitatively compare different multi-layer models, it is necessary
to transform equations from different sources to use a common set
of variables. The surface number densities $\rho_{i}$ used in equations
\ref{eq:Adsorption Transition Kinetics} and \ref{eq:Adsorption Transition Kinetics 2}
can be converted to fractional surface coverage by multiplying by
the area covered by an adsorbed particle in state $i$, $\sigma_{i}$,
to obtain:\begin{align}
\frac{d\theta_{\alpha}}{dt} & =k_{a}\,\sigma_{\alpha}c\,\Phi_{\alpha}-k_{s}\theta_{\alpha}\Psi_{\alpha\beta}-k_{d}\theta_{\alpha}\label{eq:dtheta_alpha dt}\\
\frac{d\theta_{\beta}}{dt} & =k_{s}\Sigma^{2}\theta_{\alpha}\Psi_{\alpha\beta}\label{eq:dtheta_beta dt}\end{align}



\subsubsection{Langmuir-Type Model of Adsorption with Transition}

A different model of adsorption with a post-adsorption transition
has been used to fit the adsorption kinetics of the fibronectin fragment
\BPChem{FNIII\_{7-10}} \cite{Lundstroem1984,Michael2003}. Written
using the same variables as equations \ref{eq:Adsorption Transition Kinetics}
and \ref{eq:Adsorption Transition Kinetics 2}, the equations that
describe this model are\begin{align}
\frac{d\rho_{\alpha}}{dt} & =k_{a}\, c\, A_{av}-k_{s}\,\rho_{\alpha}\, A_{av}-k_{d}\,\rho_{\alpha}\label{eq:Michael kinetics 1}\\
\frac{d\rho_{\beta}}{dt} & =k_{s}\,\rho_{\alpha}\, A{}_{av}\label{eq:Michael kinetics 2}\end{align}
where $Y_{i}$ is the surface density ($ng/cm^{2}$) of adsorbed protein
in each state. $A_{av}$ is the surface area ($cm^{2}$) available
for adsorption, which is given by\[
A_{av}=A_{total}\left(1-f\,\sigma_{1}\,\rho_{1}-f\, b\,\sigma_{1}\,\rho_{2}\right)\]
where $b=\sigma_{\beta}/\sigma_{\alpha}$. Since $\theta_{i}=\sigma_{i}\, f\,\rho_{i}$
and $f=N_{A}/M$ ($molecules/ng$), this expression can be written
$A_{av}=A_{total}\left(1-\theta\right)$, where $\theta=\theta_{\alpha}+\theta_{\beta}$.
Equations \ref{eq:Michael kinetics 1} and \ref{eq:Michael kinetics 2}
can be written in terms of fractional surface coverage by multiplying
both sides by $\sigma_{i}\, f$ to obtain:\begin{align}
\frac{d\theta_{1}}{dt} & =k\,\sigma_{1}\,\sigma A_{total}\, f\, c\left(1-\theta\right)-s\, A_{total}\,\theta{}_{1}\left(1-\theta\right)-r\,\theta_{1}\label{eq:Langmuir two stage 1}\\
\frac{d\theta_{2}}{dt} & =s\, A_{total}\, b\,\theta_{1}\,\left(1-\theta\right)\label{eq:Langmuir two stage 2}\end{align}
It is clear that the blocking function for this model is actually
the Langmuir blocking function, with $\theta_{\infty}=1$.


\subsubsection{Langmuir-Type Model of Two-Layer Adsorption}

A two-layer adsorption model was formulated, based on the assumptions
of the Langmuir adsorption model. The two-layer adsorption model can
be represented by the chemical equations \[ \cee{A + B <->[k_{a1},k_{d1}] AB} \]
and \[ \cee{A + AB <->[k_{a2},k_{d2}] AAB} \] A represents a molecule
in solution, B represents an available adsorption site, AB represents
a single molecule adsorbed on the surface, and AAB represents a {}``stack''
of two adsorbed molecules. The kinetics of adsorption can be modeled
by a set of coupled ordinary differential equations:\begin{align}
\frac{d\theta_{AB}}{dt} & =k_{a1}c\,\left(\theta_{\infty}-\theta_{AB}-\theta_{AAB}\right)-k_{d1}\theta_{AB}-k_{a2}c\,\theta_{AB}\label{eq:Two Layer AB}\\
\frac{d\theta_{AAB}}{dt} & =k_{a2}c\,\theta_{AB}-k_{d2}-\theta_{AAB}\label{eq:Two Layer AAB}\end{align}
$\theta_{AB}$ is the fraction of the surface covered by a single
particle and $\theta_{AAB}$ is the fraction of the surface covered
by two layers of particles. $\theta=\theta_{AB}+\theta_{AAB}$ is
the total fractional surface coverage. Although it is straightforward
to solve this set of equations analytically, the resulting formulae
are complicated, and the analysis of the solutions is beyond the scope
of this work. 


\subsubsection{Calculating the Surface Density of Active GO }

Since the Langmuir and RSA models only allow protein to exist in one
state, all of the adsorbed protein must be treated as active or inactive.
For the two adsorption models that incorporate a post-adsorption transition,
enzyme in the initial state was assumed to be active, and enzyme in
the denatured state was assumed to be inactive. For the two-layer
model, it was assumed that enzyme molecules in the upper layer prevented
the substrate solution from interacting with molecules adsorbed in
the lower layer. Enzyme in the lower layer was therefore treated as
inactive. The activity of adsorbed enzyme in contact with the surface
which is not screened by an upper layer depends upon the nature of
the surface. It was assumed that hydrophilic surfaces (glass and DETA)
did not induce denaturation upon adsorption, so single-layer protein
was assumed to be active. The surface density of active enzyme was
calculated~$\rho_{active}=\rho_{AB}+\rho_{AAB}$. For 13F (a hydrophobic
surface), it was assumed that adsorption induces denaturation and
destroys the activity of the enzyme. Only enzyme in the upper layer
was considered to be active, so $\rho_{active}=\rho_{AAB}$ The surface
densities of active protein and total protein predicted by the models
were plotted and compared to experimental results. 


\subsection{Implementation of Models and Fitting to Experimental Data}

A general single-layer adsorption simulation based on Equation \ref{eq:Single Layer Kinetics}
was implemented using the Python programming language. Various blocking
functions, such as the Langmuir and RSA blocking functions, could
be plugged into the simulation. Another simulation was created based
on equations \ref{eq:dtheta_alpha dt} and \ref{eq:dtheta_beta dt}
that could utilize either the Langmuir or SPT-derived blocking function.
Equations \ref{eq:Langmuir two stage 1} and \ref{eq:Two Layer AAB}
were used to create a two-layer adsorption simulation. The differential
equations were solved numerically using the odeint routine from SciPy~\cite{Jones2001-}.
All equations were solved in terms of fractional surface coverage
to avoid numerical difficulties that may occur when working with small
floating-point numbers. For comparison with experimental data, the
fractional surface coverage predicted by each model was converted
to surface density ($ng\, cm^{-1}$) using $\rho_{i}=\theta_{i}\,\sigma_{i}^{-1}\, f^{-1}$. 

Each model was fitted to the experimental data from the WGM biosensor
for glass, DETA, 13F, and SiPEG by adjusting the parameters until
the best possible fit was achieved, according to the least-squares
criterion. The concentration of protein in solution near the surface
was assumed to be constant. For each surface, kinetic curves for all
solution concentrations were fitted simultaneously with a single set
of parameters using the \emph{leastsq} fitting routine from SciPy,
which uses a modified version of the Levenberg-Marquardt algorithm.
Third-order splines were used to interpolate the averaged experimental
data to the same time points used in the model. The quality of fit
was quantified by computing the sum of squared errors (SSE) for the
total surface concentration of adsorbed protein measured by the WGM
sensor and the surface concentration predicted by the model. The SSE
for each experiment was divided by the total number of data points
in the data set so that the quality of fit could be compared between
data sets with different numbers of time points. 


\section{Results}

A comparison of the blocking functions of the Langmuir and RSA models
is shown in Figure~\ref{fig:Blocking function comparison}.%
\begin{figure}
\includegraphics{Protein_Adsorption/Plots/ASF_comparison}

\caption{\label{fig:Blocking function comparison}Blocking functions from the
RSA model ($\phi_{FIT,3}$) and scaled particle theory ($\Phi_{\alpha}$)}
%
\end{figure}
 The blocking function for protein in the initial state for the model
derived from scaled particle theory is also shown. It can be seen
that the first-level blocking function derived from SPT is similar
but not identical to the RSA blocking function, especially at higher
values of fractional surface coverage.

The kinetics of adsorption based on the Langmuir blocking function
were compared to the kinetics modeled by the SPT blocking functions,
and the results are shown in Figure~\,\ref{fig:SPT vs Langmuir kinetics}.
%
\begin{figure}[h]
\subfloat[\label{fig:SPT kinetics}Blocking function from scaled particle theory]{\includegraphics[scale=0.75]{Protein_Adsorption/Plots/SPT_kinetics}}\subfloat[\label{fig:Langmuir two-stage kinetics}Langmuir two-stage model]{\includegraphics[scale=0.75]{Protein_Adsorption/Plots/Langmuir_twostage_kinetics}}

\caption{\label{fig:SPT vs Langmuir kinetics}Comparison of kinetics predicted
by SPT blocking function (a) and Langmuir blocking function (b) for
$k_{a}=1$, $k_{s}=\pi$, $k_{d}=\pi$, $r_{\alpha}=1$, $\Sigma=1.2$,
and $c=1$.}
%
\end{figure}
 The parameters used for the comparison were taken from Figure 2a
from \cite{Brusatori1999}. Note that Figure 2a from \cite{Brusatori1999}
cannot be directly compared directly with Figure \ref{fig:SPT vs Langmuir kinetics}
in this work, because $\theta_{\beta}\neq\overline{\rho_{\beta}}$.


\subsection{Transport Analysis of the WGM Biosensor}

The steady-state velocity magnitude predicted by the model of the
whole flow cell is shown in Figure~\ref{fig:CFD 3D velocity}.%
\begin{figure}
\includegraphics{Protein_Adsorption/Figures/CFD_3D_velocity}\includegraphics{Protein_Adsorption/Figures/CFD_3D_velocity_xcut}

\caption{\label{fig:CFD 3D velocity}The magnitude of velocity predicted by
CFD simulations in the vicinity of the WGM resonator.}
%
\end{figure}
It can be seen that the flow field around the resonator was symmetric
about the long axis of the resonator, with only a minor perturbation
caused by the presence of the waveguide. This configuration ensured
that the shear rate was constant in the region where the evanescent
wave was excited, minimizing any shear rate effects on the adsorption
of protein from solution. These results also confirmed that the axisymmetric
model was a reasonable choice for simulating the depletion region
near the surface of the resonator.


\subsubsection{Fibronectin}

The evolution of the concentration of FN in solution near the surface
of the WGM resonator over time is shown in Figure \ref{fig:CFD Bulk Conc}.%
\begin{figure}
\subfloat[\label{fig:CFD FN near surf conc}Near-surface concentration]{\includegraphics{Protein_Adsorption/Plots/CFD_FN_DETA_near_surf_conc}



}\subfloat[\label{fig:CFD FN near surf conc rescaled}Same data, rescaled to
show lower concentrations]{\includegraphics{Protein_Adsorption/Plots/CFD_FN_DETA_near_surf_conc_rescaled}



}

\includegraphics{Protein_Adsorption/Plots/CFD_FN_legend}\caption{\label{fig:CFD Bulk Conc}CFD prediction of the concentration of FN
very close to the surface of the resonator.}


%
\end{figure}
 At $10\,\mu g/ml$ the concentration near the surface of the resonator
required about 150 seconds to reach its final value. Figure \ref{fig:CFD FN near surf conc rescaled}
indicates that the near-surface concentration takes longer to reach
its final value as the bulk concentration is decreased. Transport
was not analyzed for the SiPEG surface because its low affinity for
protein was not expected to deplete protein in solution near the surface
significantly. The surface density of adsorbed FN predicted by the
CFD model is shown in Figure \ref{fig:CFD near surface conc}, along
with the average experimental data.%
\begin{figure}
\subfloat[\label{fig:CFD FN DETA surf conc}DETA]{\includegraphics{Protein_Adsorption/Plots/CFD_FN_DETA_surf_conc}}\subfloat[\label{fig:CFD FN 13F surf conc}13F]{\includegraphics{Protein_Adsorption/Plots/CFD_FN_13F_surf_conc}}

\includegraphics{Protein_Adsorption/Plots/CFD_FN_legend}

\caption{\label{fig:CFD near surface conc}CFD predictions and experimental
measurements of the surface density of adsorbed FN. Thick lines indicate
CFD predictions, while thin lines indicate average experimental data.}


%
\end{figure}



\subsubsection{Glucose Oxidase}

A similar analysis was performed for the transport of glucose oxidase
in the flow cell. The predicted concentration near the surface of
the resonator with adsorption parameters for GO on glass is shown
in Figure \ref{fig:CFD GO near surf conc}.%
\begin{figure}
\includegraphics{Protein_Adsorption/Plots/CFD_GO_glass_near_surf_conc}

\caption{\label{fig:CFD GO near surf conc}CFD prediction of the concentration
of GO near the surface of the resonator.}
%
\end{figure}
 The results for DETA and 13F are virtually identical. To verify that
the correct kinetic constants were used in the Langmuir adsorption
model, the surface density of adsorbed protein over time predicted
by the CFD simulations was plotted along with the measured surface
density. These results are shown in Figure \ref{fig:CFD GO near surf conc}.%
\begin{figure}
\subfloat[\label{fig:CFD GO DETA surf conc}DETA]{\includegraphics{Protein_Adsorption/Plots/CFD_GO_DETA_surf_conc}}\subfloat[\label{fig:CFD GO 13F surf conc}13F]{\includegraphics{Protein_Adsorption/Plots/CFD_GO_13F_surf_conc}

}

\subfloat[\label{fig:CFD GO glass surf conc}Glass]{\includegraphics{Protein_Adsorption/Plots/CFD_GO_glass_surf_conc}

}\includegraphics{Protein_Adsorption/Plots/CFD_surf_conc_legend}\caption{\label{fig:CFD GO surface density}Surface density of adsorbed GO
predicted by CFD simulation and measured by WGM biosensor.}
%
\end{figure}
 Transport was not analyzed for the SiPEG surface because its low
affinity for protein was not expected to deplete protein in solution
near the surface.


\subsection{Modeling the Adsorption of Fibronectin on Silane Surfaces}

The kinetics of adsorption of fibronectin on 13F, DETA, and SiPEG
are shown in Figure \ref{fig:FN experiments}.%
\begin{figure}
\includegraphics[width=1\columnwidth]{Protein_Adsorption/Plots/FN_Experimental_Data}

\caption{\label{fig:FN experiments}Measured adsorption kinetics for fibronectin
on 13F, DETA, and SiPEG surfaces.}


%
\end{figure}
 The kinetics predicted by the models fitted to adsorption on DETA
are shown in Figure \ref{fig:FN DETA fitted}, and the fitted parameter
values are shown in Table \ref{tab:FN on DETA params}.%
\begin{table}
\caption{\label{tab:FN on DETA params}Fitted parameter values for FN on DETA.}
\begin{tabular}{>{\raggedright}p{0.75in}cccccc}
 & $k_{a}\left(cm^{3}\, ng^{-1}\, s^{-1}\right)$ & $k_{s}\left(s^{-1}\right)$ & $k_{d}\left(s^{-1}\right)$ & $\sigma_{\alpha}\left(nm^{2}\right)$ & $\sigma_{\beta}\left(nm^{2}\right)$ & $SSE$\tabularnewline[\doublerulesep]
\cline{2-7} 
\noalign{\vskip\doublerulesep}
RSA & $2.05\times10^{-6}$ &  & $3.16\times10^{-4}$ & $182$ &  & $86.1$\tabularnewline
\noalign{\vskip\doublerulesep}
RSA with transition & $2.37\times10^{-6}$ & $1.14\times10^{-4}$ & $2.12\times10^{-4}$ & $203$ & $380$ & $86.0$\tabularnewline
\noalign{\vskip\doublerulesep}
\end{tabular}%
\end{table}
%
\begin{figure}
\includegraphics{Protein_Adsorption/Plots/FN_DETA_RSA_CFD}

\caption{\label{fig:FN DETA fitted}RSA model fitted to experimental data for
FN on DETA.}


%
\end{figure}
 The kinetics predicted by models fitted to FN adsorption on 13F are
shown in Figure \ref{fig:FN 13F fitted}, and the parameter values
for the fitted models are shown in Table \ref{tab:FN OEG fitted params}.%
\begin{table}
\caption{\label{tab:FN on 13F}Fitted parameter values for FN on 13F.}
\begin{tabular}{>{\raggedright}p{0.75in}cccccc}
 & $k_{a}\left(cm^{3}\, ng^{-1}\, s^{-1}\right)$ & $k_{s}\left(s^{-1}\right)$ & $k_{d}\left(s^{-1}\right)$ & $\sigma_{\alpha}\left(nm^{2}\right)$ & $\sigma_{\beta}\left(nm^{2}\right)$ & $SSE$\tabularnewline[\doublerulesep]
\cline{2-7} 
\noalign{\vskip\doublerulesep}
RSA & $2.24\times10^{-6}$ &  & $2.49\times10^{-4}$ & $196$ &  & $121$\tabularnewline
\noalign{\vskip\doublerulesep}
RSA with transition & $1.99\times10^{-6}$ & $8.00\times10^{-3}$ & $9.95\times10^{-5}$ & $176$ & $303$ & $85.5$\tabularnewline
\noalign{\vskip\doublerulesep}
\end{tabular}%
\end{table}
%
\begin{figure}
\includegraphics{Protein_Adsorption/Plots/FN_13F_VanTassel_CFD}

\caption{\label{fig:FN 13F fitted}Adsorption model with post-adsorption transition
fitted to experimental data for FN on 13F.}
%
\end{figure}
 The kinetics predicted by models fitted to FN adsorption on SiPEG
are shown in Figure \ref{fig:FN OEG fitted}, and the parameter values
for the fitted models are shown in Table \ref{tab:FN OEG fitted params}.%
\begin{table}
\caption{\label{tab:FN OEG fitted params}Parameter values fitted to FN on
SiPEG.}
\begin{tabular}{>{\raggedright}p{0.75in}cccccc}
 & $k_{a}\left(cm^{3}\, ng^{-1}\, s^{-1}\right)$ & $k_{s}\left(s^{-1}\right)$ & $k_{d}\left(s^{-1}\right)$ & $\sigma_{\alpha}\left(nm^{2}\right)$ & $\sigma_{\beta}\left(nm^{2}\right)$ & $SSE$\tabularnewline[\doublerulesep]
\cline{2-7} 
\noalign{\vskip\doublerulesep}
RSA & $9.25\times10^{-7}$ &  & $7.14\times10^{-4}$ & $1240$ &  & $2.7$\tabularnewline
\noalign{\vskip\doublerulesep}
RSA with transition & $1.12\times10^{-6}$ & $1.16\times10^{-2}$ & $1.04\times10^{-4}$ & $1240$ & $3420$ & $1.86$\tabularnewline
\noalign{\vskip\doublerulesep}
\end{tabular}%
\end{table}
%
\begin{figure}
\includegraphics{Protein_Adsorption/Plots/FN_OEG_RSA_CFD}

\caption{\label{fig:FN OEG fitted}RSA model fitted to experimental data for
FN on SiPEG.}
%
\end{figure}
 Results of the cell culture experiments are shown in Table~\ref{tab:Cell counts on FN}.%
\begin{table}
\caption{\label{tab:Cell counts on FN}Cell counts ($mm^{-2}$) for embryonic
hippocampal neurons and embryonic skeletal muscle cultured on silane
surfaces.}


\begin{tabular}{cccccc}
Cell Type & Parameter & N & DETA & 13F & SiPEG\tabularnewline[\doublerulesep]
\hline
\noalign{\vskip\doublerulesep}
Hippocampal & Live & $9$ & $212\pm102$ & $1\pm3$ & $4\pm5$\tabularnewline
\noalign{\vskip\doublerulesep}
Hippocampal & Dead & $9$ & $340\pm75$ & $218\pm81$ & $265\pm86$\tabularnewline
\noalign{\vskip\doublerulesep}
Muscle & Live & $4$ & $178\pm43$ & $50\pm32$ & $0\pm0$\tabularnewline
\noalign{\vskip\doublerulesep}
Muscle & Dead & $4$ & $63\pm66$ & $18\pm14$ & $111\pm59$\tabularnewline
\noalign{\vskip\doublerulesep}
Muscle & Myotubes & $4$ & $35\pm13$ & $0\pm0$ & $0\pm0$\tabularnewline
\noalign{\vskip\doublerulesep}
\end{tabular}%
\end{table}



\subsection{Modeling the Adsorption of Glucose Oxidase on Silane Surfaces}

The experimentally measured kinetic curves for glucose oxidase on
glass, DETA, 13F, and SiPEG, along with the model that achieved the
best fit to each data set, are shown in Figures \ref{fig:GO glass plot},
\ref{fig:GO on DETA}, \ref{fig:GO on 13F} and \ref{fig:GO on PEG},
respectively. The initial rate of adsorption was highest on the DETA
and 13F surfaces and lowest on the SiPEG surface. The surface density
of adsorbed protein reached the highest saturation value on the DETA
surface at $100\,\mu g/ml$, while the highest saturation at $10\,\mu g/ml$
occurred on 13F. On the 13F surface the saturation surface density
at $10\,\mu g/ml$ was nearly as high as $100\,\mu g/ml$. Adsorption
on the glass surface showed an {}``overshoot'' profile in which
the surface density reached a maximum and then decreased gradually
with time. The parameter values for the five models fitted to the
experimental measurements are shown in Tables \ref{tab:GO params glass},
\ref{tab:GO params DETA}, and \ref{tab:GO params 13F}, respectively.
Only the RSA and Langmuir models were fitted to GO on SiPEG, and the
fitted parameter values are shown in Table \ref{tab:GO params PEG}.

%
\begin{figure}
\includegraphics{Protein_Adsorption/Plots/GO_glass}\caption{\label{fig:GO glass plot}Langmuir model with post-adsorption transition
fitted to data for GO on glass from the WGM biosensor.}
%
\end{figure}
%
\begin{figure}
\includegraphics{Protein_Adsorption/Plots/GO_DETA}

\caption{\label{fig:GO on DETA}Langmuir two-layer adsorption model fitted
to data for GO on DETA from the WGM biosensor.}
%
\end{figure}
%
\begin{figure}
\includegraphics{Protein_Adsorption/Plots/GO_13F}

\caption{\label{fig:GO on 13F}Langmuir model with post-adsorption transition
fitted to data for GO on 13F from the WGM biosensor.}
%
\end{figure}
%
\begin{figure}
\includegraphics{Protein_Adsorption/Plots/GO_PEG}

\caption{\label{fig:GO on PEG}RSA model fitted to data for GO on SiPEG from
the WGM biosensor.}
%
\end{figure}
%
\begin{figure}
\includegraphics{Protein_Adsorption/Plots/GO_glass_activity}

\caption{\label{fig:GO glass activity}Experimental measurements and model
predictions of GO activity on a glass surface. The {*} denotes the
model with the best fit to the kinetic data.}


%
\end{figure}
%
\begin{figure}
\includegraphics{Protein_Adsorption/Plots/GO_DETA_activity}

\caption{\label{fig:GO DETA activity}Experimental measurements and model predictions
of GO activity on a DETA surface. The {*} denotes the model with the
best fit to the kinetic data.}
%
\end{figure}
%
\begin{figure}
\includegraphics{Protein_Adsorption/Plots/GO_13F_activity}

\caption{\label{fig:GO 13F activity}Experimental measurements and model predictions
of GO activity on a 13F surface. The {*} denotes the model with the
best fit to the kinetic data.}
%
\end{figure}
%
\begin{table}
\caption{\label{tab:GO params glass}Parameters fitted to data for GO on glass.}
\begin{tabular}{>{\raggedright}p{0.75in}cccccc}
 & $k_{a}\left(cm^{3}\, ng^{-1}\, s^{-1}\right)$ & $k_{s}\left(s^{-1}\right)$ & $k_{d}\left(s^{-1}\right)$ & $\sigma_{\alpha}\left(nm^{2}\right)$ & $\sigma_{\beta}\left(nm^{2}\right)$ & $SSE$\tabularnewline[\doublerulesep]
\cline{2-7} 
\noalign{\vskip\doublerulesep}
Langmuir & $3.33\times10^{-8}$ &  & $1.15\times10^{-3}$ & $257$ &  & $6.15$\tabularnewline
Langmuir with transition & $1.35\times10^{-8}$ & $1.48\times10^{-3}$ & $1.75\times10^{-3}$ & $122$ & $414$ & $0.73$\tabularnewline
\noalign{\vskip\doublerulesep}
RSA & $1.16\times10^{-8}$ &  & $1.34\times10^{-3}$ & $80.8$ &  & $7.81$\tabularnewline
\noalign{\vskip\doublerulesep}
RSA with transition & $3.80\times10^{-9}$ & $2.20\times10^{-3}$ & $1.99\times10^{-3}$ & $31.5$ & $349$ & $1.19$\tabularnewline
\noalign{\vskip\doublerulesep}
\noalign{\vskip\doublerulesep}
 & $k_{a1}$ & $k_{a2}$ & $k_{d1}$ & $k_{d2}$ & $\sigma$ & $SSE$\tabularnewline
Two layer model & $1.85\times10^{-8}$ & $5.46\times10^{-8}$ & $2.18\times10^{-4}$ & $4.60\times10^{-3}$ & $224$ & $3.15$\tabularnewline
\end{tabular}%
\end{table}
%
\begin{table}
\caption{\label{tab:GO params DETA}Parameters fitted to data for GO on DETA.}
\begin{tabular}{>{\raggedright}p{0.75in}cccccc}
 & $k_{a}\left(cm^{3}\, ng^{-1}\, s^{-1}\right)$ & $k_{s}\left(s^{-1}\right)$ & $k_{d}\left(s^{-1}\right)$ & $\sigma_{\alpha}\left(nm^{2}\right)$ & $\sigma_{\beta}\left(nm^{2}\right)$ & $SSE$\tabularnewline[\doublerulesep]
\cline{2-7} 
\noalign{\vskip\doublerulesep}
Langmuir & $3.10\times10^{-8}$ &  & $1.83\times10^{-4}$ & $190$ &  & $66.1$\tabularnewline
\noalign{\vskip\doublerulesep}
Langmuir with transition & $1.15\times10^{-7}$ & $9.41\times10^{-3}$ & $3.22\times10^{-2}$ & $161$ & $161$ & $38.7$\tabularnewline
\noalign{\vskip\doublerulesep}
RSA & $1.47\times10^{-8}$ &  & $2.15\times10^{-4}$ & $72.6$ &  & $39.3$\tabularnewline
\noalign{\vskip\doublerulesep}
RSA with transition & $1.19\times10^{-8}$ & $0.412$ & $2.33\times10^{-4}$ & $28.0$ & $457$ & $6.47$\tabularnewline
\noalign{\vskip\doublerulesep}
\noalign{\vskip\doublerulesep}
 & $k_{a1}$ & $k_{a2}$ & $k_{d1}$ & $k_{d2}$ & $\sigma$ & $SSE$\tabularnewline
\noalign{\vskip\doublerulesep}
Two layer model & $1.44\times10^{-7}$ & $1.26\times10^{-8}$ & $1.19\times10^{-3}$ & $3.42\times10^{-9}$ & $363$ & $12.7$\tabularnewline
\end{tabular}%
\end{table}
%
\begin{table}
\caption{\label{tab:GO params 13F}Parameters fitted to data for GO on 13F.}
\begin{tabular}{>{\raggedright}p{0.75in}cccccc}
 & $k_{a}\left(cm^{3}\, ng^{-1}\, s^{-1}\right)$ & $k_{s}\left(s^{-1}\right)$ & $k_{d}\left(s^{-1}\right)$ & $\sigma_{\alpha}\left(nm^{2}\right)$ & $\sigma_{\beta}\left(nm^{2}\right)$ & $SSE$\tabularnewline[\doublerulesep]
\cline{2-7} 
\noalign{\vskip\doublerulesep}
Langmuir & $7.20\times10^{-8}$ &  & $1.17\times10^{-4}$ & $247$ &  & $2.45$\tabularnewline
Langmuir with transition & $5.96\times10^{-8}$ & $1.14\times10^{-2}$ & $6.74\times10^{-5}$ & $209$ & $291$ & $2.23$\tabularnewline
\noalign{\vskip\doublerulesep}
RSA & $3.79\times10^{-8}$ &  & $3.81\times10^{-5}$ & $103$ &  & $12.8$\tabularnewline
\noalign{\vskip\doublerulesep}
RSA with transition & $1.27\times10^{-9}$ & $0.377$ & $9.73\times10^{-3}$ & $4.6$ & $175$ & $2.3$\tabularnewline
\noalign{\vskip\doublerulesep}
\noalign{\vskip\doublerulesep}
 & $k_{a1}$ & $k_{a2}$ & $k_{d1}$ & $k_{d2}$ & $\sigma$ & $SSE$\tabularnewline
Two layer model & $7.92\times10^{-8}$ & $7.18\times10^{-7}$ & $1.51\times10^{-4}$ & $1.28\times10^{-4}$ & $496$ & $1.82$\tabularnewline
\end{tabular}%
\end{table}
%
\begin{table}
\caption{\label{tab:GO params PEG}Fitted parameters for GO on SiPEG.}
\begin{tabular}{>{\raggedright}p{0.75in}cccc}
 & $k_{a}\left(cm^{3}\, ng^{-1}\, s^{-1}\right)$ & $k_{d}\left(s^{-1}\right)$ & $\sigma\,\left(nm^{2}\right)$ & $SSE$\tabularnewline[\doublerulesep]
\cline{2-5} 
\noalign{\vskip\doublerulesep}
Langmuir & $4.06\times10^{-8}$ & $3.67\times10^{-4}$ & $684$ & $0.81$\tabularnewline
\noalign{\vskip\doublerulesep}
RSA & $2.18\times10^{-8}$ & $2.55\times10^{-4}$ & $482$ & $0.73$\tabularnewline
\noalign{\vskip\doublerulesep}
\end{tabular}

%
\end{table}


For GO on glass, the Langmuir-derived models fitted the data slightly
better than the RSA-derived models. The Langmuir-type model with post-adsorption
transition provided the best fit to the kinetic data. Both models
with post-adsorption transition predicted the amount of active protein
at $10\,\mu g/ml$, but not at $100\,\mu g/ml$, as shown in Figure
\ref{fig:GO glass activity}. The activity predicted by both models
at $100\,\mu g/ml$ matched the experimental value much more closely
if it was assumed that the protein was active in both the initial
and final states. For GO on DETA, the RSA-derived models fitted the
data better than the Langmuir-type models. The best fit to the kinetic
curves was achieved by the RSA-type model with post-adsorption transition.
However, this model did not predict the amount of active enzyme at
$10\,\mu g/ml$ or $100\,\mu g/ml$, as shown in Figure \ref{fig:GO DETA activity}
The two-layer model does not fit the kinetic data quite as well, but
it does predict the amount of active enzyme very well. For GO on 13F,
the two-layer model and the Langmuir-type model with post-adsorption
transition fitted the kinetic data well. Although the two-layer model
fitted the kinetic data slightly better, Figure \ref{fig:GO 13F activity}
indicates that it does not accurately predict the amount of active
adsorbed protein. The Langmuir-type model with a post-adsorption transition
predicted the amount of active protein very well.


\section{Discussion}

Two different kinetic models have been proposed in the literature
to model the adsorption of protein with a post-adsorption transition.
The model by Brusatori and Van Tassel \cite{Brusatori1999} is analogous
to the RSA model for adsorbing disks because it was assumed that the
particles were circular before and after the transition. Ideally,
the blocking function used in this model (equations~\ref{eq:Phi_alpha}
and~\ref{eq:Psi_alpha_beta}) should produce the same results as
the RSA blocking function (equation\,\ref{eq:RSA Blocking Function})
when the radius of particles in state $\beta$ is identical to the
radius of particles in state~$\alpha$. Figure~\ref{fig:Blocking function comparison}
demonstrates that there is a small difference between the blocking
functions. This difference is probably because neither blocking function
is exact; the equation for the RSA blocking function is fitted to
simulation data and the two-stage blocking function is also an approximation.
If the RSA model and the transition model are fitted to the same data
set, the fitted parameters will not be identical, even if the transition
rate constant is very small.

When the model used by Michael et. al. \cite{Michael2003} was written
in terms of a different set of variables, it was revealed that the
model uses a Langmuir-like blocking function. This result shows that
the relationship between the kinetic models stated and Michael et.
al. and Brusatori and Van Tassel is analogous to the relationship
between the RSA model and the Langmuir model. This is illustrated
in Figure~\ref{fig:Adsorption Model Matrix}.%
\begin{figure}
\includegraphics{Protein_Adsorption/Figures/Adsorption_Model_Matrix}

\caption{\label{fig:Adsorption Model Matrix}Relationship between adsorption
models.}


%
\end{figure}
 Although the kinetics of the total fractional surface coverage $\theta$
are similar for both models, Figure~\ref{fig:SPT vs Langmuir kinetics}
shows that the kinetics of $\theta_{\alpha}$ and $\theta_{\beta}$
are very different for the two models. This difference has significant
implications for modeling experimental data if a protein in state~$\alpha$
functions differently than a protein in state~$\beta$.

Langmuir-type and RSA-type models imply different types of adsorption
behavior. RSA models assume that proteins can be approximated by spheres
that act like hard spheres. Because the spheres adsorb randomly on
the surface, the surface packing is inefficient and the fractional
surface coverage reaches its maximum value around 0.547. Although
it is well known that proteins such as fibronectin are nothing like
hard spheres, the RSA model is surprising useful for fitting adsorption
data. The Langmuir model makes no assumptions about the shape of the
adsorbing particles and assumes that the surface available for adsorption
is a linear function of the surface coverage. The Langmuir model also
assumes that the surface reaches 100\% coverage. Reality is probably
somewhere in between the two extremes, with proteins adsorbing in
a random sequential fashion but packing somewhat more effectively
than spheres due their flexibility. Fitting the RSA and Langmuir models
to the same data set and seeing which model fits better can indicate
whether a particular protein is hard or flexible (on a particular
surface under specific solvent conditions).


\subsection{Transport Analysis}

Although the flow cell was designed to minimize transport limitations,
CFD simulations showed that the near-surface concentration of fibronectin
was influenced by transport limitations early in the adsorption process
on DETA and 13F surfaces. Initially, transport was limited by the
large amount of buffer that had to be displaced from the flow system
before the protein solution could reach the resonator at full concentration.
This limitation was due to the prototype nature of the system, and
can be easily eliminated in future systems. After the protein solution
in the bulk of the flow cell reached its target concentration, the
rate of adsorption onto the resonator was high enough to deplete the
fibronectin in a boundary layer near the resonator. Because of the
no-slip boundary condition at the surface and the absence of turbulent
mixing, diffusion was the only way that fibronectin can be transported
to the surface to adsorb. Fibronectin is a relatively large protein
with a correspondingly low diffusion coefficient, so its rate of diffusive
transport is rather low. This type of limitation is virtually unavoidable
in microfluidic systems when the rate of adsorption is high and the
solution concentration is low. The near-surface concentration predicted
by CFD was used when fitting the adsorption models to the fibronectin
data, enabling accurate kinetic parameters to be extracted even in
the presence of transport limitations.

In the experiments reported here, transport limitation had much less
influence on the adsorption of glucose oxidase. GO is a smaller protein
than FN, its initial rate of adsorption is lower and the concentrations
used in these experiments were higher than the concentrations of FN.
GO reached 99\% of full concentration at the resonator surface within
20 seconds, so the near-surface depletion had minimal impact on the
kinetics. Constant concentration was assumed when fitting the kinetic
adsorption models to the GO data. The association rate constants from
fitted models (Tables \ref{fig:FN DETA fitted}-\ref{fig:FN OEG fitted}
and \ref{tab:GO params glass}-\ref{tab:GO params PEG}) demonstrate
that GO has a much lower affinity for the surfaces than FN, which
explains why its maximum rate of adsorption is lower and transport
limitation is much less significant.

Our modeling method was approximate in that the evolution of the near-surface
concentration over time was computed once, based on the estimated
adsorption rate, and was not modified during the curve-fitting procedure.
A more accurate method would be to incorporate the CFD model into
the fitting routine so that the near-surface concentration would be
updated as the adsorption rate changed \cite{Jenkins2004}. However,
this method is only practical when the CFD simulation is simple enough
to run very quickly. It was also assumed that the near-surface concentration
increased linearly from sixty seconds until the target concentration
was reached. Although this probably had a slight impact on the modeled
kinetics, it was better than assuming that the concentration near
the resonator remained constant.


\subsection{Fibronectin Adsorption}

%
\begin{table}
\begin{tabular}{>{\centering}p{1in}>{\centering}p{1in}>{\centering}p{0.8in}>{\centering}p{0.8in}>{\centering}p{0.65in}>{\centering}p{1.25in}}
Solution Concentration ($\mu g/ml$) & Hydrophobic Surface & Hydrophilic Charged Surface & Hydrophilic Neutral Surface & Ref & Notes\tabularnewline[\doublerulesep]
\hline
\noalign{\vskip\doublerulesep}
$10$ & $200$ (13F) & $190$ & $25$ & This work & \tabularnewline
\noalign{\vskip\doublerulesep}
$10$ & $170$ (\ce{CH3}) & $170$ & $30$ & \cite{Keselowsky2003} & \tabularnewline
\noalign{\vskip\doublerulesep}
$10$ & $160$ (\ce{CH3}) &  & $30$ & \cite{Capadona2003} & \tabularnewline
\noalign{\vskip\doublerulesep}
$10$ & $140$ & $140$ & $50$ & \cite{Michael2003} & \ce{FN III_{7-10}} fragment\tabularnewline
\noalign{\vskip\doublerulesep}
$10$ & $110$ & $70$ & $25$ & \cite{Lee2006} & \tabularnewline
\noalign{\vskip\doublerulesep}
$10$ & $\sim175$ & $\sim175$ &  & \cite{Baujard-Lamotte2008} & Saturation value inferred\tabularnewline
\noalign{\vskip\doublerulesep}
$1$ & $135$ (13F) & $137$ & $14$ & This work & \tabularnewline
\noalign{\vskip\doublerulesep}
$1$ & $20$(\ce{CH3}) & $20$ & $10$ & \cite{Keselowsky2003} & \tabularnewline
\noalign{\vskip\doublerulesep}
$1$ & $20$(\ce{CH3}) &  & $5$ & \cite{Capadona2003} & \tabularnewline
\noalign{\vskip\doublerulesep}
$1$ & $30$ & $25$ & $10$ & \cite{Michael2003} & \tabularnewline
\noalign{\vskip\doublerulesep}
$1$ & $10$ & $10$ & $10$ & \cite{Lee2006} & \tabularnewline
\noalign{\vskip\doublerulesep}
\end{tabular}

\caption{\label{tab:FN saturation values}Saturation surface density of adsorbed
fibronectin ($ng/cm^{2})$ from this work and previous studies reported
in the literature.}
%
\end{table}
The saturation surface concentration of FN, as measured by the WGM
system, was compared to previously published results for solution
concentrations of $1\,\mu g/ml$ and $10\,\mu g/ml$ as shown in Table
\ref{tab:FN saturation values}. The WGM measurements compare favorably
with previously published results for a solution concentration of
$10\,\mu g/ml$. Although different measurement techniques and surface
preparations were used, the surface concentrations on hydrophobic
and hydrophilic surfaces were fairly similar among the various references
(reference \cite{Lee2006} being the only exception). The saturation
surface concentration of FN on hydrophilic neutral surfaces measured
by the WGM sensor also indicated excellent agreement with previously
published results. However, the amount of adsorbed protein measured
by the WGM system at lower concentrations was significantly greater
than the amount reported in previously published results for hydrophobic
and hydrophilic charged surfaces. In contrast, the surface concentration
value for a neutral hydrophobic surface was in good agreement with
previous results. The discrepancy between the WGM sensor results and
the other methods at low concentrations may be explained by differences
in the measurement system, the surface chemistry, or the adsorption
process. Since the limiting surface coverages measured by the WGM
sensor agree well with other techniques for $10\,\mu g/ml$ and for
neutral hydrophilic surfaces at $1\,\mu g/ml$, the results from the
WGM sensor can be considered reliable. It is likely that if systematic
errors were inherent to the WGM method, those errors would be reflected
throughout all solution concentrations measured. 

One possible interpretation of the higher saturation values measured
at $1\,\mu g/ml$ and lower on 13F and DETA could be the relative
packing order of silane monolayers compared to alkanethiol monolayers.
Alkanethiol SAMs are known to create highly ordered monolayers due
to their tight packing on highly ordered gold films \cite{Prime1991}.
Because of this tight packing only the terminal functional groups
of the alkanethiol are presented at the surface, resulting in highly
defined surface chemistries. Alkylsilane monolayers, which are formed
on silica surfaces, are less tightly packed and therefore form less
ordered monolayers, which can potentially present more than just the
terminal functional group. It has been hypothesized that this may
result from interaction of electron donating groups of the silane
side chain with silanol groups at the surface resulting in reaction
site-limited substrates \cite{Stenger1992}. This can lead to incomplete
monolayers that allow interaction of protein with the unreacted substrate
or allow sufficient degrees of freedom for the silane side-chains
to adopt multiple conformations, creating less ordered monolayers
that can rearrange to accommodate the native protein structure. At
the high concentrations the amount of protein available to bind would
swamp out these effects but they would be present at the lower concentrations.
This could explain why the differences between the silane chemistry
used in this work and the alkanethiol chemistry used in previous work
\cite{Keselowsky2003} did not show up at $10\,\mu g/ml$. Thus, this
additional degree of freedom would allow the long side chains to rearrange
to accommodate greater protein interaction for structural stabilization
and higher coverage or to expose new surface sites for increased protein
adsorption. This would not be possible with the tightly packed alkane
thiol monolayers. This effect could also have major consequences for
protein function, as described later. 

Another significant difference between the WGM measurement system
and previous work was that the protein was deposited on the WGM resonator
under flow conditions and the measurement was continued until saturation
was reached, while previous measurements were made after exposure
to a static FN solution for a fixed amount of time (30-60 minutes.)
The combination of high-affinity surfaces and low solution concentration
is conducive to transport-limited adsorption, which could explain
the discrepancy between WGM and static experiments for hydrophobic
and charged hydrophilic surfaces. In contrast, neutral hydrophilic
surfaces have a much lower affinity for protein, so depletion of protein
near the surface would be much less of a factor, resulting in good
agreement between the WGM and static measurements. 


\subsubsection{Fibronectin on DETA and 13F}

The fitted parameters of the RSA model were very similar for FN adsorption
on DETA and 13F surfaces, reflecting the similar shapes of the experimental
curves. The RSA model fitted the DETA data well, indicating that the
assumptions of the RSA model were valid for the process of adsorption
on DETA. This result was confirmed by the fitting results for the
two-stage adsorption model. The mean squared error for the model with
transition was only slightly lower than the error for the fitted RSA
model. We conclude that fibronectin adsorbs on DETA with a well-defined
footprint, which does not change significantly after adsorption. This
result is consistent with the well-established theory that proteins
generally experience less denaturation on a hydrophilic surface than
on a hydrophobic surface \cite{Latour2005}. The experimental results
for FN on 13F showed significant deviations from the RSA model in
the saturation region, especially at higher solution concentrations.
The two-stage model allows particles to change size after adsorption,
which significantly improved the fit of the model to the data for
FN on 13F. The fitted values for the association constant were quite
similar for DETA and 13F, but the transition rate constant $k_{s}$
for the 13F surface was an order of magnitude larger than $k_{s}$
for the DETA surface. The results from the fitting process indicated
that FN denatured after adsorption on 13F, which had been previously
postulated for certain hydrophobic surfaces \cite{Lan2005,Sivaraman2009}. 


\subsubsection{Fibronectin on SiPEG}

The RSA model fitted the SiPEG data well. The association rate constant
for FN adsorption on SiPEG was lower than the values for DETA and
13F while the dissociation rate constant was higher, which is expected
for a protein-resistant surface. This result is consistent with findings
that SiPEG is an electrostatically neutral surface that does not exhibit
coulombic attraction for proteins in solution. Surprisingly, the fitted
radius of FN adsorbed on SiPEG was more than twice the fitted radius
of FN adsorbed on DETA or 13F. For the two-stage model, the transition
rate constant for adsorption on SiPEG was significantly higher than
for the other surfaces. The fitted pre-transition radius and post-transition
radius of adsorbed FN were also larger for SiPEG than DETA or 13F.
The large radius predicted by the RSA model and the significant transition
predicted by the two-stage model seemed to indicate that FN denatures
after it adsorbs to PEG. This prediction was not consistent with the
well-known observation that proteins in contact with hydrophobic surfaces
tend to denature, while proteins in contact with hydrophilic, charged
surfaces tend to retain their native conformations. However, it also
may indicate that the SiPEG surface could be promoting the denaturation
of adsorbed proteins, which could explain why it is a cell-resistant
surface despite being hydrophilic. 

Although the SSE of the fitted two-stage model was about 30\% lower
than the SSE for the RSA model, the absolute change in SSE was relatively
small, and may not be significant. It is possible that the two extra
variable parameters (transition rate constant and post-transition
radius) are redundant for the SiPEG surface, in which case their fitted
values should not be considered significant. It is also possible that
the radius predicted by the fitting process for SiPEG is an artifact
caused by fitting the data with a model that is not well suited to
the surface chemistry. Given the assumptions of the RSA model, surface
coverage can reach saturation in only two ways: either the rate of
desorption equals the rate of adsorption, or there is no space left
on the surface for another protein to adsorb. The second case may
not apply to an adsorption-resistant surface like SiPEG. However,
combinations of parameters and that fitted the initial adsorption
kinetics did not predict the low saturation level of protein observed
in our experiments. One possible explanation is that FN adsorbed to
a small number of defects in the SiPEG monolayer, which could explain
both the rapid initial adsorption and the small amount of adsorbed
protein when the surface is saturated. If this were the case, a site-limited
adsorption model like the Langmuir model may be better for modeling
adsorption on SiPEG. Our prototype instrument did not have the sensitivity
to perform a more thorough study of adsorption on SiPEG at low solution
concentrations. Future systems based on whispering gallery mode technology
have the potential to study the adsorption of proteins on SiPEG surfaces
in greater detail, which could lead to greater understanding as to
why SiPEG resists protein adsorption. 


\subsubsection{Cell Growth and Survival on FN-Coated Alkylsilane Monolayers}

Both the embryonic hippocampal neurons and myocytes showed significantly
better survival on DETA surfaces than 13F surfaces. However, measurements
with the WGM biosensor indicated that the amount of adsorbed protein
measured on the 13F surfaces was comparable to that of DETA, indicating
that the conformation of adsorbed FN, and its function, was just as
important as the quantity of FN for cell survival. This is consistent
with the postulate made above that the silane monolayers are able
to rearrange to accommodate more protein and that the DETA surface,
being charged and hydrophilic, could accommodate the functional conformation
of the FN so little or no denaturation would occur. Conversely, the
hydrophobic side chains of the 13F could rearrange to allow for the
adsorption of more protein but would also promote the exposure of
the protein's hydrophobic core, thus denaturing the protein and deactivating
its biological activity as postulated previously for hydrophobic surfaces
\cite{Keselowsky2003}. 

Results from the skeletal myocyte culture (Table~\ref{tab:Cell counts on FN})
provided further information about the bioactivity of absorbed FN.
Skeletal myocytes are precursor cells that fuse and differentiate
into contractile myotubes. This differentiation is mediated by, among
other factors, the interaction of the $\alpha5\beta1$ integrin receptors
on the surface of the myocytes with the cell binding domain of the
FN molecule \cite{Michael2003}. Without this interaction, myotubes
do not form. The muscle cell culture on 13F indicated that while a
significant number of cells survived, no myotubes formed. The number
of dead cells was actually less than that of SiPEG or DETA, and the
fact that so many cells survived on the 13F substrate indicates that
there was enough protein adsorbed to the surface to promote adhesion.
However, the lack of myotube formation indicates that FN adsorbed
on 13F had reduced biological activity due to its denaturation and
did not activate the $\alpha5\beta1$ integrin signaling pathways
necessary for myotube differentiation. These proliferation and differentiation
results are consistent with previously reported results \cite{Michael2003}.
The lack of survival of cells on SiPEG surfaces can be attributed
to the small amount of adsorbed FN and the possibility that the protein
was also denatured. 


\subsection{Glucose Oxidase Adsorption}

The highest saturation surface density of adsorbed protein was found
on the hydrophobic 13F surface, and the lowest was found on the hydrophilic
SiPEG surface. Intermediate amounts of protein adsorbed on the hydrophilic
charged surfaces, glass and DETA. These results are consistent with
the general consensus of previous protein adsorption studies~\cite{Rabe2010}.
The initial kinetics of adsorption were slowest on the SiPEG surface,
followed by glass. The 13F surface had the highest rate of adsorption
at $100\,\mu g/ml$, but the initial rate of adsorption on the DETA
surface was slightly higher at $10\,\mu g/ml$.

In a previous study, glucose oxidase was adsorbed onto plasma-polymerized
thin films of hexamethyldisiloxane (HDMS) on silica substrates~\cite{Muguruma2006}.
HDMS is a hydrophobic polymer, and its surface properties were modified
by exposure to nitrogen and oxygen plasma, as shown in Table \ref{tab:Surface Properties}.
Adsorption was measured with a quartz crystal microbalance (QCM),
although only the saturation values were quantified. The size and
shape of adsorbed GO was also measured by AFM. It was assumed that
GO in solution could be approximated by an ellipsoid with a major
axis of $10-14\, nm$ and a minor axis of $6-8\, nm$. It was found
that GO adsorbed to native HDMS surfaces with the major axis parallel
to the surface, while GO adsorbed to HDMS-O and HDMS-N with the major
axis normal the surface. The maximum area occupied by an adsorbed
molecule (its ''footprint'') would be $88\, nm^{2}$ when adsorbed
with major axis parallel to surface, while the minimum area would
be $28\, nm^{2}$ when adsorbed with the major axis normal to surface.
Previous AFM studies of GO adsorbed on gold also found that it could
be represented by an ellipse with a major axis of $14-18\, nm$ and
a minor axis of $5-8\, nm$~\cite{Quinto1998}. Estimates of protein
size from crystal structure data or AFM measurements on gold surfaces
should be considered {}``order of magnitude'' estimates, since the
size of proteins is heavily dependent upon their environment. 

%
\begin{table}
\caption{\label{tab:Surface Properties}Properties of surfaces used in these
experiments and similar surfaces used in previous experiments.}
\begin{tabular}{lccc}
 & Water contact angle & Zeta potential ($mV$) & Reference\tabularnewline[\doublerulesep]
\cline{2-4} 
\noalign{\vskip\doublerulesep}
Glass & $<5^{\circ}$ & $-25$ & \cite{Kirby2004,Wilson2011a}\tabularnewline
\noalign{\vskip\doublerulesep}
DETA & $49\pm2^{\circ}$ & $10$ & \cite{Wilson2011a,Metwalli2006}\tabularnewline
\noalign{\vskip\doublerulesep}
13F & $94\pm2^{\circ}$ & $-15$ & \cite{Stenger1992,Tandon2008}\tabularnewline
\noalign{\vskip\doublerulesep}
SiPEG & $38\pm2^{\circ}$ &  & \cite{Wilson2011a}\tabularnewline
\noalign{\vskip\doublerulesep}
HDMS & $>90^{\circ}$ &  & \cite{Muguruma2006}\tabularnewline
\noalign{\vskip\doublerulesep}
HDMS-N & $<50^{\circ}$ & $20$ & \cite{Muguruma2006}\tabularnewline
\noalign{\vskip\doublerulesep}
HDMS-O & Hydrophilic & $-40$ & \cite{Muguruma2006}\tabularnewline
\noalign{\vskip\doublerulesep}
\end{tabular}%
\end{table}



\subsubsection{Glucose Oxidase on Glass}

The Langmuir-type models fitted the kinetics of GO on glass slightly
better than the RSA-type models, and the two-stage models fitted the
data better than the single-stage models. The enzyme activity data
indicated that GO adsorbed on bare glass retained about 25\% of its
activity at a solution concentration of $10\,\mu g/ml$. When the
solution concentration was increased to $100\,\mu g/ml$, almost all
of the adsorbed enzyme remained active. Both of the two-stage models
predicted the amount of active enzyme correctly at $10\,\mu g/ml$,
but under-predicted the amount of active enzyme at $100\,\mu g/ml$
with the assumption that enzyme in state $\alpha$ was active and
enzyme in state $\beta$ was inactive. If it was assumed that enzyme
in both states remains active, both of the two-stage models would
accurately predict the amount of active enzyme at $100\,\mu g/ml$.
It has been shown that some {}``soft'' enzymes lose most of their
structure and activity when adsorbed to glass, while {}``hard''
enzymes retain their structure and function after adsorption~\cite{Zoungrana1997,Welzel2002}.
An alternative hypothesis is that an initial layer of enzyme adsorbed
to the surface in a highly spread-out state, resulting in a very low
surface density and low activity, followed by a second layer of adsorbed
enzyme that retained its activity. This could explain the low fraction
of active enzyme at $10\,\mu g/ml$, and the high fraction of active
enzyme at $100\,\mu g/ml$. Modeling this process would require a
model that incorporated both a post-adsorption transition and multi-layer
adsorption.


\subsubsection{Glucose Oxidase on DETA}

The model fits to the kinetic data for GO on DETA were not as good
as the fits to the kinetic data for GO on glass. The RSA-type model
with a post-adsorption transition provided the best fit to the kinetic
data, as determined by the sum of squared error. It was assumed that
the DETA surface did not induce proteins to denature upon adsorption,
and this assumption was consistent with the data from the enzyme activity
assay at $10\,\mu g/ml$. When the solution concentration was increased
to $100\,\mu g/ml$, approximately half of the adsorbed GO apparently
lost its activity. It is unlikely that this reduction was caused by
denaturation, since adsorbed proteins are more likely to retain their
native conformation when the rate of adsorption is higher~\cite{Latour2005}.
Further, the fitted parameter values for the two-stage RSA-type model
were fairly implausible. The fitted transition rate constant was two
orders of magnitude higher than the transition rate constant for the
glass surface, and the surface area occupied by a molecule in state
$\beta$ was 16 times larger than a molecule in state $\alpha$. A
more likely hypothesis for the 50\% reduction in activity is that
a second layer of adsorbed enzyme formed, preventing the substrate
solution from reaching the lower layer of enzyme. The two-layer model
provided the best prediction of the amount of active adsorbed enzyme
at both concentrations, although it did not fit the kinetic data quite
as well as the RSA-type two-stage model. 


\subsubsection{Glucose Oxidase on 13F}

The kinetic data for GO on 13F was fitted well by both the Langmuir-type
model with a post-adsorption transition and the two-layer model. Although
the SSE was slightly lower for the two-layer model, the Langmuir-type
model provided a much better prediction of the amount of active adsorbed
enzyme. The experimental data showed that almost none of the adsorbed
enzyme was active at a solution concentration of $10\,\mu g/ml$.
When the solution concentration was increased to $100\,\mu g/ml$,
about 75\% of the adsorbed enzyme retained its activity. GO that adsorbs
from a higher solution concentration may experience less denaturation
because it is sterically hindered from spreading out on the surface
by the high density of adsorbed enzyme molecules. Surprisingly, the
fitted parameters for the Langmuir-type model with a post-adsorption
transition indicated the area occupied by a molecule in the final
state was not much larger than that of a molecule in the initial state.


\subsubsection{Glucose Oxidase on SiPEG}

Despite its sensitivity, the WGM sensor was unable to detect the adsorption
of GO on an SiPEG surface reliably at a solution concentration of
$10\,\mu g/ml$. At $100\,\mu g/ml$, GO adsorbed on the SiPEG surface,
but at a relatively low surface density. The enzyme activity assay
did not detect any significant amount of active enzyme for either
solution concentration. This may be because the amount of activity
was below the detection limit of the assay, or because the enzyme
did not retain its activity on SiPEG. It is well known that SiPEG
surfaces resist protein adsorption, but denaturation on SiPEG has
not been previously reported. Further investigation is necessary to
answer this question. Both the RSA and Langmuir single-layer models
fitted this data set well.


\subsection{Fitting Kinetic Models to Experimental Data}

Fitting models to the experimental data enabled us to test hypotheses
about the mechanisms of adsorption on various surfaces. Formulating
a hypothesis as a model, fitting it to the experimental data and evaluating
the quality of fit provided a quantitative methodology to compare
hypotheses. When several models fitted the data equally well, the
quantitative enzyme activity data for adsorbed glucose oxidase provided
a valuable test to support or reject a particular model. 

It was assumed that the kinetic constants did not vary across the
limited concentration range in this study. Therefore, a single set
of parameters was fitted to multiple concentrations for a single surface.
Fitting more concentrations, while holding the number of fitted parameters
constant, increased the possibility of finding a unique combination
of parameters that minimized the mean-squared error. A model with
too many parameters can have multiple parameter sets with equivalent
optimal fits, much like an under-determined system of linear equations
with multiple solutions. Although a lower SSE could have been achieved
by fitting data at each concentration individually, the likelihood
of finding non-unique parameters would have increased.

Finding a unique combination of parameters that provides a {}``best
fit'' to the data is equivalent to finding a global minimum in an
optimization problem. While this is not too difficult for the basic
Langmuir and RSA models (which have only three parameters each), it
becomes more challenging for the multi-stage and multi-layer models
(with five parameters each). It is easy to formulate even more complex
adsorption models, but the addition of each parameter increases the
difficulty of finding a global minimum in the error function. The
danger of a very complex model with many variable parameters is that
it can be fitted to almost any data set but few useful conclusions
can be drawn from the results! If models with more parameters must
be used, it will be necessary to find additional experimental methods
that can provide additional data or decrease the size of the search
space. It may also be necessary to use novel nonlinear optimization
methods, such as genetic algorithms, to optimize the fit of the model
to the data.


\subsection{Conclusions}

Utilization of the WGM sensor enabled, for the first time, quantitative
analysis of protein adsorption on silane monolayers, which are commonly
used as substrates for cell culture. The sensitivity of this technology
can be readily enhanced by a number of methods, such as fabricating
smaller microspheres and using a laser with a shorter wavelength \cite{Vollmer2008a}
or coating the glass microsphere with a high-index wave-guiding layer
\cite{Teraoka2006}. WGM biosensors have the potential to help answer
difficult questions in biomaterials research. For example, PEG/OEG
monolayers are resistant to protein adsorption but have not been shown
to be useful as long-term biocompatible coatings for implants and
medical devices. Many people would benefit from the development of
a surface coating that reduces or eliminates the immune response to
a foreign body, such as an implanted medical device. Computational
fluid dynamics simulations were used in the design and characterization
of the flow cell for the WGM biosensor. Many previous experimental
studies used simple approximate models of transport limitations or
neglected them altogether. The results of the CFD simulations were
incorporated into the process of fitting kinetic models to fibronectin
adsorption data, although this step was unnecessary for glucose oxidase.
In this work four existing adsorption models were formulated in terms
of the same variables, quantitatively compared and fitted to experimental
data. This process enabled definitive conclusions to be drawn about
the mechanisms of adsorption and inspired new directions for future
research.
