%% LyX 1.6.10 created this file.  For more info, see http://www.lyx.org/.
%% Do not edit unless you really know what you are doing.
\documentclass[letterpaper,english,PhD]{UCFthesis}
\usepackage[T1]{fontenc}
\usepackage[latin9]{inputenc}
\setcounter{secnumdepth}{3}
\setcounter{tocdepth}{3}
\usepackage{amsmath}
\usepackage{graphicx}
\usepackage{amssymb}

\makeatletter

%%%%%%%%%%%%%%%%%%%%%%%%%%%%%% LyX specific LaTeX commands.
\pdfpageheight\paperheight
\pdfpagewidth\paperwidth

\DeclareRobustCommand{\greektext}{%
  \fontencoding{LGR}\selectfont\def\encodingdefault{LGR}}
\DeclareRobustCommand{\textgreek}[1]{\leavevmode{\greektext #1}}
\DeclareFontEncoding{LGR}{}{}
\DeclareTextSymbol{\~}{LGR}{126}
%% Because html converters don't know tabularnewline
\providecommand{\tabularnewline}{\\}

\@ifundefined{showcaptionsetup}{}{%
 \PassOptionsToPackage{caption=false}{subfig}}
\usepackage{subfig}
\makeatother

\usepackage{babel}

\begin{document}

\chapter{MODELING THE KINETICS OF PROTEIN ADSORPTION}


\section{Introduction}

Many studies to date have focused on the adsorption behavior of proteins
using alkanethiol SAMs as model surfaces, and multiple theoretical
models with varying degrees of complexity have been proposed to describe
the kinetics observed. These surfaces are convenient in that they
are relatively easy to prepare, present highly ordered monolayers
with well-defined composition, and are compatible with integrated
electrodes and other sensor systems utilizing metal coated surfaces,
such as surface plasmon resonance (SPR). In-depth discussions of protein
adsorption on alkanethiol SAMS are easily found in the literature.
Less attention, however, has been given to alkylsilane monolayers,
perhaps because they lack the highly ordered packing formed by alkanethiol
SAMs (resulting in less well defined surfaces), or perceived difficulty
in the preparation of well-characterized alkylsilane surfaces. This
is somewhat unfortunate as alkylsilanes represent a broad and useful
class of compounds that are used in an increasing variety of biomedical
and biotechnological applications. For this reason a biosensor system
utilizing whispering gallery mode (WGM) technology, where the active
sensor is typically a silica disk, ring, toroid, or sphere/spheroid,
can provide new insights into the adsorption behavior of biomolecules
onto alkylsilane-modified surfaces. 


\subsection{Whispering Gallery Mode Biosensing}

WGM biosensing is based on monitoring the frequency shift of an optical
resonance excited inside a glass microsphere \cite{Vollmer2005,Vollmer2002}.
Near-infrared light is evanescently coupled to a glass microsphere
with a radius of 50-200 \textgreek{m}m from a tapered optical fiber,
which is connected to a tunable distributed feedback (DFB) laser at
one end and a photodetector at the other. The laser and detector are
used to precisely monitor changes in the resonant wavelength of the
microsphere. As proteins or other material accrete at the surface
of the microsphere, the effective radius of the sphere increases,
resulting in a red shift of the resonant wavelength that can be quantified
and used to calculate the average surface density of adsorbed material.
Even with a simple experimental configuration \cite{Vollmer2002}
it has been shown that a detection limit of $\sim1pg/mm^{2}$ can
be readily achieved. This is ten times more sensitive than an SPR
biosensor and theoretical calculations predict the ultimate detection
limit of the method to be close to the single molecule level \cite{Armani2003,Vollmer2008,Vollmer2008a,Arnold2003}.
These qualities make WGM biosensing an ideal method for studying protein
adsorption, as the dynamic range of the method allows measurements
to be performed in concentration regimes that previously were unattainable.
To date, optical resonators of this kind have been applied to a variety
of biosensing applications with great effect. In addition to the inherent
sensitivity of the method, standard CMOS technology can be applied
to fabricate arrays of resonators on silicon wafers that provide a
scalable multiplex sensing capability for detecting multiple biological
or chemical markers from a single sample in parallel \cite{Luchansky2010,Washburn2009}.
Furthermore it has been shown that these measurements can be performed
in complex samples such a blood plasma and serum \cite{Luchansky2011}.
This provides an added level of complexity as {}``real-world'' samples
such a plasma and serum contain a mixture of hundreds of proteins,
which may non-specifically bind to a sensor giving inaccurate readings
or false positive measurements. For this reason it is critical to
understand the underlying processes that govern the adsorption of
biomolecules to silica substrates and the coatings used to functionalize
them and their subsequent interactions with cells and tissue constructs.


\subsection{Fibronectin}

The unique capabilities of the WGM sensor were utilized to quantify
the kinetics of the adsorption of fibronectin (FN) onto engineered
surfaces. Cell studies were then done to evaluate the biological activity
of the adsorbed FN \cite{Keselowsky2004}. Fibronectin is an important
protein in the extracellular matrix (ECM) that mediates the interaction
of cells with surfaces, but its activity has been shown to be influenced
by its surface structure \cite{Lan2005,Michael2003}. Since the adsorption
of FN has been widely studied, the results from the WGM instrument
could be compared to an extensive amount of data. 

Fibronectin (FN) is a physiologically important protein in vertebrates.
It is abundant in plasma and other bodily fluids, and plays an important
role in the extracellular matrix. The structure of fibronectin is
complex. The primary structure of FN is a chain composed of three
types of repeated modules. Although only one gene codes for FN, alternative
splicing of the pre-mRNA results in numerous variants in which modules
are added or deleted \cite{Pankov2002}. X-ray crystallography has
been used to determine the secondary and tertiary structure of the
three types of modules. The complete FN molecule has not been crystallized,
so its secondary and tertiary structure are unknown. It is likely
that the secondary and tertiary structure are highly dependent on
the local environment. In solution, fibronectin exists as a dimer
with two identical subunits linked by disulfide bonds. In the extracellular
matrix, fibronectin is assembled into a fibrillar network \cite{Mao2005}.


\subsection{Glucose Oxidase}

The WGM biosensor was also used to quantify the kinetics of adsorption
of glucose oxidase (GO) onto alkylsilane surfaces. The activity of
the adsorbed enzyme was measured to provide more information about
its conformation on the surface. Glucose oxidase is an important and
useful enzyme from a technological standpoint. GO which has been adsorbed
or covalently attached to electrodes forms the basis for amperometric
glucose sensors\cite{Wang2007}. Advances in portable blood glucose
sensors have enabled diabetics to monitor and control their blood
glucose levels, minimizing the risk of complications from the disease
\cite{Oliver2009}. Perhaps because of its important role in biosensors,
glucose oxidase has been thoroughly characterized. This makes it an
ideal candidate for testing and validating the accuracy of the WGM
biosensor system. 

GO is a dimeric glycoprotein that is composed of two identical subunits
\cite{Wohlfahrt1999}. The crystal structure of GO from Aspergillus
niger is available at the Protein Data Bank (1CF3), and its bounding
box is 6.0 nm x 5.2nm x 7.7 nm. Reported values for the molecular
mass of GO range from 152kDa \cite{Keilin1948} to 186kDa \cite{Swoboda1965},
depending upon the method of purification that was used. The isoelectric
point of GO is 4.2 \cite{Pazur1964}. Many previous studies have focused
on the effect of different surface chemistries on the structure and
activity of adsorbed glucose oxidase \cite{Fears2009,Dong1997}. Atomic
force microscopy (AFM) studies have been performed to determine the
size and shape of GO adsorbed on various surfaces \cite{Muguruma2006,Otsuka2004}.
Although the immobilization of glucose oxidase has been widely studied,
only qualitative results have recently been reported for the kinetics
of adsorption \cite{Muguruma2006}.


\subsection{Kinetic Models of Protein Adsorption}

Kinetic models have been widely used to model protein adsorption.


\subsection{Summary}

The glass WGM resonator was modified with alkylsilane monolayers presenting
well-defined surface chemistries: 2-{[}Methoxypoly(ethyleneoxy)propyl{]}trimethoxysilane
(SiPEG), (3-trimethoxysilyl propyl) diethyltriamine (DETA), and 1,1,2,2-perfluorooctyl
trichlorosilane (13F). The WGM biosensor was then used to quantitatively
study the adsorption of FN at varying concentrations onto SiPEG, DETA
and 13F alkylsilane SAMs. The WGM biosensor incorporated a flow cell,
which minimized the effect of transport limitations on protein adsorption,
and this, along with the inherent sensitivity of the method, allowed
the kinetics of adsorption of FN and GO to be measured at concentrations
lower than those that have previously been reported {[}13{]}. The
kinetic data provided additional information about the mechanism of
adsorption that is not available from equilibrium experiments. The
well-known RSA model {[}15{]} was fitted to the measured kinetic curves,
and the resulting parameters were used to assist in interpreting the
results. In addition, a more sophisticated model featuring a post-adsorption
transition was fitted to the experimental data {[}16, 17{]}. To maximize
the accuracy of the kinetic parameters, computational fluid dynamics
(CFD) simulations were used to study limitations in the transport
of protein to the sensor surface. For determining the biological activity
of the adsorbed FN, neuronal and skeletal muscle cells were cultured
on the SiPEG, DETA, and 13F surfaces in a serum-free culture system
{[}18, 19{]}. Glass and silicon oxide surfaces are much more common
than metal surfaces in cell culture applications, so silane surface
modification is of greater practical importance than thiol surface
modification for tissue engineering. WGM biosensing provides a unique
capability to quantify protein adsorption on silane-modified surfaces
for the purpose of understanding the interactions between tissues
and tailored interfaces. 


\section{Methods and Materials}


\subsection{Experimental Methods}

A whispering gallery mode sensor system was constructed as described
in \cite{Wilson2011,Wilson2009}. A schematic overview of the system
is shown in Figure \ref{fig:WGM System Diagram}.%
\begin{figure}
\includegraphics{Figures/WGM_System_Overview}

\caption{\label{fig:WGM System Diagram}Schematic diagram of the whispering
gallery mode biosensor}


%
\end{figure}
 Bovine plasma fibronectin in solution was obtained from Sigma-Aldrich
(F1141) and diluted to the desired concentrations with 50mM phosphate
buffered saline (PBS) at pH 7.4.


\subsection{Analysis of Transport in the WGM Biosensor}

The rate of adsorption on a surface may be limited by the rate at
which the chemistry of adsorption takes place, the rate of transport
to the surface, or both. Computational fluid dynamics (CFD) was used
to model the transport of protein in the flow cell of the WGM biosensor.
CFD simulations were run with a CFD-ACE+ multiphysics solver (ESI
Software, Huntsville, AL) to determine the influence of transport
on the adsorption of protein on the resonator surface. Before beginning,
a Reynolds number calculation was performed to confirm that flow in
the device would be entirely laminar. For all CFD simulations, the
size of the mesh and the time step were refined until the simulation
results did not change significantly. Upwind differencing was used
to approximate velocity derivatives, a 2nd order limiter was used
to approximate concentration derivatives, and the Euler method was
used for time stepping in transient simulations. A multi-scale approach
was used to obtain high-resolution results while keeping the simulation
run time reasonable. A large-scale model of the flow system, including
the 40 cm of tubing between the three-way stopcock and the flow cell,
was created using CFD-GEOM and discretized using a structured mesh.
This model did not include the resonator and waveguide, as they have
a minimal effect on the overall flow in the channel. A steady-state
simulation was performed first to determine the flow field. Transient
simulations were then performed in which the concentration at the
inlet was suddenly increased from zero to the target concentration.
The concentration of protein was monitored at a point in the flow
cell at the location of the resonator, and the simulation was run
until the concentration at this point reached the target value. It
was assumed that the low concentrations of protein used in these experiments
did not affect the flow field significantly, so the transient simulations
utilized the flow field from the steady-state simulation to reduce
computation time. 

Two additional CFD models were created to simulate the transport of
protein near the resonator. A detailed three-dimensional model was
used to model the flow field around the resonator and the waveguide.
This model ran too slowly to be used for transient simulations of
adsorption, so a simplified model was created with a two-dimensional
axisymmetric geometry. Because the axisymmetric geometry models a
channel with a circular cross section while the actual channel has
a square cross section, the flow velocity at the inlet was adjusted
so the velocity near the resonator matched the results from the three-dimensional
simulation. Data from the simulation of the whole flow cell was used
to set the concentration over time at the inlet of the axisymmetric
model. A surface reaction was defined on the surface of the resonator
using the Langmuir model built in to the Biochemistry module of ACE+.
The purpose of this reaction was to deplete the protein near the resonator
at an appropriate rate, rather than to model the actual chemistry
of adsorption. An association rate constant of $1.44\times10^{6}L\, mol{}^{-1}s^{-1}$,
a dissociation rate constant of $8.9\times10^{-4}s^{-1}$, and a maximum
density of adsorption sites of $4.4\times10^{-9}mol\, m^{-2}$ were
used to match the maximum adsorption rate that was observed in the
experimental data for DETA and 13F. A steady-state simulation was
used to establish the flow field and a transient simulation was used
to determine the concentration of protein near the surface of the
resonator. The adsorption simulation mandated a fairly small time
step (on the order of $10^{-4}s$), so it was impractical to run the
simulation to equilibrium due to the large number of time steps required.
The system was simulated for 150-300 seconds (of simulated time).
It was assumed that the near-surface concentration increased linearly
from the end of the CFD simulation to the target concentration.


\subsection{Modeling the Kinetics of Protein Adsorption}

It is well established that single-layer adsorption models can be
stated in the general form\begin{equation}
\frac{d\theta}{dt}=k_{a}c\Phi\left(\theta\right)-k_{d}\theta\label{eq:Single Layer Kinetics}\end{equation}
where $\theta$ is the fraction of the surface covered by adsorbed
particles, $c$ is the concentration of protein in solution near the
surface, $k_{a}$ is the adsorption rate constant, and $k_{d}$ is
the desorption rate constant \cite{Andrade1986}. The function $\Phi$
represents the blocking effect of adsorbed particles. For the Langmuir
model, the blocking function is simply $\Phi\left(\theta\right)=1-\theta/\theta_{\infty}$.
The blocking function for the random sequential adsorption (RSA) model
is not available in analytic form. A widely used approximation for
the random sequential adsorption of spherical particles is\begin{equation}
\Phi\left(\theta\right)=\frac{\left(1-\bar{\theta}\right)^{3}}{1.0-0.812\bar{\theta}+0.2335\bar{\theta}^{2}+0.0845\bar{\theta}^{3}}\label{eq:RSA Blocking Function}\end{equation}
where $\bar{\theta}=\theta/\theta_{\infty}$ \cite{Schaaf1989}.


\subsubsection{RSA-type model of adsorption with transition}

Single-layer adsorption models cannot adequately predict the kinetics
of adsorption for many combinations of proteins and surfaces. More
complex models have been developed to account for these experimental
results. Because many surfaces cause proteins to denature upon adsorption,
it is common to model adsorption with a post-adsorption transition.
This process can be described by the kinetic equations\begin{align}
\frac{d\rho_{\alpha}}{dt} & =k_{a}c\Phi_{\alpha}-k_{s}\rho_{\alpha}\Psi_{\alpha\beta}-k_{d}\rho_{\alpha}\label{eq:Adsorption Transition Kinetics}\\
\frac{d\rho_{\beta}}{dt} & =k_{s}\rho_{\alpha}\Psi_{\alpha\beta}\label{eq:Adsorption Transition Kinetics 2}\end{align}
The form of the blocking functions depends upon the assumptions of
the model. When the particles are spherical, scaled particle theory
(SPT) can be used to derive the following blocking functions \cite{Brusatori1999}:
\begin{align}
\Phi_{\alpha} & =\left(1-\theta\right)\exp\left[-\frac{2\left(\overline{\rho_{\alpha}}+\Sigma\overline{\rho_{\beta}}\right)}{1-\theta}-\frac{\overline{\rho_{\alpha}}+\overline{\rho_{\beta}}+\left(\Sigma-1\right)^{2}\overline{\rho_{\alpha}}\overline{\rho_{\beta}}}{\left(1-\theta\right)^{2}}\right]\label{eq:Phi_alpha}\\
\Psi_{\alpha\beta} & =\exp\left[-\frac{2\left(\Sigma-1\right)\left(\overline{\rho_{\alpha}}+\Sigma\overline{\rho_{\beta}}\right)}{1-\theta}-\frac{\left(\Sigma^{2}-1\right)\left[\overline{\rho_{\alpha}}+\overline{\rho_{\beta}}+\left(\Sigma-1\right)^{2}\overline{\rho_{\alpha}}\overline{\rho_{\beta}}\right]}{\left(1-\theta\right)^{2}}\right]\label{eq:Psi_alpha_beta}\end{align}
The following non-dimensional variables are defined: $\overline{\rho_{\alpha}}=\rho_{\alpha}\pi R_{\alpha}^{2}$,
$\overline{\rho_{\beta}}=\rho_{\beta}\pi R_{\alpha}^{2}$, $\theta=\overline{\rho_{\alpha}}+\Sigma^{2}\overline{\rho_{\beta}}$,
and $\Sigma=R_{\beta}/R_{\alpha}$.

Since the blocking functions derived from SPT describe spherical particles,
they can be directly compared to the RSA blocking function. When $k_{s}=0$,
equations \ref{eq:Adsorption Transition Kinetics} and \ref{eq:Adsorption Transition Kinetics 2}
reduce to equation \ref{eq:Single Layer Kinetics}. By setting $\rho_{\beta}=0$
and $\Sigma=1$, equations \ref{eq:RSA Blocking Function} and \ref{eq:Phi_alpha}
can be compared.

To quantitatively compare different multi-layer models, it is necessary
to transform equations from different sources to use a common set
of variables. The surface number densities $\rho_{i}$ used in equations
\ref{eq:Adsorption Transition Kinetics} and \ref{eq:Adsorption Transition Kinetics 2}
can be converted to fractional surface coverage by multiplying by
the area covered by an adsorbed particle in state $i$, $\sigma_{i}$,
to obtain:\begin{align}
\frac{d\theta_{\alpha}}{dt} & =k_{a}\,\sigma_{\alpha}c\Phi_{\alpha}-k_{s}\theta_{\alpha}\Psi_{\alpha\beta}-k_{d}\theta_{\alpha}\label{eq:dtheta_alpha dt}\\
\frac{d\theta_{\beta}}{dt} & =k_{s}\Sigma^{2}\theta_{\alpha}\Psi_{\alpha\beta}\label{eq:dtheta_beta dt}\end{align}



\subsubsection{Langmuir-type model of adsorption with transition}

A different model of adsorption with a post-adsorption transition
has been used to fit the adsorption kinetics of the fibronectin fragment
$\textrm{FNII\ensuremath{I_{7-10}}}$ \cite{Michael2003}. Written
using the same variables as equations \ref{eq:Adsorption Transition Kinetics}
and \ref{eq:Adsorption Transition Kinetics 2}, the equations that
describe this model are\begin{align}
\frac{d\rho_{\alpha}}{dt} & =k_{a}\, c\, A_{av}-k_{s}\,\rho_{\alpha}\, A_{av}-k_{d}\,\rho_{\alpha}\label{eq:Michael kinetics 1}\\
\frac{d\rho_{\beta}}{dt} & =k_{s}\,\rho_{\alpha}\, A{}_{av}\label{eq:Michael kinetics 2}\end{align}
where $Y_{i}$ is the surface density ($ng/cm^{2}$) of adsorbed protein
in each state. $A_{av}$ is the surface area ($cm^{2}$) available
for adsorption, which is given by\[
A_{av}=A_{total}\left(1-f\,\sigma_{1}\,\rho_{1}-f\, b\,\sigma_{1}\,\rho_{2}\right)\]
where $b=\sigma_{\beta}/\sigma_{\alpha}$. Since $\theta_{i}=\sigma_{i}\, f\,\rho_{i}$
and $f=N_{A}/M$ ($molecules/ng$), this expression can be written
$A_{av}=A_{total}\left(1-\theta\right)$, where $\theta=\theta_{\alpha}+\theta_{\beta}$.
Equations \ref{eq:Michael kinetics 1} and \ref{eq:Michael kinetics 2}
can be written in terms of fractional surface coverage by multiplying
both sides by $\sigma_{i}\, f$ to obtain:\begin{align}
\frac{d\theta_{1}}{dt} & =k\,\sigma_{1}\,\sigma A_{total}\, f\, c\left(1-\theta\right)-s\, A_{total}\,\theta{}_{1}\left(1-\theta\right)-r\,\theta_{1}\label{eq:Langmuir two stage 1}\\
\frac{d\theta_{2}}{dt} & =s\, A_{total}\, b\,\theta_{1}\,\left(1-\theta\right)\label{eq:Langmuir two stage 2}\end{align}
It is clear that the blocking function for this model is actually
the Langmuir blocking function, with $\theta_{\infty}=1$.


\subsection{Fitting Kinetic Models to Experimental Data}

The modeled adsorption curves were fitted to the experimentally measured
curve using the least squares fitting routine from Scipy, which uses
a modified version of the Levenberg-Marquardt algorithm \cite{Jones2001-}.
For each surface, the parameters $k_{a}$, $k_{d}$, $k_{s}$, $r_{\alpha}$
and $r_{\beta}$ were fitted simultaneously for all of the available
concentrations. The sum of squared errors (SSE) was computed for each
type of surface and normalized by dividing by the total number of
data points in each data set so that the SSE values from different
data sets could be compared.


\section{Results}

A comparison of the blocking functions of the Langmuir and RSA models
is shown in Figure \ref{fig:Blocking function comparison}.%
\begin{figure}
\includegraphics{Plots/ASF_comparison}

\caption{\label{fig:Blocking function comparison}Blocking functions from the
RSA model and scaled particle theory}
%
\end{figure}
 The blocking function for protein in the initial state for the model
derived from scaled particle theory is also shown. It can be seen
that the first-level blocking function derived from SPT is similar
but not identical to the RSA blocking function, especially at higher
values of fractional surface coverage.

The kinetics of adsorption based on the Langmuir blocking function
were compared to the kinetics modeled by the SPT blocking functions,
and the results are shown in Figure \ref{fig:SPT vs Langmuir kinetics}.
%
\begin{figure}[h]
\subfloat[\label{fig:SPT kinetics}Blocking function from scaled particle theory]{\includegraphics[scale=0.75]{Plots/SPT_kinetics}}\subfloat[\label{fig:Langmuir two-stage kinetics}Langmuir two-stage model]{\includegraphics[scale=0.75]{Plots/Langmuir_twostage_kinetics}}

\caption{\label{fig:SPT vs Langmuir kinetics}Comparison of kinetics predicted
by SPT blocking function (a) and Langmuir blocking function (b) for
$k_{a}=1$, $k_{s}=\pi$, $k_{d}=\pi$, $r_{\alpha}=1$, $\Sigma=1.2$,
and $c=1$.}
%
\end{figure}
 The parameters used for the comparison were taken from Figure 2a
from \cite{Brusatori1999}. Note that Figure 2a from \cite{Brusatori1999}
cannot be directly compared directly with Figure \ref{fig:SPT vs Langmuir kinetics}
in this work, because $\theta_{\beta}\neq\overline{\rho_{\beta}}$.
For the Langmuir model, $\theta_{\infty}=0.547$ was used for consistency
with the RSA and SPT blocking functions.


\subsection{Transport Analysis of the WGM Biosensor}

The steady-state velocity magnitude predicted by the model of the
whole flow cell is shown in Figure , along with a close-up view of
the simulated flow field near the junction of the resonator and waveguide.
It can be seen that the flow field around the resonator was symmetric
about the long axis of the resonator, as intended. This configuration
also ensured that the shear rate was constant in the region where
the evanescent wave was excited, minimizing any shear rate effects
on the adsorption of protein from solution. Also, it should be noted
that the waveguide coupled to the resonator created a negligible disturbance
of the flow field. The evolution of bulk concentration in the flow
cell over time is shown in Figure \ref{fig:CFD Bulk Conc}.%
\begin{figure}
\includegraphics[width=0.75\columnwidth]{Plots/CFD_flow_cell_inlet_concentration}

\caption{\label{fig:CFD Bulk Conc}CFD prediction of concentration in the flow
cell}


%
\end{figure}
 The solution concentration in the flow cell in the vicinity of the
resonator required less than 60 seconds to reach the bulk solution
concentration. The near-surface concentration predicted by the high-resolution
CFD model is shown in Figure \ref{fig:CFD near surface conc}.%
\begin{figure}
\includegraphics[width=0.75\columnwidth]{Plots/CFD_near_surface_concentration}

\caption{\label{fig:CFD near surface conc}CFD prediction of concentration
very close to the surface of the resonator}


%
\end{figure}
 The evolution of the protein concentration near the surface of the
resonator required considerably more time to reach its final value.
The effects of this difference in near surface concentration with
respect to the bulk were taken into account when fitting the model
to the experimental data.


\subsection{Modeling the Adsorption of Fibronectin on Silane Surfaces}

The kinetics of adsorption of fibronection on 13F, DETA, and SiPEG
are shown in Figure \ref{fig:FN experiments}.%
\begin{figure}
\includegraphics[width=1\columnwidth]{Plots/FN_Experimental_Data}

\caption{\label{fig:FN experiments}Measured adsorption kinetics for fibronectin
on 13F, DETA, and OEG surfaces}


%
\end{figure}
 The kinetics predicted by the models fitted to adsorption on DETA
are shown in Figure \ref{fig:FN DETA fitted}, and the fitted parameter
values are shown in Table \ref{tab:FN on DETA params}.%
\begin{table}
\caption{\label{tab:FN on DETA params}Parameter values fitted to FN on DETA}
\begin{tabular}{lccccc}
 & Langmuir & Langmuir & RSA & RSA & units\tabularnewline
 &  & two-stage &  & two-stage & \tabularnewline[\doublerulesep]
\hline
\noalign{\vskip\doublerulesep}
$k_{a}$ & $ $ & $\times10^{-6}$ & $2.05\times10^{-6}$ & $2.37\times10^{-6}$ & $cm^{3}ng^{-1}s^{-1}$\tabularnewline
$k_{s}$ & $ $ & $\times10^{-6}$ & $ $ & $1.14\times10^{-4}$ & $s^{-1}$\tabularnewline
$k_{d}$ & $ $ & $\times10^{-6}$ & $3.16\times10^{-4}$ & $2.12\times10^{-4}$ & $s^{-1}$\tabularnewline
$\sigma_{\alpha}$ & $ $ & $\times10^{-6}$ & $1.82\times10^{-12}$ & $2.03\times10^{-12}$ & $cm^{2}$\tabularnewline
$\sigma_{\beta}$ & $ $ & $\times10^{-6}$ & $ $ & $3.80\times10^{-12}$ & $cm^{2}$\tabularnewline
$SSE$ & $ $ & $ $ & $86.1$ & $86.0$ & \tabularnewline
\end{tabular}%
\end{table}
%
\begin{figure}
\includegraphics{Plots/FN_DETA_RSA_CFD}

\caption{\label{fig:FN DETA fitted}RSA model fitted to experimental data for
FN on DETA}


%
\end{figure}


The kinetics predicted by models fitted to FN adsorption on 13F are
shown in Figure \ref{fig:FN 13F fitted}, and the parameter values
for the fitted models are shown in Table \ref{tab:FN OEG fitted params}.%
\begin{table}
\caption{\label{tab:FN on 13F}Parameter values fitted to FN on 13F}
\begin{tabular}{lccccc}
 & Langmuir & Langmuir two-stage & RSA & RSA two-stage & Units\tabularnewline
\hline
$k_{a}$ & $ $ & $\times10^{-6}$ & $2.24\times10^{-6}$ & $1.99\times10^{-6}$ & $cm^{3}ng^{-1}s^{-1}$\tabularnewline
$k_{s}$ & $ $ & $\times10^{-6}$ & $ $ & $8.00\times10^{-3}$ & $s^{-1}$\tabularnewline
$k_{d}$ & $ $ & $\times10^{-6}$ & $2.49\times10^{-4}$ & $9.95\times10^{-5}$ & $s^{-1}$\tabularnewline
$\sigma_{\alpha}$ & $ $ & $\times10^{-6}$ & $1.96\times10^{-12}$ & $1.76\times10^{-12}$ & $cm^{2}$\tabularnewline
$\sigma_{\beta}$ & $ $ & $\times10^{-6}$ & $ $ & $3.03\times10^{-12}$ & $cm^{2}$\tabularnewline
$SSE$ & $ $ & $ $ & $121$ & $85.5$ & \tabularnewline
\end{tabular}%
\end{table}
%
\begin{figure}
\includegraphics{Plots/FN_13F_RSA_CFD}

\caption{\label{fig:FN 13F fitted}RSA model fitted to experimental data for
FN on 13F}
%
\end{figure}


The kinetics predicted by models fitted to FN adsorption on 13F are
shown in Figure \ref{fig:FN OEG fitted}, and the parameter values
for the fitted models are shown in Table \ref{tab:FN OEG fitted params}.%
\begin{table}
\caption{\label{tab:FN OEG fitted params}Parameter values fitted to FN on
SiOEG}
\begin{tabular}{lccccc}
 & Langmuir & Langmuir two-stage & RSA & RSA two-stage & Units\tabularnewline
\hline
$k_{a}$ & $ $ & $\times10^{-6}$ & $9.25\times10^{-7}$ & $1.12\times10^{-6}$ & $cm^{3}ng^{-1}s^{-1}$\tabularnewline
$k_{s}$ & $ $ & $\times10^{-6}$ & $ $ & $1.16\times10^{-2}$ & $s^{-1}$\tabularnewline
$k_{d}$ & $ $ & $\times10^{-6}$ & $7.14\times10^{-4}$ & $1.04\times10^{-4}$ & $s^{-1}$\tabularnewline
$\sigma_{\alpha}$ & $ $ & $\times10^{-6}$ & $1.24\times10^{-11}$ & $1.24\times10^{-11}$ & $cm^{2}$\tabularnewline
$\sigma_{\beta}$ & $ $ & $\times10^{-6}$ & $ $ & $3.42\times10^{-11}$ & $cm^{2}$\tabularnewline
$SSE$ & $ $ & $ $ & $2.7$ & $1.86$ & \tabularnewline
\end{tabular}%
\end{table}
%
\begin{figure}
\includegraphics{Plots/FN_OEG_RSA_CFD}

\caption{\label{fig:FN OEG fitted}RSA model fitted to experimental data for
FN on OEG}
%
\end{figure}



\subsection{Modeling the Adsorption of Glucose Oxidase on Silane Surfaces}


\section{Discussion}

It has been shown that the relationship between the kinetic models
stated by Brusatori and Van Tassel \cite{Brusatori1999} and Michael
et. al. \cite{Michael2003} is analogous to the relationship between
the RSA model and the Langmuir model. Although the kinetics of the
total fractional surface coverage $\theta$ are similar for both models,
the kinetics of $\theta_{\alpha}$ and $\theta_{\beta}$ are very
different. This difference could have significant implications for
modeling experimental data, if protein in state $\alpha$ functions
differently than protein in state $\beta$.

The equilibrium surface concentration of FN, as measured by the WGM
system, compares favorably with previously published results for a
solution concentration of $10\mu g/ml$. Although different measurement
techniques and surface preparations were used, the surface concentrations
on hydrophobic and hydrophilic surfaces were fairly similar among
the various references (reference \cite{Lee2006} being the only exception).
The saturation surface concentration of FN on hydrophilic neutral
surfaces measured by the WGM sensor also indicated excellent agreement
with previously published results. However, the amount of adsorbed
protein measured by the WGM system at lower concentrations was significantly
greater than the amount reported in previously published results for
hydrophobic and hydrophilic charged surfaces. In contrast, the surface
concentration value for a neutral hydrophobic surface was in good
agreement with previous results. Table 1 also shows the saturation
surface concentration of FN at a solution concentration of $1\mu g/ml$.
The discrepancy between the WGM sensor results and the other methods
at low concentrations may be explained by differences in the measurement
system, the surface chemistry, or the adsorption process. Since the
limiting surface coverages measured by the WGM sensor agree well with
other techniques for $10\mu g/ml$ and for neutral hydrophilic surfaces
at $1\mu g/ml$, the results from the WGM sensor can be considered
reliable. It is likely that if systematic errors were inherent to
the WGM method, those errors would be reflected throughout all solution
concentrations measured. 

One possible interpretation of the higher saturation values measured
at $1\mu g/ml$ and lower on 13F and DETA could be the relative packing
order of silane monolayers compared to alkanethiol monolayers. Alkanethiol
SAMs are known to create highly ordered monolayers due to their tight
packing on highly ordered gold films \cite{Prime1991}. Because of
this tight packing only the terminal functional groups of the alkanethiol
are presented at the surface, resulting in highly defined surface
chemistries. Alkylsilane monolayers, which are formed on silica surfaces,
are less tightly packed and therefore form less ordered monolayers,
which can potentially present more than just the terminal functional
group. It has been hypothesized that this may result from interaction
of electron donating groups of the silane side chain with silanol
groups at the surface resulting in reaction site-limited substrates
\cite{Stenger1992}. This can lead to incomplete monolayers that allow
interaction of protein with the unreacted substrate or allow sufficient
degrees of freedom for the silane side-chains to adopt multiple conformations,
creating less ordered monolayers that can rearrange to accommodate
the native protein structure. At the high concentrations the amount
of protein available to bind would swamp out these effects but they
would be present at the lower concentrations. This could explain why
the differences between the silane chemistry used in this work and
the alkanethiol chemistry used in previous work \cite{Keselowsky2003}
did not show up at $10\mu g/ml$. Thus, this additional degree of
freedom would allow the long side chains to rearrange to accommodate
greater protein interaction for structural stabilization and higher
coverage or to expose new surface sites for increased protein adsorption.
This would not be possible with the tightly packed alkane thiol monolayers.
However, if the observation of high saturation values from low solution
concentrations were due strictly to monolayer packing, this would
be reflected in the literature. Thus, SAM surface structure differences
are not seen as the only possible explanation, but it should be noted
that this effect could also have major consequences for protein function,
as described later. 

Another significant difference between the WGM measurement system
and previous work was that the protein was deposited on the WGM resonator
under flow conditions and the measurement was continued until saturation
was reached, while previous measurements were made after exposure
to a static FN solution for a fixed amount of time (30-60 minutes.)
The combination of high-affinity surfaces and low solution concentration
is conducive to transport-limited adsorption, which could explain
the discrepancy between WGM and static experiments for hydrophobic
and charged hydrophilic surfaces, but also give new insight into the
reasons behind why silane monolayers seem to be better cell culture
substrates than thiol monolayers. In contrast, neutral hydrophilic
surfaces have a much lower affinity for protein, so depletion of protein
near the surface would be much less of a factor, resulting in a good
match between the WGM and static measurements. 

Although the flow cell was designed to minimize transport limitations,
CFD analysis indicated that transport had some influence on the rate
of adsorption on DETA and 13F. Initially, transport was limited by
the large amount of buffer that had to be displaced from the flow
system before the protein solution could reach the resonator at full
concentration. This limitation was due to the prototype nature of
the system, and can be easily eliminated in future systems. After
the protein solution in the channel reached its target concentration,
transport was limited by the rate of diffusion across a depletion
layer that formed at the surface of the resonator. This type of limitation
is virtually unavoidable in microfluidic systems when the rate of
adsorption is high and the solution concentration is low. The flow
of water is laminar at small length scales, and the no-slip boundary
condition at the resonator surface means that transport to the surface
is almost entirely diffusive. Our modeling method was approximate
in that the evolution of the near-surface concentration over time
was computed once, based on the estimated adsorption rate, and was
not modified during the curve-fitting procedure. A more accurate method
would be to incorporate the CFD model into the fitting routine, so
that the near-surface concentration would be updated as the adsorption
rate changed \cite{Jenkins2004}. However, this method is only practical
when the CFD simulation is simple enough to run very quickly. It was
also assumed that the near-surface concentration increased linearly
from sixty seconds until the target concentration was reached. Although
this probably had some impact on the modeled kinetics, it was better
than assuming that the concentration near the resonator remained constant.
The near-surface concentration can have a large effect on the kinetics
of protein adsorption, and must be taken into account when fitting
kinetic parameters.

Both the embryonic hippocampal neurons and myocytes showed significantly
better survival on DETA surfaces than 13F surfaces. However, the amount
of adsorbed protein measured on the 13F surfaces was comparable to
that of DETA, indicating that the conformation of adsorbed FN, and
its function, was just as important as the quantity of FN for cell
survival. This is consistent with the postulate made above that the
silane monolayers are able to rearrange to accommodate more protein
and that the DETA surface, being charged and hydrophilic, could accommodate
the functional conformation of the FN so little or no denaturation
would occur. Conversely, the hydrophobic side chains of the 13F could
rearrange to allow for the adsorption of more protein but would also
promote the exposure of the protein's hydrophobic core, thus denaturing
the protein and deactivating its biological activity as postulated
previously for hydrophobic surfaces \cite{Keselowsky2003}. Results
from the skeletal myocyte culture provided further information about
the bioactivity of absorbed FN. Skeletal myocytes are precursor cells
that fuse and differentiate into contractile myotubes. This differentiation
is mediated by, among other factors, the interaction of the $\alpha5\beta1$
integrin receptors on the surface of the myocytes with the cell binding
domain of the FN molecule \cite{Michael2003}. Without this interaction,
myotubes do not form. The muscle cell culture on 13F indicated that
while a significant number of cells survived, no myotubes formed.
The number of dead cells was actually less than that of SiPEG or DETA,
and the fact that so many cells survived on the 13F substrate indicates
that there was enough protein adsorbed to the surface to promote adhesion.
However, the lack of myotube formation indicates that FN adsorbed
on 13F had reduced biological activity due to its denaturation and
did not activate the $\alpha5\beta1$ integrin signaling pathways
necessary for myotube differentiation. These proliferation and differentiation
results are consistent with previously reported results \cite{Michael2003}.
The lack of survival of cells on SiPEG surfaces can be attributed
to the small amount of adsorbed FN and the possibility that the protein
was also denatured. Utilization of the WGM sensor enables, for the
first time, quantitative analysis of protein adsorption on silane
monolayers, which are more commonly used as substrates for cell culture
than thiol monolayers. The sensitivity of this bench-top setup also
can be readily enhanced by a number of methods, such as fabricating
smaller microspheres and using a laser with a shorter wavelength \cite{Vollmer2008a}
or coating the glass microsphere with a high-index wave-guiding layer
\cite{Teraoka2006}. 

Fitting models to the experimental data provided additional insight
about the process of adsorption. It was assumed that the kinetic constants
did not vary across the limited concentration range in this study.
Therefore, a single set of parameters was fitted to multiple concentrations
for a single surface. Fitting more concentrations, while holding the
number of parameters constant, increased the possibility of finding
a unique combination of parameters that minimized the SSE. A model
with too many parameters can have multiple parameter sets with equivalent
optimal fits, much like an under-determined system of linear equations.
Although a lower SSE could have been achieved by fitting each concentration
individually, the likelihood of finding non-unique parameters would
have increased. 

The fitted parameters of the RSA model were very similar for FN adsorption
on DETA and 13F surfaces, reflecting the similar shapes of the experimental
curves. The RSA model fitted the DETA data well, indicating that the
assumptions of the RSA model were valid for the process of adsorption
on DETA. This result was confirmed by the fitting results for the
two-stage adsorption model. The sum of squared errors was only slightly
lower than the SSE value for the fitted RSA model. We concluded that
fibronectin adsorbed on DETA with a well-defined footprint, which
does not change significantly after adsorption. This result is consistent
with the well-established theory that proteins generally experience
less denaturation on a hydrophilic surface than on a hydrophobic surface
{[}31{]}. The experimental results for FN on 13F showed significant
deviations from the RSA model in the saturation region, especially
at higher solution concentrations. The two-stage model allows particles
to change size after adsorption, which significantly improved the
fit of the model to the data for FN on 13F. The fitted values for
the association constant were quite similar for DETA and 13F, but
the transition rate constant ks for the 13F surface was an order of
magnitude larger than ks for the DETA surface. The results from the
fitting process indicated that FN denatured after adsorption on 13F,
which had been previously postulated for certain hydrophobic surfaces
{[}13, 32{]} and this was also consistent with our cell culture results. 

The RSA model fitted the SiPEG data well. The value for FN adsorption
on SiPEG was lower than the values for DETA and 13F while the dissociation
rate constant was higher, which is expected for a protein-resistant
surface. This result is consistent with findings that SiPEG is an
electrostatically neutral surface that does not exhibit coulombic
attraction for proteins in solution. Surprisingly, the fitted radius
of FN adsorbed on SiPEG was more than twice the fitted radius of FN
adsorbed on DETA or 13F. For the two-stage model, the transition rate
constant for adsorption on SiPEG was significantly higher than for
the other surfaces. The fitted pre-transition radius and post-transition
radius of adsorbed FN were also larger for SiPEG than DETA or 13F.
The large radius predicted by the RSA model and the significant transition
predicted by the two-stage model seemed to indicate that FN denatures
after it adsorbs to PEG. This prediction was not consistent with the
well-known observation that proteins in contact with hydrophobic surfaces
tend to denature, while proteins in contact with hydrophilic, charged
surfaces tend to retain their native conformations. However, it also
may indicate that the SiPEG surface could be promoting the denaturation
of adsorbed proteins, which could explain why it is a cell-resistant
surface despite being hydrophilic. 

Although the SSE of the fitted two-stage model was about 30\% lower
than the SSE for the RSA model, the absolute change in SSE was relatively
small, and may not be significant. It is possible that the two extra
variable parameters (transition rate constant and post-transition
radius) are redundant for the SiPEG surface, in which case their fitted
values should not be considered significant. It is also possible that
the radius predicted by the fitting process for SiPEG is an artifact
caused by fitting the data with a model that is not well suited to
the surface chemistry. Given the assumptions of the RSA model, surface
coverage can reach saturation in only two ways: either the rate of
desorption equals the rate of adsorption, or there is no space left
on the surface for another protein to adsorb. The second case may
not apply to an adsorption-resistant surface like SiPEG. However,
combinations of parameters and that fitted the initial adsorption
kinetics did not predict the low saturation level of protein observed
in our experiments. One possible explanation is that FN adsorbed to
a small number of defects in the SiPEG monolayer, which could explain
both the rapid initial adsorption and the small amount of adsorbed
protein when the surface is saturated. If this were the case, a site-limited
adsorption model like the Langmuir model may be better for modeling
adsorption on SiPEG. Our prototype instrument did not have the sensitivity
to perform a more thorough study of adsorption on SiPEG at low solution
concentrations. Future systems based on whispering gallery mode technology
have the potential to study the adsorption of proteins on SiPEG surfaces
in greater detail, which could lead to greater understanding as to
why SiPEG resists protein adsorption. 
\end{document}
